\documentclass[8pt,aspectratio=169]{beamer} 
% Paquetes
\usepackage[utf8]{inputenc}
\usepackage[spanish]{babel} % (puedes cambiar a spanish si quieres hyphenation en español)

\usepackage{amsmath, amssymb, bm}
\usepackage{booktabs}
\usepackage{csquotes}
\usepackage{hyperref}


% Paquetes

\usepackage{amsmath,amssymb,amsthm}
\usepackage{graphicx}
\usepackage{booktabs}
\usepackage{multirow}
\usepackage{xcolor}
\usepackage{tikz}
\usepackage{algorithm}
\usepackage{algorithmic}
\usepackage{listings}


% Tema y colores
\usepackage{fontspec}
\setmainfont{EB Garamond}[
  UprightFont = * Regular,
  ItalicFont = * Italic,
  BoldFont = * SemiBold,
  BoldItalicFont = * SemiBold Italic,
]
\setsansfont{EB Garamond}[
  UprightFont = * Regular,
  ItalicFont = * Italic,
  BoldFont = * SemiBold,
  BoldItalicFont = * SemiBold Italic,
]
\usetheme{metropolis}

\usecolortheme{default}
\definecolor{primarycolor}{HTML}{0b0a40}
\setbeamercolor{structure}{fg=blue!70!black}
\setbeamercolor{palette primary}{bg=primarycolor,fg=white} 
\setbeamercolor{palette secondary}{bg=blue!60!black,fg=white}
\setbeamercolor{palette tertiary}{bg=blue!50!black,fg=white}
\setbeamercolor{background canvas}{bg=white}


% Configuración de listings para código Python
\lstset{
    language=Python,
    basicstyle=\tiny\ttfamily,
    keywordstyle=\color{blue},
    commentstyle=\color{gray},
    stringstyle=\color{red},
    showstringspaces=false,
    breaklines=true,
    frame=single,
    numbers=left,
    numberstyle=\tiny\color{gray}
}

% set root directory for images at 09 Metropolis-Hastings para Inferencia Bayesiana en Black-Scholes/figures
\graphicspath{{../figures/}}

% Información del documento
\title[Metropolis-Hastings: Opciones Asiáticas]{Implementación del Método de Metropolis--Hastings\\en una Aplicación Actuarial}
\subtitle{Valoración Bayesiana de Opciones Asiáticas}
\author[Equipo]{Temas Selectos 1: Simulación Actuarial}
\institute{Equipo: 1}
\date{\today}

% Logo (opcional - comentar si no se usa)
% \logo{\includegraphics[height=0.8cm]{logo_universidad.pdf}}

\begin{document}

% ============================================================================
% PORTADA
% ============================================================================
\begin{frame}
\titlepage
\end{frame}

% ============================================================================
% TABLA DE CONTENIDOS
% ============================================================================
\begin{frame}{Contenido}
\tableofcontents
\end{frame}

% ============================================================================
\section{Introducción y Motivación}
% ============================================================================

\begin{frame}{Contexto: Opciones Asiáticas}

\begin{columns}[T]
\begin{column}{0.55\textwidth}
\textbf{¿Qué son?}
\begin{itemize}
    \item Derivados cuyo \textit{payoff} depende del \alert{promedio} del precio del activo
    \item Más comunes que opciones europeas en ciertos mercados
\end{itemize}

\vspace{0.3cm}
\textbf{Payoff (Call asiática):}
\begin{equation*}
\text{Payoff} = \max\left(\bar{S}_T - K, 0\right)
\end{equation*}

donde $\bar{S}_T = \frac{1}{T}\int_0^T S_t \, dt$ (promedio continuo)
\end{column}

\begin{column}{0.42\textwidth}
\textbf{Aplicaciones:}
\begin{itemize}
    \item \textcolor{blue}{Commodities}: Petróleo, gas, metales
    \item \textcolor{blue}{Cobertura}: Flujos continuos
    \item \textcolor{blue}{Gestión de riesgo}: Reduce volatilidad
\end{itemize}

\vspace{0.3cm}
\begin{block}{Ventaja clave}
Menor costo que opciones europeas
\end{block}
\end{column}
\end{columns}

\end{frame}

% ============================================================================

\begin{frame}{Problema Actuarial}

\begin{alertblock}{Desafío Principal}
Valorar opciones asiáticas cuando los parámetros del proceso estocástico ($\mu, \sigma$) son \alert{desconocidos} e inciertos.
\end{alertblock}

\vspace{0.5cm}

\begin{columns}[T]
\begin{column}{0.48\textwidth}
\textbf{Enfoque Clásico (MLE):}
\begin{itemize}
    \item[$\times$] Estimaciones puntuales
    \item[$\times$] Ignora incertidumbre
    \item[$\times$] Subestima riesgo
\end{itemize}
\end{column}

\begin{column}{0.48\textwidth}
\textbf{Enfoque Bayesiano (MCMC):}
\begin{itemize}
    \item[\checkmark] Distribuciones completas
    \item[\checkmark] Cuantifica incertidumbre
    \item[\checkmark] Propaga riesgo al precio
\end{itemize}
\end{column}
\end{columns}

\vspace{0.5cm}

\begin{block}{Objetivo del Proyecto}
Implementar Metropolis--Hastings para inferencia bayesiana de parámetros y valoración robusta de opciones asiáticas.
\end{block}

\end{frame}

% ============================================================================
\section{Modelo Estadístico}
% ============================================================================

\begin{frame}{Modelo del Activo Subyacente}

\textbf{Proceso estocástico: Geometric Brownian Motion (GBM)}

\begin{equation*}
dS_t = \mu S_t \, dt + \sigma S_t \, dW_t
\end{equation*}

\begin{itemize}
    \item $\mu$: \textcolor{blue}{drift} (retorno esperado anualizado)
    \item $\sigma$: \textcolor{blue}{volatilidad} (desviación estándar anualizada)
    \item $W_t$: Movimiento Browniano estándar
\end{itemize}

\vspace{0.3cm}

\textbf{Discretización (Lema de Itô):}

Aplicando a $\log S_t$:
\begin{equation*}
r_t := \log\left(\frac{S_{t+\Delta t}}{S_t}\right) \sim N\left(\left(\mu - \frac{\sigma^2}{2}\right)\Delta t, \sigma^2 \Delta t\right)
\end{equation*}

\begin{alertblock}{Parámetros a estimar}
$\theta = (\mu, \sigma)$ usando datos de log-retornos observados
\end{alertblock}

\end{frame}

% ============================================================================

\begin{frame}{Modelo Bayesiano}

\begin{block}{Verosimilitud}
Dada una muestra de $n$ log-retornos $\mathbf{r} = (r_1, \ldots, r_n)$:
\begin{equation*}
p(\mathbf{r} \mid \mu, \sigma) = \prod_{i=1}^n \frac{1}{\sqrt{2\pi \sigma^2 \Delta t}} \exp\left(-\frac{(r_i - m)^2}{2\sigma^2 \Delta t}\right)
\end{equation*}
donde $m = \left(\mu - \frac{\sigma^2}{2}\right)\Delta t$
\end{block}

%\vspace{0.3cm}

\begin{columns}[T]
\begin{column}{0.48\textwidth}
\textbf{Priors (no informativos):}
\begin{align*}
\mu &\sim N(0, 1) \\
\sigma &\sim \text{InvGamma}(2, 0.1)
\end{align*}
\end{column}

\begin{column}{0.48\textwidth}
\textbf{Posterior (Teorema de Bayes):}
\begin{equation*}
p(\mu, \sigma \mid \mathbf{r}) \propto p(\mathbf{r} \mid \mu, \sigma) \times p(\mu) \times p(\sigma)
\end{equation*}
\end{column}
\end{columns}

%\vspace{0.3cm}

\begin{alertblock}{Problema}
\alert{No tiene forma cerrada} $\Rightarrow$ Requiere métodos MCMC
\end{alertblock}

\end{frame}

% ============================================================================
\section{Método de Metropolis--Hastings}
% ============================================================================

\begin{frame}{Fundamento del Algoritmo}

\textbf{Objetivo:} Generar muestras de la distribución posterior $p(\theta \mid y)$ cuando solo conocemos $p(\theta \mid y)$ hasta una constante normalizante.

\vspace{0.5cm}

\begin{block}{Idea Central}
Construir una \alert{cadena de Markov} cuya distribución estacionaria sea la posterior deseada.
\end{block}

\vspace{0.3cm}

\textbf{Componentes clave:}
\begin{enumerate}
    \item \textbf{Distribución objetivo:} $\pi(\theta) = p(\theta \mid y)$ (posterior)
    \item \textbf{Distribución de propuesta:} $q(\theta^* \mid \theta^{(t)})$
    \item \textbf{Razón de aceptación:} $\alpha$ que garantiza balance detallado
\end{enumerate}

\vspace{0.3cm}

\begin{alertblock}{Propiedad Clave}
Bajo condiciones regulares, la cadena converge a la distribución posterior sin importar el punto inicial.
\end{alertblock}

\end{frame}

% ============================================================================

\begin{frame}[fragile]{Algoritmo de Metropolis--Hastings}

\begin{algorithm}[H]
\caption{Metropolis--Hastings}
\begin{algorithmic}[1]
\STATE Inicializar $\theta^{(0)}$
\FOR{$t = 1$ to $N$}
    \STATE Proponer $\theta^* \sim q(\theta^* \mid \theta^{(t-1)})$
    \STATE Calcular razón de aceptación:
    \begin{equation*}
    \alpha = \min\left(1, \frac{p(\theta^* \mid y) \, q(\theta^{(t-1)} \mid \theta^)}{p(\theta^{(t-1)} \mid y) \, q(\theta^ \mid \theta^{(t-1)})}\right)
    \end{equation*}
    \STATE Generar $u \sim \text{Uniform}(0,1)$
    \IF{$u < \alpha$}
        \STATE $\theta^{(t)} = \theta^*$ \quad \textcolor{green}{(aceptar)}
    \ELSE
        \STATE $\theta^{(t)} = \theta^{(t-1)}$ \quad \textcolor{red}{(rechazar)}
    \ENDIF
\ENDFOR
\end{algorithmic}
\end{algorithm}

%\vspace{0.2cm}

\textbf{Propuesta Random Walk:} $\theta^* = \theta^{(t-1)} + \epsilon$, \quad $\epsilon \sim N(0, \Sigma)$

$\Rightarrow$ Simétrica: $q(\theta^* \mid \theta) = q(\theta \mid \theta^*)$ $\Rightarrow$ Se simplifica $\alpha$

\end{frame}

% ============================================================================

\begin{frame}{Configuración de la Implementación}

\begin{columns}[T]
\begin{column}{0.48\textwidth}
\textbf{Datos Simulados:}
\begin{itemize}
    \item $\mu_{\text{true}} = 0.08$
    \item $\sigma_{\text{true}} = 0.25$
    \item $S_0 = 100$
    \item $T = 1$ año
    \item $n = 252$ observaciones
\end{itemize}

%\vspace{0.3cm}

\textbf{Parámetros MCMC:}
\begin{itemize}
    \item Iteraciones: 30,000
    \item Burn-in: 5,000
    \item Propuesta SD: [0.01, 0.01]
\end{itemize}
\end{column}

\begin{column}{0.48\textwidth}
\begin{figure}
\centering
\includegraphics[width=\textwidth]{fig1_trayectoria_precios.pdf}
\caption{Trayectoria simulada del activo subyacente}
\end{figure}
\end{column}
\end{columns}

\begin{block}{Resultado}
Tasa de aceptación: \textbf{72.1\%} \quad \textcolor{green}{\checkmark Óptimo} (rango ideal: 40--70\%)
\end{block}

\end{frame}

% ============================================================================
\section{Resultados}
% ============================================================================

\begin{frame}{Diagnósticos MCMC: Convergencia}

\begin{figure}
\centering
\includegraphics[width=0.65\textwidth]{fig2_traceplots.pdf}
\caption{Traceplots de los parámetros $\mu$ y $\sigma$. Convergencia alcanzada después del burn-in.}
\end{figure}

\begin{itemize}
    \item \textcolor{green}{\checkmark} Cadena de $\sigma$ mezcla bien
    \item \textcolor{orange}{!} Cadena de $\mu$ muestra alta autocorrelación (exploración lenta)
\end{itemize}

\end{frame}

% ============================================================================

\begin{frame}{Distribuciones Posteriores}

\begin{figure}
\centering
\includegraphics[width=0.85\textwidth]{fig3_distribuciones_posteriores.pdf}
\caption{Distribuciones posteriores marginales con intervalos creíbles del 95\%}
\end{figure}

\vspace{-0.2cm}

\begin{itemize}
    \item \textbf{Volatilidad $\sigma$:} Estimación \alert{muy precisa}, IC estrecho
    \item \textbf{Drift $\mu$:} Mayor \alert{incertidumbre}, IC amplio (consistente con teoría)
\end{itemize}

\end{frame}

% ============================================================================

\begin{frame}{Resultados Numéricos}

\begin{table}
\centering
\caption{Estimaciones Bayesianas vs. Clásicas (MLE)}
\begin{tabular}{lccccc}
\toprule
\textbf{Parámetro} & \textbf{Verdadero} & \textbf{Media Post.} & \textbf{IC 95\%} & \textbf{MLE} \\
\midrule
$\mu$ (drift)      & 0.0800 & 0.2041 & [$-0.15$, $0.59$] & 0.0629 \\
$\sigma$ (vol.)    & 0.2500 & 0.2422 & [$0.22$, $0.26$]  & 0.2413 \\
\bottomrule
\end{tabular}
\end{table}

\vspace{0.3cm}

\begin{block}{Observaciones Clave}
\begin{itemize}
    \item MLE y media bayesiana casi \textbf{idénticos para $\sigma$}
    \item \alert{Gran diferencia en $\mu$}: refleja incertidumbre no capturada por MLE
    \item IC para $\mu$ muy amplio: difícil estimar retorno esperado con datos limitados
\end{itemize}
\end{block}

\end{frame}

% ============================================================================

\begin{frame}{Diagnósticos MCMC: Autocorrelación}

\begin{figure}
\centering
\includegraphics[width=0.85\textwidth]{fig4_autocorrelacion.pdf}
\caption{Funciones de autocorrelación para ambos parámetros}
\end{figure}

\begin{columns}[T]
\begin{column}{0.48\textwidth}
\textbf{Tamaño Efectivo (ESS):}
\begin{itemize}
    \item $\mu$: 258 / 25,000
    \item $\sigma$: 2,601 / 25,000
\end{itemize}
\end{column}

\begin{column}{0.48\textwidth}
\textbf{Implicación:}
\begin{itemize}
    \item[$\times$] Alta autocorr. en $\mu$
    \item[\checkmark] Buena mezcla en $\sigma$
\end{itemize}
\end{column}
\end{columns}

\end{frame}

% ============================================================================

\begin{frame}{Distribución Conjunta Posterior}

\begin{figure}
\centering
\includegraphics[width=0.40\textwidth]{fig6_joint_posterior.pdf}
\caption{Distribución conjunta posterior $(\mu, \sigma)$. Correlación débil entre parámetros.}
\end{figure}

\begin{itemize}
    \item \textcolor{red}{Estrella roja}: Valores verdaderos
    \item \textcolor{orange}{Círculo naranja}: Media posterior
    \item Nube de puntos: Región de alta probabilidad posterior
\end{itemize}

\end{frame}

% ============================================================================
\section{Aplicación: Valoración de Opciones}
% ============================================================================

\begin{frame}{Valoración de Opción Asiática}

\textbf{Especificación de la opción:}
\begin{itemize}
    \item Tipo: Call asiática (promedio aritmético continuo)
    \item $S_0 = 100$, $K = 100$ (at-the-money), $T = 1$ año, $r = 3\%$
\end{itemize}

%\vspace{0.3cm}

\textbf{Metodología:}
\begin{enumerate}
    \item Para cada muestra posterior $(\mu^{(i)}, \sigma^{(i)})$
    \item Simular 1,000 trayectorias del precio (Monte Carlo)
    \item Calcular precio de la opción
    \item Obtener distribución posterior del precio
\end{enumerate}

\vspace{0.3cm}

\begin{table}
\centering
\caption{Resultados de Valoración}
\begin{tabular}{lccc}
\toprule
\textbf{Método} & \textbf{Precio (\$)} & \textbf{SD/SE} & \textbf{IC 95\%} \\
\midrule
Parámetros Verdaderos & 7.74 & 0.10 & --- \\
Estimación MLE        & 7.16 & 0.10 & --- \\
Inferencia Bayesiana  & 14.11 & 9.76 & [2.39, 35.11] \\
\bottomrule
\end{tabular}
\end{table}

\end{frame}

% ============================================================================

\begin{frame}{Distribución Posterior del Precio}

\begin{figure}
\centering
\includegraphics[width=0.70\textwidth]{fig5_precio_opcion_posterior.pdf}
\caption{Distribución posterior del precio de la opción asiática. Bimodal y muy dispersa.}
\end{figure}

\begin{alertblock}{¡Enorme Incertidumbre!}
IC 95\%: [\$2.39, \$35.11] $\Rightarrow$ El precio podría ser hasta \alert{5 veces mayor} que la estimación MLE
\end{alertblock}

\end{frame}

% ============================================================================

\begin{frame}{Interpretación Actuarial}

\begin{block}{Implicaciones para Gestión de Capital}
\begin{enumerate}
    \item \textbf{MLE subestima riesgo}: Precio puntual \$7.16 ignora incertidumbre
    \item \textbf{Capital requerido}: Usar percentil 95\% (\$35.11) $\Rightarrow$ 5× mayor
    \item \textbf{Pricing conservador}: Percentil 75--90 (\$17--\$25) como precio de venta
    \item \textbf{Reservas}: Necesarias significativamente mayores que sugiere MLE
\end{enumerate}
\end{block}

\vspace{0.3cm}

\begin{columns}[T]
\begin{column}{0.48\textwidth}
\textbf{Medidas de Riesgo:}
\begin{itemize}
    \item VaR$_{95\%}$ = \$35.11
    \item ES$_{95\%}$ $\approx$ \$38
\end{itemize}
\end{column}

\begin{column}{0.48\textwidth}
\textbf{Fuente de incertidumbre:}
\begin{itemize}
    \item Principalmente: drift $\mu$
    \item $\mu$ alto $\Rightarrow$ trayectorias crecientes $\Rightarrow$ payoffs grandes
\end{itemize}
\end{column}
\end{columns}

\end{frame}

% ============================================================================

\begin{frame}{Comparación: Bayesiano vs. Clásico}

\begin{table}
\centering
\caption{Enfoque Bayesiano vs. MLE en Valoración}
\begin{tabular}{p{3cm}p{5cm}p{5cm}}
\toprule
\textbf{Aspecto} & \textbf{MLE (Clásico)} & \textbf{Bayesiano (MCMC)} \\
\midrule
Estimación & Puntual: \$7.16 & Distribución: \$14.11 $\pm$ \$9.76 \\
\midrule
Incertidumbre & No cuantificada & IC: [\$2.39, \$35.11] \\
\midrule
Capital requerido & \$7.16 & \$35.11 (perc. 95\%) \\
\midrule
Decisiones & Potencialmente riesgosas & Conservadoras y robustas \\
\midrule
Gestión de riesgo & Inadecuada & \textcolor{green}{Completa} \\
\bottomrule
\end{tabular}
\end{table}

\vspace{0.3cm}

\begin{alertblock}{Conclusión Clave}
El enfoque clásico \alert{subestima dramáticamente} el riesgo real de la posición.
\end{alertblock}

\end{frame}

% ============================================================================
\section{Conclusiones}
% ============================================================================

\begin{frame}{Conclusiones Principales}

\begin{block}{1. Éxito de la Implementación}
\begin{itemize}
    \item[\checkmark] Algoritmo M--H implementado \textbf{desde cero} en Python
    \item[\checkmark] Convergencia verificada con tasa de aceptación óptima (72\%)
    \item[\checkmark] Diagnósticos MCMC confirman calidad de las muestras
\end{itemize}
\end{block}

\vspace{0.2cm}

\begin{block}{2. Valor del Enfoque Bayesiano}
\begin{itemize}
    \item[\checkmark] \textbf{Cuantifica incertidumbre} paramétrica completa
    \item[\checkmark] \textbf{Revela riesgos ocultos} no capturados por MLE
    \item[\checkmark] \textbf{Permite decisiones robustas} en gestión de capital
\end{itemize}
\end{block}

\vspace{0.2cm}

\begin{alertblock}{3. Implicación Actuarial Crítica}
La valoración clásica (MLE) puede llevar a \alert{reservas insuficientes} y \alert{capital inadecuado}, exponiendo a la institución a riesgos catastróficos.
\end{alertblock}

\end{frame}

% ============================================================================

\begin{frame}{Limitaciones y Trabajo Futuro}

\begin{columns}[T]
\begin{column}{0.48\textwidth}
\textbf{Limitaciones:}
\begin{itemize}
    \item Alta autocorrelación en $\mu$
    \item Datos sintéticos (idealmente: datos reales de mercado)
    \item Modelo simple (GBM)
\end{itemize}

\vspace{0.3cm}

\textbf{Mejoras Posibles:}
\begin{itemize}
    \item Hamiltonian MC (HMC)
    \item Reparametrización
    \item Propuestas adaptativas
\end{itemize}
\end{column}

\begin{column}{0.48\textwidth}
\textbf{Extensiones:}
\begin{itemize}
    \item Volatilidad estocástica (Heston)
    \item Saltos (Merton)
    \item Opciones sobre canastas
    \item Modelos multivariados
\end{itemize}

\vspace{0.3cm}

\textbf{Aplicaciones Adicionales:}
\begin{itemize}
    \item Reserving bayesiano
    \item Modelos de credibilidad
    \item Riesgo de crédito
    \item Pérdidas agregadas
\end{itemize}
\end{column}
\end{columns}

\end{frame}

% ============================================================================

\begin{frame}{Relevancia Actuarial del MCMC}

\begin{block}{¿Por qué MCMC es esencial en Actuaría moderna?}
\begin{enumerate}
    \item \textbf{Complejidad creciente}: Modelos sin soluciones analíticas
    \item \textbf{Requerimientos regulatorios}: Solvencia II, ORSA
    \item \textbf{Gestión de riesgos}: Cuantificación completa de incertidumbre
    \item \textbf{Robustez}: Decisiones que resisten incertidumbre de modelo
\end{enumerate}
\end{block}

\vspace{0.3cm}

\textbf{Áreas de aplicación:}
\begin{itemize}
    \item Reservas técnicas (Chain Ladder bayesiano)
    \item Pricing de productos complejos
    \item Modelos de dependencia (cópulas)
    \item Riesgo operacional y catastrófico
    \item Stress testing y análisis de escenarios
\end{itemize}

\vspace{0.3cm}

\begin{alertblock}{Mensaje Final}
\textbf{Metropolis--Hastings} es una herramienta fundamental que todo actuario debe dominar para enfrentar los desafíos del siglo XXI.
\end{alertblock}

\end{frame}

% ============================================================================

\begin{frame}{Contribuciones del Proyecto}

\begin{block}{Logros Técnicos}
\begin{enumerate}
    \item Implementación completa de M--H sin librerías especializadas
    \item Aplicación a problema actuarial realista (opciones asiáticas)
    \item Análisis exhaustivo con 6 visualizaciones de calidad
    \item Comparación rigurosa: Bayesiano vs. Clásico
\end{enumerate}
\end{block}

%\vspace{0.3cm}

\begin{block}{Aportaciones Conceptuales}
\begin{itemize}
    \item Demostración de \textbf{subestimación de riesgo} en métodos puntuales
    \item Cuantificación de \textbf{impacto financiero} de incertidumbre paramétrica
    \item Metodología replicable para otros derivados y productos
\end{itemize}
\end{block}

%\vspace{0.3cm}

\begin{exampleblock}{Código y Documentación}
%Proyecto completo disponible con:
\begin{itemize}
    \item Código Python documentado (500+ líneas)
    \item Reporte técnico LaTeX (18 páginas)
    \item 9 referencias académicas
\end{itemize}
\end{exampleblock}

\end{frame}

% ============================================================================

\begin{frame}[standout]
\begin{center}
{\Huge \textbf{¡Gracias!}}

\vspace{1cm}

{\Large Preguntas y Comentarios}

\vspace{1.5cm}

\begin{tikzpicture}
\node[draw, rounded corners, fill=blue!20, text width=10cm, align=center] {
\textbf{Proyecto:} Método de Metropolis--Hastings\\
\textbf{Aplicación:} Valoración Bayesiana de Opciones Asiáticas\\
\textbf{Materia:} Temas Selectos 1 - Simulación Actuarial
};
\end{tikzpicture}

\end{center}
\end{frame}

% ============================================================================
% APÉNDICE (opcional)
% ============================================================================

\appendix

\begin{frame}[allowframebreaks]{Referencias}

\begin{thebibliography}{99}

\bibitem{metropolis1953}
Metropolis, N., Rosenbluth, A. W., Rosenbluth, M. N., Teller, A. H., \& Teller, E. (1953).
\textit{Equation of state calculations by fast computing machines}.
The Journal of Chemical Physics, 21(6), 1087--1092.

\bibitem{hastings1970}
Hastings, W. K. (1970).
\textit{Monte Carlo sampling methods using Markov chains and their applications}.
Biometrika, 57(1), 97--109.

\bibitem{robert2004}
Robert, C. P., \& Casella, G. (2004).
\textit{Monte Carlo Statistical Methods} (2nd ed.). Springer.

\bibitem{hull2018}
Hull, J. C. (2018).
\textit{Options, Futures, and Other Derivatives} (10th ed.). Pearson.

\bibitem{glasserman2004}
Glasserman, P. (2004).
\textit{Monte Carlo Methods in Financial Engineering}. Springer.

\bibitem{klugman2012}
Klugman, S. A., Panjer, H. H., \& Willmot, G. E. (2012).
\textit{Loss Models: From Data to Decisions} (4th ed.). Wiley.

\end{thebibliography}

\end{frame}

% ============================================================================

\begin{frame}{Apéndice: Detalles Técnicos del Código}

\textbf{Estructura del código Python (500+ líneas):}

\begin{enumerate}
    \item \textbf{Generación de datos}: Simulación GBM con parámetros conocidos
    \item \textbf{Función log-posterior}: Likelihood + Priors
    \item \textbf{Algoritmo M--H}: Implementación desde cero con Random Walk
    \item \textbf{Diagnósticos}: Traceplots, autocorrelación, ESS
    \item \textbf{Valoración}: Monte Carlo para opciones asiáticas
    \item \textbf{Visualización}: 6 gráficas de alta calidad
\end{enumerate}

\vspace{0.3cm}

\textbf{Herramientas utilizadas:}
\begin{itemize}
    \item NumPy (cálculos numéricos y álgebra lineal)
    \item SciPy (distribuciones de probabilidad)
    \item Matplotlib (visualización)
    \item Pandas (manejo de datos tabulares)
\end{itemize}

\end{frame}

% ============================================================================

\begin{frame}[fragile]{Apéndice: Fragmento de Código (Log-Posterior)}

\begin{lstlisting}[language=Python]
def log_posterior(theta, log_retornos, dt):
    """
    Calcula el log de la posterior (hasta constante normalizante)
    theta = [mu, sigma]
    """
    mu, sigma = theta
    
    # Restricción: sigma > 0
    if sigma <= 0:
        return -np.inf
    
    # Log-likelihood
    media_retornos = (mu - 0.5 * sigma**2) * dt
    var_retornos = sigma**2 * dt
    
    log_lik = -0.5 * len(log_retornos) * np.log(2 * np.pi * var_retornos)
    log_lik -= 0.5 * np.sum((log_retornos - media_retornos)**2) / var_retornos
    
    # Log-prior para mu: N(0, 1)
    log_prior_mu = -0.5 * np.log(2 * np.pi) - 0.5 * mu**2
    
    # Log-prior para sigma: InverseGamma(2, 0.1)
    alpha, beta = 2, 0.1
    log_prior_sigma = alpha * np.log(beta) - (alpha + 1) * np.log(sigma) - beta / sigma
    
    return log_lik + log_prior_mu + log_prior_sigma
\end{lstlisting}

\end{frame}

% ============================================================================

\begin{frame}{Apéndice: Fórmulas Adicionales}

\textbf{Razón de aceptación (Random Walk simétrico):}

Como $q(\theta^* \mid \theta^{(t)}) = q(\theta^{(t)} \mid \theta^*)$, se simplifica a:

\begin{equation*}
\alpha = \min\left(1, \frac{p(\theta^* \mid y)}{p(\theta^{(t)} \mid y)}\right) = \min\left(1, \exp\left(\log p(\theta^* \mid y) - \log p(\theta^{(t)} \mid y)\right)\right)
\end{equation*}

\vspace{0.5cm}

\textbf{Tamaño Efectivo de Muestra (ESS):}

Aproximación mediante autocorrelación:

\begin{equation*}
\text{ESS} \approx \frac{N}{1 + 2\sum_{k=1}^{K} \rho_k}
\end{equation*}

donde $\rho_k$ es la autocorrelación en lag $k$.

\end{frame}

\end{document}