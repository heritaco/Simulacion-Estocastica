\documentclass[9pt,letterpaper]{article} % Compila con lualatex o xelatex

% --- Layout & links ---
\usepackage[margin=4cm]{geometry}
\usepackage[hidelinks]{hyperref}

% Paquetes
\usepackage[utf8]{inputenc}
\usepackage[spanish]{babel} % (puedes cambiar a spanish si quieres hyphenation en español)
\usepackage{amsmath,amssymb,amsthm}
\usepackage{graphicx}
\usepackage{float}
\usepackage{booktabs}
\usepackage{multirow}
\usepackage{caption}
\usepackage{subcaption}
\usepackage{hyperref}
\usepackage[numbers,sort&compress]{natbib}
\usepackage{algorithm}
\usepackage{algorithmic}
\usepackage{xcolor}
\usepackage{setspace}

% Configuración
\onehalfspacing
\hypersetup{
    colorlinks=true,
    linkcolor=blue,
    citecolor=blue,
    urlcolor=blue
}

% Teoremas
\newtheorem{definition}{Definición}
\newtheorem{theorem}{Teorema}



% --- Fuentes (texto y matemáticas) ---
\usepackage{newtxmath}
\usepackage{fontspec}
\setmainfont{EB Garamond}[
    UprightFont = * Regular,
    ItalicFont = * Italic,
    BoldFont = * SemiBold,
    BoldItalicFont = * SemiBold Italic,
]


% --- Secciones y subsecciones ---
\usepackage{titlesec}
\newfontface\garamondbold{EB Garamond SemiBold}
\newfontface\garamondbolditalic{EB Garamond Bold Italic}
\titleformat{\section}
  {\garamondbold\Large}{\thesection}{1em}{}
\titleformat{\subsection}
  {\garamondbold\large}{\thesubsection}{1em}{}

% --- Encabezado con nombre de la sección ---
\usepackage{fancyhdr}
\pagestyle{fancy}
\fancyhf{}
\fancyhead[L]{\small\leftmark}
\fancyhead[R]{\small\thepage}
\renewcommand{\headrulewidth}{0.1pt}

% --- Espaciado ---
\usepackage{setspace}

% --- Otros paquetes útiles ---
\usepackage{tocloft}  % Para personalizar la tabla de contenidos
\usepackage{xcolor}   % Para colores en el código
\usepackage{booktabs} % Para tablas bonitas

% --- Código fuente ---
\usepackage{listings}
\lstset{
  basicstyle=\ttfamily\small,
  backgroundcolor=\color{gray!10},
  frame=single,
  breaklines=true
}

\hypersetup{
    colorlinks=true,
    linkcolor=black, % <- TOC + \ref links
    citecolor=blue,
    urlcolor=blue
}


\graphicspath{{../figures/}}

\begin{document}

% Portada
\begin{titlepage}
    \centering
    \vspace*{2cm}

    {\LARGE\bfseries Implementación del Método de Metropolis--Hastings\\[0.3cm]
    en una Aplicación Actuarial\par}

    \vspace{1.5cm}

    {\Large Valoración Bayesiana de Opciones Asiáticas\\
    mediante Inferencia MCMC\par}

    \vspace{2cm}

    {\large Temas Selectos 1: Simulación Actuarial\par}

    \vspace{1cm}

    {\large Equipo: 1\par}

    \vspace{1cm}

    {\large \today\par}

    \vfill
\end{titlepage}

\tableofcontents
\newpage

% ============================================================================
\section{Introducción y Contexto Actuarial}
% ============================================================================

\subsection{Motivación}

Las opciones asiáticas son instrumentos derivados cuyo \textit{payoff} depende del promedio del precio del activo subyacente durante la vida de la opción, en lugar del precio spot al vencimiento. Este detalle vuelve a estos contratos especialmente útiles cuando el riesgo relevante no es el precio final, sino la evolución promedio del subyacente. En práctica, estas opciones aparecen en:

\begin{itemize}
    \item \textbf{Mercados de commodities}: petróleo, gas natural y metales. El promedio reduce incentivos para manipular el precio en fechas específicas y refleja mejor el costo promedio de insumos.
    \item \textbf{Cobertura corporativa}: empresas con flujos de efectivo continuos que enfrentan riesgo de precio promedio (por ejemplo, aerolíneas con combustible).
    \item \textbf{Gestión de riesgos financieros}: suelen tener primas menores que opciones europeas estándar, porque el promedio suaviza la volatilidad efectiva.
\end{itemize}

\subsection{Problema Actuarial}

El problema central consiste en la \textbf{valoración} de opciones asiáticas cuando existe incertidumbre sobre los parámetros del proceso estocástico del activo subyacente. En particular:

\begin{itemize}
    \item El precio del activo sigue un Movimiento Browniano Geométrico (GBM).
    \item Los parámetros $\mu$ (drift) y $\sigma$ (volatilidad) son desconocidos y deben estimarse desde datos históricos.
    \item La valoración clásica mediante Máxima Verosimilitud (MLE) entrega estimaciones puntuales, pero ignora \textbf{riesgo de modelo} e incertidumbre paramétrica.
\end{itemize}

Desde una perspectiva actuarial, esta incertidumbre importa porque el \textit{payoff} es una función convexa del promedio; por lo tanto, la dispersión en los parámetros puede generar colas de riesgo que no se observan bajo un único estimador puntual.

\subsection{Enfoque Bayesiano}

La inferencia bayesiana permite:

\begin{enumerate}
    \item \textbf{Cuantificar la incertidumbre} sobre los parámetros mediante distribuciones posteriores $p(\mu,\sigma\mid \mathbf r)$.
    \item \textbf{Propagar esta incertidumbre} al precio de la opción mediante simulación posterior predictiva.
    \item \textbf{Obtener intervalos creíbles} que funcionen como medidas directas de riesgo de modelo.
\end{enumerate}

Dado que la posterior conjunta no tiene forma cerrada analítica, recurrimos a métodos de \textbf{Monte Carlo vía Cadenas de Markov (MCMC)}, específicamente al algoritmo de Metropolis--Hastings.

% ============================================================================
\section{Descripción del Método Metropolis--Hastings}
% ============================================================================

\subsection{Fundamento Teórico}

El método de Metropolis--Hastings \citep{metropolis1953equation, hastings1970monte} genera muestras de una distribución objetivo $\pi(\theta)$ cuando dicha distribución solo es conocida hasta una constante normalizante. En nuestro caso:

$$
\pi(\theta) = p(\mu,\sigma \mid \mathbf r) \propto p(\mathbf r\mid \mu,\sigma)\,p(\mu)\,p(\sigma),
$$

donde $\theta=(\mu,\sigma)$.

\begin{definition}[Distribución estacionaria]
Una cadena de Markov con matriz de transición $P$ tiene distribución estacionaria $\pi$ si satisface:
$$
\pi P = \pi.
$$
\end{definition}

La estrategia es construir $P$ de forma que la posterior sea estacionaria. Una condición suficiente es el \textbf{balance detallado}:
$$
\pi(\theta)\,P(\theta\to\theta')=\pi(\theta')\,P(\theta'\to\theta).
$$

\subsection{Algoritmo}

El núcleo del algoritmo es proponer un nuevo estado $\theta^\ast$ desde una distribución $q(\theta^\ast \mid \theta^{(t-1)})$ y aceptarlo con probabilidad:

$$
\alpha(\theta^{(t-1)},\theta^\ast)=
\min\!\left(1,\,
\frac{\pi(\theta^\ast)\,q(\theta^{(t-1)}\mid\theta^\ast)}
{\pi(\theta^{(t-1)})\,q(\theta^\ast\mid\theta^{(t-1)})}
\right).
$$

\begin{algorithm}[H]
\caption{Metropolis--Hastings}
\begin{algorithmic}[1]
\STATE Inicializar $\theta^{(0)}$
\FOR{$t = 1$ to $N$}
    \STATE Proponer $\theta^* \sim q(\theta^* \mid \theta^{(t-1)})$
    \STATE Calcular $\alpha(\theta^{(t-1)},\theta^*)$
    \STATE Generar $u \sim \text{Uniform}(0,1)$
    \IF{$u < \alpha$}
        \STATE $\theta^{(t)} = \theta^*$ \quad (aceptar)
    \ELSE
        \STATE $\theta^{(t)} = \theta^{(t-1)}$ \quad (rechazar)
    \ENDIF
\ENDFOR
\end{algorithmic}
\end{algorithm}

\subsection{Propuesta Random Walk}

En este proyecto usamos un esquema \textbf{Random Walk Metropolis}:

$$
\theta^\ast=\theta^{(t-1)}+\epsilon,\qquad \epsilon\sim N(0,\Sigma),
$$

lo que implica simetría $q(\theta^\ast\mid\theta)=q(\theta\mid\theta^\ast)$. Por tanto:

$$
\alpha(\theta^{(t-1)},\theta^\ast)=
\min\!\left(1,\,
\frac{\pi(\theta^\ast)}{\pi(\theta^{(t-1)})}
\right)
=
\min\!\left(1,\,
\exp\big(\log\pi(\theta^\ast)-\log\pi(\theta^{(t-1)})\big)
\right).
$$

Este detalle es clave para la implementación computacional: basta evaluar la log-posterior no normalizada.

% ============================================================================
\section{Modelo Estadístico/Bayesiano}
% ============================================================================

\subsection{Modelo del Activo Subyacente}

Suponemos que el precio del activo $S_t$ sigue un GBM bajo la medida física $\mathbb{P}$:

$$
dS_t = \mu S_t\,dt + \sigma S_t\,dW_t,
$$

donde $\mu$ representa retorno esperado anualizado, $\sigma$ la volatilidad anualizada y $W_t$ es Browniano estándar.

\subsection{Discretización y Log-retornos}

Aplicando el Lema de Itô a $\log S_t$:

$$
d(\log S_t)=\left(\mu-\frac{\sigma^2}{2}\right)dt+\sigma dW_t.
$$

En tiempo discreto con paso $\Delta t$:

$$
r_t:=\log\left(\frac{S_{t+\Delta t}}{S_t}\right)
\sim N\!\left(\left(\mu-\frac{\sigma^2}{2}\right)\Delta t,\ \sigma^2\Delta t\right).
$$

Esta formulación muestra explícitamente que $\sigma$ controla la varianza de los retornos, mientras que $\mu$ afecta solo la media, lo cual anticipa mayor dificultad para estimar drift con muestras finitas.

\subsection{Verosimilitud}

Sea $\mathbf r=(r_1,\dots,r_n)$ con $n$ log-retornos observados. Entonces:

\begin{align}
p(\mathbf r \mid \mu,\sigma)
&=\prod_{i=1}^n \frac{1}{\sqrt{2\pi\sigma^2\Delta t}}
\exp\!\left(
-\frac{(r_i-m)^2}{2\sigma^2\Delta t}
\right),\\
m &= \left(\mu-\frac{\sigma^2}{2}\right)\Delta t.
\end{align}

La log-verosimilitud, ignorando constantes, es:

$$
\log p(\mathbf r\mid\mu,\sigma)
=
-\frac{n}{2}\log(\sigma^2\Delta t)
-\frac{1}{2\sigma^2\Delta t}\sum_{i=1}^n (r_i-m)^2.
$$

\subsection{Distribuciones a Priori}

Usamos priors débiles:

\begin{align}
\mu &\sim N(0,1),\\
\sigma &\sim \text{InverseGamma}(2,0.1).
\end{align}

\textbf{Justificación actuarial}:
\begin{itemize}
    \item El prior para $\mu$ centrado en 0 refleja neutralidad direccional sin imponer tendencia fuerte.
    \item El prior para $\sigma$ favorece valores positivos y rangos plausibles (de orden 10\%–50\% anual), evitando pesos exagerados en volatilidades irreales.
\end{itemize}

\subsection{Posterior}

Por Bayes:

$$
p(\mu,\sigma\mid \mathbf r)\propto p(\mathbf r\mid \mu,\sigma)\,p(\mu)\,p(\sigma).
$$

Como la constante normalizante es intractable, M--H opera sobre la densidad no normalizada vía la log-posterior.

% ============================================================================
\section{Resultados Numéricos}
% ============================================================================

\subsection{Datos y Configuración}

Se simuló una trayectoria sintética con parámetros verdaderos:

$$
\mu_{\text{true}}=0.08,\qquad \sigma_{\text{true}}=0.25,\qquad S_0=100,
$$
durante $T=1$ año con $n=252$ observaciones diarias ($\Delta t=T/n$). La trayectoria observada arrojó:

\begin{itemize}
    \item Precio final $S_T=103.44$
    \item Precio promedio $\bar S = 92.65$
    \item Retorno anual observado $\log(S_T/S_0)/T = 0.0338$
\end{itemize}

El MCMC se ejecutó con 30,000 iteraciones, burn-in de 5,000 y propuesta con SD $[0.01,0.01]$.

\subsection{Estimaciones Bayesianas vs. Clásicas}

\begin{table}[H]
\centering
\caption{Resultados de Inferencia Bayesiana}
\label{tab:resultados}
\begin{tabular}{lcccccc}
\toprule
\textbf{Parámetro} & \textbf{Verdadero} & \textbf{Media Post.} & \textbf{Mediana Post.} & \textbf{IC 2.5\%} & \textbf{IC 97.5\%} & \textbf{MLE} \\
\midrule
$\mu$ (drift)       & 0.0800 & 0.2041 & 0.1641 & $-0.1542$ & 0.5946 & 0.0629 \\
$\sigma$ (vol.)     & 0.2500 & 0.2422 & 0.2418 & 0.2219 & 0.2650 & 0.2413 \\
\bottomrule
\end{tabular}
\end{table}

\textbf{Observaciones}:
\begin{enumerate}
    \item $\sigma$ se recupera con alta precisión: la posterior es estrecha y cercana al verdadero valor. Esto es coherente con que la varianza de retornos identifica bien la volatilidad.
    \item $\mu$ muestra intervalo creíble amplio: incluso con 252 retornos diarios, el drift anual es difícil de estimar porque su señal es pequeña frente al ruido.
    \item La MLE de $\sigma$ coincide casi exactamente con la media posterior; para $\mu$ aparecen diferencias relevantes debido a la asimetría/dispersión posterior.
\end{enumerate}

\subsection{Diagnósticos MCMC}

\begin{table}[H]
\centering
\caption{Diagnósticos del Algoritmo MCMC}
\label{tab:diagnosticos}
\begin{tabular}{lc}
\toprule
\textbf{Métrica} & \textbf{Valor} \\
\midrule
Tasa de aceptación & 0.721 \\
ESS$_\mu$ (tamaño efectivo) & 258 \\
ESS$_\sigma$ (tamaño efectivo) & 2,601 \\
Autocorr(1) $\mu$ & 0.9992 \\
Autocorr(1) $\sigma$ & 0.8059 \\
\bottomrule
\end{tabular}
\end{table}

\textbf{Interpretación ampliada}:
\begin{itemize}
    \item Una aceptación de 72\% es alta para random-walk; sugiere pasos pequeños y, por ende, autocorrelación elevada.
    \item La autocorrelación extremadamente alta de $\mu$ implica mezcla lenta: aunque hay 25{,}000 muestras post burn-in, la información efectiva equivale a unas pocas centenas.
    \item Para $\sigma$, la autocorrelación decae mucho más rápido: la cadena explora ese parámetro con mayor eficiencia.
\end{itemize}

\subsection{Análisis Visual}

\begin{figure}[H]
\centering
\includegraphics[width=0.95\textwidth]{fig1_trayectoria_precios.pdf}
\caption{Trayectoria simulada del precio del activo subyacente. El promedio aritmético ($\bar{S} = 92.65$) es inferior al strike ($K=100$), lo que afecta la valoración de la call asiática.}
\label{fig:trayectoria}
\end{figure}

\begin{figure}[H]
\centering
\includegraphics[width=0.95\textwidth]{fig2_traceplots.pdf}
\caption{Traceplots de $\mu$ y $\sigma$. Tras burn-in, $\sigma$ estaciona y mezcla bien; $\mu$ presenta persistencia y cambios lentos.}
\label{fig:traceplots}
\end{figure}

\begin{figure}[H]
\centering
\includegraphics[width=0.95\textwidth]{fig3_distribuciones_posteriores.pdf}
\caption{Posteriores marginales. $\mu$ es amplia y sesgada; $\sigma$ es concentrada y casi simétrica alrededor del verdadero valor.}
\label{fig:posteriores}
\end{figure}

\begin{figure}[H]
\centering
\includegraphics[width=0.95\textwidth]{fig4_autocorrelacion.pdf}
\caption{Autocorrelaciones. Persistencia elevada en $\mu$ (decaimiento lento); mezcla moderada en $\sigma$.}
\label{fig:autocorr}
\end{figure}

\begin{figure}[H]
\centering
\includegraphics[width=0.8\textwidth]{fig6_joint_posterior.pdf}
\caption{Posterior conjunta $(\mu,\sigma)$. Se aprecia región de alta densidad y proximidad de la media posterior a los valores verdaderos.}
\label{fig:joint}
\end{figure}

% ============================================================================
\section{Interpretación Actuarial: Valoración de Opción Asiática}
% ============================================================================

\subsection{Especificación de la Opción}

Se valora una \textbf{opción call asiática de promedio aritmético continuo} con:

\begin{itemize}
    \item $S_0=100$
    \item $K=100$ (at-the-money)
    \item madurez $T=1$ año
    \item tasa libre de riesgo $r=0.03$
\end{itemize}

Payoff:

$$
\text{Payoff}=\max(\bar S_T-K,0),\qquad
\bar S_T=\frac{1}{T}\int_0^T S_t\,dt.
$$

\subsection{Metodología de Valoración}

\begin{enumerate}
    \item \textbf{Enfoque bayesiano:} para cada muestra posterior $(\mu^{(i)},\sigma^{(i)})$, se simulan trayectorias GBM y se calcula un precio $C^{(i)}$ por Monte Carlo.
    \item \textbf{Enfoque MLE:} se usa el precio puntual $C_{\text{MLE}}$ evaluando en $(\hat\mu_{\text{MLE}},\hat\sigma_{\text{MLE}})$.
    \item \textbf{Referencia teórica:} precio $C_{\text{true}}$ con $(\mu_{\text{true}},\sigma_{\text{true}})$.
\end{enumerate}

La distribución posterior del precio se interpreta como:

$$
p(C\mid \mathbf r)
=
\int p(C\mid\mu,\sigma)\,p(\mu,\sigma\mid\mathbf r)\,d\mu\,d\sigma.
$$

\subsection{Resultados de Valoración}

\begin{table}[H]
\centering
\caption{Precio de la Opción Asiática Call}
\label{tab:valoracion}
\begin{tabular}{lccc}
\toprule
\textbf{Método} & \textbf{Precio Medio (\$)} & \textbf{SD/SE} & \textbf{IC 95\%} \\
\midrule
Parámetros Verdaderos & 7.74 & 0.10 & --- \\
Estimación MLE        & 7.16 & 0.10 & --- \\
Inferencia Bayesiana  & 14.11 & 9.76 & [2.39, 35.11] \\
\bottomrule
\end{tabular}
\end{table}

\begin{figure}[H]
\centering
\includegraphics[width=0.9\textwidth]{fig5_precio_opcion_posterior.pdf}
\caption{Posterior del precio. Dispersión y cola derecha pesada por incertidumbre de $\mu$.}
\label{fig:precio_opcion}
\end{figure}

\subsection{Implicaciones Actuariales (ampliadas)}

\begin{enumerate}
    \item \textbf{Riesgo de modelo explícito:} el precio no es un escalar sino una variable aleatoria inducida por $p(\mu,\sigma\mid \mathbf r)$. La MLE, al ser puntual, equivale a asumir conocimiento perfecto de parámetros.
    
    \item \textbf{Subestimación con MLE:} aunque $C_{\text{MLE}}=7.16$ es cercano a $C_{\text{true}}=7.74$, la posterior contiene escenarios plausibles donde el precio supera 30. Esto ocurre porque el payoff es convexo en la tendencia, amplificando valores altos de $\mu$.
    
    \item \textbf{Capital económico y reservas:} para una institución que vende estos contratos, reservar solo $C_{\text{MLE}}$ puede ser insuficiente. Un criterio prudente sería usar percentiles altos de $p(C\mid\mathbf r)$ como capital: por ejemplo, el 95\% sugiere un capital cercano a 35.
    
    \item \textbf{Pricing conservador:} en práctica actuarial/financiera, un precio de venta puede fijarse en percentiles 75–90 de la posterior (aprox. 17–25), incorporando margen de seguridad y apetito de riesgo.
    
    \item \textbf{Sensibilidad al drift:} el driver dominante de incertidumbre es $\mu$, no $\sigma$. Esto es estructural: derivados dependientes de promedios aritméticos son especialmente sensibles a la tendencia media de las trayectorias.
\end{enumerate}

\subsection{Análisis de Riesgo}

\textbf{Value-at-Risk (VaR) del precio}:
\begin{itemize}
    \item VaR$_{95\%} \approx 35.11$
    \item VaR$_{99\%}$ mayor a 40 (cola derecha pesada)
\end{itemize}

\textbf{Expected Shortfall}:
$$
\text{ES}_{95\%}=\mathbb{E}[C\mid C>\text{VaR}_{95\%}]\approx 38.
$$

Estas medidas muestran que el riesgo extremo está gobernado por la incertidumbre en tendencia. Para gestión de solvencia, ES es preferible a VaR por ser coherente bajo colas pesadas.

% ============================================================================
\section{Conclusiones}
% ============================================================================

\subsection{Síntesis de Resultados}

\begin{enumerate}
    \item Se implementó Metropolis--Hastings desde cero para inferir parámetros GBM bajo un marco bayesiano.
    \item La cadena converge, con aceptación alta (72\%), pero con mezcla lenta en $\mu$ reflejada en ESS bajo.
    \item $\sigma$ se estima con alta precisión (posterior estrecha, ESS alto); $\mu$ conserva incertidumbre inherente a la naturaleza de los datos financieros anuales.
    \item La valoración bayesiana produce una posterior del precio amplia, evidenciando riesgo de modelo ausente en MLE.
    \item Para decisiones actuariales prudentes, el precio debe considerarse distribucional y no puntual.
\end{enumerate}

\subsection{Ventajas del Enfoque Bayesiano}

\begin{itemize}
    \item \textbf{Cuantificación completa de incertidumbre:} se obtiene $p(\mu,\sigma\mid\mathbf r)$ y, por extensión, $p(C\mid\mathbf r)$.
    \item \textbf{Decisiones robustas:} permite pricing y capital basados en percentiles/ES coherentes con aversión al riesgo.
    \item \textbf{Actualización secuencial:} los priors se actualizan naturalmente al incorporar nueva información.
\end{itemize}

\subsection{Limitaciones y Trabajo Futuro}

\begin{enumerate}
    \item \textbf{Autocorrelación alta en $\mu$:} se puede mejorar con propuestas adaptativas, reparametrización o HMC/NUTS.
    \item \textbf{Datos sintéticos:} en aplicación real sería necesario calibrar con datos de mercado, donde hay colas pesadas, volatilidad estocástica y posibles saltos.
    \item \textbf{Extensiones:}
    \begin{itemize}
        \item Heston (volatilidad estocástica)
        \item Merton (saltos)
        \item Opciones asiáticas de promedio geométrico
        \item Opciones sobre canastas multivariadas
    \end{itemize}
    \item \textbf{Validación externa:} comparar precios con métodos PDE, Fourier o benchmarks de mercado para cuantificar sesgo y robustez.
\end{enumerate}

\subsection{Relevancia Actuarial}

Este proyecto muestra que MCMC es crucial en acturía moderna cuando:

\begin{itemize}
    \item Los modelos financieros carecen de soluciones analíticas.
    \item La regulación exige cuantificar incertidumbre (Solvencia II, ORSA).
    \item Las decisiones de pricing y reservas deben ser robustas ante riesgo de modelo.
\end{itemize}

Metropolis--Hastings, pese a su simplicidad, ofrece un puente práctico entre teoría estocástica, inferencia estadística y gestión prudente de derivados.

% ============================================================================
% REFERENCIAS
% ============================================================================
\newpage
\thispagestyle{empty}
\bibliographystyle{apalike}
\begin{thebibliography}{99}

\bibitem{metropolis1953equation}
Metropolis, N., Rosenbluth, A. W., Rosenbluth, M. N., Teller, A. H., \& Teller, E. (1953).
\textit{Equation of state calculations by fast computing machines}.
The Journal of Chemical Physics, 21(6), 1087--1092.

\bibitem{hastings1970monte}
Hastings, W. K. (1970).
\textit{Monte Carlo sampling methods using Markov chains and their applications}.
Biometrika, 57(1), 97--109.

\bibitem{robert2004monte}
Robert, C. P., \& Casella, G. (2004).
\textit{Monte Carlo Statistical Methods} (2nd ed.).
Springer.

\bibitem{hull2018options}
Hull, J. C. (2018).
\textit{Options, Futures, and Other Derivatives} (10th ed.).
Pearson.

\bibitem{glasserman2004monte}
Glasserman, P. (2004).
\textit{Monte Carlo Methods in Financial Engineering}.
Springer.

\bibitem{klugman2012loss}
Klugman, S. A., Panjer, H. H., \& Willmot, G. E. (2012).
\textit{Loss Models: From Data to Decisions} (4th ed.).
Wiley.

\bibitem{ntzoufras2011bayesian}
Ntzoufras, I. (2011).
\textit{Bayesian Modeling Using WinBUGS}.
Wiley.

\bibitem{scollnik2001implementation}
Scollnik, D. P. (2001).
\textit{Implementing a Bayesian analysis of the compound Poisson-gamma distribution using WinBUGS}.
Journal of Actuarial Practice, 9, 5--28.

\bibitem{vehtari2000bayesian}
Vehtari, A., \& Ojanen, J. (2012).
\textit{A survey of Bayesian predictive methods for model assessment, selection and comparison}.
Statistics Surveys, 6, 142--228.

\end{thebibliography}

\end{document}
