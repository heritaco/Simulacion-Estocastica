\documentclass[12pt]{article}
\usepackage{amsmath,amssymb,mathtools}
\usepackage[margin=1in]{geometry}

\begin{document}




\[
I=\int_{0}^{2}\frac{6x-5}{\sqrt{10+(3x-2)^{2}}}\,dx
\]

\paragraph{Change-of-variable theorem (definite integrals).}
If $g:[A,B]\to\mathbb{R}$ is $C^1$ and strictly monotone and $\phi$ is integrable, then
\[
\int_{A}^{B}\phi\!\big(g(x)\big)\,g'(x)\,dx
=\int_{g(A)}^{g(B)}\phi(t)\,dt,
\qquad t=g(x),\ dt=g'(x)\,dx.
\]
This is the rigorous form of $u$-substitution for definite integrals.

\paragraph{Solution linking $u$-sub and change of variable.}
Apply the affine substitution
\[
y=g(x)=3x-2,\qquad g'(x)=3,\qquad dx=\frac{dy}{3},\qquad g(0)=-2,\ g(2)=4.
\]
Rewrite the numerator to expose $y$:
\[
6x-5=2(3x-2)-1=2y-1.
\]
Thus
\begin{align*}
I
&=\frac{1}{3}\int_{-2}^{4}\frac{2y-1}{\sqrt{10+y^{2}}}\,dy
\\
&=\frac{1}{3}\left(\int_{-2}^{4}\frac{2y}{\sqrt{10+y^{2}}}\,dy
-\int_{-2}^{4}\frac{1}{\sqrt{10+y^{2}}}\,dy\right).
\end{align*}

\emph{First piece} ($u$-sub via the theorem): with $t=10+y^{2}$, $dt=2y\,dy$,
\[
\int \frac{2y}{\sqrt{10+y^{2}}}\,dy=\int t^{-1/2}\,dt=2\sqrt{t}=2\sqrt{10+y^{2}}.
\]
Evaluating on $[-2,4]$ and including the prefactor $1/3$ gives
\[
\frac{1}{3}\cdot 2\big(\sqrt{26}-\sqrt{14}\big)=\frac{2}{3}\big(\sqrt{26}-\sqrt{14}\big).
\]

\emph{Second piece} (standard primitive):
\[
\int\frac{dy}{\sqrt{a^{2}+y^{2}}}=\operatorname{arsinh}\!\Big(\frac{y}{a}\Big)
=\ln\!\Big(y+\sqrt{y^{2}+a^{2}}\Big)\quad(a>0).
\]
With $a=\sqrt{10}$,
\[
\frac{1}{3}\int_{-2}^{4}\frac{dy}{\sqrt{10+y^{2}}}
=\frac{1}{3}\left[\operatorname{arsinh}\!\Big(\frac{y}{\sqrt{10}}\Big)\right]_{-2}^{4}
=\frac{1}{3}\left(\operatorname{arsinh}\!\frac{4}{\sqrt{10}}+\operatorname{arsinh}\!\frac{2}{\sqrt{10}}\right).
\]

\emph{Combine:}
\[
\boxed{\,I=\frac{2}{3}\big(\sqrt{26}-\sqrt{14}\big)\;-\;\frac{1}{3}\!\left[\operatorname{arsinh}\!\frac{4}{\sqrt{10}}+\operatorname{arsinh}\!\frac{2}{\sqrt{10}}\right]\, }.
\]

\paragraph{Equivalent logarithmic form.}
Using $\operatorname{arsinh}(x)=\ln\!\big(x+\sqrt{1+x^{2}}\big)$,
\[
\operatorname{arsinh}\!\frac{4}{\sqrt{10}}+\operatorname{arsinh}\!\frac{2}{\sqrt{10}}
=\ln\!\left(\frac{4+\sqrt{26}}{\sqrt{10}}\right)+\ln\!\left(\frac{2+\sqrt{14}}{\sqrt{10}}\right)
=\ln\!\left(\frac{(4+\sqrt{26})(2+\sqrt{14})}{10}\right).
\]
Hence
\[
I=\frac{2}{3}\big(\sqrt{26}-\sqrt{14}\big)
-\frac{1}{3}\ln\!\left(\frac{(4+\sqrt{26})(2+\sqrt{14})}{10}\right).
\]

\paragraph{Derivative check.}
Define
\[
F(x)=\frac{2}{3}\sqrt{10+(3x-2)^{2}}-\frac{1}{3}\operatorname{arsinh}\!\frac{3x-2}{\sqrt{10}}.
\]
Then
\[
F'(x)=\frac{2}{3}\cdot\frac{(3x-2)\cdot 3}{\sqrt{10+(3x-2)^2}}
-\frac{1}{3}\cdot\frac{3}{\sqrt{10}}\cdot\frac{1}{\sqrt{1+\frac{(3x-2)^2}{10}}}
=\frac{6x-5}{\sqrt{10+(3x-2)^{2}}}.
\]
\end{document}
