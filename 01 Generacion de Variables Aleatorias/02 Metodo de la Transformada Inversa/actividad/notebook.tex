\documentclass[11pt]{article}

    \usepackage[breakable]{tcolorbox}
    \usepackage{parskip} % Stop auto-indenting (to mimic markdown behaviour)
    

    % Basic figure setup, for now with no caption control since it's done
    % automatically by Pandoc (which extracts ![](path) syntax from Markdown).
    \usepackage{graphicx}
    % Keep aspect ratio if custom image width or height is specified
    \setkeys{Gin}{keepaspectratio}
    % Maintain compatibility with old templates. Remove in nbconvert 6.0
    \let\Oldincludegraphics\includegraphics
    % Ensure that by default, figures have no caption (until we provide a
    % proper Figure object with a Caption API and a way to capture that
    % in the conversion process - todo).
    \usepackage{caption}
    \DeclareCaptionFormat{nocaption}{}
    \captionsetup{format=nocaption,aboveskip=0pt,belowskip=0pt}

    \usepackage{float}
    \floatplacement{figure}{H} % forces figures to be placed at the correct location
    \usepackage{xcolor} % Allow colors to be defined
    \usepackage{enumerate} % Needed for markdown enumerations to work
    \usepackage{geometry} % Used to adjust the document margins
    \usepackage{amsmath} % Equations
    \usepackage{amssymb} % Equations
    \usepackage{textcomp} % defines textquotesingle
    % Hack from http://tex.stackexchange.com/a/47451/13684:
    \AtBeginDocument{%
        \def\PYZsq{\textquotesingle}% Upright quotes in Pygmentized code
    }
    \usepackage{upquote} % Upright quotes for verbatim code
    \usepackage{eurosym} % defines \euro

    \usepackage{iftex}
    \ifPDFTeX
        \usepackage[T1]{fontenc}
        \IfFileExists{alphabeta.sty}{
              \usepackage{alphabeta}
          }{
              \usepackage[mathletters]{ucs}
              \usepackage[utf8x]{inputenc}
          }
    \else
        \usepackage{fontspec}
        \usepackage{unicode-math}
    \fi

    \usepackage{fancyvrb} % verbatim replacement that allows latex
    \usepackage{grffile} % extends the file name processing of package graphics
                         % to support a larger range
    \makeatletter % fix for old versions of grffile with XeLaTeX
    \@ifpackagelater{grffile}{2019/11/01}
    {
      % Do nothing on new versions
    }
    {
      \def\Gread@@xetex#1{%
        \IfFileExists{"\Gin@base".bb}%
        {\Gread@eps{\Gin@base.bb}}%
        {\Gread@@xetex@aux#1}%
      }
    }
    \makeatother
    \usepackage[Export]{adjustbox} % Used to constrain images to a maximum size
    \adjustboxset{max size={0.9\linewidth}{0.9\paperheight}}

    % The hyperref package gives us a pdf with properly built
    % internal navigation ('pdf bookmarks' for the table of contents,
    % internal cross-reference links, web links for URLs, etc.)
    \usepackage{hyperref}
    % The default LaTeX title has an obnoxious amount of whitespace. By default,
    % titling removes some of it. It also provides customization options.
    \usepackage{titling}
    \usepackage{longtable} % longtable support required by pandoc >1.10
    \usepackage{booktabs}  % table support for pandoc > 1.12.2
    \usepackage{array}     % table support for pandoc >= 2.11.3
    \usepackage{calc}      % table minipage width calculation for pandoc >= 2.11.1
    \usepackage[inline]{enumitem} % IRkernel/repr support (it uses the enumerate* environment)
    \usepackage[normalem]{ulem} % ulem is needed to support strikethroughs (\sout)
                                % normalem makes italics be italics, not underlines
    \usepackage{soul}      % strikethrough (\st) support for pandoc >= 3.0.0
    \usepackage{mathrsfs}
    

    
    % Colors for the hyperref package
    \definecolor{urlcolor}{rgb}{0,.145,.698}
    \definecolor{linkcolor}{rgb}{.71,0.21,0.01}
    \definecolor{citecolor}{rgb}{.12,.54,.11}

    % ANSI colors
    \definecolor{ansi-black}{HTML}{3E424D}
    \definecolor{ansi-black-intense}{HTML}{282C36}
    \definecolor{ansi-red}{HTML}{E75C58}
    \definecolor{ansi-red-intense}{HTML}{B22B31}
    \definecolor{ansi-green}{HTML}{00A250}
    \definecolor{ansi-green-intense}{HTML}{007427}
    \definecolor{ansi-yellow}{HTML}{DDB62B}
    \definecolor{ansi-yellow-intense}{HTML}{B27D12}
    \definecolor{ansi-blue}{HTML}{208FFB}
    \definecolor{ansi-blue-intense}{HTML}{0065CA}
    \definecolor{ansi-magenta}{HTML}{D160C4}
    \definecolor{ansi-magenta-intense}{HTML}{A03196}
    \definecolor{ansi-cyan}{HTML}{60C6C8}
    \definecolor{ansi-cyan-intense}{HTML}{258F8F}
    \definecolor{ansi-white}{HTML}{C5C1B4}
    \definecolor{ansi-white-intense}{HTML}{A1A6B2}
    \definecolor{ansi-default-inverse-fg}{HTML}{FFFFFF}
    \definecolor{ansi-default-inverse-bg}{HTML}{000000}

    % common color for the border for error outputs.
    \definecolor{outerrorbackground}{HTML}{FFDFDF}

    % commands and environments needed by pandoc snippets
    % extracted from the output of `pandoc -s`
    \providecommand{\tightlist}{%
      \setlength{\itemsep}{0pt}\setlength{\parskip}{0pt}}
    \DefineVerbatimEnvironment{Highlighting}{Verbatim}{commandchars=\\\{\}}
    % Add ',fontsize=\small' for more characters per line
    \newenvironment{Shaded}{}{}
    \newcommand{\KeywordTok}[1]{\textcolor[rgb]{0.00,0.44,0.13}{\textbf{{#1}}}}
    \newcommand{\DataTypeTok}[1]{\textcolor[rgb]{0.56,0.13,0.00}{{#1}}}
    \newcommand{\DecValTok}[1]{\textcolor[rgb]{0.25,0.63,0.44}{{#1}}}
    \newcommand{\BaseNTok}[1]{\textcolor[rgb]{0.25,0.63,0.44}{{#1}}}
    \newcommand{\FloatTok}[1]{\textcolor[rgb]{0.25,0.63,0.44}{{#1}}}
    \newcommand{\CharTok}[1]{\textcolor[rgb]{0.25,0.44,0.63}{{#1}}}
    \newcommand{\StringTok}[1]{\textcolor[rgb]{0.25,0.44,0.63}{{#1}}}
    \newcommand{\CommentTok}[1]{\textcolor[rgb]{0.38,0.63,0.69}{\textit{{#1}}}}
    \newcommand{\OtherTok}[1]{\textcolor[rgb]{0.00,0.44,0.13}{{#1}}}
    \newcommand{\AlertTok}[1]{\textcolor[rgb]{1.00,0.00,0.00}{\textbf{{#1}}}}
    \newcommand{\FunctionTok}[1]{\textcolor[rgb]{0.02,0.16,0.49}{{#1}}}
    \newcommand{\RegionMarkerTok}[1]{{#1}}
    \newcommand{\ErrorTok}[1]{\textcolor[rgb]{1.00,0.00,0.00}{\textbf{{#1}}}}
    \newcommand{\NormalTok}[1]{{#1}}

    % Additional commands for more recent versions of Pandoc
    \newcommand{\ConstantTok}[1]{\textcolor[rgb]{0.53,0.00,0.00}{{#1}}}
    \newcommand{\SpecialCharTok}[1]{\textcolor[rgb]{0.25,0.44,0.63}{{#1}}}
    \newcommand{\VerbatimStringTok}[1]{\textcolor[rgb]{0.25,0.44,0.63}{{#1}}}
    \newcommand{\SpecialStringTok}[1]{\textcolor[rgb]{0.73,0.40,0.53}{{#1}}}
    \newcommand{\ImportTok}[1]{{#1}}
    \newcommand{\DocumentationTok}[1]{\textcolor[rgb]{0.73,0.13,0.13}{\textit{{#1}}}}
    \newcommand{\AnnotationTok}[1]{\textcolor[rgb]{0.38,0.63,0.69}{\textbf{\textit{{#1}}}}}
    \newcommand{\CommentVarTok}[1]{\textcolor[rgb]{0.38,0.63,0.69}{\textbf{\textit{{#1}}}}}
    \newcommand{\VariableTok}[1]{\textcolor[rgb]{0.10,0.09,0.49}{{#1}}}
    \newcommand{\ControlFlowTok}[1]{\textcolor[rgb]{0.00,0.44,0.13}{\textbf{{#1}}}}
    \newcommand{\OperatorTok}[1]{\textcolor[rgb]{0.40,0.40,0.40}{{#1}}}
    \newcommand{\BuiltInTok}[1]{{#1}}
    \newcommand{\ExtensionTok}[1]{{#1}}
    \newcommand{\PreprocessorTok}[1]{\textcolor[rgb]{0.74,0.48,0.00}{{#1}}}
    \newcommand{\AttributeTok}[1]{\textcolor[rgb]{0.49,0.56,0.16}{{#1}}}
    \newcommand{\InformationTok}[1]{\textcolor[rgb]{0.38,0.63,0.69}{\textbf{\textit{{#1}}}}}
    \newcommand{\WarningTok}[1]{\textcolor[rgb]{0.38,0.63,0.69}{\textbf{\textit{{#1}}}}}
    \makeatletter
    \newsavebox\pandoc@box
    \newcommand*\pandocbounded[1]{%
      \sbox\pandoc@box{#1}%
      % scaling factors for width and height
      \Gscale@div\@tempa\textheight{\dimexpr\ht\pandoc@box+\dp\pandoc@box\relax}%
      \Gscale@div\@tempb\linewidth{\wd\pandoc@box}%
      % select the smaller of both
      \ifdim\@tempb\p@<\@tempa\p@
        \let\@tempa\@tempb
      \fi
      % scaling accordingly (\@tempa < 1)
      \ifdim\@tempa\p@<\p@
        \scalebox{\@tempa}{\usebox\pandoc@box}%
      % scaling not needed, use as it is
      \else
        \usebox{\pandoc@box}%
      \fi
    }
    \makeatother

    % Define a nice break command that doesn't care if a line doesn't already
    % exist.
    \def\br{\hspace*{\fill} \\* }
    % Math Jax compatibility definitions
    \def\gt{>}
    \def\lt{<}
    \let\Oldtex\TeX
    \let\Oldlatex\LaTeX
    \renewcommand{\TeX}{\textrm{\Oldtex}}
    \renewcommand{\LaTeX}{\textrm{\Oldlatex}}
    % Document parameters
    % Document title
    \title{notebook}
    
    
    
    
    
    
    
% Pygments definitions
\makeatletter
\def\PY@reset{\let\PY@it=\relax \let\PY@bf=\relax%
    \let\PY@ul=\relax \let\PY@tc=\relax%
    \let\PY@bc=\relax \let\PY@ff=\relax}
\def\PY@tok#1{\csname PY@tok@#1\endcsname}
\def\PY@toks#1+{\ifx\relax#1\empty\else%
    \PY@tok{#1}\expandafter\PY@toks\fi}
\def\PY@do#1{\PY@bc{\PY@tc{\PY@ul{%
    \PY@it{\PY@bf{\PY@ff{#1}}}}}}}
\def\PY#1#2{\PY@reset\PY@toks#1+\relax+\PY@do{#2}}

\@namedef{PY@tok@w}{\def\PY@tc##1{\textcolor[rgb]{0.73,0.73,0.73}{##1}}}
\@namedef{PY@tok@c}{\let\PY@it=\textit\def\PY@tc##1{\textcolor[rgb]{0.24,0.48,0.48}{##1}}}
\@namedef{PY@tok@cp}{\def\PY@tc##1{\textcolor[rgb]{0.61,0.40,0.00}{##1}}}
\@namedef{PY@tok@k}{\let\PY@bf=\textbf\def\PY@tc##1{\textcolor[rgb]{0.00,0.50,0.00}{##1}}}
\@namedef{PY@tok@kp}{\def\PY@tc##1{\textcolor[rgb]{0.00,0.50,0.00}{##1}}}
\@namedef{PY@tok@kt}{\def\PY@tc##1{\textcolor[rgb]{0.69,0.00,0.25}{##1}}}
\@namedef{PY@tok@o}{\def\PY@tc##1{\textcolor[rgb]{0.40,0.40,0.40}{##1}}}
\@namedef{PY@tok@ow}{\let\PY@bf=\textbf\def\PY@tc##1{\textcolor[rgb]{0.67,0.13,1.00}{##1}}}
\@namedef{PY@tok@nb}{\def\PY@tc##1{\textcolor[rgb]{0.00,0.50,0.00}{##1}}}
\@namedef{PY@tok@nf}{\def\PY@tc##1{\textcolor[rgb]{0.00,0.00,1.00}{##1}}}
\@namedef{PY@tok@nc}{\let\PY@bf=\textbf\def\PY@tc##1{\textcolor[rgb]{0.00,0.00,1.00}{##1}}}
\@namedef{PY@tok@nn}{\let\PY@bf=\textbf\def\PY@tc##1{\textcolor[rgb]{0.00,0.00,1.00}{##1}}}
\@namedef{PY@tok@ne}{\let\PY@bf=\textbf\def\PY@tc##1{\textcolor[rgb]{0.80,0.25,0.22}{##1}}}
\@namedef{PY@tok@nv}{\def\PY@tc##1{\textcolor[rgb]{0.10,0.09,0.49}{##1}}}
\@namedef{PY@tok@no}{\def\PY@tc##1{\textcolor[rgb]{0.53,0.00,0.00}{##1}}}
\@namedef{PY@tok@nl}{\def\PY@tc##1{\textcolor[rgb]{0.46,0.46,0.00}{##1}}}
\@namedef{PY@tok@ni}{\let\PY@bf=\textbf\def\PY@tc##1{\textcolor[rgb]{0.44,0.44,0.44}{##1}}}
\@namedef{PY@tok@na}{\def\PY@tc##1{\textcolor[rgb]{0.41,0.47,0.13}{##1}}}
\@namedef{PY@tok@nt}{\let\PY@bf=\textbf\def\PY@tc##1{\textcolor[rgb]{0.00,0.50,0.00}{##1}}}
\@namedef{PY@tok@nd}{\def\PY@tc##1{\textcolor[rgb]{0.67,0.13,1.00}{##1}}}
\@namedef{PY@tok@s}{\def\PY@tc##1{\textcolor[rgb]{0.73,0.13,0.13}{##1}}}
\@namedef{PY@tok@sd}{\let\PY@it=\textit\def\PY@tc##1{\textcolor[rgb]{0.73,0.13,0.13}{##1}}}
\@namedef{PY@tok@si}{\let\PY@bf=\textbf\def\PY@tc##1{\textcolor[rgb]{0.64,0.35,0.47}{##1}}}
\@namedef{PY@tok@se}{\let\PY@bf=\textbf\def\PY@tc##1{\textcolor[rgb]{0.67,0.36,0.12}{##1}}}
\@namedef{PY@tok@sr}{\def\PY@tc##1{\textcolor[rgb]{0.64,0.35,0.47}{##1}}}
\@namedef{PY@tok@ss}{\def\PY@tc##1{\textcolor[rgb]{0.10,0.09,0.49}{##1}}}
\@namedef{PY@tok@sx}{\def\PY@tc##1{\textcolor[rgb]{0.00,0.50,0.00}{##1}}}
\@namedef{PY@tok@m}{\def\PY@tc##1{\textcolor[rgb]{0.40,0.40,0.40}{##1}}}
\@namedef{PY@tok@gh}{\let\PY@bf=\textbf\def\PY@tc##1{\textcolor[rgb]{0.00,0.00,0.50}{##1}}}
\@namedef{PY@tok@gu}{\let\PY@bf=\textbf\def\PY@tc##1{\textcolor[rgb]{0.50,0.00,0.50}{##1}}}
\@namedef{PY@tok@gd}{\def\PY@tc##1{\textcolor[rgb]{0.63,0.00,0.00}{##1}}}
\@namedef{PY@tok@gi}{\def\PY@tc##1{\textcolor[rgb]{0.00,0.52,0.00}{##1}}}
\@namedef{PY@tok@gr}{\def\PY@tc##1{\textcolor[rgb]{0.89,0.00,0.00}{##1}}}
\@namedef{PY@tok@ge}{\let\PY@it=\textit}
\@namedef{PY@tok@gs}{\let\PY@bf=\textbf}
\@namedef{PY@tok@ges}{\let\PY@bf=\textbf\let\PY@it=\textit}
\@namedef{PY@tok@gp}{\let\PY@bf=\textbf\def\PY@tc##1{\textcolor[rgb]{0.00,0.00,0.50}{##1}}}
\@namedef{PY@tok@go}{\def\PY@tc##1{\textcolor[rgb]{0.44,0.44,0.44}{##1}}}
\@namedef{PY@tok@gt}{\def\PY@tc##1{\textcolor[rgb]{0.00,0.27,0.87}{##1}}}
\@namedef{PY@tok@err}{\def\PY@bc##1{{\setlength{\fboxsep}{\string -\fboxrule}\fcolorbox[rgb]{1.00,0.00,0.00}{1,1,1}{\strut ##1}}}}
\@namedef{PY@tok@kc}{\let\PY@bf=\textbf\def\PY@tc##1{\textcolor[rgb]{0.00,0.50,0.00}{##1}}}
\@namedef{PY@tok@kd}{\let\PY@bf=\textbf\def\PY@tc##1{\textcolor[rgb]{0.00,0.50,0.00}{##1}}}
\@namedef{PY@tok@kn}{\let\PY@bf=\textbf\def\PY@tc##1{\textcolor[rgb]{0.00,0.50,0.00}{##1}}}
\@namedef{PY@tok@kr}{\let\PY@bf=\textbf\def\PY@tc##1{\textcolor[rgb]{0.00,0.50,0.00}{##1}}}
\@namedef{PY@tok@bp}{\def\PY@tc##1{\textcolor[rgb]{0.00,0.50,0.00}{##1}}}
\@namedef{PY@tok@fm}{\def\PY@tc##1{\textcolor[rgb]{0.00,0.00,1.00}{##1}}}
\@namedef{PY@tok@vc}{\def\PY@tc##1{\textcolor[rgb]{0.10,0.09,0.49}{##1}}}
\@namedef{PY@tok@vg}{\def\PY@tc##1{\textcolor[rgb]{0.10,0.09,0.49}{##1}}}
\@namedef{PY@tok@vi}{\def\PY@tc##1{\textcolor[rgb]{0.10,0.09,0.49}{##1}}}
\@namedef{PY@tok@vm}{\def\PY@tc##1{\textcolor[rgb]{0.10,0.09,0.49}{##1}}}
\@namedef{PY@tok@sa}{\def\PY@tc##1{\textcolor[rgb]{0.73,0.13,0.13}{##1}}}
\@namedef{PY@tok@sb}{\def\PY@tc##1{\textcolor[rgb]{0.73,0.13,0.13}{##1}}}
\@namedef{PY@tok@sc}{\def\PY@tc##1{\textcolor[rgb]{0.73,0.13,0.13}{##1}}}
\@namedef{PY@tok@dl}{\def\PY@tc##1{\textcolor[rgb]{0.73,0.13,0.13}{##1}}}
\@namedef{PY@tok@s2}{\def\PY@tc##1{\textcolor[rgb]{0.73,0.13,0.13}{##1}}}
\@namedef{PY@tok@sh}{\def\PY@tc##1{\textcolor[rgb]{0.73,0.13,0.13}{##1}}}
\@namedef{PY@tok@s1}{\def\PY@tc##1{\textcolor[rgb]{0.73,0.13,0.13}{##1}}}
\@namedef{PY@tok@mb}{\def\PY@tc##1{\textcolor[rgb]{0.40,0.40,0.40}{##1}}}
\@namedef{PY@tok@mf}{\def\PY@tc##1{\textcolor[rgb]{0.40,0.40,0.40}{##1}}}
\@namedef{PY@tok@mh}{\def\PY@tc##1{\textcolor[rgb]{0.40,0.40,0.40}{##1}}}
\@namedef{PY@tok@mi}{\def\PY@tc##1{\textcolor[rgb]{0.40,0.40,0.40}{##1}}}
\@namedef{PY@tok@il}{\def\PY@tc##1{\textcolor[rgb]{0.40,0.40,0.40}{##1}}}
\@namedef{PY@tok@mo}{\def\PY@tc##1{\textcolor[rgb]{0.40,0.40,0.40}{##1}}}
\@namedef{PY@tok@ch}{\let\PY@it=\textit\def\PY@tc##1{\textcolor[rgb]{0.24,0.48,0.48}{##1}}}
\@namedef{PY@tok@cm}{\let\PY@it=\textit\def\PY@tc##1{\textcolor[rgb]{0.24,0.48,0.48}{##1}}}
\@namedef{PY@tok@cpf}{\let\PY@it=\textit\def\PY@tc##1{\textcolor[rgb]{0.24,0.48,0.48}{##1}}}
\@namedef{PY@tok@c1}{\let\PY@it=\textit\def\PY@tc##1{\textcolor[rgb]{0.24,0.48,0.48}{##1}}}
\@namedef{PY@tok@cs}{\let\PY@it=\textit\def\PY@tc##1{\textcolor[rgb]{0.24,0.48,0.48}{##1}}}

\def\PYZbs{\char`\\}
\def\PYZus{\char`\_}
\def\PYZob{\char`\{}
\def\PYZcb{\char`\}}
\def\PYZca{\char`\^}
\def\PYZam{\char`\&}
\def\PYZlt{\char`\<}
\def\PYZgt{\char`\>}
\def\PYZsh{\char`\#}
\def\PYZpc{\char`\%}
\def\PYZdl{\char`\$}
\def\PYZhy{\char`\-}
\def\PYZsq{\char`\'}
\def\PYZdq{\char`\"}
\def\PYZti{\char`\~}
% for compatibility with earlier versions
\def\PYZat{@}
\def\PYZlb{[}
\def\PYZrb{]}
\makeatother


    % For linebreaks inside Verbatim environment from package fancyvrb.
    \makeatletter
        \newbox\Wrappedcontinuationbox
        \newbox\Wrappedvisiblespacebox
        \newcommand*\Wrappedvisiblespace {\textcolor{red}{\textvisiblespace}}
        \newcommand*\Wrappedcontinuationsymbol {\textcolor{red}{\llap{\tiny$\m@th\hookrightarrow$}}}
        \newcommand*\Wrappedcontinuationindent {3ex }
        \newcommand*\Wrappedafterbreak {\kern\Wrappedcontinuationindent\copy\Wrappedcontinuationbox}
        % Take advantage of the already applied Pygments mark-up to insert
        % potential linebreaks for TeX processing.
        %        {, <, #, %, $, ' and ": go to next line.
        %        _, }, ^, &, >, - and ~: stay at end of broken line.
        % Use of \textquotesingle for straight quote.
        \newcommand*\Wrappedbreaksatspecials {%
            \def\PYGZus{\discretionary{\char`\_}{\Wrappedafterbreak}{\char`\_}}%
            \def\PYGZob{\discretionary{}{\Wrappedafterbreak\char`\{}{\char`\{}}%
            \def\PYGZcb{\discretionary{\char`\}}{\Wrappedafterbreak}{\char`\}}}%
            \def\PYGZca{\discretionary{\char`\^}{\Wrappedafterbreak}{\char`\^}}%
            \def\PYGZam{\discretionary{\char`\&}{\Wrappedafterbreak}{\char`\&}}%
            \def\PYGZlt{\discretionary{}{\Wrappedafterbreak\char`\<}{\char`\<}}%
            \def\PYGZgt{\discretionary{\char`\>}{\Wrappedafterbreak}{\char`\>}}%
            \def\PYGZsh{\discretionary{}{\Wrappedafterbreak\char`\#}{\char`\#}}%
            \def\PYGZpc{\discretionary{}{\Wrappedafterbreak\char`\%}{\char`\%}}%
            \def\PYGZdl{\discretionary{}{\Wrappedafterbreak\char`\$}{\char`\$}}%
            \def\PYGZhy{\discretionary{\char`\-}{\Wrappedafterbreak}{\char`\-}}%
            \def\PYGZsq{\discretionary{}{\Wrappedafterbreak\textquotesingle}{\textquotesingle}}%
            \def\PYGZdq{\discretionary{}{\Wrappedafterbreak\char`\"}{\char`\"}}%
            \def\PYGZti{\discretionary{\char`\~}{\Wrappedafterbreak}{\char`\~}}%
        }
        % Some characters . , ; ? ! / are not pygmentized.
        % This macro makes them "active" and they will insert potential linebreaks
        \newcommand*\Wrappedbreaksatpunct {%
            \lccode`\~`\.\lowercase{\def~}{\discretionary{\hbox{\char`\.}}{\Wrappedafterbreak}{\hbox{\char`\.}}}%
            \lccode`\~`\,\lowercase{\def~}{\discretionary{\hbox{\char`\,}}{\Wrappedafterbreak}{\hbox{\char`\,}}}%
            \lccode`\~`\;\lowercase{\def~}{\discretionary{\hbox{\char`\;}}{\Wrappedafterbreak}{\hbox{\char`\;}}}%
            \lccode`\~`\:\lowercase{\def~}{\discretionary{\hbox{\char`\:}}{\Wrappedafterbreak}{\hbox{\char`\:}}}%
            \lccode`\~`\?\lowercase{\def~}{\discretionary{\hbox{\char`\?}}{\Wrappedafterbreak}{\hbox{\char`\?}}}%
            \lccode`\~`\!\lowercase{\def~}{\discretionary{\hbox{\char`\!}}{\Wrappedafterbreak}{\hbox{\char`\!}}}%
            \lccode`\~`\/\lowercase{\def~}{\discretionary{\hbox{\char`\/}}{\Wrappedafterbreak}{\hbox{\char`\/}}}%
            \catcode`\.\active
            \catcode`\,\active
            \catcode`\;\active
            \catcode`\:\active
            \catcode`\?\active
            \catcode`\!\active
            \catcode`\/\active
            \lccode`\~`\~
        }
    \makeatother

    \let\OriginalVerbatim=\Verbatim
    \makeatletter
    \renewcommand{\Verbatim}[1][1]{%
        %\parskip\z@skip
        \sbox\Wrappedcontinuationbox {\Wrappedcontinuationsymbol}%
        \sbox\Wrappedvisiblespacebox {\FV@SetupFont\Wrappedvisiblespace}%
        \def\FancyVerbFormatLine ##1{\hsize\linewidth
            \vtop{\raggedright\hyphenpenalty\z@\exhyphenpenalty\z@
                \doublehyphendemerits\z@\finalhyphendemerits\z@
                \strut ##1\strut}%
        }%
        % If the linebreak is at a space, the latter will be displayed as visible
        % space at end of first line, and a continuation symbol starts next line.
        % Stretch/shrink are however usually zero for typewriter font.
        \def\FV@Space {%
            \nobreak\hskip\z@ plus\fontdimen3\font minus\fontdimen4\font
            \discretionary{\copy\Wrappedvisiblespacebox}{\Wrappedafterbreak}
            {\kern\fontdimen2\font}%
        }%

        % Allow breaks at special characters using \PYG... macros.
        \Wrappedbreaksatspecials
        % Breaks at punctuation characters . , ; ? ! and / need catcode=\active
        \OriginalVerbatim[#1,codes*=\Wrappedbreaksatpunct]%
    }
    \makeatother

    % Exact colors from NB
    \definecolor{incolor}{HTML}{303F9F}
    \definecolor{outcolor}{HTML}{D84315}
    \definecolor{cellborder}{HTML}{CFCFCF}
    \definecolor{cellbackground}{HTML}{F7F7F7}

    % prompt
    \makeatletter
    \newcommand{\boxspacing}{\kern\kvtcb@left@rule\kern\kvtcb@boxsep}
    \makeatother
    \newcommand{\prompt}[4]{
        {\ttfamily\llap{{\color{#2}[#3]:\hspace{3pt}#4}}\vspace{-\baselineskip}}
    }
    

    
    % Prevent overflowing lines due to hard-to-break entities
    \sloppy
    % Setup hyperref package
    \hypersetup{
      breaklinks=true,  % so long urls are correctly broken across lines
      colorlinks=true,
      urlcolor=urlcolor,
      linkcolor=linkcolor,
      citecolor=citecolor,
      }
    % Slightly bigger margins than the latex defaults
    
    \geometry{verbose,tmargin=1in,bmargin=1in,lmargin=1in,rmargin=1in}
    
    

\begin{document}
    
    \maketitle
    
    

    
    \begin{tcolorbox}[breakable, size=fbox, boxrule=1pt, pad at break*=1mm,colback=cellbackground, colframe=cellborder]
\prompt{In}{incolor}{6}{\boxspacing}
\begin{Verbatim}[commandchars=\\\{\}]
\PY{k+kn}{import}\PY{+w}{ }\PY{n+nn}{math}
\PY{k+kn}{import}\PY{+w}{ }\PY{n+nn}{random}
\PY{k+kn}{from}\PY{+w}{ }\PY{n+nn}{collections}\PY{+w}{ }\PY{k+kn}{import} \PY{n}{Counter}

\PY{k+kn}{import}\PY{+w}{ }\PY{n+nn}{numpy}\PY{+w}{ }\PY{k}{as}\PY{+w}{ }\PY{n+nn}{np}
\PY{k+kn}{import}\PY{+w}{ }\PY{n+nn}{matplotlib}\PY{n+nn}{.}\PY{n+nn}{pyplot}\PY{+w}{ }\PY{k}{as}\PY{+w}{ }\PY{n+nn}{plt}
\PY{k+kn}{import}\PY{+w}{ }\PY{n+nn}{seaborn}\PY{+w}{ }\PY{k}{as}\PY{+w}{ }\PY{n+nn}{sns}
\end{Verbatim}
\end{tcolorbox}

    \begin{tcolorbox}[breakable, size=fbox, boxrule=1pt, pad at break*=1mm,colback=cellbackground, colframe=cellborder]
\prompt{In}{incolor}{4}{\boxspacing}
\begin{Verbatim}[commandchars=\\\{\}]
\PY{k+kn}{import}\PY{+w}{ }\PY{n+nn}{matplotlib}\PY{+w}{ }\PY{k}{as}\PY{+w}{ }\PY{n+nn}{mpl}
\PY{n}{mpl}\PY{o}{.}\PY{n}{rcParams}\PY{o}{.}\PY{n}{update}\PY{p}{(}\PY{p}{\PYZob{}}
    \PY{l+s+s2}{\PYZdq{}}\PY{l+s+s2}{text.usetex}\PY{l+s+s2}{\PYZdq{}}\PY{p}{:} \PY{k+kc}{True}\PY{p}{,}          \PY{c+c1}{\PYZsh{} route all text through LaTeX}
    \PY{l+s+s2}{\PYZdq{}}\PY{l+s+s2}{pgf.texsystem}\PY{l+s+s2}{\PYZdq{}}\PY{p}{:} \PY{l+s+s2}{\PYZdq{}}\PY{l+s+s2}{xelatex}\PY{l+s+s2}{\PYZdq{}}\PY{p}{,}  \PY{c+c1}{\PYZsh{} or \PYZdq{}xelatex\PYZdq{}}
    \PY{l+s+s2}{\PYZdq{}}\PY{l+s+s2}{pgf.rcfonts}\PY{l+s+s2}{\PYZdq{}}\PY{p}{:} \PY{k+kc}{False}\PY{p}{,}         \PY{c+c1}{\PYZsh{} do not override with mpl fonts}
    \PY{l+s+s2}{\PYZdq{}}\PY{l+s+s2}{font.family}\PY{l+s+s2}{\PYZdq{}}\PY{p}{:} \PY{l+s+s2}{\PYZdq{}}\PY{l+s+s2}{serif}\PY{l+s+s2}{\PYZdq{}}\PY{p}{,}
    \PY{l+s+s2}{\PYZdq{}}\PY{l+s+s2}{text.latex.preamble}\PY{l+s+s2}{\PYZdq{}}\PY{p}{:} \PY{l+s+sa}{r}\PY{l+s+s2}{\PYZdq{}\PYZdq{}\PYZdq{}}
\PY{l+s+s2}{\PYZbs{}}\PY{l+s+s2}{usepackage}\PY{l+s+si}{\PYZob{}libertinust1math\PYZcb{}}
\PY{l+s+s2}{\PYZdq{}\PYZdq{}\PYZdq{}}\PY{p}{,}
\PY{p}{\PYZcb{}}\PY{p}{)}

\PY{k+kn}{import}\PY{+w}{ }\PY{n+nn}{matplotlib}\PY{n+nn}{.}\PY{n+nn}{pyplot}\PY{+w}{ }\PY{k}{as}\PY{+w}{ }\PY{n+nn}{plt}
\PY{n}{plt}\PY{o}{.}\PY{n}{plot}\PY{p}{(}\PY{p}{[}\PY{l+m+mi}{0}\PY{p}{,}\PY{l+m+mi}{1}\PY{p}{]}\PY{p}{,}\PY{p}{[}\PY{l+m+mi}{0}\PY{p}{,}\PY{l+m+mi}{1}\PY{p}{]}\PY{p}{)}
\PY{n}{plt}\PY{o}{.}\PY{n}{title}\PY{p}{(}\PY{l+s+sa}{r}\PY{l+s+s2}{\PYZdq{}}\PY{l+s+s2}{\PYZdl{}}\PY{l+s+s2}{\PYZbs{}}\PY{l+s+s2}{int\PYZus{}0\PYZca{}1 x\PYZca{}2}\PY{l+s+s2}{\PYZbs{}}\PY{l+s+s2}{,dx=}\PY{l+s+s2}{\PYZbs{}}\PY{l+s+s2}{tfrac13\PYZdl{}}\PY{l+s+s2}{\PYZdq{}}\PY{p}{)}
\PY{n}{plt}\PY{o}{.}\PY{n}{xlabel}\PY{p}{(}\PY{l+s+sa}{r}\PY{l+s+s2}{\PYZdq{}}\PY{l+s+s2}{\PYZdl{}}\PY{l+s+s2}{\PYZbs{}}\PY{l+s+s2}{alpha,}\PY{l+s+s2}{\PYZbs{}}\PY{l+s+s2}{ }\PY{l+s+s2}{\PYZbs{}}\PY{l+s+s2}{beta,}\PY{l+s+s2}{\PYZbs{}}\PY{l+s+s2}{ }\PY{l+s+s2}{\PYZbs{}}\PY{l+s+s2}{Gamma(z) ,X\PYZdl{}}\PY{l+s+s2}{\PYZdq{}}\PY{p}{)}
\PY{n}{plt}\PY{o}{.}\PY{n}{show}\PY{p}{(}\PY{p}{)}
\end{Verbatim}
\end{tcolorbox}

    \begin{center}
    \adjustimage{max size={0.9\linewidth}{0.9\paperheight}}{notebook_files/notebook_1_0.png}
    \end{center}
    { \hspace*{\fill} \\}
    
    \begin{tcolorbox}[breakable, size=fbox, boxrule=1pt, pad at break*=1mm,colback=cellbackground, colframe=cellborder]
\prompt{In}{incolor}{11}{\boxspacing}
\begin{Verbatim}[commandchars=\\\{\}]
\PY{k+kn}{from}\PY{+w}{ }\PY{n+nn}{IPython}\PY{n+nn}{.}\PY{n+nn}{display}\PY{+w}{ }\PY{k+kn}{import} \PY{n}{set\PYZus{}matplotlib\PYZus{}formats}
\PY{n}{set\PYZus{}matplotlib\PYZus{}formats}\PY{p}{(}\PY{l+s+s1}{\PYZsq{}}\PY{l+s+s1}{svg}\PY{l+s+s1}{\PYZsq{}}\PY{p}{)}                           \PY{c+c1}{\PYZsh{} or \PYZsq{}svg\PYZsq{},\PYZsq{}pdf\PYZsq{}}
\end{Verbatim}
\end{tcolorbox}

    \begin{Verbatim}[commandchars=\\\{\}, frame=single, framerule=2mm, rulecolor=\color{outerrorbackground}]
\textcolor{ansi-red}{---------------------------------------------------------------------------}
\textcolor{ansi-red}{ImportError}                               Traceback (most recent call last)
\textcolor{ansi-cyan}{Cell}\textcolor{ansi-cyan}{ }\textcolor{ansi-green}{In[11]}\textcolor{ansi-green}{, line 1}
\textcolor{ansi-green}{----> }\textcolor{ansi-green}{1} \def\tcRGB{\textcolor[RGB]}\expandafter\tcRGB\expandafter{\detokenize{0,135,0}}{\textbf{from}}\def\tcRGB{\textcolor[RGB]}\expandafter\tcRGB\expandafter{\detokenize{188,188,188}}{ }\textcolor{ansi-blue-intense}{\textbf{IPython}}\textcolor{ansi-blue-intense}{\textbf{.}}\textcolor{ansi-blue-intense}{\textbf{display}}\def\tcRGB{\textcolor[RGB]}\expandafter\tcRGB\expandafter{\detokenize{188,188,188}}{ }\def\tcRGB{\textcolor[RGB]}\expandafter\tcRGB\expandafter{\detokenize{0,135,0}}{\textbf{import}} set\_matplotlib\_formats
\textcolor{ansi-green}{      2} set\_matplotlib\_formats(\textcolor{ansi-yellow}{'}\textcolor{ansi-yellow}{svg}\textcolor{ansi-yellow}{'})

\textcolor{ansi-red}{ImportError}: cannot import name 'set\_matplotlib\_formats' from 'IPython.display' (c:\textbackslash{}Users\textbackslash{}herie\textbackslash{}miniconda3\textbackslash{}envs\textbackslash{}u\textbackslash{}Lib\textbackslash{}site-packages\textbackslash{}IPython\textbackslash{}display.py)
    \end{Verbatim}

    \begin{tcolorbox}[breakable, size=fbox, boxrule=1pt, pad at break*=1mm,colback=cellbackground, colframe=cellborder]
\prompt{In}{incolor}{7}{\boxspacing}
\begin{Verbatim}[commandchars=\\\{\}]
\PY{n}{color} \PY{o}{=} \PY{n}{sns}\PY{o}{.}\PY{n}{color\PYZus{}palette}\PY{p}{(}\PY{l+s+s2}{\PYZdq{}}\PY{l+s+s2}{muted}\PY{l+s+s2}{\PYZdq{}}\PY{p}{)}
\PY{n}{np}\PY{o}{.}\PY{n}{random}\PY{o}{.}\PY{n}{shuffle}\PY{p}{(}\PY{n}{color}\PY{p}{)}
\PY{n}{sns}\PY{o}{.}\PY{n}{set}\PY{p}{(}\PY{n}{style}\PY{o}{=}\PY{l+s+s2}{\PYZdq{}}\PY{l+s+s2}{whitegrid}\PY{l+s+s2}{\PYZdq{}}\PY{p}{,} \PY{n}{context}\PY{o}{=}\PY{l+s+s2}{\PYZdq{}}\PY{l+s+s2}{paper}\PY{l+s+s2}{\PYZdq{}}\PY{p}{,} \PY{n}{palette}\PY{o}{=}\PY{n}{color}\PY{p}{,} \PY{n}{font}\PY{o}{=}\PY{l+s+s2}{\PYZdq{}}\PY{l+s+s2}{Inconsolata}\PY{l+s+s2}{\PYZdq{}}\PY{p}{)}
\PY{n}{sns}\PY{o}{.}\PY{n}{color\PYZus{}palette}\PY{p}{(}\PY{p}{)}
\end{Verbatim}
\end{tcolorbox}

            \begin{tcolorbox}[breakable, size=fbox, boxrule=.5pt, pad at break*=1mm, opacityfill=0]
\prompt{Out}{outcolor}{7}{\boxspacing}
\begin{Verbatim}[commandchars=\\\{\}]
[(0.5490196078431373, 0.3803921568627451, 0.23529411764705882),
 (0.9333333333333333, 0.5215686274509804, 0.2901960784313726),
 (0.5098039215686274, 0.7764705882352941, 0.8862745098039215),
 (0.2823529411764706, 0.47058823529411764, 0.8156862745098039),
 (0.8352941176470589, 0.7333333333333333, 0.403921568627451),
 (0.8392156862745098, 0.37254901960784315, 0.37254901960784315),
 (0.8627450980392157, 0.49411764705882355, 0.7529411764705882),
 (0.5843137254901961, 0.4235294117647059, 0.7058823529411765),
 (0.41568627450980394, 0.8, 0.39215686274509803),
 (0.4745098039215686, 0.4745098039215686, 0.4745098039215686)]
\end{Verbatim}
\end{tcolorbox}
        
    \hypertarget{distribuciuxf3n-uniforme-ab}{%
\section{\texorpdfstring{Distribución Uniforme
\((a,b)\)}{Distribución Uniforme (a,b)}}\label{distribuciuxf3n-uniforme-ab}}

    Sea \textbf{\(U\sim\mathrm{Unif}(0,1)\)}. Si
\(X\sim\mathrm{Unif}(a,b)\), entonces su funcion de distribucion
acumulada es:

\[
F_X(x)=\frac{x-a}{b-a}\,\mathbf{1}_{[a,b]}(x) + \mathbf{1}_{(b,\infty)}(x)
\]

Encontrando la inversa:

\[
\begin{aligned}
F_X(x)=u &\quad\iff\quad \frac{x-a}{b-a}=u,\\
         &\quad\iff\quad x-a=(b-a)\,u,\\
         &\quad\iff\quad x=a+(b-a)\,u.
\end{aligned}
\]

Entonces:

\[
F_X^{-1}(u)=a+(b-a)\,u.
\]

    \hypertarget{distribuciuxf3n-discreta}{%
\section{Distribución Discreta}\label{distribuciuxf3n-discreta}}

    \hypertarget{a}{%
\subsection{a)}\label{a}}

    \[
f_X(x)=\mathbb P(X = x)=
\begin{cases}
0.4,& x =1,\\
0.3,& x =2,\\
0.2,& x =3,\\
0.1,& x =4,\\
0,& \text{en otro caso}
\end{cases}
\]

    \begin{tcolorbox}[breakable, size=fbox, boxrule=1pt, pad at break*=1mm,colback=cellbackground, colframe=cellborder]
\prompt{In}{incolor}{12}{\boxspacing}
\begin{Verbatim}[commandchars=\\\{\}]
\PY{n}{x\PYZus{}vals} \PY{o}{=} \PY{p}{[}\PY{l+m+mi}{1}\PY{p}{,} \PY{l+m+mi}{2}\PY{p}{,} \PY{l+m+mi}{3}\PY{p}{,} \PY{l+m+mi}{4}\PY{p}{]}
\PY{n}{pmf} \PY{o}{=} \PY{p}{[}\PY{l+m+mf}{0.4}\PY{p}{,} \PY{l+m+mf}{0.3}\PY{p}{,} \PY{l+m+mf}{0.2}\PY{p}{,} \PY{l+m+mf}{0.1}\PY{p}{]}

\PY{n}{plt}\PY{o}{.}\PY{n}{vlines}\PY{p}{(}\PY{n}{x\PYZus{}vals}\PY{p}{,} \PY{l+m+mi}{0}\PY{p}{,} \PY{n}{pmf}\PY{p}{,} \PY{n}{linestyles}\PY{o}{=}\PY{l+s+s1}{\PYZsq{}}\PY{l+s+s1}{\PYZhy{}\PYZhy{}}\PY{l+s+s1}{\PYZsq{}}\PY{p}{)}
\PY{n}{plt}\PY{o}{.}\PY{n}{plot}\PY{p}{(}\PY{n}{x\PYZus{}vals}\PY{p}{,} \PY{n}{pmf}\PY{p}{,} \PY{l+s+s1}{\PYZsq{}}\PY{l+s+s1}{o}\PY{l+s+s1}{\PYZsq{}}\PY{p}{)}
\PY{n}{plt}\PY{o}{.}\PY{n}{xlabel}\PY{p}{(}\PY{l+s+s1}{\PYZsq{}}\PY{l+s+s1}{\PYZdl{}X\PYZdl{}}\PY{l+s+s1}{\PYZsq{}}\PY{p}{)}
\PY{n}{plt}\PY{o}{.}\PY{n}{ylabel}\PY{p}{(}\PY{l+s+s1}{\PYZsq{}}\PY{l+s+s1}{\PYZdl{}f\PYZus{}X(x)\PYZdl{}}\PY{l+s+s1}{\PYZsq{}}\PY{p}{)}
\PY{n}{plt}\PY{o}{.}\PY{n}{savefig}\PY{p}{(}\PY{l+s+s1}{\PYZsq{}}\PY{l+s+s1}{pmf\PYZus{}plot.svg}\PY{l+s+s1}{\PYZsq{}}\PY{p}{)}
\PY{n}{plt}\PY{o}{.}\PY{n}{show}\PY{p}{(}\PY{p}{)}
\end{Verbatim}
\end{tcolorbox}

    \begin{center}
    \adjustimage{max size={0.9\linewidth}{0.9\paperheight}}{notebook_files/notebook_9_0.png}
    \end{center}
    { \hspace*{\fill} \\}
    
    \hypertarget{b}{%
\subsection{b)}\label{b}}

    \[
F_X(x)=\mathbb P(X\le x)=
\begin{cases}
0,& x<1,\\
0.4,& 1\le x<2,\\
0.7,& 2\le x<3,\\
0.9,& 3\le x<4,\\
1,& x\ge 4.
\end{cases}
\]

    \begin{tcolorbox}[breakable, size=fbox, boxrule=1pt, pad at break*=1mm,colback=cellbackground, colframe=cellborder]
\prompt{In}{incolor}{9}{\boxspacing}
\begin{Verbatim}[commandchars=\\\{\}]
\PY{n}{x\PYZus{}cdf} \PY{o}{=} \PY{p}{[}\PY{o}{\PYZhy{}}\PY{l+m+mi}{1}\PY{p}{,} \PY{l+m+mi}{1}\PY{p}{,} \PY{l+m+mi}{2}\PY{p}{,} \PY{l+m+mi}{3}\PY{p}{,} \PY{l+m+mi}{4}\PY{p}{,} \PY{l+m+mi}{5}\PY{p}{]}
\PY{n}{F\PYZus{}cdf} \PY{o}{=} \PY{p}{[}\PY{l+m+mi}{0}\PY{p}{,} \PY{l+m+mf}{0.4}\PY{p}{,} \PY{l+m+mf}{0.7}\PY{p}{,} \PY{l+m+mf}{0.9}\PY{p}{,} \PY{l+m+mi}{1}\PY{p}{,} \PY{l+m+mi}{1}\PY{p}{]}

\PY{n}{plt}\PY{o}{.}\PY{n}{step}\PY{p}{(}\PY{n}{x\PYZus{}cdf}\PY{p}{,} \PY{n}{F\PYZus{}cdf}\PY{p}{,} \PY{n}{where}\PY{o}{=}\PY{l+s+s1}{\PYZsq{}}\PY{l+s+s1}{post}\PY{l+s+s1}{\PYZsq{}}\PY{p}{)}
\PY{n}{plt}\PY{o}{.}\PY{n}{xlabel}\PY{p}{(}\PY{l+s+s1}{\PYZsq{}}\PY{l+s+s1}{x}\PY{l+s+s1}{\PYZsq{}}\PY{p}{)}
\PY{n}{plt}\PY{o}{.}\PY{n}{ylabel}\PY{p}{(}\PY{l+s+s1}{\PYZsq{}}\PY{l+s+s1}{\PYZdl{}F\PYZus{}X(x)\PYZdl{}}\PY{l+s+s1}{\PYZsq{}}\PY{p}{)}
\PY{n}{plt}\PY{o}{.}\PY{n}{show}\PY{p}{(}\PY{p}{)}
\end{Verbatim}
\end{tcolorbox}

    \begin{center}
    \adjustimage{max size={0.9\linewidth}{0.9\paperheight}}{notebook_files/notebook_12_0.png}
    \end{center}
    { \hspace*{\fill} \\}
    
    \hypertarget{c}{%
\subsection{c)}\label{c}}

    Sea \textbf{\(U\sim\mathrm{Unif}(0,1)\)}.

\[
\begin{aligned}
F_X(x)=u 
&\;\iff\; u\in(0,0.4] \;\Rightarrow\; x=1,\\
&\;\iff\; u\in(0.4,0.7] \;\Rightarrow\; x=2,\\
&\;\iff\; u\in(0.7,0.9] \;\Rightarrow\; x=3,\\
&\;\iff\; u\in(0.9,1] \;\Rightarrow\; x=4.
\end{aligned}
\]

Entonces:

\[
F_X^{-1}(u)=
\begin{cases}
1,& 0<u\le 0.4,\\
2,& 0.4<u\le 0.7,\\
3,& 0.7<u\le 0.9,\\
4,& 0.9<u\le 1.
\end{cases}
\]

    \begin{tcolorbox}[breakable, size=fbox, boxrule=1pt, pad at break*=1mm,colback=cellbackground, colframe=cellborder]
\prompt{In}{incolor}{5}{\boxspacing}
\begin{Verbatim}[commandchars=\\\{\}]
\PY{n}{u\PYZus{}vals} \PY{o}{=} \PY{p}{[}\PY{l+m+mi}{0}\PY{p}{,} \PY{l+m+mi}{0}\PY{p}{,} \PY{l+m+mf}{0.4}\PY{p}{,} \PY{l+m+mf}{0.7}\PY{p}{,} \PY{l+m+mf}{0.9}\PY{p}{,} \PY{l+m+mi}{1}\PY{p}{]}
\PY{n}{F\PYZus{}inv\PYZus{}vals} \PY{o}{=} \PY{p}{[}\PY{o}{\PYZhy{}}\PY{l+m+mi}{1}\PY{p}{,} \PY{l+m+mi}{1}\PY{p}{,} \PY{l+m+mi}{2}\PY{p}{,} \PY{l+m+mi}{3}\PY{p}{,} \PY{l+m+mi}{4}\PY{p}{,} \PY{l+m+mi}{5}\PY{p}{]}

\PY{n}{plt}\PY{o}{.}\PY{n}{step}\PY{p}{(}\PY{n}{u\PYZus{}vals}\PY{p}{,} \PY{n}{F\PYZus{}inv\PYZus{}vals}\PY{p}{,} \PY{n}{where}\PY{o}{=}\PY{l+s+s2}{\PYZdq{}}\PY{l+s+s2}{post}\PY{l+s+s2}{\PYZdq{}}\PY{p}{)}
\PY{n}{plt}\PY{o}{.}\PY{n}{xlabel}\PY{p}{(}\PY{l+s+s2}{\PYZdq{}}\PY{l+s+s2}{u}\PY{l+s+s2}{\PYZdq{}}\PY{p}{)}
\PY{n}{plt}\PY{o}{.}\PY{n}{ylabel}\PY{p}{(}\PY{l+s+s2}{\PYZdq{}}\PY{l+s+s2}{\PYZdl{}F\PYZus{}X\PYZca{}}\PY{l+s+s2}{\PYZob{}}\PY{l+s+s2}{\PYZhy{}1\PYZcb{}(u)\PYZdl{}}\PY{l+s+s2}{\PYZdq{}}\PY{p}{)}
\PY{n}{plt}\PY{o}{.}\PY{n}{show}\PY{p}{(}\PY{p}{)}
\end{Verbatim}
\end{tcolorbox}

    \begin{center}
    \adjustimage{max size={0.9\linewidth}{0.9\paperheight}}{notebook_files/notebook_15_0.png}
    \end{center}
    { \hspace*{\fill} \\}
    
    \hypertarget{d}{%
\subsection{d)}\label{d}}

    \begin{tcolorbox}[breakable, size=fbox, boxrule=1pt, pad at break*=1mm,colback=cellbackground, colframe=cellborder]
\prompt{In}{incolor}{7}{\boxspacing}
\begin{Verbatim}[commandchars=\\\{\}]
\PY{k}{def}\PY{+w}{ }\PY{n+nf}{F\PYZus{}inv}\PY{p}{(}\PY{n}{u}\PY{p}{)}\PY{p}{:}
    \PY{k}{if} \PY{n}{u} \PY{o}{\PYZlt{}}\PY{o}{=} \PY{l+m+mf}{0.4}\PY{p}{:}
        \PY{k}{return} \PY{l+m+mi}{1}
    \PY{k}{elif} \PY{n}{u} \PY{o}{\PYZlt{}}\PY{o}{=} \PY{l+m+mf}{0.7}\PY{p}{:}
        \PY{k}{return} \PY{l+m+mi}{2}
    \PY{k}{elif} \PY{n}{u} \PY{o}{\PYZlt{}}\PY{o}{=} \PY{l+m+mf}{0.9}\PY{p}{:}
        \PY{k}{return} \PY{l+m+mi}{3}
    \PY{k}{else}\PY{p}{:}
        \PY{k}{return} \PY{l+m+mi}{4}

\PY{n}{u\PYZus{}samples} \PY{o}{=} \PY{n}{np}\PY{o}{.}\PY{n}{random}\PY{o}{.}\PY{n}{uniform}\PY{p}{(}\PY{l+m+mi}{0}\PY{p}{,} \PY{l+m+mi}{1}\PY{p}{,} \PY{l+m+mi}{500}\PY{p}{)}
\PY{n}{x\PYZus{}samples} \PY{o}{=} \PY{p}{[}\PY{n}{F\PYZus{}inv}\PY{p}{(}\PY{n}{u}\PY{p}{)} \PY{k}{for} \PY{n}{u} \PY{o+ow}{in} \PY{n}{u\PYZus{}samples}\PY{p}{]}
\end{Verbatim}
\end{tcolorbox}

    \hypertarget{e}{%
\subsection{e)}\label{e}}

    \begin{tcolorbox}[breakable, size=fbox, boxrule=1pt, pad at break*=1mm,colback=cellbackground, colframe=cellborder]
\prompt{In}{incolor}{8}{\boxspacing}
\begin{Verbatim}[commandchars=\\\{\}]
\PY{n}{plt}\PY{o}{.}\PY{n}{hist}\PY{p}{(}\PY{n}{x\PYZus{}samples}\PY{p}{,} \PY{n}{bins}\PY{o}{=}\PY{n}{np}\PY{o}{.}\PY{n}{arange}\PY{p}{(}\PY{l+m+mf}{0.5}\PY{p}{,} \PY{l+m+mf}{5.5}\PY{p}{,} \PY{l+m+mi}{1}\PY{p}{)}\PY{p}{,} \PY{n}{density}\PY{o}{=}\PY{k+kc}{True}\PY{p}{,} \PY{n}{alpha}\PY{o}{=}\PY{l+m+mf}{0.3}\PY{p}{)}
\PY{n}{plt}\PY{o}{.}\PY{n}{vlines}\PY{p}{(}\PY{n}{x\PYZus{}vals}\PY{p}{,} \PY{l+m+mi}{0}\PY{p}{,} \PY{n}{pmf}\PY{p}{,} \PY{n}{linestyles}\PY{o}{=}\PY{l+s+s1}{\PYZsq{}}\PY{l+s+s1}{\PYZhy{}\PYZhy{}}\PY{l+s+s1}{\PYZsq{}}\PY{p}{,} \PY{n}{color}\PY{o}{=}\PY{n}{color}\PY{p}{[}\PY{l+m+mi}{1}\PY{p}{]}\PY{p}{)}
\PY{n}{plt}\PY{o}{.}\PY{n}{plot}\PY{p}{(}\PY{n}{x\PYZus{}vals}\PY{p}{,} \PY{n}{pmf}\PY{p}{,} \PY{l+s+s1}{\PYZsq{}}\PY{l+s+s1}{o}\PY{l+s+s1}{\PYZsq{}}\PY{p}{,} \PY{n}{color}\PY{o}{=}\PY{n}{color}\PY{p}{[}\PY{l+m+mi}{1}\PY{p}{]}\PY{p}{)}
\PY{n}{plt}\PY{o}{.}\PY{n}{xlabel}\PY{p}{(}\PY{l+s+s2}{\PYZdq{}}\PY{l+s+s2}{\PYZdl{}X\PYZdl{}}\PY{l+s+s2}{\PYZdq{}}\PY{p}{)}
\PY{n}{plt}\PY{o}{.}\PY{n}{ylabel}\PY{p}{(}\PY{l+s+s2}{\PYZdq{}}\PY{l+s+s2}{Densidad}\PY{l+s+s2}{\PYZdq{}}\PY{p}{)}
\PY{n}{plt}\PY{o}{.}\PY{n}{title}\PY{p}{(}\PY{l+s+s2}{\PYZdq{}}\PY{l+s+s2}{Histograma de la Variable Aleatoria X a partir de \PYZdl{}F\PYZus{}X\PYZca{}}\PY{l+s+s2}{\PYZob{}}\PY{l+s+s2}{\PYZhy{}1\PYZcb{}(u)\PYZdl{}}\PY{l+s+s2}{\PYZdq{}}\PY{p}{)}
\PY{n}{plt}\PY{o}{.}\PY{n}{show}\PY{p}{(}\PY{p}{)}
\end{Verbatim}
\end{tcolorbox}

    \begin{center}
    \adjustimage{max size={0.9\linewidth}{0.9\paperheight}}{notebook_files/notebook_19_0.png}
    \end{center}
    { \hspace*{\fill} \\}
    
    \hypertarget{exponencial-mathrmexplambda}{%
\section{\texorpdfstring{Exponencial
\(\mathrm{Exp}(\lambda)\)}{Exponencial \textbackslash mathrm\{Exp\}(\textbackslash lambda)}}\label{exponencial-mathrmexplambda}}

    Sea \textbf{\(U\sim\mathrm{Unif}(0,1)\)}. Si
\(X\sim\mathrm{Exp}(\lambda)\), entonces su funcion de distribucion
acumulada es:

\[
F_X(x)=\bigl(1-e^{-\lambda x}\bigr)\,\mathbf{1}_{[0,\infty)}(x)
\]

Encontrando la inversa:

\[
\begin{aligned}
F_X(x)=u &\;\iff\; 1-e^{-\lambda x}=u,\\
         &\;\iff\; e^{-\lambda x}=1-u,\\
         &\;\iff\; -\lambda x=\ln(1-u),\\
         &\;\iff\; x=-\frac{1}{\lambda}\,\ln(1-u).
\end{aligned}
\]

Entonces:

\[
F_X^{-1}(u)=-\frac{1}{\lambda}\,\ln(1-u).
\]

Por lo tanto, con \(U\sim\mathrm{Unif}(0,1)\),

\[
X=F_X^{-1}(U)=-\frac{1}{\lambda}\ln(1-U)\sim\mathrm{Exp}(\lambda).
\]

    \hypertarget{weibull-rlambda}{%
\section{\texorpdfstring{Weibull
\((r,\lambda)\)}{Weibull (r,\textbackslash lambda)}}\label{weibull-rlambda}}

    Sea \textbf{\(U\sim\mathrm{Unif}(0,1)\)}, con \(r>0\) y \(\lambda>0\).
Si \(X\sim\mathrm{Weibull}(r,\lambda)\), entonces su funcion de
distribucion acumulada es:

\[
F_X(x)=\bigl(1-e^{-(\lambda x)^r}\bigr)\,\mathbf{1}_{[0,\infty)}(x)
\]

Encontrando la inversa:

\[
\begin{aligned}
F_X(x)=u &\;\iff\; 1-e^{-(\lambda x)^r}=u,\\
         &\;\iff\; e^{-(\lambda x)^r}=1-u,\\
         &\;\iff\; -(\lambda x)^r=\ln(1-u),\\
         &\;\iff\; (\lambda x)^r=-\ln(1-u),\\
         &\;\iff\; x=\frac{1}{\lambda}\,\bigl[-\ln(1-u)\bigr]^{1/r}.
\end{aligned}
\]

Entonces:

\[
F_X^{-1}(u)=\frac{1}{\lambda}\,\bigl[-\ln(1-u)\bigr]^{1/r}.
\]

Por lo tanto, con \(U\sim\mathrm{Unif}(0,1)\):

\[
X=F_X^{-1}(U)=\frac{1}{\lambda}\,\bigl[-\ln(1-U)\bigr]^{1/r}\sim\mathrm{Weibull}(r,\lambda).
\]

    \hypertarget{cauchy-ab}{%
\section{\texorpdfstring{Cauchy
\((a,b)\)}{Cauchy (a,b)}}\label{cauchy-ab}}

    Sea \textbf{\(U\sim\mathrm{Unif}(0,1)\)}. Si
\(X\sim\mathrm{Cauchy}(a,b)\), entonces su funcion de distribucion
acumulada es:

\[
F_X(x)=\frac{1}{\pi}\arctan\!\Big(\frac{x-a}{b}\Big)+\frac{1}{2},\qquad x\in\mathbb{R},\; b>0.
\]

Encontrando la inversa:

\[
\begin{aligned}
F_X(x)=u 
&\;\iff\; \frac{1}{\pi}\arctan\!\Big(\frac{x-a}{b}\Big)+\frac{1}{2}=u,\\
&\;\iff\; \arctan\!\Big(\frac{x-a}{b}\Big)=\pi\!\left(u-\tfrac{1}{2}\right),\\
&\;\iff\; \frac{x-a}{b}=\tan\!\big(\pi(u-\tfrac{1}{2})\big),\\
&\;\iff\; x=a+b\,\tan\!\big(\pi(u-\tfrac{1}{2})\big).
\end{aligned}
\]

Entonces:

\[
F_X^{-1}(u)=a+b\,\tan\!\big(\pi(u-\tfrac{1}{2})\big).
\]

Por lo tanto,
\(X=F_X^{-1}(U)=a+b\,\tan(\pi(U-\tfrac{1}{2}))\sim\mathrm{Cauchy}(a,b)\).

    \hypertarget{pareto-i-ab}{%
\section{\texorpdfstring{Pareto I
\((a,b)\)}{Pareto I (a,b)}}\label{pareto-i-ab}}

    Sea \textbf{\(U\sim\mathrm{Unif}(0,1)\)}, con \(a>0\) y \(b>0\). Si
\(X\sim\mathrm{Pareto\ I}(a,b)\), entonces su funcion de distribucion
acumulada es:

\[
F_X(x)=\bigl(1-(b/x)^a\bigr)\,\mathbf{1}_{[b,\infty)}(x)
\]

Encontrando la inversa:

\[
\begin{aligned}
F_X(x)=u &\;\iff\; 1-\left(\frac{b}{x}\right)^a=u,\\
         &\;\iff\; \left(\frac{b}{x}\right)^a=1-u,\\
         &\;\iff\; \frac{b}{x}=(1-u)^{1/a},\\
         &\;\iff\; x=b\,(1-u)^{-1/a}.
\end{aligned}
\]

Entonces:

\[
F_X^{-1}(u)=b\,(1-u)^{-1/a}.
\]

Por lo tanto,

\[
X=F_X^{-1}(U)=b\,(1-U)^{-1/a}\sim\mathrm{Pareto\ I}(a,b).
\]

    \hypertarget{muxednimo-x_1minx_1dotsx_n}{%
\section{\texorpdfstring{Mínimo
\(X_{(1)}=\min\{X_1,\dots,X_n\}\)}{Mínimo X\_\{(1)\}=\textbackslash min\textbackslash\{X\_1,\textbackslash dots,X\_n\textbackslash\}}}\label{muxednimo-x_1minx_1dotsx_n}}

    Sea \textbf{\(U\sim\mathrm{Unif}(0,1)\)} y
\(X_{(1)}:=\min\{X_1,\dots,X_n\}\) con \(X_i\) i.i.d. de CDF \(F\).

\[
F_{X_{(1)}}(x)=\mathbb P(X_{(1)}\le x)
=1-\mathbb P(X_1>x,\dots,X_n>x)
=1-\bigl(1-F(x)\bigr)^n.
\]

\textbf{Encontrando la inversa:}

\[
\begin{aligned}
F_{X_{(1)}}(x)=u 
&\;\iff\; 1-\bigl(1-F(x)\bigr)^n=u,\\
&\;\iff\; \bigl(1-F(x)\bigr)^n=1-u,\\
&\;\iff\; 1-F(x)=(1-u)^{1/n},\\
&\;\iff\; F(x)=1-(1-u)^{1/n},\\
&\;\iff\; x=F^{-1}\!\bigl(1-(1-u)^{1/n}\bigr).
\end{aligned}
\]

Entonces:

\[
F_{X_{(1)}}^{-1}(u)=F^{-1}\!\bigl(1-(1-u)^{1/n}\bigr),\quad 0<u<1.
\]

    \hypertarget{mixta-xminym-con-ysimmathrmexplambda}{%
\section{\texorpdfstring{Mixta \(X=\min\{Y,M\}\) con
\(Y\sim\mathrm{Exp}(\lambda)\)}{Mixta X=\textbackslash min\textbackslash\{Y,M\textbackslash\} con Y\textbackslash sim\textbackslash mathrm\{Exp\}(\textbackslash lambda)}}\label{mixta-xminym-con-ysimmathrmexplambda}}

    Caracteriza la cdf, localiza la masa en \(x=M\) y construye \(F^{-1}\)
con caso discreto/continuo. Implementa
\texttt{sample\_min\_exp\_M(lam,\ M,\ N)}.

    \begin{tcolorbox}[breakable, size=fbox, boxrule=1pt, pad at break*=1mm,colback=cellbackground, colframe=cellborder]
\prompt{In}{incolor}{29}{\boxspacing}
\begin{Verbatim}[commandchars=\\\{\}]
\PY{c+c1}{\PYZsh{} TODO: implementa aquí}
\end{Verbatim}
\end{tcolorbox}

    \hypertarget{mixta-xmaxym-con-ysimmathrmexplambda}{%
\section{\texorpdfstring{Mixta \(X=\max\{Y,M\}\) con
\(Y\sim\mathrm{Exp}(\lambda)\)}{Mixta X=\textbackslash max\textbackslash\{Y,M\textbackslash\} con Y\textbackslash sim\textbackslash mathrm\{Exp\}(\textbackslash lambda)}}\label{mixta-xmaxym-con-ysimmathrmexplambda}}

    Sea \textbf{\(Y\sim\mathrm{Exp}(\lambda)\)} y \(M>0\). Defina
\(X=\max\{Y,M\}\).

\textbf{CDF}

\[
F_X(x)=\mathbb P(X\le x)=
\begin{cases}
0,& x<M,\\[4pt]
1-e^{-\lambda x},& x\ge M,
\end{cases}
\]

con salto en \(x=M\) de tamaño \(1-e^{-\lambda M}\).

\textbf{Inversa (cuantil generalizado
\(F_X^{-1}(u)=\inf\{x:F_X(x)\ge u\}\))}

Sea \textbf{\(U\sim\mathrm{Unif}(0,1)\)}.

\[
F_X^{-1}(u)=
\begin{cases}
M,& 0<u\le 1-e^{-\lambda M},\\[6pt]
-\dfrac{1}{\lambda}\ln(1-u),& 1-e^{-\lambda M}<u<1,
\end{cases}
\]

y además \(F_X^{-1}(0^+)=M\) y \(F_X^{-1}(1)=+\infty\).

\begin{quote}
Para muestrear: si \(U\le 1-e^{-\lambda M}\) devuelve \(M\); en caso
contrario devuelve \(-\frac1\lambda\ln(1-U)\).
\end{quote}

    \begin{enumerate}
\def\labelenumi{\alph{enumi})}
\setcounter{enumi}{2}
\tightlist
\item
  \textbf{Probar \(F(F^{-1}(u))\ge u\), \(0<u<1\).} Defina el cuantil
  generalizado \(F^{-1}(u):=\inf\{x:\,F(x)\ge u\}\). Sea
  \(S_u=\{x:\,F(x)\ge u\}\). Para todo \(\varepsilon>0\) existe
  \(x_\varepsilon\in S_u\) con
  \(x_\varepsilon\le F^{-1}(u)+\varepsilon\). Entonces
\end{enumerate}

\[
F\big(F^{-1}(u)+\varepsilon\big)\ \ge\ u.
\]

Por derecha-continuidad,

\[
F\!\big(F^{-1}(u)\big)=\lim_{\varepsilon\downarrow0}F\big(F^{-1}(u)+\varepsilon\big)\ \ge\ u.
\]

\textbf{Verificación para \(X=\max\{Y,M\}\).} Con
\(p_0:=1-e^{-\lambda M}\),

\[
F^{-1}(u)=
\begin{cases}
M,& 0<u\le p_0,\\
-\frac1\lambda\ln(1-u),& p_0<u<1.
\end{cases}
\]

Luego \(F(F^{-1}(u))=F(M)=p_0\ge u\) si \(u\le p_0\), y
\(F(-\tfrac1\lambda\ln(1-u))=u\) si \(u>p_0\).

\begin{enumerate}
\def\labelenumi{\alph{enumi})}
\setcounter{enumi}{3}
\tightlist
\item
  \textbf{Probar \(F^{-1}(F(x))\le x\) cuando \(0<F(x)<1\).} Tome
  \(u=F(x)\). El conjunto \(S_u=\{t:\,F(t)\ge u\}\) contiene a \(x\)
  (trivialmente \(F(x)\ge u\)). Por lo tanto
\end{enumerate}

\[
F^{-1}(F(x))=\inf S_u\ \le\ x.
\]

La igualdad se da cuando \(F\) es continua en \(x\).

\textbf{Verificación para \(X=\max\{Y,M\}\).} Si \(x>M\),

\[
F^{-1}(F(x))=-\frac1\lambda\ln\!\big(1-(1-e^{-\lambda x})\big)=x.
\]

Si \(x=M\), \(F^{-1}(F(M))=F^{-1}(p_0)=M=x\).

\begin{enumerate}
\def\labelenumi{\alph{enumi})}
\setcounter{enumi}{4}
\tightlist
\item
  \textbf{Generación por transformada inversa.} Con
  \(p_0:=1-e^{-\lambda M}\) y \(U\sim\mathrm{Unif}(0,1)\):
\end{enumerate}

\[
X=
\begin{cases}
M, & U\le p_0,\\[4pt]
-\dfrac1\lambda\ln(1-U), & U>p_0.
\end{cases}
\]

Es todo.

    \hypertarget{variable-con-cdf-por-tramos}{%
\section{Variable con CDF por
tramos}\label{variable-con-cdf-por-tramos}}

    \hypertarget{a}{%
\subsection{a)}\label{a}}

    \[
F_X(x)=
\begin{cases}
0,& x\le -2,\\[2pt]
\dfrac{x+2}{2},& -2<x<-1,\\[8pt]
\dfrac12,& -1\le x<1,\\[6pt]
\dfrac{x}{2},& 1\le x<2,\\[6pt]
1,& x\ge 2.
\end{cases}
\]

    \begin{tcolorbox}[breakable, size=fbox, boxrule=1pt, pad at break*=1mm,colback=cellbackground, colframe=cellborder]
\prompt{In}{incolor}{30}{\boxspacing}
\begin{Verbatim}[commandchars=\\\{\}]
\PY{c+c1}{\PYZsh{} Define a fine grid of x values}
\PY{n}{x} \PY{o}{=} \PY{n}{np}\PY{o}{.}\PY{n}{linspace}\PY{p}{(}\PY{o}{\PYZhy{}}\PY{l+m+mi}{3}\PY{p}{,} \PY{l+m+mi}{3}\PY{p}{,} \PY{l+m+mi}{400}\PY{p}{)}

\PY{c+c1}{\PYZsh{} Define the piecewise function for F\PYZus{}X(x)}
\PY{n}{F\PYZus{}x} \PY{o}{=} \PY{n}{np}\PY{o}{.}\PY{n}{piecewise}\PY{p}{(}\PY{n}{x}\PY{p}{,}
                   \PY{p}{[}\PY{n}{x} \PY{o}{\PYZlt{}}\PY{o}{=} \PY{o}{\PYZhy{}}\PY{l+m+mi}{2}\PY{p}{,} \PY{p}{(}\PY{n}{x} \PY{o}{\PYZgt{}} \PY{o}{\PYZhy{}}\PY{l+m+mi}{2}\PY{p}{)} \PY{o}{\PYZam{}} \PY{p}{(}\PY{n}{x} \PY{o}{\PYZlt{}} \PY{o}{\PYZhy{}}\PY{l+m+mi}{1}\PY{p}{)}\PY{p}{,} \PY{p}{(}\PY{n}{x} \PY{o}{\PYZgt{}}\PY{o}{=} \PY{o}{\PYZhy{}}\PY{l+m+mi}{1}\PY{p}{)} \PY{o}{\PYZam{}} \PY{p}{(}\PY{n}{x} \PY{o}{\PYZlt{}} \PY{l+m+mi}{1}\PY{p}{)}\PY{p}{,} \PY{p}{(}\PY{n}{x} \PY{o}{\PYZgt{}}\PY{o}{=} \PY{l+m+mi}{1}\PY{p}{)} \PY{o}{\PYZam{}} \PY{p}{(}\PY{n}{x} \PY{o}{\PYZlt{}} \PY{l+m+mi}{2}\PY{p}{)}\PY{p}{,} \PY{n}{x} \PY{o}{\PYZgt{}}\PY{o}{=} \PY{l+m+mi}{2}\PY{p}{]}\PY{p}{,}
                   \PY{p}{[}\PY{l+m+mi}{0}\PY{p}{,} \PY{k}{lambda} \PY{n}{x}\PY{p}{:} \PY{p}{(}\PY{n}{x} \PY{o}{+} \PY{l+m+mi}{2}\PY{p}{)} \PY{o}{/} \PY{l+m+mi}{2}\PY{p}{,} \PY{l+m+mf}{0.5}\PY{p}{,} \PY{k}{lambda} \PY{n}{x}\PY{p}{:} \PY{n}{x} \PY{o}{/} \PY{l+m+mi}{2}\PY{p}{,} \PY{l+m+mi}{1}\PY{p}{]}\PY{p}{)}

\PY{c+c1}{\PYZsh{} Plot the CDF}
\PY{n}{plt}\PY{o}{.}\PY{n}{plot}\PY{p}{(}\PY{n}{x}\PY{p}{,} \PY{n}{F\PYZus{}x}\PY{p}{,} \PY{n}{lw}\PY{o}{=}\PY{l+m+mi}{2}\PY{p}{,} \PY{n}{color}\PY{o}{=}\PY{n}{color}\PY{p}{[}\PY{l+m+mi}{0}\PY{p}{]}\PY{p}{)}
\PY{n}{plt}\PY{o}{.}\PY{n}{xlabel}\PY{p}{(}\PY{l+s+s1}{\PYZsq{}}\PY{l+s+s1}{x}\PY{l+s+s1}{\PYZsq{}}\PY{p}{)}
\PY{n}{plt}\PY{o}{.}\PY{n}{ylabel}\PY{p}{(}\PY{l+s+s1}{\PYZsq{}}\PY{l+s+s1}{\PYZdl{}F\PYZus{}X(x)\PYZdl{}}\PY{l+s+s1}{\PYZsq{}}\PY{p}{)}
\PY{n}{plt}\PY{o}{.}\PY{n}{title}\PY{p}{(}\PY{l+s+s1}{\PYZsq{}}\PY{l+s+s1}{CDF Function \PYZdl{}F\PYZus{}X(x)\PYZdl{}}\PY{l+s+s1}{\PYZsq{}}\PY{p}{)}
\PY{n}{plt}\PY{o}{.}\PY{n}{grid}\PY{p}{(}\PY{k+kc}{True}\PY{p}{)}
\PY{n}{plt}\PY{o}{.}\PY{n}{show}\PY{p}{(}\PY{p}{)}
\end{Verbatim}
\end{tcolorbox}

    \begin{Verbatim}[commandchars=\\\{\}]
findfont: Font family 'Inconsolata' not found.
findfont: Font family 'Inconsolata' not found.
findfont: Font family 'Inconsolata' not found.
findfont: Font family 'Inconsolata' not found.
findfont: Font family 'Inconsolata' not found.
findfont: Font family 'Inconsolata' not found.
findfont: Font family 'Inconsolata' not found.
findfont: Font family 'Inconsolata' not found.
findfont: Font family 'Inconsolata' not found.
findfont: Font family 'Inconsolata' not found.
findfont: Font family 'Inconsolata' not found.
findfont: Font family 'Inconsolata' not found.
findfont: Font family 'Inconsolata' not found.
findfont: Font family 'Inconsolata' not found.
findfont: Font family 'Inconsolata' not found.
findfont: Font family 'Inconsolata' not found.
findfont: Font family 'Inconsolata' not found.
findfont: Font family 'Inconsolata' not found.
findfont: Font family 'Inconsolata' not found.
findfont: Font family 'Inconsolata' not found.
findfont: Font family 'Inconsolata' not found.
findfont: Font family 'Inconsolata' not found.
findfont: Font family 'Inconsolata' not found.
findfont: Font family 'Inconsolata' not found.
findfont: Font family 'Inconsolata' not found.
findfont: Font family 'Inconsolata' not found.
findfont: Font family 'Inconsolata' not found.
findfont: Font family 'Inconsolata' not found.
findfont: Font family 'Inconsolata' not found.
findfont: Font family 'Inconsolata' not found.
findfont: Font family 'Inconsolata' not found.
findfont: Font family 'Inconsolata' not found.
findfont: Font family 'Inconsolata' not found.
findfont: Font family 'Inconsolata' not found.
findfont: Font family 'Inconsolata' not found.
findfont: Font family 'Inconsolata' not found.
findfont: Font family 'Inconsolata' not found.
findfont: Font family 'Inconsolata' not found.
findfont: Font family 'Inconsolata' not found.
findfont: Font family 'Inconsolata' not found.
findfont: Font family 'Inconsolata' not found.
findfont: Font family 'Inconsolata' not found.
findfont: Font family 'Inconsolata' not found.
findfont: Font family 'Inconsolata' not found.
findfont: Font family 'Inconsolata' not found.
findfont: Font family 'Inconsolata' not found.
    \end{Verbatim}

    \begin{center}
    \adjustimage{max size={0.9\linewidth}{0.9\paperheight}}{notebook_files/notebook_39_1.png}
    \end{center}
    { \hspace*{\fill} \\}
    
    \hypertarget{b}{%
\subsection{b)}\label{b}}

    Sea \textbf{\(U\sim\mathrm{Unif}(0,1)\)}.

\textbf{Encontrando la inversa:}

\begin{itemize}
\item
  Para \(x\le -2\), \(u\in\{0\}\): \(F_X^{-1}(0)=-2\) \emph{(convención
  en el extremo)}.
\item
  Para \(-2<x<-1\), \(u\in(0,\tfrac12)\):

  \[
  \begin{aligned}
  F_X(x)=u &\iff \frac{x+2}{2}=u,\\
           &\iff x=2u-2.
  \end{aligned}
  \]
\item
  Para \(-1\le x<1\), \(u\in\{\tfrac12\}\): meseta
  \(\Rightarrow F_X^{-1}(\tfrac12)=-1\) (borde izquierdo).
\item
  Para \(1\le x<2\), \(u\in(\tfrac12,1)\):

  \[
  \begin{aligned}
  F_X(x)=u &\iff \frac{x}{2}=u,\\
           &\iff x=2u.
  \end{aligned}
  \]
\item
  Para \(x\ge 2\), \(u\in\{1\}\): \(F_X^{-1}(1)=2\).
\end{itemize}

\textbf{Entonces:}

\[
F_X^{-1}(u)=
\begin{cases}
-2,& u=0,\\[4pt]
2u-2,& 0<u<\tfrac12,\\[4pt]
-1,& u=\tfrac12,\\[4pt]
2u,& \tfrac12<u<1,\\[4pt]
2,& u=1.
\end{cases}
\]

    \begin{tcolorbox}[breakable, size=fbox, boxrule=1pt, pad at break*=1mm,colback=cellbackground, colframe=cellborder]
\prompt{In}{incolor}{31}{\boxspacing}
\begin{Verbatim}[commandchars=\\\{\}]
\PY{c+c1}{\PYZsh{} Create a fine grid of u values in [0,1] ensuring u=0 and u=1 are included}
\PY{n}{u} \PY{o}{=} \PY{n}{np}\PY{o}{.}\PY{n}{linspace}\PY{p}{(}\PY{l+m+mi}{0}\PY{p}{,} \PY{l+m+mi}{1}\PY{p}{,} \PY{l+m+mi}{500}\PY{p}{)}

\PY{k}{def}\PY{+w}{ }\PY{n+nf}{F\PYZus{}inv\PYZus{}func}\PY{p}{(}\PY{n}{u}\PY{p}{)}\PY{p}{:}
    \PY{k}{if} \PY{n}{u} \PY{o}{\PYZlt{}}\PY{o}{=} \PY{l+m+mf}{0.4}\PY{p}{:}
        \PY{k}{return} \PY{l+m+mi}{1}
    \PY{k}{elif} \PY{n}{u} \PY{o}{\PYZlt{}}\PY{o}{=} \PY{l+m+mf}{0.7}\PY{p}{:}
        \PY{k}{return} \PY{l+m+mi}{2}
    \PY{k}{elif} \PY{n}{u} \PY{o}{\PYZlt{}}\PY{o}{=} \PY{l+m+mf}{0.9}\PY{p}{:}
        \PY{k}{return} \PY{l+m+mi}{3}
    \PY{k}{else}\PY{p}{:}
        \PY{k}{return} \PY{l+m+mi}{4}

\PY{n}{F\PYZus{}inv} \PY{o}{=} \PY{n}{np}\PY{o}{.}\PY{n}{vectorize}\PY{p}{(}\PY{n}{F\PYZus{}inv\PYZus{}func}\PY{p}{)}\PY{p}{(}\PY{n}{u}\PY{p}{)}

\PY{n}{plt}\PY{o}{.}\PY{n}{figure}\PY{p}{(}\PY{n}{figsize}\PY{o}{=}\PY{p}{(}\PY{l+m+mi}{8}\PY{p}{,} \PY{l+m+mi}{5}\PY{p}{)}\PY{p}{)}
\PY{n}{plt}\PY{o}{.}\PY{n}{plot}\PY{p}{(}\PY{n}{u}\PY{p}{,} \PY{n}{F\PYZus{}inv}\PY{p}{,} \PY{n}{color}\PY{o}{=}\PY{n}{color}\PY{p}{[}\PY{l+m+mi}{0}\PY{p}{]}\PY{p}{,} \PY{n}{lw}\PY{o}{=}\PY{l+m+mi}{2}\PY{p}{)}
\PY{n}{plt}\PY{o}{.}\PY{n}{xlabel}\PY{p}{(}\PY{l+s+s1}{\PYZsq{}}\PY{l+s+s1}{\PYZdl{}u\PYZdl{}}\PY{l+s+s1}{\PYZsq{}}\PY{p}{)}
\PY{n}{plt}\PY{o}{.}\PY{n}{ylabel}\PY{p}{(}\PY{l+s+s1}{\PYZsq{}}\PY{l+s+s1}{\PYZdl{}F\PYZus{}X\PYZca{}}\PY{l+s+s1}{\PYZob{}}\PY{l+s+s1}{\PYZhy{}1\PYZcb{}(u)\PYZdl{}}\PY{l+s+s1}{\PYZsq{}}\PY{p}{)}
\PY{n}{plt}\PY{o}{.}\PY{n}{show}\PY{p}{(}\PY{p}{)}
\end{Verbatim}
\end{tcolorbox}

    \begin{Verbatim}[commandchars=\\\{\}]
findfont: Font family 'Inconsolata' not found.
findfont: Font family 'Inconsolata' not found.
findfont: Font family 'Inconsolata' not found.
findfont: Font family 'Inconsolata' not found.
findfont: Font family 'Inconsolata' not found.
findfont: Font family 'Inconsolata' not found.
findfont: Font family 'Inconsolata' not found.
findfont: Font family 'Inconsolata' not found.
findfont: Font family 'Inconsolata' not found.
findfont: Font family 'Inconsolata' not found.
findfont: Font family 'Inconsolata' not found.
findfont: Font family 'Inconsolata' not found.
findfont: Font family 'Inconsolata' not found.
findfont: Font family 'Inconsolata' not found.
findfont: Font family 'Inconsolata' not found.
findfont: Font family 'Inconsolata' not found.
findfont: Font family 'Inconsolata' not found.
findfont: Font family 'Inconsolata' not found.
findfont: Font family 'Inconsolata' not found.
findfont: Font family 'Inconsolata' not found.
findfont: Font family 'Inconsolata' not found.
findfont: Font family 'Inconsolata' not found.
findfont: Font family 'Inconsolata' not found.
findfont: Font family 'Inconsolata' not found.
findfont: Font family 'Inconsolata' not found.
findfont: Font family 'Inconsolata' not found.
findfont: Font family 'Inconsolata' not found.
findfont: Font family 'Inconsolata' not found.
findfont: Font family 'Inconsolata' not found.
findfont: Font family 'Inconsolata' not found.
findfont: Font family 'Inconsolata' not found.
findfont: Font family 'Inconsolata' not found.
findfont: Font family 'Inconsolata' not found.
findfont: Font family 'Inconsolata' not found.
findfont: Font family 'Inconsolata' not found.
findfont: Font family 'Inconsolata' not found.
findfont: Font family 'Inconsolata' not found.
findfont: Font family 'Inconsolata' not found.
findfont: Font family 'Inconsolata' not found.
findfont: Font family 'Inconsolata' not found.
findfont: Font family 'Inconsolata' not found.
    \end{Verbatim}

    \begin{center}
    \adjustimage{max size={0.9\linewidth}{0.9\paperheight}}{notebook_files/notebook_42_1.png}
    \end{center}
    { \hspace*{\fill} \\}
    
    \hypertarget{bernoulli-p-desde-u01}{%
\section{\texorpdfstring{Bernoulli \((p)\) desde
\(U(0,1)\)}{Bernoulli (p) desde U(0,1)}}\label{bernoulli-p-desde-u01}}

    Sea \textbf{\(U\sim\mathrm{Unif}(0,1)\)} y \(0<p<1\). Defina

\[
X=\mathbf 1_{(0,p]}(U)=
\begin{cases}
1,& U\le p,\\
0,& U>p.
\end{cases}
\]

\[
\mathbb P(X=1)=\mathbb P(U\le p)=p,\qquad
\mathbb P(X=0)=\mathbb P(U>p)=1-p,
\]

usando que \(\mathbb P(U=p)=0\). Por tanto
\(X\sim \mathrm{Bernoulli}(p)\).

    \hypertarget{variable-aleatoria-discreta}{%
\section{Variable aleatoria
discreta}\label{variable-aleatoria-discreta}}

    \hypertarget{a}{%
\subsection{a)}\label{a}}

    \[
f_X(x)=
\begin{cases}
0.3& x=1,\\
0.5,& x=2,\\
0.2,& x=3,\\
0,& \text{en otro caso.}
\end{cases}
\]

    \begin{tcolorbox}[breakable, size=fbox, boxrule=1pt, pad at break*=1mm,colback=cellbackground, colframe=cellborder]
\prompt{In}{incolor}{32}{\boxspacing}
\begin{Verbatim}[commandchars=\\\{\}]
\PY{n}{x\PYZus{}vals} \PY{o}{=} \PY{p}{[}\PY{l+m+mi}{1}\PY{p}{,} \PY{l+m+mi}{2}\PY{p}{,} \PY{l+m+mi}{3}\PY{p}{]}
\PY{n}{pmf} \PY{o}{=} \PY{p}{[}\PY{l+m+mf}{0.3}\PY{p}{,} \PY{l+m+mf}{0.5}\PY{p}{,} \PY{l+m+mf}{0.2}\PY{p}{]}

\PY{n}{plt}\PY{o}{.}\PY{n}{vlines}\PY{p}{(}\PY{n}{x\PYZus{}vals}\PY{p}{,} \PY{l+m+mi}{0}\PY{p}{,} \PY{n}{pmf}\PY{p}{,} \PY{n}{linestyle}\PY{o}{=}\PY{l+s+s1}{\PYZsq{}}\PY{l+s+s1}{\PYZhy{}\PYZhy{}}\PY{l+s+s1}{\PYZsq{}}\PY{p}{)}
\PY{n}{plt}\PY{o}{.}\PY{n}{plot}\PY{p}{(}\PY{n}{x\PYZus{}vals}\PY{p}{,} \PY{n}{pmf}\PY{p}{,} \PY{l+s+s1}{\PYZsq{}}\PY{l+s+s1}{o}\PY{l+s+s1}{\PYZsq{}}\PY{p}{)}
\PY{n}{plt}\PY{o}{.}\PY{n}{xlabel}\PY{p}{(}\PY{l+s+s1}{\PYZsq{}}\PY{l+s+s1}{\PYZdl{}X\PYZdl{}}\PY{l+s+s1}{\PYZsq{}}\PY{p}{)}
\PY{n}{plt}\PY{o}{.}\PY{n}{ylabel}\PY{p}{(}\PY{l+s+s1}{\PYZsq{}}\PY{l+s+s1}{\PYZdl{}f\PYZus{}X(x)\PYZdl{}}\PY{l+s+s1}{\PYZsq{}}\PY{p}{)}
\PY{n}{plt}\PY{o}{.}\PY{n}{show}\PY{p}{(}\PY{p}{)}
\end{Verbatim}
\end{tcolorbox}

    \begin{Verbatim}[commandchars=\\\{\}]
findfont: Font family 'Inconsolata' not found.
findfont: Font family 'Inconsolata' not found.
findfont: Font family 'Inconsolata' not found.
findfont: Font family 'Inconsolata' not found.
findfont: Font family 'Inconsolata' not found.
findfont: Font family 'Inconsolata' not found.
findfont: Font family 'Inconsolata' not found.
findfont: Font family 'Inconsolata' not found.
findfont: Font family 'Inconsolata' not found.
findfont: Font family 'Inconsolata' not found.
findfont: Font family 'Inconsolata' not found.
findfont: Font family 'Inconsolata' not found.
findfont: Font family 'Inconsolata' not found.
findfont: Font family 'Inconsolata' not found.
findfont: Font family 'Inconsolata' not found.
findfont: Font family 'Inconsolata' not found.
findfont: Font family 'Inconsolata' not found.
findfont: Font family 'Inconsolata' not found.
findfont: Font family 'Inconsolata' not found.
findfont: Font family 'Inconsolata' not found.
findfont: Font family 'Inconsolata' not found.
findfont: Font family 'Inconsolata' not found.
findfont: Font family 'Inconsolata' not found.
findfont: Font family 'Inconsolata' not found.
findfont: Font family 'Inconsolata' not found.
findfont: Font family 'Inconsolata' not found.
findfont: Font family 'Inconsolata' not found.
findfont: Font family 'Inconsolata' not found.
findfont: Font family 'Inconsolata' not found.
findfont: Font family 'Inconsolata' not found.
findfont: Font family 'Inconsolata' not found.
findfont: Font family 'Inconsolata' not found.
findfont: Font family 'Inconsolata' not found.
findfont: Font family 'Inconsolata' not found.
findfont: Font family 'Inconsolata' not found.
findfont: Font family 'Inconsolata' not found.
findfont: Font family 'Inconsolata' not found.
findfont: Font family 'Inconsolata' not found.
findfont: Font family 'Inconsolata' not found.
findfont: Font family 'Inconsolata' not found.
findfont: Font family 'Inconsolata' not found.
findfont: Font family 'Inconsolata' not found.
findfont: Font family 'Inconsolata' not found.
findfont: Font family 'Inconsolata' not found.
findfont: Font family 'Inconsolata' not found.
findfont: Font family 'Inconsolata' not found.
findfont: Font family 'Inconsolata' not found.
findfont: Font family 'Inconsolata' not found.
findfont: Font family 'Inconsolata' not found.
    \end{Verbatim}

    \begin{center}
    \adjustimage{max size={0.9\linewidth}{0.9\paperheight}}{notebook_files/notebook_48_1.png}
    \end{center}
    { \hspace*{\fill} \\}
    
    \hypertarget{b}{%
\subsection{b)}\label{b}}

    \[
F_X(x)=\mathbb P(X\le x)=
\begin{cases}
0,& x<1,\\
0.3,& 1\le x<2,\\
0.8,& 2\le x<3,\\
1,& x\ge 3.
\end{cases}
\]

    \begin{tcolorbox}[breakable, size=fbox, boxrule=1pt, pad at break*=1mm,colback=cellbackground, colframe=cellborder]
\prompt{In}{incolor}{33}{\boxspacing}
\begin{Verbatim}[commandchars=\\\{\}]
\PY{n}{x\PYZus{}cdf} \PY{o}{=} \PY{p}{[}\PY{o}{\PYZhy{}}\PY{l+m+mi}{1}\PY{p}{,} \PY{l+m+mi}{1}\PY{p}{,} \PY{l+m+mi}{2}\PY{p}{,} \PY{l+m+mi}{3}\PY{p}{,} \PY{l+m+mi}{4}\PY{p}{]}
\PY{n}{F\PYZus{}cdf} \PY{o}{=} \PY{p}{[}\PY{l+m+mi}{0}\PY{p}{,} \PY{l+m+mf}{0.3}\PY{p}{,} \PY{l+m+mf}{0.5}\PY{p}{,} \PY{l+m+mi}{1}\PY{p}{,} \PY{l+m+mi}{1}\PY{p}{]}

\PY{n}{plt}\PY{o}{.}\PY{n}{step}\PY{p}{(}\PY{n}{x\PYZus{}cdf}\PY{p}{,} \PY{n}{F\PYZus{}cdf}\PY{p}{,} \PY{n}{where}\PY{o}{=}\PY{l+s+s1}{\PYZsq{}}\PY{l+s+s1}{post}\PY{l+s+s1}{\PYZsq{}}\PY{p}{,} \PY{n}{linestyle}\PY{o}{=}\PY{l+s+s1}{\PYZsq{}}\PY{l+s+s1}{\PYZhy{}}\PY{l+s+s1}{\PYZsq{}}\PY{p}{)}
\PY{n}{plt}\PY{o}{.}\PY{n}{xlabel}\PY{p}{(}\PY{l+s+s1}{\PYZsq{}}\PY{l+s+s1}{\PYZdl{}X\PYZdl{}}\PY{l+s+s1}{\PYZsq{}}\PY{p}{)}
\PY{n}{plt}\PY{o}{.}\PY{n}{ylabel}\PY{p}{(}\PY{l+s+s1}{\PYZsq{}}\PY{l+s+s1}{\PYZdl{}F\PYZus{}X(x)\PYZdl{}}\PY{l+s+s1}{\PYZsq{}}\PY{p}{)}
\PY{n}{plt}\PY{o}{.}\PY{n}{show}\PY{p}{(}\PY{p}{)}
\end{Verbatim}
\end{tcolorbox}

    \begin{Verbatim}[commandchars=\\\{\}]
findfont: Font family 'Inconsolata' not found.
findfont: Font family 'Inconsolata' not found.
findfont: Font family 'Inconsolata' not found.
findfont: Font family 'Inconsolata' not found.
findfont: Font family 'Inconsolata' not found.
findfont: Font family 'Inconsolata' not found.
findfont: Font family 'Inconsolata' not found.
findfont: Font family 'Inconsolata' not found.
findfont: Font family 'Inconsolata' not found.
findfont: Font family 'Inconsolata' not found.
findfont: Font family 'Inconsolata' not found.
findfont: Font family 'Inconsolata' not found.
findfont: Font family 'Inconsolata' not found.
findfont: Font family 'Inconsolata' not found.
findfont: Font family 'Inconsolata' not found.
findfont: Font family 'Inconsolata' not found.
findfont: Font family 'Inconsolata' not found.
findfont: Font family 'Inconsolata' not found.
findfont: Font family 'Inconsolata' not found.
findfont: Font family 'Inconsolata' not found.
findfont: Font family 'Inconsolata' not found.
findfont: Font family 'Inconsolata' not found.
findfont: Font family 'Inconsolata' not found.
findfont: Font family 'Inconsolata' not found.
findfont: Font family 'Inconsolata' not found.
findfont: Font family 'Inconsolata' not found.
findfont: Font family 'Inconsolata' not found.
findfont: Font family 'Inconsolata' not found.
findfont: Font family 'Inconsolata' not found.
findfont: Font family 'Inconsolata' not found.
findfont: Font family 'Inconsolata' not found.
findfont: Font family 'Inconsolata' not found.
findfont: Font family 'Inconsolata' not found.
findfont: Font family 'Inconsolata' not found.
findfont: Font family 'Inconsolata' not found.
findfont: Font family 'Inconsolata' not found.
findfont: Font family 'Inconsolata' not found.
findfont: Font family 'Inconsolata' not found.
findfont: Font family 'Inconsolata' not found.
findfont: Font family 'Inconsolata' not found.
    \end{Verbatim}

    \begin{center}
    \adjustimage{max size={0.9\linewidth}{0.9\paperheight}}{notebook_files/notebook_51_1.png}
    \end{center}
    { \hspace*{\fill} \\}
    
    \hypertarget{c}{%
\subsection{c)}\label{c}}

    Sea \textbf{\(U\sim\mathrm{Unif}(0,1)\)}.

\[
\begin{aligned}
F_X(x)=u &\;\iff\; u\in(0,0.3]\ \Rightarrow\ x=1,\\
         &\;\iff\; u\in(0.3,0.8]\ \Rightarrow\ x=2,\\
         &\;\iff\; u\in(0.8,1]\ \Rightarrow\ x=3.
\end{aligned}
\]

Entonces:

\[
F_X^{-1}(u)=
\begin{cases}
1,& 0<u\le 0.3,\\
2,& 0.3<u\le 0.8,\\
3,& 0.8<u\le 1.
\end{cases}
\]

    \begin{tcolorbox}[breakable, size=fbox, boxrule=1pt, pad at break*=1mm,colback=cellbackground, colframe=cellborder]
\prompt{In}{incolor}{34}{\boxspacing}
\begin{Verbatim}[commandchars=\\\{\}]
\PY{n}{u\PYZus{}vals} \PY{o}{=} \PY{p}{[}\PY{l+m+mi}{0}\PY{p}{,} \PY{l+m+mi}{0}\PY{p}{,} \PY{l+m+mf}{0.3}\PY{p}{,} \PY{l+m+mf}{0.8}\PY{p}{,} \PY{l+m+mf}{1.0}\PY{p}{]}
\PY{n}{F\PYZus{}inv\PYZus{}values} \PY{o}{=} \PY{p}{[}\PY{o}{\PYZhy{}}\PY{l+m+mi}{1}\PY{p}{,} \PY{l+m+mi}{1}\PY{p}{,} \PY{l+m+mi}{2}\PY{p}{,} \PY{l+m+mi}{3}\PY{p}{,} \PY{l+m+mi}{4}\PY{p}{]}

\PY{n}{plt}\PY{o}{.}\PY{n}{step}\PY{p}{(}\PY{n}{u\PYZus{}vals}\PY{p}{,} \PY{n}{F\PYZus{}inv\PYZus{}values}\PY{p}{,} \PY{n}{where}\PY{o}{=}\PY{l+s+s1}{\PYZsq{}}\PY{l+s+s1}{post}\PY{l+s+s1}{\PYZsq{}}\PY{p}{)}
\PY{n}{plt}\PY{o}{.}\PY{n}{xlabel}\PY{p}{(}\PY{l+s+s1}{\PYZsq{}}\PY{l+s+s1}{\PYZdl{}u\PYZdl{}}\PY{l+s+s1}{\PYZsq{}}\PY{p}{)}
\PY{n}{plt}\PY{o}{.}\PY{n}{ylabel}\PY{p}{(}\PY{l+s+s1}{\PYZsq{}}\PY{l+s+s1}{\PYZdl{}F\PYZca{}}\PY{l+s+s1}{\PYZob{}}\PY{l+s+s1}{\PYZhy{}1\PYZcb{}(u)\PYZdl{}}\PY{l+s+s1}{\PYZsq{}}\PY{p}{)}
\PY{n}{plt}\PY{o}{.}\PY{n}{show}\PY{p}{(}\PY{p}{)}
\end{Verbatim}
\end{tcolorbox}

    \begin{Verbatim}[commandchars=\\\{\}]
findfont: Font family 'Inconsolata' not found.
findfont: Font family 'Inconsolata' not found.
    \end{Verbatim}

    \begin{Verbatim}[commandchars=\\\{\}]
findfont: Font family 'Inconsolata' not found.
findfont: Font family 'Inconsolata' not found.
findfont: Font family 'Inconsolata' not found.
findfont: Font family 'Inconsolata' not found.
findfont: Font family 'Inconsolata' not found.
findfont: Font family 'Inconsolata' not found.
findfont: Font family 'Inconsolata' not found.
findfont: Font family 'Inconsolata' not found.
findfont: Font family 'Inconsolata' not found.
findfont: Font family 'Inconsolata' not found.
findfont: Font family 'Inconsolata' not found.
findfont: Font family 'Inconsolata' not found.
findfont: Font family 'Inconsolata' not found.
findfont: Font family 'Inconsolata' not found.
findfont: Font family 'Inconsolata' not found.
findfont: Font family 'Inconsolata' not found.
findfont: Font family 'Inconsolata' not found.
findfont: Font family 'Inconsolata' not found.
findfont: Font family 'Inconsolata' not found.
findfont: Font family 'Inconsolata' not found.
findfont: Font family 'Inconsolata' not found.
findfont: Font family 'Inconsolata' not found.
findfont: Font family 'Inconsolata' not found.
findfont: Font family 'Inconsolata' not found.
findfont: Font family 'Inconsolata' not found.
findfont: Font family 'Inconsolata' not found.
findfont: Font family 'Inconsolata' not found.
findfont: Font family 'Inconsolata' not found.
findfont: Font family 'Inconsolata' not found.
findfont: Font family 'Inconsolata' not found.
findfont: Font family 'Inconsolata' not found.
findfont: Font family 'Inconsolata' not found.
findfont: Font family 'Inconsolata' not found.
findfont: Font family 'Inconsolata' not found.
findfont: Font family 'Inconsolata' not found.
findfont: Font family 'Inconsolata' not found.
findfont: Font family 'Inconsolata' not found.
findfont: Font family 'Inconsolata' not found.
    \end{Verbatim}

    \begin{center}
    \adjustimage{max size={0.9\linewidth}{0.9\paperheight}}{notebook_files/notebook_54_2.png}
    \end{center}
    { \hspace*{\fill} \\}
    
    \hypertarget{d-valor-esperado}{%
\subsection{d) Valor esperado}\label{d-valor-esperado}}

\[
\mathbb E[X]=\sum_{x=1}^{3}x\cdot f_X(x)=1\cdot 0.3+2\cdot 0.5+3\cdot 0.2=1.9.
\]

\hypertarget{e-varianza}{%
\subsection{e) Varianza}\label{e-varianza}}

\[
\mathbb V[X]=\sum_{x=1}^{3}(x-\mathbb E[X])^2\cdot f_X(x)=
(1-1.9)^2\cdot 0.3+(2-1.9)^2\cdot 0.5+(3-1.9)^2\cdot 0.2=0.49.
\]

\[
\mathbb V[X] = \mathbb E[X^2] - (\mathbb E[X])^2 = 
0.3\cdot 1^2 + 0.5\cdot 2^2 + 0.2\cdot 3^2 - (1.9)^2 = 2.79 - (1.9)^2 = 0.49.
\]

    \hypertarget{f}{%
\subsection{f)}\label{f}}

    \begin{tcolorbox}[breakable, size=fbox, boxrule=1pt, pad at break*=1mm,colback=cellbackground, colframe=cellborder]
\prompt{In}{incolor}{35}{\boxspacing}
\begin{Verbatim}[commandchars=\\\{\}]
\PY{k}{def}\PY{+w}{ }\PY{n+nf}{F\PYZus{}inv}\PY{p}{(}\PY{n}{u}\PY{p}{)}\PY{p}{:}
    \PY{k}{if} \PY{l+m+mf}{0.3} \PY{o}{\PYZlt{}} \PY{n}{u} \PY{o}{\PYZlt{}}\PY{o}{=} \PY{l+m+mf}{0.8}\PY{p}{:}
        \PY{k}{return} \PY{l+m+mi}{2}
    \PY{k}{elif} \PY{n}{u} \PY{o}{\PYZlt{}}\PY{o}{=} \PY{l+m+mf}{0.3}\PY{p}{:}
        \PY{k}{return} \PY{l+m+mi}{1}
    \PY{k}{else}\PY{p}{:}
        \PY{k}{return} \PY{l+m+mi}{3}
\end{Verbatim}
\end{tcolorbox}

    \begin{tcolorbox}[breakable, size=fbox, boxrule=1pt, pad at break*=1mm,colback=cellbackground, colframe=cellborder]
\prompt{In}{incolor}{36}{\boxspacing}
\begin{Verbatim}[commandchars=\\\{\}]
\PY{n}{u\PYZus{}samples} \PY{o}{=} \PY{n}{np}\PY{o}{.}\PY{n}{random}\PY{o}{.}\PY{n}{uniform}\PY{p}{(}\PY{l+m+mi}{0}\PY{p}{,} \PY{l+m+mi}{1}\PY{p}{,} \PY{l+m+mi}{500}\PY{p}{)}
\PY{n}{x\PYZus{}samples} \PY{o}{=} \PY{p}{[}\PY{n}{F\PYZus{}inv}\PY{p}{(}\PY{n}{u}\PY{p}{)} \PY{k}{for} \PY{n}{u} \PY{o+ow}{in} \PY{n}{u\PYZus{}samples}\PY{p}{]}
\end{Verbatim}
\end{tcolorbox}

    \hypertarget{g}{%
\subsection{g)}\label{g}}

    \begin{tcolorbox}[breakable, size=fbox, boxrule=1pt, pad at break*=1mm,colback=cellbackground, colframe=cellborder]
\prompt{In}{incolor}{37}{\boxspacing}
\begin{Verbatim}[commandchars=\\\{\}]
\PY{n}{plt}\PY{o}{.}\PY{n}{hist}\PY{p}{(}\PY{n}{x\PYZus{}samples}\PY{p}{,} \PY{n}{bins}\PY{o}{=}\PY{n}{np}\PY{o}{.}\PY{n}{arange}\PY{p}{(}\PY{l+m+mf}{0.5}\PY{p}{,} \PY{l+m+mf}{4.5}\PY{p}{,} \PY{l+m+mi}{1}\PY{p}{)}\PY{p}{,} \PY{n}{density}\PY{o}{=}\PY{k+kc}{True}\PY{p}{,} \PY{n}{alpha}\PY{o}{=}\PY{l+m+mf}{0.3}\PY{p}{)}
\PY{n}{plt}\PY{o}{.}\PY{n}{vlines}\PY{p}{(}\PY{n}{x\PYZus{}vals}\PY{p}{,} \PY{l+m+mi}{0}\PY{p}{,} \PY{n}{pmf}\PY{p}{,} \PY{n}{linestyle}\PY{o}{=}\PY{l+s+s1}{\PYZsq{}}\PY{l+s+s1}{\PYZhy{}\PYZhy{}}\PY{l+s+s1}{\PYZsq{}}\PY{p}{)}
\PY{n}{plt}\PY{o}{.}\PY{n}{plot}\PY{p}{(}\PY{n}{x\PYZus{}vals}\PY{p}{,} \PY{n}{pmf}\PY{p}{,} \PY{l+s+s1}{\PYZsq{}}\PY{l+s+s1}{o}\PY{l+s+s1}{\PYZsq{}}\PY{p}{)}
\PY{n}{plt}\PY{o}{.}\PY{n}{xlabel}\PY{p}{(}\PY{l+s+s2}{\PYZdq{}}\PY{l+s+s2}{\PYZdl{}X\PYZdl{}}\PY{l+s+s2}{\PYZdq{}}\PY{p}{)}
\PY{n}{plt}\PY{o}{.}\PY{n}{ylabel}\PY{p}{(}\PY{l+s+s2}{\PYZdq{}}\PY{l+s+s2}{Densidad}\PY{l+s+s2}{\PYZdq{}}\PY{p}{)}
\PY{n}{plt}\PY{o}{.}\PY{n}{title}\PY{p}{(}\PY{l+s+s2}{\PYZdq{}}\PY{l+s+s2}{Histograma de \PYZdl{}F\PYZus{}X\PYZca{}}\PY{l+s+s2}{\PYZob{}}\PY{l+s+s2}{\PYZhy{}1\PYZcb{}(U)\PYZdl{}}\PY{l+s+s2}{\PYZdq{}}\PY{p}{)}
\PY{n}{plt}\PY{o}{.}\PY{n}{show}\PY{p}{(}\PY{p}{)}
\end{Verbatim}
\end{tcolorbox}

    \begin{Verbatim}[commandchars=\\\{\}]
findfont: Font family 'Inconsolata' not found.
findfont: Font family 'Inconsolata' not found.
findfont: Font family 'Inconsolata' not found.
findfont: Font family 'Inconsolata' not found.
findfont: Font family 'Inconsolata' not found.
findfont: Font family 'Inconsolata' not found.
findfont: Font family 'Inconsolata' not found.
findfont: Font family 'Inconsolata' not found.
findfont: Font family 'Inconsolata' not found.
findfont: Font family 'Inconsolata' not found.
findfont: Font family 'Inconsolata' not found.
findfont: Font family 'Inconsolata' not found.
findfont: Font family 'Inconsolata' not found.
findfont: Font family 'Inconsolata' not found.
findfont: Font family 'Inconsolata' not found.
findfont: Font family 'Inconsolata' not found.
findfont: Font family 'Inconsolata' not found.
findfont: Font family 'Inconsolata' not found.
findfont: Font family 'Inconsolata' not found.
findfont: Font family 'Inconsolata' not found.
findfont: Font family 'Inconsolata' not found.
findfont: Font family 'Inconsolata' not found.
findfont: Font family 'Inconsolata' not found.
findfont: Font family 'Inconsolata' not found.
findfont: Font family 'Inconsolata' not found.
findfont: Font family 'Inconsolata' not found.
findfont: Font family 'Inconsolata' not found.
findfont: Font family 'Inconsolata' not found.
findfont: Font family 'Inconsolata' not found.
findfont: Font family 'Inconsolata' not found.
findfont: Font family 'Inconsolata' not found.
findfont: Font family 'Inconsolata' not found.
findfont: Font family 'Inconsolata' not found.
findfont: Font family 'Inconsolata' not found.
findfont: Font family 'Inconsolata' not found.
findfont: Font family 'Inconsolata' not found.
findfont: Font family 'Inconsolata' not found.
findfont: Font family 'Inconsolata' not found.
findfont: Font family 'Inconsolata' not found.
findfont: Font family 'Inconsolata' not found.
findfont: Font family 'Inconsolata' not found.
findfont: Font family 'Inconsolata' not found.
findfont: Font family 'Inconsolata' not found.
findfont: Font family 'Inconsolata' not found.
    \end{Verbatim}

    \begin{center}
    \adjustimage{max size={0.9\linewidth}{0.9\paperheight}}{notebook_files/notebook_60_1.png}
    \end{center}
    { \hspace*{\fill} \\}
    
    \hypertarget{h}{%
\subsection{h)}\label{h}}

    \begin{tcolorbox}[breakable, size=fbox, boxrule=1pt, pad at break*=1mm,colback=cellbackground, colframe=cellborder]
\prompt{In}{incolor}{38}{\boxspacing}
\begin{Verbatim}[commandchars=\\\{\}]
\PY{n}{x\PYZus{}mean} \PY{o}{=} \PY{n}{np}\PY{o}{.}\PY{n}{mean}\PY{p}{(}\PY{n}{x\PYZus{}samples}\PY{p}{)}
\PY{n+nb}{print}\PY{p}{(}\PY{l+s+sa}{f}\PY{l+s+s2}{\PYZdq{}}\PY{l+s+s2}{Media muestral de X: }\PY{l+s+si}{\PYZob{}}\PY{n}{x\PYZus{}mean}\PY{l+s+si}{\PYZcb{}}\PY{l+s+s2}{\PYZdq{}}\PY{p}{)}
\PY{n+nb}{print}\PY{p}{(}\PY{l+s+sa}{f}\PY{l+s+s2}{\PYZdq{}}\PY{l+s+s2}{Media teórica de X: 1.9}\PY{l+s+s2}{\PYZdq{}}\PY{p}{)}
\PY{n+nb}{print}\PY{p}{(}\PY{l+s+sa}{f}\PY{l+s+s2}{\PYZdq{}}\PY{l+s+s2}{Diferencia: }\PY{l+s+si}{\PYZob{}}\PY{n}{x\PYZus{}mean}\PY{+w}{ }\PY{o}{\PYZhy{}}\PY{+w}{ }\PY{l+m+mf}{1.9}\PY{l+s+si}{:}\PY{l+s+s2}{.4f}\PY{l+s+si}{\PYZcb{}}\PY{l+s+s2}{\PYZdq{}}\PY{p}{)}
\end{Verbatim}
\end{tcolorbox}

    \begin{Verbatim}[commandchars=\\\{\}]
Media muestral de X: 1.842
Media teórica de X: 1.9
Diferencia: -0.0580
    \end{Verbatim}

    \hypertarget{i}{%
\subsection{i)}\label{i}}

    \begin{tcolorbox}[breakable, size=fbox, boxrule=1pt, pad at break*=1mm,colback=cellbackground, colframe=cellborder]
\prompt{In}{incolor}{39}{\boxspacing}
\begin{Verbatim}[commandchars=\\\{\}]
\PY{n}{x\PYZus{}var} \PY{o}{=} \PY{n}{np}\PY{o}{.}\PY{n}{var}\PY{p}{(}\PY{n}{x\PYZus{}samples}\PY{p}{)}
\PY{n+nb}{print}\PY{p}{(}\PY{l+s+sa}{f}\PY{l+s+s2}{\PYZdq{}}\PY{l+s+s2}{Varianza muestral de X: }\PY{l+s+si}{\PYZob{}}\PY{n}{x\PYZus{}var}\PY{l+s+si}{\PYZcb{}}\PY{l+s+s2}{\PYZdq{}}\PY{p}{)}
\PY{n+nb}{print}\PY{p}{(}\PY{l+s+sa}{f}\PY{l+s+s2}{\PYZdq{}}\PY{l+s+s2}{Varianza teórica de X: 0.49}\PY{l+s+s2}{\PYZdq{}}\PY{p}{)}
\PY{n+nb}{print}\PY{p}{(}\PY{l+s+sa}{f}\PY{l+s+s2}{\PYZdq{}}\PY{l+s+s2}{Diferencia: }\PY{l+s+si}{\PYZob{}}\PY{n}{x\PYZus{}var}\PY{+w}{ }\PY{o}{\PYZhy{}}\PY{+w}{ }\PY{l+m+mf}{0.49}\PY{l+s+si}{:}\PY{l+s+s2}{.4f}\PY{l+s+si}{\PYZcb{}}\PY{l+s+s2}{\PYZdq{}}\PY{p}{)}
\end{Verbatim}
\end{tcolorbox}

    \begin{Verbatim}[commandchars=\\\{\}]
Varianza muestral de X: 0.4850359999999999
Varianza teórica de X: 0.49
Diferencia: -0.0050
    \end{Verbatim}

    \hypertarget{binomial-geomuxe9trica-y-poisson}{%
\section{Binomial, Geométrica y
Poisson}\label{binomial-geomuxe9trica-y-poisson}}

    Implementa generadores por inversión para: - \(X\sim\mathrm{Bin}(m,p)\)
con \(m = 10\) y \(p = 1/3\) - \(X\sim\mathrm{Geo}(p)\) con
\(\mathbb P(X=k)=p(1-p)^{k-1}\), \(k\ge 1\) con \(p = 3/4\) -
\(X\sim\mathrm{Poisson}(\lambda)\) con \(\lambda = 2\)

    \hypertarget{a-binomial}{%
\subsection{a) Binomial}\label{a-binomial}}

    \begin{tcolorbox}[breakable, size=fbox, boxrule=1pt, pad at break*=1mm,colback=cellbackground, colframe=cellborder]
\prompt{In}{incolor}{40}{\boxspacing}
\begin{Verbatim}[commandchars=\\\{\}]
\PY{k}{def}\PY{+w}{ }\PY{n+nf}{rbinom\PYZus{}inv}\PY{p}{(}\PY{n}{ntrials}\PY{p}{:} \PY{n+nb}{int}\PY{p}{,} \PY{n}{p}\PY{p}{:} \PY{n+nb}{float}\PY{p}{,} \PY{n}{size}\PY{p}{:} \PY{n+nb}{int}\PY{p}{)}\PY{p}{:}
\PY{+w}{    }\PY{l+s+sd}{\PYZdq{}\PYZdq{}\PYZdq{}Binomial(ntrials, p) por transformada inversa.\PYZdq{}\PYZdq{}\PYZdq{}}
    \PY{k}{if} \PY{o+ow}{not} \PY{p}{(}\PY{l+m+mi}{0} \PY{o}{\PYZlt{}}\PY{o}{=} \PY{n}{p} \PY{o}{\PYZlt{}}\PY{o}{=} \PY{l+m+mi}{1}\PY{p}{)}\PY{p}{:} 
        \PY{k}{raise} \PY{n+ne}{ValueError}\PY{p}{(}\PY{l+s+s2}{\PYZdq{}}\PY{l+s+s2}{p debe estar en [0,1].}\PY{l+s+s2}{\PYZdq{}}\PY{p}{)}
    \PY{k}{if} \PY{n}{p} \PY{o}{==} \PY{l+m+mi}{0}\PY{p}{:} 
        \PY{k}{return} \PY{p}{[}\PY{l+m+mi}{0}\PY{p}{]}\PY{o}{*}\PY{n}{size}
    \PY{k}{if} \PY{n}{p} \PY{o}{==} \PY{l+m+mi}{1}\PY{p}{:} 
        \PY{k}{return} \PY{p}{[}\PY{n}{ntrials}\PY{p}{]}\PY{o}{*}\PY{n}{size}

    \PY{n}{c} \PY{o}{=} \PY{n}{p}\PY{o}{/}\PY{p}{(}\PY{l+m+mf}{1.0} \PY{o}{\PYZhy{}} \PY{n}{p}\PY{p}{)}                 \PY{c+c1}{\PYZsh{} paso 2: c}
    \PY{n}{out} \PY{o}{=} \PY{p}{[}\PY{p}{]}
    \PY{k}{for} \PY{n}{\PYZus{}} \PY{o+ow}{in} \PY{n+nb}{range}\PY{p}{(}\PY{n}{size}\PY{p}{)}\PY{p}{:}
        \PY{n}{U}  \PY{o}{=} \PY{n}{random}\PY{o}{.}\PY{n}{random}\PY{p}{(}\PY{p}{)}        \PY{c+c1}{\PYZsh{} paso 1}
        \PY{n}{i}  \PY{o}{=} \PY{l+m+mi}{0}                      \PY{c+c1}{\PYZsh{} paso 2}
        \PY{n}{pr} \PY{o}{=} \PY{p}{(}\PY{l+m+mf}{1.0} \PY{o}{\PYZhy{}} \PY{n}{p}\PY{p}{)}\PY{o}{*}\PY{o}{*}\PY{n}{ntrials}     \PY{c+c1}{\PYZsh{} pr = P(X=0)}
        \PY{n}{F}  \PY{o}{=} \PY{n}{pr}
        \PY{k}{if} \PY{n}{U} \PY{o}{\PYZlt{}} \PY{n}{F}\PY{p}{:}                   \PY{c+c1}{\PYZsh{} paso 3–5}
            \PY{n}{out}\PY{o}{.}\PY{n}{append}\PY{p}{(}\PY{n}{i}\PY{p}{)}\PY{p}{;} \PY{k}{continue}
        \PY{k}{while} \PY{k+kc}{True}\PY{p}{:}                 \PY{c+c1}{\PYZsh{} paso 6}
            \PY{n}{pr} \PY{o}{*}\PY{o}{=} \PY{n}{c} \PY{o}{*} \PY{p}{(}\PY{n}{ntrials} \PY{o}{\PYZhy{}} \PY{n}{i}\PY{p}{)} \PY{o}{/} \PY{p}{(}\PY{n}{i} \PY{o}{+} \PY{l+m+mi}{1}\PY{p}{)}  \PY{c+c1}{\PYZsh{} paso 7}
            \PY{n}{F}  \PY{o}{+}\PY{o}{=} \PY{n}{pr}
            \PY{n}{i}  \PY{o}{+}\PY{o}{=} \PY{l+m+mi}{1}
            \PY{k}{if} \PY{n}{U} \PY{o}{\PYZlt{}} \PY{n}{F} \PY{o+ow}{or} \PY{n}{i} \PY{o}{==} \PY{n}{ntrials}\PY{p}{:}          \PY{c+c1}{\PYZsh{} paso 8–11 (con tope)}
                \PY{n}{out}\PY{o}{.}\PY{n}{append}\PY{p}{(}\PY{n}{i}\PY{p}{)}
                \PY{k}{break}
    \PY{k}{return} \PY{n}{out}

\PY{n}{m} \PY{o}{=} \PY{n}{rbinom\PYZus{}inv}\PY{p}{(}\PY{n}{ntrials}\PY{o}{=}\PY{l+m+mi}{10}\PY{p}{,} \PY{n}{p}\PY{o}{=}\PY{l+m+mf}{0.3}\PY{p}{,} \PY{n}{size}\PY{o}{=}\PY{l+m+mi}{100}\PY{p}{)}
\PY{n+nb}{print}\PY{p}{(}\PY{n}{m}\PY{p}{[}\PY{p}{:}\PY{l+m+mi}{20}\PY{p}{]}\PY{p}{)}
\end{Verbatim}
\end{tcolorbox}

    \begin{Verbatim}[commandchars=\\\{\}]
[3, 1, 4, 2, 3, 4, 5, 2, 4, 3, 3, 3, 5, 3, 3, 3, 4, 3, 5, 3]
    \end{Verbatim}

    \hypertarget{b-geometrica}{%
\subsection{b) Geometrica}\label{b-geometrica}}

    \begin{tcolorbox}[breakable, size=fbox, boxrule=1pt, pad at break*=1mm,colback=cellbackground, colframe=cellborder]
\prompt{In}{incolor}{41}{\boxspacing}
\begin{Verbatim}[commandchars=\\\{\}]
\PY{k}{def}\PY{+w}{ }\PY{n+nf}{rgeom\PYZus{}inv\PYZus{}trials}\PY{p}{(}\PY{n}{p}\PY{p}{:} \PY{n+nb}{float}\PY{p}{,} \PY{n}{n}\PY{p}{:} \PY{n+nb}{int}\PY{p}{)}\PY{p}{:}
\PY{+w}{    }\PY{l+s+sd}{\PYZdq{}\PYZdq{}\PYZdq{}Geom(p) en \PYZob{}1,2,...\PYZcb{} por inversión: floor(log(U)/log(1\PYZhy{}p))+1.\PYZdq{}\PYZdq{}\PYZdq{}}
    \PY{n}{out} \PY{o}{=} \PY{p}{[}\PY{p}{]}
    \PY{k}{for} \PY{n}{\PYZus{}} \PY{o+ow}{in} \PY{n+nb}{range}\PY{p}{(}\PY{n}{n}\PY{p}{)}\PY{p}{:}
        \PY{n}{u} \PY{o}{=} \PY{n}{random}\PY{o}{.}\PY{n}{random}\PY{p}{(}\PY{p}{)}       
        \PY{n}{x} \PY{o}{=} \PY{n}{math}\PY{o}{.}\PY{n}{ceil}\PY{p}{(}\PY{n}{math}\PY{o}{.}\PY{n}{log}\PY{p}{(}\PY{n}{u}\PY{p}{)} \PY{o}{/} \PY{n}{math}\PY{o}{.}\PY{n}{log}\PY{p}{(}\PY{l+m+mf}{1.0} \PY{o}{\PYZhy{}} \PY{n}{p}\PY{p}{)}\PY{p}{)}
        \PY{n}{out}\PY{o}{.}\PY{n}{append}\PY{p}{(}\PY{n}{x}\PY{p}{)}
    \PY{k}{return} \PY{n}{out}

\PY{c+c1}{\PYZsh{} Ejemplo: p=0.3, n=100}
\PY{n}{muestras} \PY{o}{=} \PY{n}{rgeom\PYZus{}inv\PYZus{}trials}\PY{p}{(}\PY{l+m+mf}{0.3}\PY{p}{,} \PY{l+m+mi}{100}\PY{p}{)}
\PY{n+nb}{print}\PY{p}{(}\PY{n}{muestras}\PY{p}{[}\PY{p}{:}\PY{l+m+mi}{20}\PY{p}{]}\PY{p}{)}
\end{Verbatim}
\end{tcolorbox}

    \begin{Verbatim}[commandchars=\\\{\}]
[3, 2, 8, 5, 2, 3, 1, 5, 9, 2, 1, 1, 6, 1, 13, 1, 3, 1, 11, 3]
    \end{Verbatim}

    \hypertarget{c-poisson}{%
\subsection{c) Poisson}\label{c-poisson}}

    \begin{tcolorbox}[breakable, size=fbox, boxrule=1pt, pad at break*=1mm,colback=cellbackground, colframe=cellborder]
\prompt{In}{incolor}{42}{\boxspacing}
\begin{Verbatim}[commandchars=\\\{\}]
\PY{k}{def}\PY{+w}{ }\PY{n+nf}{rpois\PYZus{}inv}\PY{p}{(}\PY{n}{lam}\PY{p}{:} \PY{n+nb}{float}\PY{p}{,} \PY{n}{n}\PY{p}{:} \PY{n+nb}{int}\PY{p}{)}\PY{p}{:}
\PY{+w}{    }\PY{l+s+sd}{\PYZdq{}\PYZdq{}\PYZdq{}Poisson(lam) por transformada inversa, siguiendo el algoritmo dado.\PYZdq{}\PYZdq{}\PYZdq{}}
    \PY{k}{if} \PY{n}{lam} \PY{o}{\PYZlt{}}\PY{o}{=} \PY{l+m+mi}{0}\PY{p}{:} 
        \PY{k}{raise} \PY{n+ne}{ValueError}\PY{p}{(}\PY{l+s+s2}{\PYZdq{}}\PY{l+s+s2}{lam debe ser \PYZgt{} 0}\PY{l+s+s2}{\PYZdq{}}\PY{p}{)}

    \PY{n}{out} \PY{o}{=} \PY{p}{[}\PY{p}{]}
    \PY{k}{for} \PY{n}{\PYZus{}} \PY{o+ow}{in} \PY{n+nb}{range}\PY{p}{(}\PY{n}{n}\PY{p}{)}\PY{p}{:}
        \PY{c+c1}{\PYZsh{} 1) U \PYZti{} U(0,1)}
        \PY{n}{U} \PY{o}{=} \PY{n}{random}\PY{o}{.}\PY{n}{random}\PY{p}{(}\PY{p}{)}
        \PY{c+c1}{\PYZsh{} 2) i=0, p=e\PYZca{}\PYZob{}\PYZhy{}lam\PYZcb{}, F=p}
        \PY{n}{i} \PY{o}{=} \PY{l+m+mi}{0}
        \PY{n}{p} \PY{o}{=} \PY{n}{math}\PY{o}{.}\PY{n}{exp}\PY{p}{(}\PY{o}{\PYZhy{}}\PY{n}{lam}\PY{p}{)}
        \PY{n}{F} \PY{o}{=} \PY{n}{p}
        \PY{c+c1}{\PYZsh{} 3) if U \PYZlt{} F: X=i}
        \PY{k}{if} \PY{n}{U} \PY{o}{\PYZlt{}} \PY{n}{F}\PY{p}{:}
            \PY{n}{out}\PY{o}{.}\PY{n}{append}\PY{p}{(}\PY{n}{i}\PY{p}{)}
            \PY{k}{continue}
        \PY{c+c1}{\PYZsh{} 6–11) loop: p = (lam/(i+1))*p; F = F + p; i = i + 1; if U \PYZlt{} F: X=i}
        \PY{k}{while} \PY{k+kc}{True}\PY{p}{:}
            \PY{n}{i} \PY{o}{+}\PY{o}{=} \PY{l+m+mi}{1}
            \PY{n}{p} \PY{o}{=} \PY{n}{p} \PY{o}{*} \PY{n}{lam} \PY{o}{/} \PY{n}{i}
            \PY{n}{F} \PY{o}{=} \PY{n}{F} \PY{o}{+} \PY{n}{p}
            \PY{k}{if} \PY{n}{U} \PY{o}{\PYZlt{}} \PY{n}{F}\PY{p}{:}
                \PY{n}{out}\PY{o}{.}\PY{n}{append}\PY{p}{(}\PY{n}{i}\PY{p}{)}
                \PY{k}{break}
    \PY{k}{return} \PY{n}{out}

\PY{n}{muestras} \PY{o}{=} \PY{n}{rpois\PYZus{}inv}\PY{p}{(}\PY{n}{lam}\PY{o}{=}\PY{l+m+mf}{2.0}\PY{p}{,} \PY{n}{n}\PY{o}{=}\PY{l+m+mi}{100}\PY{p}{)}
\PY{n+nb}{print}\PY{p}{(}\PY{n}{muestras}\PY{p}{[}\PY{p}{:}\PY{l+m+mi}{20}\PY{p}{]}\PY{p}{)}
\end{Verbatim}
\end{tcolorbox}

    \begin{Verbatim}[commandchars=\\\{\}]
[5, 1, 1, 2, 2, 4, 0, 3, 4, 2, 1, 2, 1, 0, 3, 0, 3, 1, 2, 0]
    \end{Verbatim}


    % Add a bibliography block to the postdoc
    
    
    
\end{document}
