% jupyter nbconvert 'notebook.ipynb' --to latex

\documentclass[10pt,a4paper]{article}

    \usepackage[spanish]{babel} % Idioma del documento, fundamental para la correcta separación de sílabas y traducción de términos
    \usepackage[unicode]{hyperref}
    \usepackage[breakable]{tcolorbox}
    \usepackage{parskip}
    \usepackage{graphicx}

    % \usepackage{amssymb}
    \AtBeginDocument{%
      \let\le\leqslant
      \let\ge\geqslant
      \let\leq\leqslant
      \let\geq\geqslant
    }



    \setkeys{Gin}{keepaspectratio}

    \let\Oldincludegraphics\includegraphics
  
    \usepackage{caption}
    \DeclareCaptionFormat{nocaption}{}
    \captionsetup{format=nocaption,aboveskip=0pt,belowskip=0pt}

    \usepackage{float}
    \floatplacement{figure}{H} % forces figures to be placed at the correct location
    \usepackage{xcolor} % Allow colors to be defined
    \usepackage{enumerate} % Needed for markdown enumerations to work
    \usepackage{geometry} % Used to adjust the document margins
    \usepackage{amsmath} % Equations
    \usepackage{amssymb}
    \usepackage{textcomp}
    \AtBeginDocument{%
        \def\PYZsq{\textquotesingle}% Upright quotes in Pygmentized code
    }
    \usepackage{upquote} % Upright quotes for verbatim code
    \usepackage{eurosym} % defines \euro

    \usepackage{iftex}
    \ifPDFTeX
        \usepackage[T1]{fontenc}
        \IfFileExists{alphabeta.sty}{
              \usepackage{alphabeta}
          }{
              \usepackage[mathletters]{ucs}
              \usepackage[utf8x]{inputenc}
          }
    \else
        \usepackage{fontspec}
        \usepackage{unicode-math}
    \fi

    % --- Change the cell font ---
    % \usepackage{fontspec} 
    \newfontface\myfont{Inconsolata}[Contextuals={WordInitial,WordFinal}]{}  

    \usepackage{fancyvrb}
    \fvset{
    commandchars=\\\{\},
    fontsize=\scriptsize,
    formatcom=\myfont
    }

    \usepackage{grffile} 
    \makeatletter 
    \@ifpackagelater{grffile}{2019/11/01}
    {
      % Do nothing on new versions
    }
    {
      \def\Gread@@xetex#1{%
        \IfFileExists{"\Gin@base".bb}%
        {\Gread@eps{\Gin@base.bb}}%
        {\Gread@@xetex@aux#1}%
      }
    }
    \makeatother

    \usepackage[Export]{adjustbox} % Used to constrain images to a maximum size
    \adjustboxset{max size={0.9\linewidth}{0.9\paperheight}}


    
    \usepackage{titling}
    \usepackage{longtable}
    \usepackage{booktabs}
    \usepackage{array}
    \usepackage{calc}
    \usepackage[inline]{enumitem}
    \usepackage[normalem]{ulem}
    \usepackage{soul} 
    \usepackage{mathrsfs}

    
    % Colors for the hyperref package
    \definecolor{urlcolor}{rgb}{0.00,0.00,1.00}
    \definecolor{linkcolor}{rgb}{0.00,0.00,0.00}
    \definecolor{citecolor}{rgb}{0.149,0.498,0.600}

    % ANSI colors
    \definecolor{ansi-black}{HTML}{3E424D}
    \definecolor{ansi-black-intense}{HTML}{282C36}
    \definecolor{ansi-red}{HTML}{E75C58}
    \definecolor{ansi-red-intense}{HTML}{B22B31}
    \definecolor{ansi-green}{HTML}{00A250}
    \definecolor{ansi-green-intense}{HTML}{007427}
    \definecolor{ansi-yellow}{HTML}{DDB62B}
    \definecolor{ansi-yellow-intense}{HTML}{B27D12}
    \definecolor{ansi-blue}{HTML}{208FFB}
    \definecolor{ansi-blue-intense}{HTML}{0065CA}
    \definecolor{ansi-magenta}{HTML}{D160C4}
    \definecolor{ansi-magenta-intense}{HTML}{A03196}
    \definecolor{ansi-cyan}{HTML}{60C6C8}
    \definecolor{ansi-cyan-intense}{HTML}{258F8F}
    \definecolor{ansi-white}{HTML}{C5C1B4}
    \definecolor{ansi-white-intense}{HTML}{A1A6B2}
    \definecolor{ansi-default-inverse-fg}{HTML}{FFFFFF}
    \definecolor{ansi-default-inverse-bg}{HTML}{000000}

    % common color for the border for error outputs.
    \definecolor{outerrorbackground}{HTML}{FFFFFF}


    % commands and environments needed by pandoc snippets
    % extracted from the output of `pandoc -s`
    \providecommand{\tightlist}{%
      \setlength{\itemsep}{0pt}\setlength{\parskip}{0pt}}
    \DefineVerbatimEnvironment{Highlighting}{Verbatim}{
        commandchars=\\\{\}
    }

    \newenvironment{Shaded}{}{}
    \newcommand{\KeywordTok}[1]{\textcolor[rgb]{0.00,0.00,1.00}{\textbf{{#1}}}}
    \newcommand{\DataTypeTok}[1]{\textcolor[rgb]{0.149,0.498,0.600}{{#1}}}
    \newcommand{\DecValTok}[1]{\textcolor[rgb]{0.035,0.525,0.345}{{#1}}}
    \newcommand{\BaseNTok}[1]{\textcolor[rgb]{0.035,0.525,0.345}{{#1}}}
    \newcommand{\FloatTok}[1]{\textcolor[rgb]{0.035,0.525,0.345}{{#1}}}
    \newcommand{\CharTok}[1]{\textcolor[rgb]{0.639,0.082,0.082}{{#1}}}
    \newcommand{\StringTok}[1]{\textcolor[rgb]{0.639,0.082,0.082}{{#1}}}
    \newcommand{\CommentTok}[1]{\textcolor[rgb]{0.000,0.502,0.000}{\textit{{#1}}}}
    \newcommand{\OtherTok}[1]{\textcolor[rgb]{0.00,0.00,1.00}{{#1}}}
    \newcommand{\AlertTok}[1]{\textcolor[rgb]{1.00,0.00,0.00}{\textbf{{#1}}}}
    \newcommand{\FunctionTok}[1]{\textcolor[rgb]{0.475,0.369,0.149}{{#1}}}
    \newcommand{\RegionMarkerTok}[1]{{#1}}
    \newcommand{\ErrorTok}[1]{\textcolor[rgb]{1.00,0.00,0.00}{\textbf{{#1}}}}
    \newcommand{\NormalTok}[1]{{#1}}

    % Additional commands for more recent versions of Pandoc
    \newcommand{\ConstantTok}[1]{\textcolor[rgb]{0.00,0.00,1.00}{{#1}}}
    \newcommand{\SpecialCharTok}[1]{\textcolor[rgb]{0.639,0.082,0.082}{{#1}}}
    \newcommand{\VerbatimStringTok}[1]{\textcolor[rgb]{0.639,0.082,0.082}{{#1}}}
    \newcommand{\SpecialStringTok}[1]{\textcolor[rgb]{0.639,0.082,0.082}{{#1}}}
    \newcommand{\ImportTok}[1]{{#1}}
    \newcommand{\DocumentationTok}[1]{\textcolor[rgb]{0.000,0.502,0.000}{\textit{{#1}}}}
    \newcommand{\AnnotationTok}[1]{\textcolor[rgb]{0.706,0.000,0.620}{\textbf{\textit{{#1}}}}}
    \newcommand{\CommentVarTok}[1]{\textcolor[rgb]{0.000,0.502,0.000}{\textbf{\textit{{#1}}}}}
    \newcommand{\VariableTok}[1]{\textcolor[rgb]{0.000,0.063,0.502}{{#1}}}
    \newcommand{\ControlFlowTok}[1]{\textcolor[rgb]{0.00,0.00,1.00}{\textbf{{#1}}}}
    \newcommand{\OperatorTok}[1]{\textcolor[rgb]{0.000,0.000,0.000}{{#1}}}
    \newcommand{\BuiltInTok}[1]{{#1}}
    \newcommand{\ExtensionTok}[1]{{#1}}
    \newcommand{\PreprocessorTok}[1]{\textcolor[rgb]{0.706,0.000,0.620}{{#1}}}
    \newcommand{\AttributeTok}[1]{\textcolor[rgb]{0.149,0.498,0.600}{{#1}}}
    \newcommand{\InformationTok}[1]{\textcolor[rgb]{0.000,0.502,0.000}{\textbf{\textit{{#1}}}}}
    \newcommand{\WarningTok}[1]{\textcolor[rgb]{0.000,0.502,0.000}{\textbf{\textit{{#1}}}}}
    \makeatletter

    \newsavebox\pandoc@box
    \newcommand*\pandocbounded[1]{%
      \sbox\pandoc@box{#1}%
      % scaling factors for width and height
      \Gscale@div\@tempa\textheight{\dimexpr\ht\pandoc@box+\dp\pandoc@box\relax}%
      \Gscale@div\@tempb\linewidth{\wd\pandoc@box}%
      % select the smaller of both
      \ifdim\@tempb\p@<\@tempa\p@
        \let\@tempa\@tempb
      \fi
      % scaling accordingly (\@tempa < 1)
      \ifdim\@tempa\p@<\p@
        \scalebox{\@tempa}{\usebox\pandoc@box}%
      % scaling not needed, use as it is
      \else
        \usebox{\pandoc@box}%
      \fi
    }
    \makeatother


    \def\br{\hspace*{\fill} \\* }
    % Math Jax compatibility definitions
    \def\gt{>}
    \def\lt{<}
    \let\Oldtex\TeX
    \let\Oldlatex\LaTeX
    \renewcommand{\TeX}{\textrm{\Oldtex}}
    \renewcommand{\LaTeX}{\textrm{\Oldlatex}}

    % Pygments definitions
    \makeatletter
    \def\PY@reset{\let\PY@it=\relax \let\PY@bf=\relax%
        \let\PY@ul=\relax \let\PY@tc=\relax%
        \let\PY@bc=\relax \let\PY@ff=\relax}
    \def\PY@tok#1{\csname PY@tok@#1\endcsname}
    \def\PY@toks#1+{\ifx\relax#1\empty\else%
        \PY@tok{#1}\expandafter\PY@toks\fi}
    \def\PY@do#1{\PY@bc{\PY@tc{\PY@ul{%
        \PY@it{\PY@bf{\PY@ff{#1}}}}}}}
    \def\PY#1#2{\PY@reset\PY@toks#1+\relax+\PY@do{#2}}

    \@namedef{PY@tok@w}{\def\PY@tc##1{\textcolor[rgb]{0.73,0.73,0.73}{##1}}}
    \@namedef{PY@tok@c}{\let\PY@it=\textit\def\PY@tc##1{\textcolor[rgb]{0.24,0.48,0.48}{##1}}}
    \@namedef{PY@tok@cp}{\def\PY@tc##1{\textcolor[rgb]{0.61,0.40,0.00}{##1}}}
    \@namedef{PY@tok@k}{\let\PY@bf=\textbf\def\PY@tc##1{\textcolor[rgb]{0.00,0.50,0.00}{##1}}}
    \@namedef{PY@tok@kp}{\def\PY@tc##1{\textcolor[rgb]{0.00,0.50,0.00}{##1}}}
    \@namedef{PY@tok@kt}{\def\PY@tc##1{\textcolor[rgb]{0.69,0.00,0.25}{##1}}}
    \@namedef{PY@tok@o}{\def\PY@tc##1{\textcolor[rgb]{0.40,0.40,0.40}{##1}}}
    \@namedef{PY@tok@ow}{\let\PY@bf=\textbf\def\PY@tc##1{\textcolor[rgb]{0.67,0.13,1.00}{##1}}}
    \@namedef{PY@tok@nb}{\def\PY@tc##1{\textcolor[rgb]{0.00,0.50,0.00}{##1}}}
    \@namedef{PY@tok@nf}{\def\PY@tc##1{\textcolor[rgb]{0.00,0.00,1.00}{##1}}}
    \@namedef{PY@tok@nc}{\let\PY@bf=\textbf\def\PY@tc##1{\textcolor[rgb]{0.00,0.00,1.00}{##1}}}
    \@namedef{PY@tok@nn}{\let\PY@bf=\textbf\def\PY@tc##1{\textcolor[rgb]{0.00,0.00,1.00}{##1}}}
    \@namedef{PY@tok@ne}{\let\PY@bf=\textbf\def\PY@tc##1{\textcolor[rgb]{0.80,0.25,0.22}{##1}}}
    \@namedef{PY@tok@nv}{\def\PY@tc##1{\textcolor[rgb]{0.10,0.09,0.49}{##1}}}
    \@namedef{PY@tok@no}{\def\PY@tc##1{\textcolor[rgb]{0.53,0.00,0.00}{##1}}}
    \@namedef{PY@tok@nl}{\def\PY@tc##1{\textcolor[rgb]{0.46,0.46,0.00}{##1}}}
    \@namedef{PY@tok@ni}{\let\PY@bf=\textbf\def\PY@tc##1{\textcolor[rgb]{0.44,0.44,0.44}{##1}}}
    \@namedef{PY@tok@na}{\def\PY@tc##1{\textcolor[rgb]{0.41,0.47,0.13}{##1}}}
    \@namedef{PY@tok@nt}{\let\PY@bf=\textbf\def\PY@tc##1{\textcolor[rgb]{0.00,0.50,0.00}{##1}}}
    \@namedef{PY@tok@nd}{\def\PY@tc##1{\textcolor[rgb]{0.67,0.13,1.00}{##1}}}
    \@namedef{PY@tok@s}{\def\PY@tc##1{\textcolor[rgb]{0.73,0.13,0.13}{##1}}}
    \@namedef{PY@tok@sd}{\let\PY@it=\textit\def\PY@tc##1{\textcolor[rgb]{0.73,0.13,0.13}{##1}}}
    \@namedef{PY@tok@si}{\let\PY@bf=\textbf\def\PY@tc##1{\textcolor[rgb]{0.64,0.35,0.47}{##1}}}
    \@namedef{PY@tok@se}{\let\PY@bf=\textbf\def\PY@tc##1{\textcolor[rgb]{0.67,0.36,0.12}{##1}}}
    \@namedef{PY@tok@sr}{\def\PY@tc##1{\textcolor[rgb]{0.64,0.35,0.47}{##1}}}
    \@namedef{PY@tok@ss}{\def\PY@tc##1{\textcolor[rgb]{0.10,0.09,0.49}{##1}}}
    \@namedef{PY@tok@sx}{\def\PY@tc##1{\textcolor[rgb]{0.00,0.50,0.00}{##1}}}
    \@namedef{PY@tok@m}{\def\PY@tc##1{\textcolor[rgb]{0.40,0.40,0.40}{##1}}}
    \@namedef{PY@tok@gh}{\let\PY@bf=\textbf\def\PY@tc##1{\textcolor[rgb]{0.00,0.00,0.50}{##1}}}
    \@namedef{PY@tok@gu}{\let\PY@bf=\textbf\def\PY@tc##1{\textcolor[rgb]{0.50,0.00,0.50}{##1}}}
    \@namedef{PY@tok@gd}{\def\PY@tc##1{\textcolor[rgb]{0.63,0.00,0.00}{##1}}}
    \@namedef{PY@tok@gi}{\def\PY@tc##1{\textcolor[rgb]{0.00,0.52,0.00}{##1}}}
    \@namedef{PY@tok@gr}{\def\PY@tc##1{\textcolor[rgb]{0.89,0.00,0.00}{##1}}}
    \@namedef{PY@tok@ge}{\let\PY@it=\textit}
    \@namedef{PY@tok@gs}{\let\PY@bf=\textbf}
    \@namedef{PY@tok@ges}{\let\PY@bf=\textbf\let\PY@it=\textit}
    \@namedef{PY@tok@gp}{\let\PY@bf=\textbf\def\PY@tc##1{\textcolor[rgb]{0.00,0.00,0.50}{##1}}}
    \@namedef{PY@tok@go}{\def\PY@tc##1{\textcolor[rgb]{0.44,0.44,0.44}{##1}}}
    \@namedef{PY@tok@gt}{\def\PY@tc##1{\textcolor[rgb]{0.00,0.27,0.87}{##1}}}
    \@namedef{PY@tok@err}{\def\PY@bc##1{{\setlength{\fboxsep}{\string -\fboxrule}\fcolorbox[rgb]{1.00,0.00,0.00}{1,1,1}{\strut ##1}}}}
    \@namedef{PY@tok@kc}{\let\PY@bf=\textbf\def\PY@tc##1{\textcolor[rgb]{0.00,0.50,0.00}{##1}}}
    \@namedef{PY@tok@kd}{\let\PY@bf=\textbf\def\PY@tc##1{\textcolor[rgb]{0.00,0.50,0.00}{##1}}}
    \@namedef{PY@tok@kn}{\let\PY@bf=\textbf\def\PY@tc##1{\textcolor[rgb]{0.00,0.50,0.00}{##1}}}
    \@namedef{PY@tok@kr}{\let\PY@bf=\textbf\def\PY@tc##1{\textcolor[rgb]{0.00,0.50,0.00}{##1}}}
    \@namedef{PY@tok@bp}{\def\PY@tc##1{\textcolor[rgb]{0.00,0.50,0.00}{##1}}}
    \@namedef{PY@tok@fm}{\def\PY@tc##1{\textcolor[rgb]{0.00,0.00,1.00}{##1}}}
    \@namedef{PY@tok@vc}{\def\PY@tc##1{\textcolor[rgb]{0.10,0.09,0.49}{##1}}}
    \@namedef{PY@tok@vg}{\def\PY@tc##1{\textcolor[rgb]{0.10,0.09,0.49}{##1}}}
    \@namedef{PY@tok@vi}{\def\PY@tc##1{\textcolor[rgb]{0.10,0.09,0.49}{##1}}}
    \@namedef{PY@tok@vm}{\def\PY@tc##1{\textcolor[rgb]{0.10,0.09,0.49}{##1}}}
    \@namedef{PY@tok@sa}{\def\PY@tc##1{\textcolor[rgb]{0.73,0.13,0.13}{##1}}}
    \@namedef{PY@tok@sb}{\def\PY@tc##1{\textcolor[rgb]{0.73,0.13,0.13}{##1}}}
    \@namedef{PY@tok@sc}{\def\PY@tc##1{\textcolor[rgb]{0.73,0.13,0.13}{##1}}}
    \@namedef{PY@tok@dl}{\def\PY@tc##1{\textcolor[rgb]{0.73,0.13,0.13}{##1}}}
    \@namedef{PY@tok@s2}{\def\PY@tc##1{\textcolor[rgb]{0.73,0.13,0.13}{##1}}}
    \@namedef{PY@tok@sh}{\def\PY@tc##1{\textcolor[rgb]{0.73,0.13,0.13}{##1}}}
    \@namedef{PY@tok@s1}{\def\PY@tc##1{\textcolor[rgb]{0.73,0.13,0.13}{##1}}}
    \@namedef{PY@tok@mb}{\def\PY@tc##1{\textcolor[rgb]{0.40,0.40,0.40}{##1}}}
    \@namedef{PY@tok@mf}{\def\PY@tc##1{\textcolor[rgb]{0.40,0.40,0.40}{##1}}}
    \@namedef{PY@tok@mh}{\def\PY@tc##1{\textcolor[rgb]{0.40,0.40,0.40}{##1}}}
    \@namedef{PY@tok@mi}{\def\PY@tc##1{\textcolor[rgb]{0.40,0.40,0.40}{##1}}}
    \@namedef{PY@tok@il}{\def\PY@tc##1{\textcolor[rgb]{0.40,0.40,0.40}{##1}}}
    \@namedef{PY@tok@mo}{\def\PY@tc##1{\textcolor[rgb]{0.40,0.40,0.40}{##1}}}
    \@namedef{PY@tok@ch}{\let\PY@it=\textit\def\PY@tc##1{\textcolor[rgb]{0.24,0.48,0.48}{##1}}}
    \@namedef{PY@tok@cm}{\let\PY@it=\textit\def\PY@tc##1{\textcolor[rgb]{0.24,0.48,0.48}{##1}}}
    \@namedef{PY@tok@cpf}{\let\PY@it=\textit\def\PY@tc##1{\textcolor[rgb]{0.24,0.48,0.48}{##1}}}
    \@namedef{PY@tok@c1}{\let\PY@it=\textit\def\PY@tc##1{\textcolor[rgb]{0.24,0.48,0.48}{##1}}}
    \@namedef{PY@tok@cs}{\let\PY@it=\textit\def\PY@tc##1{\textcolor[rgb]{0.24,0.48,0.48}{##1}}}

    \def\PYZbs{\char`\\}
    \def\PYZus{\char`\_}
    \def\PYZob{\char`\{}
    \def\PYZcb{\char`\}}
    \def\PYZca{\char`\^}
    \def\PYZam{\char`\&}
    \def\PYZlt{\char`\<}
    \def\PYZgt{\char`\>}
    \def\PYZsh{\char`\#}
    \def\PYZpc{\char`\%}
    \def\PYZdl{\char`\$}
    \def\PYZhy{\char`\-}
    \def\PYZsq{\char`\'}
    \def\PYZdq{\char`\"}
    \def\PYZti{\char`\~}
    \def\PYZat{@}
    \def\PYZlb{[}
    \def\PYZrb{]}
    \makeatother


    % For linebreaks inside Verbatim environment from package fancyvrb.
    \makeatletter
        \newbox\Wrappedcontinuationbox
        \newbox\Wrappedvisiblespacebox
        \newcommand*\Wrappedvisiblespace {\textcolor{red}{\textvisiblespace}}
        \newcommand*\Wrappedcontinuationsymbol {\textcolor{red}{\llap{\tiny$\m@th\hookrightarrow$}}}
        \newcommand*\Wrappedcontinuationindent {3ex }
        \newcommand*\Wrappedafterbreak {\kern\Wrappedcontinuationindent\copy\Wrappedcontinuationbox}
        \newcommand*\Wrappedbreaksatspecials {%
            \def\PYGZus{\discretionary{\char`\_}{\Wrappedafterbreak}{\char`\_}}%
            \def\PYGZob{\discretionary{}{\Wrappedafterbreak\char`\{}{\char`\{}}%
            \def\PYGZcb{\discretionary{\char`\}}{\Wrappedafterbreak}{\char`\}}}%
            \def\PYGZca{\discretionary{\char`\^}{\Wrappedafterbreak}{\char`\^}}%
            \def\PYGZam{\discretionary{\char`\&}{\Wrappedafterbreak}{\char`\&}}%
            \def\PYGZlt{\discretionary{}{\Wrappedafterbreak\char`\<}{\char`\<}}%
            \def\PYGZgt{\discretionary{\char`\>}{\Wrappedafterbreak}{\char`\>}}%
            \def\PYGZsh{\discretionary{}{\Wrappedafterbreak\char`\#}{\char`\#}}%
            \def\PYGZpc{\discretionary{}{\Wrappedafterbreak\char`\%}{\char`\%}}%
            \def\PYGZdl{\discretionary{}{\Wrappedafterbreak\char`\$}{\char`\$}}%
            \def\PYGZhy{\discretionary{\char`\-}{\Wrappedafterbreak}{\char`\-}}%
            \def\PYGZsq{\discretionary{}{\Wrappedafterbreak\textquotesingle}{\textquotesingle}}%
            \def\PYGZdq{\discretionary{}{\Wrappedafterbreak\char`\"}{\char`\"}}%
            \def\PYGZti{\discretionary{\char`\~}{\Wrappedafterbreak}{\char`\~}}%
        }
        \newcommand*\Wrappedbreaksatpunct {%
            \lccode`\~`\.\lowercase{\def~}{\discretionary{\hbox{\char`\.}}{\Wrappedafterbreak}{\hbox{\char`\.}}}%
            \lccode`\~`\,\lowercase{\def~}{\discretionary{\hbox{\char`\,}}{\Wrappedafterbreak}{\hbox{\char`\,}}}%
            \lccode`\~`\;\lowercase{\def~}{\discretionary{\hbox{\char`\;}}{\Wrappedafterbreak}{\hbox{\char`\;}}}%
            \lccode`\~`\:\lowercase{\def~}{\discretionary{\hbox{\char`\:}}{\Wrappedafterbreak}{\hbox{\char`\:}}}%
            \lccode`\~`\?\lowercase{\def~}{\discretionary{\hbox{\char`\?}}{\Wrappedafterbreak}{\hbox{\char`\?}}}%
            \lccode`\~`\!\lowercase{\def~}{\discretionary{\hbox{\char`\!}}{\Wrappedafterbreak}{\hbox{\char`\!}}}%
            \lccode`\~`\/\lowercase{\def~}{\discretionary{\hbox{\char`\/}}{\Wrappedafterbreak}{\hbox{\char`\/}}}%
            \catcode`\.\active
            \catcode`\,\active
            \catcode`\;\active
            \catcode`\:\active
            \catcode`\?\active
            \catcode`\!\active
            \catcode`\/\active
            \lccode`\~`\~
        }
    \makeatother

    \let\OriginalVerbatim=\Verbatim
    \makeatletter
    \renewcommand{\Verbatim}[1][1]{%
        %\parskip\z@skip
        \sbox\Wrappedcontinuationbox {\Wrappedcontinuationsymbol}%
        \sbox\Wrappedvisiblespacebox {\FV@SetupFont\Wrappedvisiblespace}%
        \def\FancyVerbFormatLine ##1{\hsize\linewidth
            \vtop{\raggedright\hyphenpenalty\z@\exhyphenpenalty\z@
                \doublehyphendemerits\z@\finalhyphendemerits\z@
                \strut ##1\strut}%
        }%
        % If the linebreak is at a space, the latter will be displayed as visible
        % space at end of first line, and a continuation symbol starts next line.
        % Stretch/shrink are however usually zero for typewriter font.
        \def\FV@Space {%
            \nobreak\hskip\z@ plus\fontdimen3\font minus\fontdimen4\font
            \discretionary{\copy\Wrappedvisiblespacebox}{\Wrappedafterbreak}
            {\kern\fontdimen2\font}%
        }%

        % Allow breaks at special characters using \PYG... macros.
        \Wrappedbreaksatspecials
        % Breaks at punctuation characters . , ; ? ! and / need catcode=\active
        \OriginalVerbatim[#1,codes*=\Wrappedbreaksatpunct]%
    }
    \makeatother

    % Exact colors from NB
    \definecolor{incolor}{HTML}{303F9F}
    \definecolor{outcolor}{HTML}{D84315}
    \definecolor{cellborder}{HTML}{F7F7F7}
    \definecolor{cellbackground}{HTML}{F7F7F7}

    % prompt
    \makeatletter
    \newcommand{\boxspacing}{\kern\kvtcb@left@rule\kern\kvtcb@boxsep}
    \makeatother
    \newcommand{\prompt}[4]{
        {\ttfamily\llap{{\color{#2}[#3]:\hspace{3pt}#4}}\vspace{-\baselineskip}}
    }
    

    
    % Prevent overflowing lines due to hard-to-break entities
    \sloppy
    % Setup hyperref package
    \hypersetup{
      breaklinks=true,  % so long urls are correctly broken across lines
      colorlinks=true,
      urlcolor=urlcolor,
      linkcolor=linkcolor,
      citecolor=citecolor,
      }
    % Slightly bigger margins than the latex defaults
        
    \geometry{verbose,tmargin=1.5in,bmargin=1in,lmargin=1.125in,rmargin=1.125in}
    


    % --- Nuevos caracteres Unicode ----
    \usepackage{newunicodechar}
    \newunicodechar{∼}{\ensuremath{\sim}}
    \newunicodechar{←}{\ensuremath{\leftarrow}}
    \newunicodechar{…}{\ldots}
    % \le y \ge se vuelven \leq y \geq
    \let\le\leq
    \let\ge\geq


    % --- Fuentes (texto y matemáticas) ---
    \usepackage{libertinust1math}
    \usepackage{fontspec}
    \setmainfont{EB Garamond}[
        UprightFont = * Regular,
        ItalicFont = * Italic,
        BoldFont = * SemiBold,
        BoldItalicFont = * SemiBold Italic,
    ]


    % --- Secciones y subsecciones ---
    \usepackage{titlesec}
    \newfontface\boldd{EB Garamond Bold}
    \newfontface\bolditalic{EB Garamond  Bold Italic}
    \newfontface\extrabold{EB Garamond ExtraBold}
    \newfontface\medium{EB Garamond Medium}



    \usepackage{bookmark}              % mejora anchors
    \hypersetup{hypertexnames=false}   % evita destinos idénticos

    \usepackage{etoolbox,needspace}

    % Rompe página solo en \section y crea ancla propia
    \pretocmd{\section}{\clearpage\phantomsection\needspace{6\baselineskip}}{}{}

    % No rompas página en \subsection; solo asegura espacio
    \pretocmd{\subsection}{\phantomsection\needspace{4\baselineskip}\clearpage}{}{}

    \setcounter{secnumdepth}{0}


    % --- Hacer los títulos de sección y subsección más grandes ---
    \titleformat{\section}
      {\boldd\fontsize{34pt}{128pt}\selectfont} % el primer {} es el formato, el segundo {} es el tamaño de línea
      {\thesection}{18em}{} % el primer {} es el formato, el segundo {} es la separación entre número y título
      \titleformat{\subsection}
        {\boldd\fontsize{18pt}{18pt}\selectfont}
        {\thesubsection}{10em}{}
    \titlespacing*{\section}{0pt}{0pt}{24pt} % el primer {} es la sangría, el segundo {} es el espacio antes, el tercero {} es el espacio después
    \titlespacing*{\subsection}{0pt}{0pt}{18pt} % el primer {} es la sangría, el segundo {} es el espacio antes, el tercero {} es el espacio después

    % --- Encabezado con nombre de la sección ---
    \usepackage{fancyhdr}
    \pagestyle{fancy}
    \fancyhf{}
    \fancyhead[L]{\small\leftmark}
    \fancyhead[R]{\small\thepage}
    \fancyhead[C]{\small\textit{\rightmark}}
    \renewcommand{\headrulewidth}{0.1pt}

    % --- Crear un nuevo estilo de pagina con el numero arriba a la derecha ---
    \fancypagestyle{myfancy}{
      \fancyhf{}
      \fancyhead[R]{\small\thepage}
      \renewcommand{\headrulewidth}{0pt}
    }

        % --- Cambia el pagestyle a 'plain' en cada \section ---
    \let\oldsection\section
    \renewcommand{\section}{%
      \clearpage
      \thispagestyle{myfancy}%
      \oldsection
    }

    % --- Espaciado ---
    \usepackage{setspace}
    \setstretch{1.2}
    \setlength{\parskip}{0.2em}

    \begin{document}

    % --- Portada ---
    \begin{titlepage}
    \centering
    \vspace*{4cm}
    {\extrabold\fontsize{28pt}{28pt}\selectfont Método de Aceptación y Rechazo\par}
    \vspace{1cm}
    
    {\medium\fontsize{12pt}{12pt}\selectfont
    {\large Curso:  Temas Selectos I: O25 LAT4032 1\par}
    {\large Profesor:  Rubén Blancas Rivera\par}
    {\large Equipo: jijiji, jujuju, jojojo\par}
    {\large Universidad de las Américas Puebla\par}
    \vfill
    {\large 11 de Septiembre de 2025 \par}}
    \end{titlepage}


    % --- Tabla de contenidos ---
    \renewcommand{\contentsname}{Índice}
    \tableofcontents
    \thispagestyle{empty}
    \newpage

    % --- Inicio del documento ---
    

    
    \hypertarget{importaciuxf3n-de-libreruxedas}{%
\section{Importación de
librerías}\label{importaciuxf3n-de-libreruxedas}}

    \begin{tcolorbox}[breakable, size=fbox, boxrule=1pt, pad at break*=1mm,colback=cellbackground, colframe=cellborder]
\prompt{In}{incolor}{1}{\boxspacing}
\begin{Verbatim}[commandchars=\\\{\}]
\PY{c+c1}{\PYZsh{} Importación de librerías}
\PY{k+kn}{import}\PY{+w}{ }\PY{n+nn}{numpy}\PY{+w}{ }\PY{k}{as}\PY{+w}{ }\PY{n+nn}{np}
\PY{k+kn}{import}\PY{+w}{ }\PY{n+nn}{matplotlib}\PY{n+nn}{.}\PY{n+nn}{pyplot}\PY{+w}{ }\PY{k}{as}\PY{+w}{ }\PY{n+nn}{plt}
\PY{k+kn}{from}\PY{+w}{ }\PY{n+nn}{style}\PY{+w}{ }\PY{k+kn}{import} \PY{n}{mpl\PYZus{}apply}
\PY{k+kn}{import}\PY{+w}{ }\PY{n+nn}{time}
\PY{k+kn}{import}\PY{+w}{ }\PY{n+nn}{math}

\PY{c+c1}{\PYZsh{} Configuración del estilo}
\PY{n}{mpl\PYZus{}apply}\PY{p}{(}\PY{p}{)}

\PY{c+c1}{\PYZsh{} Generador de números aleatorios}
\PY{n}{rng} \PY{o}{=} \PY{n}{np}\PY{o}{.}\PY{n}{random}\PY{o}{.}\PY{n}{default\PYZus{}rng}\PY{p}{(}\PY{p}{)}
\end{Verbatim}
\end{tcolorbox}

    \hypertarget{ejercicio-1}{%
\section{Ejercicio 1}\label{ejercicio-1}}

    Queremos simular una variable aleatoria X con distribución discreta:

\[
P(X = k) = p_k,\quad k\in\{1,2,3,4\}.
\]

donde \[
p_1=\tfrac{1}{2},\quad p_2=\tfrac{1}{4},\quad p_3=\tfrac{1}{8},\quad p_4=\tfrac{1}{8}.
\]

    \hypertarget{a-distribuciuxf3n-de-referencia-qk}{%
\subsection{a) Distribución de referencia
q(k)}\label{a-distribuciuxf3n-de-referencia-qk}}

    Proponga una distribución de referencia \(q(k)\) sobre
\(\{1, 2, 3, 4\}\) que sea fácil de simular.

    Se elige \(q(k)=\tfrac14\), uniforme en \(\{1,2,3,4\}\)

    \hypertarget{b-cota-con-constante-c}{%
\subsection{b) Cota con constante c}\label{b-cota-con-constante-c}}

    Determine la constante \(c\) tal que

\[
p_k \le c\, q(k),\ \forall k.
\]

    Se necesita \(p_k\le c\,q(k)\ \forall k\), es decir
\(c\ge \max_k \frac{p_k}{q(k)}\).

\[
\frac{p_1}{q(1)}=\frac{1/2}{1/4}=2,\quad
\frac{p_2}{q(2)}=\frac{1/4}{1/4}=1,\quad
\frac{p_3}{q(3)}=\frac{1/8}{1/4}=0.5,\quad
\frac{p_4}{q(4)}=\frac{1/8}{1/4}=0.5.
\]

\[
\Rightarrow\ \boxed{c=\max\{2,1,0.5,0.5\}=2.}
\]

\textbf{Regla de aceptación}

Se genera \(Y\sim q\) y \(U\sim \mathrm{Unif}(0,1)\). Se acepta \(X=Y\)
si

\[
U\le \frac{p_Y}{c\,q(Y)}=\frac{p_Y}{2\cdot(1/4)}=2\,p_Y.
\]

\begin{itemize}
\tightlist
\item
  Si \(Y=1\): acepta si \(U\le 1\) (siempre).
\item
  Si \(Y=2\): acepta si \(U\le 1/2\).
\item
  Si \(Y=3\) o \(4\): acepta si \(U\le 1/4\).
\end{itemize}

\textbf{Eficiencia}

La probabilidad de aceptar en una propuesta es

\[
\sum_{k} q(k)\,\min\!\Big(1,\frac{p_k}{c\,q(k)}\Big)
=\sum_k q(k)\,\frac{p_k}{c\,q(k)}=\frac{1}{c}\sum_k p_k=\frac{1}{c}.
\]

Aquí \(1/c=1/2\). El número esperado de intentos por muestra aceptada es
\(c=2\).

    \hypertarget{c-algoritmo-de-aceptaciuxf3nrechazo}{%
\subsection{c) Algoritmo de
aceptación--rechazo}\label{c-algoritmo-de-aceptaciuxf3nrechazo}}

    Describa el algoritmo de aceptación--rechazo para generar una
realización de \(X\).

    \begin{verbatim}
1. `repeat:`

2.  Genera $Y\in\{1,2,3,4\}$ con prob. $1/4$ c/u.

3.  Genera $U\sim U(0,1)$.

4.  Si $U\le p_Y/(2\cdot 1/4)=2p_Y$, **acepta** y devuelve $X=Y$; si no, regresa a 1.
\end{verbatim}

    \hypertarget{d-programa-de-simulaciuxf3n-y-estimaciuxf3n-de-c}{%
\subsection{d) Programa de simulación y estimación de
c}\label{d-programa-de-simulaciuxf3n-y-estimaciuxf3n-de-c}}

    Elabore un programa de cómputo que simule la distribución anterior y
compare el valor teórico de \(c\) con un valor aproximado obtenido de
las simulaciones.

    \begin{tcolorbox}[breakable, size=fbox, boxrule=1pt, pad at break*=1mm,colback=cellbackground, colframe=cellborder]
\prompt{In}{incolor}{2}{\boxspacing}
\begin{Verbatim}[commandchars=\\\{\}]
\PY{n}{p} \PY{o}{=} \PY{p}{\PYZob{}}\PY{l+m+mi}{1}\PY{p}{:} \PY{l+m+mi}{1}\PY{o}{/}\PY{l+m+mi}{2}\PY{p}{,} \PY{l+m+mi}{2}\PY{p}{:} \PY{l+m+mi}{1}\PY{o}{/}\PY{l+m+mi}{4}\PY{p}{,} \PY{l+m+mi}{3}\PY{p}{:} \PY{l+m+mi}{1}\PY{o}{/}\PY{l+m+mi}{8}\PY{p}{,} \PY{l+m+mi}{4}\PY{p}{:} \PY{l+m+mi}{1}\PY{o}{/}\PY{l+m+mi}{8}\PY{p}{\PYZcb{}}
\PY{n}{q} \PY{o}{=} \PY{p}{\PYZob{}}\PY{l+m+mi}{1}\PY{p}{:} \PY{l+m+mi}{1}\PY{o}{/}\PY{l+m+mi}{4}\PY{p}{,} \PY{l+m+mi}{2}\PY{p}{:} \PY{l+m+mi}{1}\PY{o}{/}\PY{l+m+mi}{4}\PY{p}{,} \PY{l+m+mi}{3}\PY{p}{:} \PY{l+m+mi}{1}\PY{o}{/}\PY{l+m+mi}{4}\PY{p}{,} \PY{l+m+mi}{4}\PY{p}{:} \PY{l+m+mi}{1}\PY{o}{/}\PY{l+m+mi}{4}\PY{p}{\PYZcb{}}
\PY{n}{c} \PY{o}{=} \PY{l+m+mf}{2.0}

\PY{k}{def}\PY{+w}{ }\PY{n+nf}{ar\PYZus{}discreto}\PY{p}{(}\PY{n}{n}\PY{o}{=}\PY{l+m+mi}{100000}\PY{p}{)}\PY{p}{:}
    \PY{n}{X}\PY{p}{,} \PY{n}{trials}\PY{p}{,} \PY{n}{acc} \PY{o}{=} \PY{p}{[}\PY{p}{]}\PY{p}{,} \PY{l+m+mi}{0}\PY{p}{,} \PY{l+m+mi}{0}
    \PY{k}{while} \PY{n+nb}{len}\PY{p}{(}\PY{n}{X}\PY{p}{)} \PY{o}{\PYZlt{}} \PY{n}{n}\PY{p}{:}
        \PY{n}{y} \PY{o}{=} \PY{n}{rng}\PY{o}{.}\PY{n}{integers}\PY{p}{(}\PY{l+m+mi}{1}\PY{p}{,}\PY{l+m+mi}{5}\PY{p}{)}         \PY{c+c1}{\PYZsh{} propuesta uniforme}
        \PY{n}{u} \PY{o}{=} \PY{n}{rng}\PY{o}{.}\PY{n}{random}\PY{p}{(}\PY{p}{)}
        \PY{n}{trials} \PY{o}{+}\PY{o}{=} \PY{l+m+mi}{1}
        \PY{k}{if} \PY{n}{u} \PY{o}{\PYZlt{}}\PY{o}{=} \PY{n}{p}\PY{p}{[}\PY{n}{y}\PY{p}{]} \PY{o}{/} \PY{p}{(}\PY{n}{c}\PY{o}{*}\PY{n}{q}\PY{p}{[}\PY{n}{y}\PY{p}{]}\PY{p}{)}\PY{p}{:}      \PY{c+c1}{\PYZsh{} umbral = 2*p[y]}
            \PY{n}{X}\PY{o}{.}\PY{n}{append}\PY{p}{(}\PY{n}{y}\PY{p}{)}\PY{p}{;} \PY{n}{acc} \PY{o}{+}\PY{o}{=} \PY{l+m+mi}{1}
    \PY{k}{return} \PY{n}{np}\PY{o}{.}\PY{n}{array}\PY{p}{(}\PY{n}{X}\PY{p}{)}\PY{p}{,} \PY{n}{acc}\PY{o}{/}\PY{n}{trials}\PY{p}{,} \PY{n}{trials}\PY{o}{/}\PY{n}{n}

\PY{n}{x}\PY{p}{,} \PY{n}{acc}\PY{p}{,} \PY{n}{intents} \PY{o}{=} \PY{n}{ar\PYZus{}discreto}\PY{p}{(}\PY{l+m+mi}{200000}\PY{p}{)}
\PY{n+nb}{print}\PY{p}{(}\PY{l+s+sa}{f}\PY{l+s+s2}{\PYZdq{}}\PY{l+s+s2}{Aceptación empírica ≈ }\PY{l+s+si}{\PYZob{}}\PY{n}{acc}\PY{l+s+si}{:}\PY{l+s+s2}{.3f}\PY{l+s+si}{\PYZcb{}}\PY{l+s+s2}{  (teórica 0.500)}\PY{l+s+s2}{\PYZdq{}}\PY{p}{)}
\PY{n+nb}{print}\PY{p}{(}\PY{l+s+sa}{f}\PY{l+s+s2}{\PYZdq{}}\PY{l+s+s2}{Intentos por muestra ≈ }\PY{l+s+si}{\PYZob{}}\PY{n}{intents}\PY{l+s+si}{:}\PY{l+s+s2}{.3f}\PY{l+s+si}{\PYZcb{}}\PY{l+s+s2}{  (teórica 2.000)}\PY{l+s+s2}{\PYZdq{}}\PY{p}{)}

\PY{n}{vals}\PY{p}{,} \PY{n}{counts} \PY{o}{=} \PY{n}{np}\PY{o}{.}\PY{n}{unique}\PY{p}{(}\PY{n}{x}\PY{p}{,} \PY{n}{return\PYZus{}counts}\PY{o}{=}\PY{k+kc}{True}\PY{p}{)}
\PY{n}{freq} \PY{o}{=} \PY{p}{\PYZob{}}\PY{n+nb}{int}\PY{p}{(}\PY{n}{k}\PY{p}{)}\PY{p}{:} \PY{n}{v}\PY{o}{/}\PY{n+nb}{len}\PY{p}{(}\PY{n}{x}\PY{p}{)} \PY{k}{for} \PY{n}{k}\PY{p}{,} \PY{n}{v} \PY{o+ow}{in} \PY{n+nb}{zip}\PY{p}{(}\PY{n}{vals}\PY{p}{,} \PY{n}{counts}\PY{p}{)}\PY{p}{\PYZcb{}}
\PY{n+nb}{print}\PY{p}{(}\PY{l+s+s2}{\PYZdq{}}\PY{l+s+s2}{Frecuencias simuladas:}\PY{l+s+s2}{\PYZdq{}}\PY{p}{,} \PY{n}{freq}\PY{p}{)}
\PY{n+nb}{print}\PY{p}{(}\PY{l+s+s2}{\PYZdq{}}\PY{l+s+s2}{Probabilidades teóricas:}\PY{l+s+s2}{\PYZdq{}}\PY{p}{,} \PY{n}{p}\PY{p}{)}

\PY{n}{bar\PYZus{}width} \PY{o}{=} \PY{l+m+mf}{0.4}
\PY{n}{keys} \PY{o}{=} \PY{n}{np}\PY{o}{.}\PY{n}{array}\PY{p}{(}\PY{n+nb}{list}\PY{p}{(}\PY{n}{freq}\PY{o}{.}\PY{n}{keys}\PY{p}{(}\PY{p}{)}\PY{p}{)}\PY{p}{)}
\PY{n}{simulated\PYZus{}values} \PY{o}{=} \PY{n}{np}\PY{o}{.}\PY{n}{array}\PY{p}{(}\PY{n+nb}{list}\PY{p}{(}\PY{n}{freq}\PY{o}{.}\PY{n}{values}\PY{p}{(}\PY{p}{)}\PY{p}{)}\PY{p}{)}
\PY{n}{theoretical\PYZus{}values} \PY{o}{=} \PY{n}{np}\PY{o}{.}\PY{n}{array}\PY{p}{(}\PY{p}{[}\PY{n}{p}\PY{p}{[}\PY{n}{k}\PY{p}{]} \PY{k}{for} \PY{n}{k} \PY{o+ow}{in} \PY{n}{keys}\PY{p}{]}\PY{p}{)}

\PY{n}{plt}\PY{o}{.}\PY{n}{bar}\PY{p}{(}\PY{n}{keys} \PY{o}{\PYZhy{}} \PY{n}{bar\PYZus{}width}\PY{o}{/}\PY{l+m+mi}{2}\PY{p}{,} \PY{n}{simulated\PYZus{}values}\PY{p}{,} \PY{n}{width}\PY{o}{=}\PY{n}{bar\PYZus{}width}\PY{p}{,} \PY{n}{alpha}\PY{o}{=}\PY{l+m+mf}{0.7}\PY{p}{,} \PY{n}{label}\PY{o}{=}\PY{l+s+s2}{\PYZdq{}}\PY{l+s+s2}{Simuladas}\PY{l+s+s2}{\PYZdq{}}\PY{p}{)}
\PY{n}{plt}\PY{o}{.}\PY{n}{bar}\PY{p}{(}\PY{n}{keys} \PY{o}{+} \PY{n}{bar\PYZus{}width}\PY{o}{/}\PY{l+m+mi}{2}\PY{p}{,} \PY{n}{theoretical\PYZus{}values}\PY{p}{,} \PY{n}{width}\PY{o}{=}\PY{n}{bar\PYZus{}width}\PY{p}{,} \PY{n}{alpha}\PY{o}{=}\PY{l+m+mf}{0.7}\PY{p}{,} \PY{n}{label}\PY{o}{=}\PY{l+s+s2}{\PYZdq{}}\PY{l+s+s2}{Teóricas}\PY{l+s+s2}{\PYZdq{}}\PY{p}{)}
\PY{n}{plt}\PY{o}{.}\PY{n}{xticks}\PY{p}{(}\PY{n}{keys}\PY{p}{)}
\PY{n}{plt}\PY{o}{.}\PY{n}{legend}\PY{p}{(}\PY{p}{)}
\PY{n}{plt}\PY{o}{.}\PY{n}{show}\PY{p}{(}\PY{p}{)}
\end{Verbatim}
\end{tcolorbox}

    \begin{Verbatim}[commandchars=\\\{\}]
Aceptación empírica ≈ 0.499  (teórica 0.500)
Intentos por muestra ≈ 2.002  (teórica 2.000)
Frecuencias simuladas: \{1: np.float64(0.49747), 2: np.float64(0.2517), 3:
np.float64(0.124885), 4: np.float64(0.125945)\}
Probabilidades teóricas: \{1: 0.5, 2: 0.25, 3: 0.125, 4: 0.125\}
    \end{Verbatim}

    \begin{center}
    \adjustimage{max size={0.9\linewidth}{0.9\paperheight}}{notebook_files/notebook_15_1.png}
    \end{center}
    { \hspace*{\fill} \\}
    
    \hypertarget{ejercicio-2}{%
\section{Ejercicio 2}\label{ejercicio-2}}

    \hypertarget{simulaciuxf3n-textbeta1-2.3-por-aceptaciuxf3nrechazo}{%
\subsection{\texorpdfstring{Simulación \(\text{Beta}(1, 2.3)\) por
aceptación--rechazo}{Simulación \textbackslash text\{Beta\}(1, 2.3) por aceptación--rechazo}}\label{simulaciuxf3n-textbeta1-2.3-por-aceptaciuxf3nrechazo}}

    Utilice el método de aceptación y rechazo para simular una variable
aleatoria con distribución \(\text{Beta}(1, 2.3)\). Elabore un programa
de cómputo que genere simulaciones de esta variable y compare resultados
con la densidad teórica.

    \begin{tcolorbox}[breakable, size=fbox, boxrule=1pt, pad at break*=1mm,colback=cellbackground, colframe=cellborder]
\prompt{In}{incolor}{3}{\boxspacing}
\begin{Verbatim}[commandchars=\\\{\}]
\PY{k}{def}\PY{+w}{ }\PY{n+nf}{rbeta\PYZus{}alpha1\PYZus{}beta}\PY{p}{(}\PY{n}{beta}\PY{o}{=}\PY{l+m+mf}{2.0}\PY{p}{,} \PY{n}{s}\PY{o}{=}\PY{l+m+mf}{3.0}\PY{p}{,} \PY{n}{n}\PY{o}{=}\PY{l+m+mi}{100000}\PY{p}{)}\PY{p}{:}
    \PY{n}{X}\PY{p}{,} \PY{n}{trials} \PY{o}{=} \PY{p}{[}\PY{p}{]}\PY{p}{,} \PY{l+m+mi}{0}
    \PY{k}{while} \PY{n+nb}{len}\PY{p}{(}\PY{n}{X}\PY{p}{)} \PY{o}{\PYZlt{}} \PY{n}{n}\PY{p}{:}
        \PY{n}{y} \PY{o}{=} \PY{n}{rng}\PY{o}{.}\PY{n}{random}\PY{p}{(}\PY{p}{)} \PY{o}{*} \PY{n}{s}           \PY{c+c1}{\PYZsh{} Unif(0,s)}
        \PY{n}{u} \PY{o}{=} \PY{n}{rng}\PY{o}{.}\PY{n}{random}\PY{p}{(}\PY{p}{)}
        \PY{k}{if} \PY{n}{u} \PY{o}{\PYZlt{}}\PY{o}{=} \PY{p}{(}\PY{l+m+mi}{1} \PY{o}{\PYZhy{}} \PY{n}{y}\PY{o}{/}\PY{n}{s}\PY{p}{)}\PY{o}{*}\PY{o}{*}\PY{p}{(}\PY{n}{beta} \PY{o}{\PYZhy{}} \PY{l+m+mi}{1}\PY{p}{)}\PY{p}{:}
            \PY{n}{X}\PY{o}{.}\PY{n}{append}\PY{p}{(}\PY{n}{y}\PY{p}{)}
        \PY{n}{trials} \PY{o}{+}\PY{o}{=} \PY{l+m+mi}{1}
    \PY{n}{X} \PY{o}{=} \PY{n}{np}\PY{o}{.}\PY{n}{array}\PY{p}{(}\PY{n}{X}\PY{p}{)}
    \PY{n}{acc} \PY{o}{=} \PY{n+nb}{len}\PY{p}{(}\PY{n}{X}\PY{p}{)}\PY{o}{/}\PY{n}{trials}
    \PY{k}{return} \PY{n}{X}\PY{p}{,} \PY{n}{acc}\PY{p}{,} \PY{n}{trials}

\PY{c+c1}{\PYZsh{} Beta(1, 2.3) en (0,1): beta=2.3, s=1}
\PY{n}{x1}\PY{p}{,} \PY{n}{acc1}\PY{p}{,} \PY{n}{tr1} \PY{o}{=} \PY{n}{rbeta\PYZus{}alpha1\PYZus{}beta}\PY{p}{(}\PY{n}{beta}\PY{o}{=}\PY{l+m+mf}{2.3}\PY{p}{,} \PY{n}{s}\PY{o}{=}\PY{l+m+mf}{1.0}\PY{p}{,} \PY{n}{n}\PY{o}{=}\PY{l+m+mi}{100000}\PY{p}{)}
\PY{n+nb}{print}\PY{p}{(}\PY{l+s+sa}{f}\PY{l+s+s2}{\PYZdq{}}\PY{l+s+s2}{Beta(1,2.3) en (0,1): aceptación empírica ≈ }\PY{l+s+si}{\PYZob{}}\PY{n}{acc1}\PY{l+s+si}{:}\PY{l+s+s2}{.3f}\PY{l+s+si}{\PYZcb{}}\PY{l+s+s2}{, teórica = }\PY{l+s+si}{\PYZob{}}\PY{l+m+mi}{1}\PY{o}{/}\PY{l+m+mf}{2.3}\PY{l+s+si}{:}\PY{l+s+s2}{.3f}\PY{l+s+si}{\PYZcb{}}\PY{l+s+s2}{, intentos/muestra ≈ }\PY{l+s+si}{\PYZob{}}\PY{n}{tr1}\PY{o}{/}\PY{n+nb}{len}\PY{p}{(}\PY{n}{x1}\PY{p}{)}\PY{l+s+si}{:}\PY{l+s+s2}{.2f}\PY{l+s+si}{\PYZcb{}}\PY{l+s+s2}{\PYZdq{}}\PY{p}{)}

\PY{n}{x} \PY{o}{=} \PY{n}{np}\PY{o}{.}\PY{n}{linspace}\PY{p}{(}\PY{l+m+mi}{0}\PY{p}{,} \PY{l+m+mi}{3}\PY{p}{,} \PY{l+m+mi}{200}\PY{p}{)}
\PY{n}{plt}\PY{o}{.}\PY{n}{plot}\PY{p}{(}\PY{n}{x}\PY{p}{[}\PY{n}{x}\PY{o}{\PYZlt{}}\PY{o}{=}\PY{l+m+mi}{1}\PY{p}{]}\PY{p}{,} \PY{l+m+mf}{2.3}\PY{o}{*}\PY{p}{(}\PY{l+m+mi}{1} \PY{o}{\PYZhy{}} \PY{n}{x}\PY{p}{[}\PY{n}{x}\PY{o}{\PYZlt{}}\PY{o}{=}\PY{l+m+mi}{1}\PY{p}{]}\PY{p}{)}\PY{o}{*}\PY{o}{*}\PY{l+m+mf}{1.3}\PY{p}{,} \PY{n}{label}\PY{o}{=}\PY{l+s+s2}{\PYZdq{}}\PY{l+s+s2}{Beta(1,2.3) en (0,1)}\PY{l+s+s2}{\PYZdq{}}\PY{p}{,} \PY{n}{color}\PY{o}{=}\PY{l+s+s1}{\PYZsq{}}\PY{l+s+s1}{C1}\PY{l+s+s1}{\PYZsq{}}\PY{p}{)}
\PY{n}{plt}\PY{o}{.}\PY{n}{hist}\PY{p}{(}\PY{n}{x1}\PY{p}{,} \PY{n}{bins}\PY{o}{=}\PY{l+m+mi}{100}\PY{p}{,} \PY{n}{density}\PY{o}{=}\PY{k+kc}{True}\PY{p}{,} \PY{n}{alpha}\PY{o}{=}\PY{l+m+mf}{0.3}\PY{p}{,} \PY{n}{color}\PY{o}{=}\PY{l+s+s1}{\PYZsq{}}\PY{l+s+s1}{C1}\PY{l+s+s1}{\PYZsq{}}\PY{p}{)}
\PY{n}{plt}\PY{o}{.}\PY{n}{legend}\PY{p}{(}\PY{p}{)}
\PY{n}{plt}\PY{o}{.}\PY{n}{show}\PY{p}{(}\PY{p}{)}
\end{Verbatim}
\end{tcolorbox}

    \begin{Verbatim}[commandchars=\\\{\}]
Beta(1,2.3) en (0,1): aceptación empírica ≈ 0.433, teórica = 0.435,
intentos/muestra ≈ 2.31
    \end{Verbatim}

    \begin{center}
    \adjustimage{max size={0.9\linewidth}{0.9\paperheight}}{notebook_files/notebook_19_1.png}
    \end{center}
    { \hspace*{\fill} \\}
    
    \hypertarget{ejercicio-3}{%
\section{Ejercicio 3}\label{ejercicio-3}}

    Considere la siguiente función de distribución acumulada

\[
F(x) = x^{n},\quad 0 \le x \le 1.
\]

    \hypertarget{a-muxe9todo-de-la-transformada-inversa}{%
\subsection{a) Método de la transformada
inversa}\label{a-muxe9todo-de-la-transformada-inversa}}

    Aplique el método de la transformada inversa para dar un algoritmo que
simule una variable aleatoria con la función de distribución anterior.

    Sea \(U\sim\mathrm{Unif}(0,1)\). Definimos

\[
X = U^{1/n}.
\]

Entonces

\[
P(X\le x)=P(U^{1/n}\le x)=P(U\le x^n)=x^n=F(x),
\]

luego \(X\) tiene CDF \(F(x)=x^n\).

\textbf{Algoritmo:} genera \(U\) y regresa \(U^{1/n}\). Aceptación:
100\%.

    \hypertarget{b-muxe9todo-de-aceptaciuxf3nrechazo}{%
\subsection{b) Método de
aceptación--rechazo}\label{b-muxe9todo-de-aceptaciuxf3nrechazo}}

    Aplique el método de aceptación y rechazo para el mismo caso.

    La densidad objetivo es \(f(x)=F'(x)=n\,x^{n-1}\) en \((0,1)\).

Toma propuesta \(g(x)=1\) (uniforme en \((0,1)\)). Entonces

\[
\frac{f(x)}{g(x)}=n\,x^{n-1}\le \max_{x\in(0,1)} n\,x^{n-1}=n \quad\Rightarrow\quad \boxed{c=n}.
\]

\textbf{Criterio:} genera \(Y\sim\mathrm{Unif}(0,1)\),
\(U\sim\mathrm{Unif}(0,1)\) y acepta si

\[
U \le \frac{f(Y)}{c\,g(Y)}=\frac{nY^{n-1}}{n}=Y^{\,n-1}.
\]

\textbf{Eficiencia:} prob. de aceptación \(=1/c=1/n\); intentos
esperados \(=c=n\).

    \hypertarget{c-implementaciuxf3n-computacional}{%
\subsection{c) Implementación
computacional}\label{c-implementaciuxf3n-computacional}}

    Elabore un programa de cómputo para implementar ambos algoritmos.

    \begin{tcolorbox}[breakable, size=fbox, boxrule=1pt, pad at break*=1mm,colback=cellbackground, colframe=cellborder]
\prompt{In}{incolor}{4}{\boxspacing}
\begin{Verbatim}[commandchars=\\\{\}]
\PY{k}{def}\PY{+w}{ }\PY{n+nf}{sample\PYZus{}inverse}\PY{p}{(}\PY{n}{n}\PY{p}{,} \PY{n}{size}\PY{p}{)}\PY{p}{:}
    \PY{n}{u} \PY{o}{=} \PY{n}{rng}\PY{o}{.}\PY{n}{random}\PY{p}{(}\PY{n}{size}\PY{p}{)}
    \PY{k}{return} \PY{n}{u}\PY{o}{*}\PY{o}{*}\PY{p}{(}\PY{l+m+mf}{1.0}\PY{o}{/}\PY{n}{n}\PY{p}{)}

\PY{k}{def}\PY{+w}{ }\PY{n+nf}{sample\PYZus{}ar}\PY{p}{(}\PY{n}{n}\PY{p}{,} \PY{n}{size}\PY{p}{)}\PY{p}{:}
    \PY{n}{X}\PY{p}{,} \PY{n}{trials} \PY{o}{=} \PY{p}{[}\PY{p}{]}\PY{p}{,} \PY{l+m+mi}{0}
    \PY{k}{while} \PY{n+nb}{len}\PY{p}{(}\PY{n}{X}\PY{p}{)} \PY{o}{\PYZlt{}} \PY{n}{size}\PY{p}{:}
        \PY{n}{y} \PY{o}{=} \PY{n}{rng}\PY{o}{.}\PY{n}{random}\PY{p}{(}\PY{p}{)}
        \PY{n}{u} \PY{o}{=} \PY{n}{rng}\PY{o}{.}\PY{n}{random}\PY{p}{(}\PY{p}{)}
        \PY{n}{trials} \PY{o}{+}\PY{o}{=} \PY{l+m+mi}{1}
        \PY{k}{if} \PY{n}{u} \PY{o}{\PYZlt{}}\PY{o}{=} \PY{n}{y}\PY{o}{*}\PY{o}{*}\PY{p}{(}\PY{n}{n}\PY{o}{\PYZhy{}}\PY{l+m+mi}{1}\PY{p}{)}\PY{p}{:}
            \PY{n}{X}\PY{o}{.}\PY{n}{append}\PY{p}{(}\PY{n}{y}\PY{p}{)}
    \PY{k}{return} \PY{n}{np}\PY{o}{.}\PY{n}{array}\PY{p}{(}\PY{n}{X}\PY{p}{)}\PY{p}{,} \PY{n+nb}{len}\PY{p}{(}\PY{n}{X}\PY{p}{)}\PY{o}{/}\PY{n}{trials}\PY{p}{,} \PY{n}{trials}

\PY{c+c1}{\PYZsh{} Demo}
\PY{n}{n}\PY{p}{,} \PY{n}{N} \PY{o}{=} \PY{l+m+mi}{7}\PY{p}{,} \PY{l+m+mi}{100\PYZus{}000}

\PY{n}{t0} \PY{o}{=} \PY{n}{time}\PY{o}{.}\PY{n}{perf\PYZus{}counter}\PY{p}{(}\PY{p}{)}
\PY{n}{x\PYZus{}inv} \PY{o}{=} \PY{n}{sample\PYZus{}inverse}\PY{p}{(}\PY{n}{n}\PY{p}{,} \PY{n}{N}\PY{p}{)}
\PY{n}{t\PYZus{}inv} \PY{o}{=} \PY{n}{time}\PY{o}{.}\PY{n}{perf\PYZus{}counter}\PY{p}{(}\PY{p}{)} \PY{o}{\PYZhy{}} \PY{n}{t0}

\PY{n}{t0} \PY{o}{=} \PY{n}{time}\PY{o}{.}\PY{n}{perf\PYZus{}counter}\PY{p}{(}\PY{p}{)}
\PY{n}{x\PYZus{}ar}\PY{p}{,} \PY{n}{acc\PYZus{}ar}\PY{p}{,} \PY{n}{props} \PY{o}{=} \PY{n}{sample\PYZus{}ar}\PY{p}{(}\PY{n}{n}\PY{p}{,} \PY{n}{N}\PY{p}{)}
\PY{n}{t\PYZus{}ar} \PY{o}{=} \PY{n}{time}\PY{o}{.}\PY{n}{perf\PYZus{}counter}\PY{p}{(}\PY{p}{)} \PY{o}{\PYZhy{}} \PY{n}{t0}

\PY{n+nb}{print}\PY{p}{(}\PY{l+s+sa}{f}\PY{l+s+s2}{\PYZdq{}}\PY{l+s+s2}{[Inversa]  tiempo=}\PY{l+s+si}{\PYZob{}}\PY{n}{t\PYZus{}inv}\PY{l+s+si}{:}\PY{l+s+s2}{.3f}\PY{l+s+si}{\PYZcb{}}\PY{l+s+s2}{s,   rechazos=0,            uniformes≈}\PY{l+s+si}{\PYZob{}}\PY{n}{N}\PY{l+s+si}{\PYZcb{}}\PY{l+s+s2}{\PYZdq{}}\PY{p}{)}
\PY{n+nb}{print}\PY{p}{(}\PY{l+s+sa}{f}\PY{l+s+s2}{\PYZdq{}}\PY{l+s+s2}{[A\PYZhy{}R]      tiempo=}\PY{l+s+si}{\PYZob{}}\PY{n}{t\PYZus{}ar}\PY{l+s+si}{:}\PY{l+s+s2}{.3f}\PY{l+s+si}{\PYZcb{}}\PY{l+s+s2}{s,   aceptación≈}\PY{l+s+si}{\PYZob{}}\PY{n}{acc\PYZus{}ar}\PY{l+s+si}{:}\PY{l+s+s2}{.3f}\PY{l+s+si}{\PYZcb{}}\PY{l+s+s2}{ (teórica }\PY{l+s+si}{\PYZob{}}\PY{l+m+mi}{1}\PY{o}{/}\PY{n}{n}\PY{l+s+si}{:}\PY{l+s+s2}{.3f}\PY{l+s+si}{\PYZcb{}}\PY{l+s+s2}{), }\PY{l+s+s2}{\PYZdq{}}
      \PY{l+s+sa}{f}\PY{l+s+s2}{\PYZdq{}}\PY{l+s+s2}{propuestas=}\PY{l+s+si}{\PYZob{}}\PY{n}{props}\PY{l+s+si}{\PYZcb{}}\PY{l+s+s2}{ (\PYZti{}}\PY{l+s+si}{\PYZob{}}\PY{n}{props}\PY{o}{/}\PY{n}{N}\PY{l+s+si}{:}\PY{l+s+s2}{.2f}\PY{l+s+si}{\PYZcb{}}\PY{l+s+s2}{ por muestra), uniformes≈}\PY{l+s+si}{\PYZob{}}\PY{l+m+mi}{2}\PY{o}{*}\PY{n}{props}\PY{l+s+si}{\PYZcb{}}\PY{l+s+s2}{\PYZdq{}}\PY{p}{)}

\PY{c+c1}{\PYZsh{} Chequeo de momentos (teóricos: E[X]=n/(n+1), Var[X]=n/[(n+2)(n+1)\PYZca{}2])}
\PY{k}{def}\PY{+w}{ }\PY{n+nf}{stats}\PY{p}{(}\PY{n}{name}\PY{p}{,} \PY{n}{x}\PY{p}{)}\PY{p}{:}
    \PY{n}{mean} \PY{o}{=} \PY{n}{x}\PY{o}{.}\PY{n}{mean}\PY{p}{(}\PY{p}{)}
    \PY{n}{var}  \PY{o}{=} \PY{n}{x}\PY{o}{.}\PY{n}{var}\PY{p}{(}\PY{p}{)}
    \PY{n}{mt}   \PY{o}{=} \PY{n}{n}\PY{o}{/}\PY{p}{(}\PY{n}{n}\PY{o}{+}\PY{l+m+mi}{1}\PY{p}{)}
    \PY{n}{vt}   \PY{o}{=} \PY{n}{n}\PY{o}{/}\PY{p}{(}\PY{p}{(}\PY{n}{n}\PY{o}{+}\PY{l+m+mi}{2}\PY{p}{)}\PY{o}{*}\PY{p}{(}\PY{n}{n}\PY{o}{+}\PY{l+m+mi}{1}\PY{p}{)}\PY{o}{*}\PY{o}{*}\PY{l+m+mi}{2}\PY{p}{)}
    \PY{n+nb}{print}\PY{p}{(}\PY{l+s+sa}{f}\PY{l+s+s2}{\PYZdq{}}\PY{l+s+si}{\PYZob{}}\PY{n}{name}\PY{l+s+si}{\PYZcb{}}\PY{l+s+s2}{: mean≈}\PY{l+s+si}{\PYZob{}}\PY{n}{mean}\PY{l+s+si}{:}\PY{l+s+s2}{.4f}\PY{l+s+si}{\PYZcb{}}\PY{l+s+s2}{ vs }\PY{l+s+si}{\PYZob{}}\PY{n}{mt}\PY{l+s+si}{:}\PY{l+s+s2}{.4f}\PY{l+s+si}{\PYZcb{}}\PY{l+s+s2}{, var≈}\PY{l+s+si}{\PYZob{}}\PY{n}{var}\PY{l+s+si}{:}\PY{l+s+s2}{.4f}\PY{l+s+si}{\PYZcb{}}\PY{l+s+s2}{ vs }\PY{l+s+si}{\PYZob{}}\PY{n}{vt}\PY{l+s+si}{:}\PY{l+s+s2}{.4f}\PY{l+s+si}{\PYZcb{}}\PY{l+s+s2}{\PYZdq{}}\PY{p}{)}

\PY{n}{stats}\PY{p}{(}\PY{l+s+s2}{\PYZdq{}}\PY{l+s+s2}{Inversa}\PY{l+s+s2}{\PYZdq{}}\PY{p}{,} \PY{n}{x\PYZus{}inv}\PY{p}{)}
\PY{n}{stats}\PY{p}{(}\PY{l+s+s2}{\PYZdq{}}\PY{l+s+s2}{A\PYZhy{}R    }\PY{l+s+s2}{\PYZdq{}}\PY{p}{,} \PY{n}{x\PYZus{}ar}\PY{p}{)}

\PY{c+c1}{\PYZsh{} Plot histograms for both methods}
\PY{n}{bins} \PY{o}{=} \PY{n}{np}\PY{o}{.}\PY{n}{linspace}\PY{p}{(}\PY{l+m+mi}{0}\PY{p}{,} \PY{l+m+mi}{1}\PY{p}{,} \PY{l+m+mi}{50}\PY{p}{)}
\PY{n}{plt}\PY{o}{.}\PY{n}{hist}\PY{p}{(}\PY{n}{x\PYZus{}inv}\PY{p}{,} \PY{n}{bins}\PY{o}{=}\PY{n}{bins}\PY{p}{,} \PY{n}{density}\PY{o}{=}\PY{k+kc}{True}\PY{p}{,} \PY{n}{alpha}\PY{o}{=}\PY{l+m+mf}{0.2}\PY{p}{,} \PY{n}{label}\PY{o}{=}\PY{l+s+sa}{f}\PY{l+s+s2}{\PYZdq{}}\PY{l+s+s2}{Inversa (t=}\PY{l+s+si}{\PYZob{}}\PY{n}{t\PYZus{}inv}\PY{l+s+si}{:}\PY{l+s+s2}{.3f}\PY{l+s+si}{\PYZcb{}}\PY{l+s+s2}{s)}\PY{l+s+s2}{\PYZdq{}}\PY{p}{)}
\PY{n}{plt}\PY{o}{.}\PY{n}{hist}\PY{p}{(}\PY{n}{x\PYZus{}ar}\PY{p}{,} \PY{n}{bins}\PY{o}{=}\PY{n}{bins}\PY{p}{,} \PY{n}{density}\PY{o}{=}\PY{k+kc}{True}\PY{p}{,} \PY{n}{alpha}\PY{o}{=}\PY{l+m+mf}{0.2}\PY{p}{,} \PY{n}{label}\PY{o}{=}\PY{l+s+sa}{f}\PY{l+s+s2}{\PYZdq{}}\PY{l+s+s2}{A\PYZhy{}R (t=}\PY{l+s+si}{\PYZob{}}\PY{n}{t\PYZus{}ar}\PY{l+s+si}{:}\PY{l+s+s2}{.3f}\PY{l+s+si}{\PYZcb{}}\PY{l+s+s2}{s)}\PY{l+s+s2}{\PYZdq{}}\PY{p}{)}

\PY{c+c1}{\PYZsh{} Overlay the theoretical density}
\PY{n}{x\PYZus{}vals} \PY{o}{=} \PY{n}{np}\PY{o}{.}\PY{n}{linspace}\PY{p}{(}\PY{l+m+mi}{0}\PY{p}{,} \PY{l+m+mi}{1}\PY{p}{,} \PY{l+m+mi}{200}\PY{p}{)}
\PY{n}{theoretical\PYZus{}density} \PY{o}{=} \PY{n}{n} \PY{o}{*} \PY{n}{x\PYZus{}vals}\PY{o}{*}\PY{o}{*}\PY{p}{(}\PY{n}{n} \PY{o}{\PYZhy{}} \PY{l+m+mi}{1}\PY{p}{)}
\PY{n}{plt}\PY{o}{.}\PY{n}{plot}\PY{p}{(}\PY{n}{x\PYZus{}vals}\PY{p}{,} \PY{n}{theoretical\PYZus{}density}\PY{p}{,} \PY{n}{label}\PY{o}{=}\PY{l+s+s2}{\PYZdq{}}\PY{l+s+s2}{Teórica}\PY{l+s+s2}{\PYZdq{}}\PY{p}{,} \PY{n}{color}\PY{o}{=}\PY{l+s+s2}{\PYZdq{}}\PY{l+s+s2}{black}\PY{l+s+s2}{\PYZdq{}}\PY{p}{,} \PY{n}{linewidth}\PY{o}{=}\PY{l+m+mi}{2}\PY{p}{)}

\PY{n}{plt}\PY{o}{.}\PY{n}{title}\PY{p}{(}\PY{l+s+s2}{\PYZdq{}}\PY{l+s+s2}{Comparación de métodos: Inversa vs A\PYZhy{}R}\PY{l+s+s2}{\PYZdq{}}\PY{p}{)}
\PY{n}{plt}\PY{o}{.}\PY{n}{xlabel}\PY{p}{(}\PY{l+s+s2}{\PYZdq{}}\PY{l+s+s2}{x}\PY{l+s+s2}{\PYZdq{}}\PY{p}{)}
\PY{n}{plt}\PY{o}{.}\PY{n}{ylabel}\PY{p}{(}\PY{l+s+s2}{\PYZdq{}}\PY{l+s+s2}{Densidad}\PY{l+s+s2}{\PYZdq{}}\PY{p}{)}
\PY{n}{plt}\PY{o}{.}\PY{n}{legend}\PY{p}{(}\PY{p}{)}
\PY{n}{plt}\PY{o}{.}\PY{n}{show}\PY{p}{(}\PY{p}{)}
\end{Verbatim}
\end{tcolorbox}

    \begin{Verbatim}[commandchars=\\\{\}]
[Inversa]  tiempo=0.003s,   rechazos=0,            uniformes≈100000
[A-R]      tiempo=0.950s,   aceptación≈0.142 (teórica 0.143), propuestas=703088
(\textasciitilde{}7.03 por muestra), uniformes≈1406176
Inversa: mean≈0.8749 vs 0.8750, var≈0.0121 vs 0.0122
A-R    : mean≈0.8749 vs 0.8750, var≈0.0122 vs 0.0122
    \end{Verbatim}

    \begin{center}
    \adjustimage{max size={0.9\linewidth}{0.9\paperheight}}{notebook_files/notebook_30_1.png}
    \end{center}
    { \hspace*{\fill} \\}
    
    \hypertarget{d-eficiencia-comparada}{%
\subsection{d) Eficiencia comparada}\label{d-eficiencia-comparada}}

    Compare la eficiencia de ambos métodos y justifique cuál es más
recomendable.

    \begin{itemize}
\item
  \textbf{Transformada inversa:} 1 uniforme por muestra, 0 rechazos →
  \textbf{siempre más eficiente} aquí.
\item
  \textbf{A--R:} aceptación \(1/n\) (requiere \(\approx n\) propuestas y
  \(\approx 2n\) uniformes por muestra aceptada).
\end{itemize}

    \hypertarget{ejercicio-4}{%
\section{Ejercicio 4}\label{ejercicio-4}}

    En el contexto del método de aceptación y rechazo para generar valores
de la distribución \(\mathcal{N}(\mu, \sigma^{2})\), demuestre
directamente los siguientes resultados:

    \hypertarget{a-densidad-de-z-para-zn01}{%
\subsection{a) Densidad de \textbar Z\textbar{} para
Z∼N(0,1)}\label{a-densidad-de-z-para-zn01}}

    Si \(Z \sim \mathcal{N}(0,1)\), entonces \(|Z|\) tiene función de
densidad

\[
f_{|Z|}(x) = \sqrt{\tfrac{2}{\pi}}\, e^{-x^{2}/2}, \qquad x > 0.
\]

    Sea \(Z\sim\mathcal N(0,1)\). Para \(x>0\):

\[
P(|Z|\le x)=P(-x\le Z\le x)=\Phi(x)-\Phi(-x)=2\Phi(x)-1.
\]

Derivando:

\[
f_{|Z|}(x)=\frac{d}{dx}\,[2\Phi(x)-1]=2\phi(x)
=\boxed{\sqrt{\frac{2}{\pi}}\,e^{-x^{2}/2}},\quad x>0.
\]

    \hypertarget{b-simetrizaciuxf3n-con-sz}{%
\subsection{b) Simetrización con
S·\textbar Z\textbar{}}\label{b-simetrizaciuxf3n-con-sz}}

    Si \(S \sim \text{Unif}\{+1,-1\}\) es independiente de \(|Z|\) (con
\(Z\) como en (a)), entonces \(S\,|Z| \sim \mathcal{N}(0,1)\).

    Sea \(S\sim\text{Unif}\{+1,-1\}\) independiente de \(|Z|\). Para
\(z\in\mathbb R\):

\[
f_{S|Z|}(z)=\tfrac12 f_{|Z|}(|z|)+\tfrac12 f_{|Z|}(|z|)=f_{|Z|}(|z|)=\phi(z),
\]

luego \(\boxed{S\,|Z|\sim \mathcal N(0,1)}\).

    \hypertarget{c-evento-de-aceptaciuxf3n-con-xexp1-y-uunif01}{%
\subsection{c) Evento de aceptación con X∼Exp(1) y
U∼Unif(0,1)}\label{c-evento-de-aceptaciuxf3n-con-xexp1-y-uunif01}}

    Sea \(X \sim \text{Exp}(\lambda)\) con \(\lambda = 1\) y
\(U \sim \text{Unif}(0,1)\) independientes. Considere el evento

\[
\big\{ U \le \exp\big(-\tfrac{(X-1)^{2}}{2}\big) \big\}.
\]

Entonces, la distribución de \(X\) condicionada a este evento tiene
densidad

\[
f(x) = \sqrt{\tfrac{2}{\pi}}\, e^{-x^{2}/2}, \qquad x > 0,
\]

la cual corresponde a la densidad del valor absoluto de una normal
estándar \(Z \sim \mathcal{N}(0,1)\).

    Propuesta \(g(x)=e^{-x}\) en \(x>0\) y objetivo \(f(x)=f_{|Z|}(x)\).

\[
\frac{f(x)}{g(x)}=\sqrt{\frac{2}{\pi}}\,\frac{e^{-x^2/2}}{e^{-x}}
=\sqrt{\frac{2}{\pi}}\,e^{-\frac{(x-1)^2}{2}}\,e^{1/2}.
\]

El máximo es en \(x=1\), así

\[
\boxed{c=\sqrt{\frac{2e}{\pi}}},\qquad
\frac{f(x)}{c\,g(x)}=e^{-\frac{(x-1)^2}{2}}.
\]

\textbf{Regla:} genera \(Y\sim\mathrm{Exp}(1)\), \(U\sim U(0,1)\);
acepta \(|Z|=Y\) si

\[
\boxed{U\le \exp\!\Big(-\tfrac{(Y-1)^2}{2}\Big)}.
\]

Luego toma \(Z=S|Z|\) con \(S\sim\text{Unif}\{\pm1\}\). Prob. de
aceptación \(=1/c=\boxed{\sqrt{\pi/(2e)}}\).

    \hypertarget{d-probabilidad-con-v1v2exp1}{%
\subsection{d) Probabilidad con
V1,V2∼Exp(1)}\label{d-probabilidad-con-v1v2exp1}}

    Sean \(V_1\) y \(V_2\) variables aleatorias independientes e
idénticamente distribuidas como \(\text{Exp}(\lambda)\) con
\(\lambda = 1\). Entonces se cumple que

\[
\mathbb{P}\!\left( V_{1} \ge \tfrac{(V_{2}-1)^{2}}{2} \right)
= \sqrt{\tfrac{\pi}{2e}}\,.
\]

    Si \(V_1,V_2\overset{iid}{\sim}\mathrm{Exp}(1)\),

\[
\begin{aligned}
P\!\left[V_1\ge \frac{(V_2-1)^2}{2}\right]
&=\int_0^\infty P\!\left[V_1\ge \tfrac{(y-1)^2}{2}\right] e^{-y}\,dy
=\int_0^\infty e^{-\frac{(y-1)^2}{2}}\,e^{-y}\,dy\\
&=\int_0^\infty \exp\!\Big(-\tfrac{y^2+1}{2}\Big)\,dy
=e^{-1/2}\!\int_0^\infty e^{-y^2/2}\,dy
=\boxed{\sqrt{\frac{\pi}{2e}}}.
\end{aligned}
\]

Coincide con la prob. de aceptación de (c).

    \hypertarget{ejercicio-5}{%
\section{Ejercicio 5}\label{ejercicio-5}}

    Implemente un algoritmo de simulación para la distribución
\(\text{Gamma}\) con los siguientes parámetros:

    Elabore un programa de cómputo que genere simulaciones de ambas
distribuciones y compare los resultados empíricos con las densidades
teóricas correspondientes.

    \hypertarget{a-textgamma1.5-3}{%
\subsection{\texorpdfstring{a)
\(\text{Gamma}(1.5, 3)\)}{a) \textbackslash text\{Gamma\}(1.5, 3)}}\label{a-textgamma1.5-3}}

    \[
\text{Gamma}(1.5, 3)
\]

    \begin{tcolorbox}[breakable, size=fbox, boxrule=1pt, pad at break*=1mm,colback=cellbackground, colframe=cellborder]
\prompt{In}{incolor}{5}{\boxspacing}
\begin{Verbatim}[commandchars=\\\{\}]
\PY{c+c1}{\PYZsh{} (i) A\PYZhy{}R para alpha\PYZgt{}1 con mu = lambda/alpha (óptimo)}
\PY{k}{def}\PY{+w}{ }\PY{n+nf}{rgamma\PYZus{}alpha\PYZus{}gt1}\PY{p}{(}\PY{n}{alpha}\PY{p}{,} \PY{n}{lam}\PY{p}{,} \PY{n}{n}\PY{o}{=}\PY{l+m+mi}{60000}\PY{p}{)}\PY{p}{:}
    \PY{n}{mu} \PY{o}{=} \PY{n}{lam} \PY{o}{/} \PY{n}{alpha}
    \PY{n}{K} \PY{o}{=} \PY{n}{lam}\PY{o}{*}\PY{o}{*}\PY{n}{alpha} \PY{o}{/} \PY{n}{math}\PY{o}{.}\PY{n}{gamma}\PY{p}{(}\PY{n}{alpha}\PY{p}{)}  \PY{c+c1}{\PYZsh{} constante de f}
    \PY{c+c1}{\PYZsh{} c(mu) evaluada en el máximo:}
    \PY{n}{c} \PY{o}{=} \PY{p}{(}\PY{n}{K}\PY{o}{/}\PY{n}{mu}\PY{p}{)} \PY{o}{*} \PY{p}{(}\PY{p}{(}\PY{n}{alpha}\PY{o}{\PYZhy{}}\PY{l+m+mi}{1}\PY{p}{)}\PY{o}{/}\PY{p}{(}\PY{n}{lam}\PY{o}{\PYZhy{}}\PY{n}{mu}\PY{p}{)}\PY{p}{)}\PY{o}{*}\PY{o}{*}\PY{p}{(}\PY{n}{alpha}\PY{o}{\PYZhy{}}\PY{l+m+mi}{1}\PY{p}{)} \PY{o}{*} \PY{n}{math}\PY{o}{.}\PY{n}{e}\PY{o}{*}\PY{o}{*}\PY{p}{(}\PY{l+m+mi}{1}\PY{o}{\PYZhy{}}\PY{n}{alpha}\PY{p}{)}
    \PY{n}{X}\PY{p}{,} \PY{n}{trials} \PY{o}{=} \PY{p}{[}\PY{p}{]}\PY{p}{,} \PY{l+m+mi}{0}
    \PY{k}{while} \PY{n+nb}{len}\PY{p}{(}\PY{n}{X}\PY{p}{)} \PY{o}{\PYZlt{}} \PY{n}{n}\PY{p}{:}
        \PY{n}{y} \PY{o}{=} \PY{n}{rng}\PY{o}{.}\PY{n}{exponential}\PY{p}{(}\PY{l+m+mi}{1}\PY{o}{/}\PY{n}{mu}\PY{p}{)}                   \PY{c+c1}{\PYZsh{} Exp(mu): media 1/mu}
        \PY{n}{u} \PY{o}{=} \PY{n}{rng}\PY{o}{.}\PY{n}{random}\PY{p}{(}\PY{p}{)}
        \PY{k}{if} \PY{n}{u} \PY{o}{\PYZlt{}}\PY{o}{=} \PY{p}{(}\PY{n}{K} \PY{o}{*} \PY{n}{y}\PY{o}{*}\PY{o}{*}\PY{p}{(}\PY{n}{alpha}\PY{o}{\PYZhy{}}\PY{l+m+mi}{1}\PY{p}{)} \PY{o}{*} \PY{n}{math}\PY{o}{.}\PY{n}{exp}\PY{p}{(}\PY{o}{\PYZhy{}}\PY{n}{lam}\PY{o}{*}\PY{n}{y}\PY{p}{)}\PY{p}{)} \PY{o}{/} \PY{p}{(}\PY{n}{c} \PY{o}{*} \PY{p}{(}\PY{n}{mu} \PY{o}{*} \PY{n}{math}\PY{o}{.}\PY{n}{exp}\PY{p}{(}\PY{o}{\PYZhy{}}\PY{n}{mu}\PY{o}{*}\PY{n}{y}\PY{p}{)}\PY{p}{)}\PY{p}{)}\PY{p}{:}
            \PY{n}{X}\PY{o}{.}\PY{n}{append}\PY{p}{(}\PY{n}{y}\PY{p}{)}
        \PY{n}{trials} \PY{o}{+}\PY{o}{=} \PY{l+m+mi}{1}
    \PY{k}{return} \PY{n}{np}\PY{o}{.}\PY{n}{array}\PY{p}{(}\PY{n}{X}\PY{p}{)}\PY{p}{,} \PY{n+nb}{len}\PY{p}{(}\PY{n}{X}\PY{p}{)}\PY{o}{/}\PY{n}{trials}\PY{p}{,} \PY{n}{c}

\PY{c+c1}{\PYZsh{} (ii) Reducción de orden para alpha in (0,1)}
\PY{k}{def}\PY{+w}{ }\PY{n+nf}{rgamma\PYZus{}alpha\PYZus{}lt1}\PY{p}{(}\PY{n}{alpha}\PY{p}{,} \PY{n}{lam}\PY{p}{,} \PY{n}{n}\PY{o}{=}\PY{l+m+mi}{60000}\PY{p}{)}\PY{p}{:}
    \PY{n}{Y}\PY{p}{,} \PY{n}{accY}\PY{p}{,} \PY{n}{cY} \PY{o}{=} \PY{n}{rgamma\PYZus{}alpha\PYZus{}gt1}\PY{p}{(}\PY{n}{alpha}\PY{o}{+}\PY{l+m+mi}{1}\PY{p}{,} \PY{n}{lam}\PY{p}{,} \PY{n}{n}\PY{p}{)}  \PY{c+c1}{\PYZsh{} primero Gamma(alpha+1, lam)}
    \PY{n}{U} \PY{o}{=} \PY{n}{rng}\PY{o}{.}\PY{n}{random}\PY{p}{(}\PY{n}{n}\PY{p}{)}
    \PY{n}{X} \PY{o}{=} \PY{n}{Y} \PY{o}{*} \PY{p}{(}\PY{n}{U} \PY{o}{*}\PY{o}{*} \PY{p}{(}\PY{l+m+mf}{1.0}\PY{o}{/}\PY{n}{alpha}\PY{p}{)}\PY{p}{)}
    \PY{k}{return} \PY{n}{X}

\PY{c+c1}{\PYZsh{} (a) Gamma(1.5, 3)}
\PY{n}{x\PYZus{}a}\PY{p}{,} \PY{n}{acc\PYZus{}a}\PY{p}{,} \PY{n}{c\PYZus{}a} \PY{o}{=} \PY{n}{rgamma\PYZus{}alpha\PYZus{}gt1}\PY{p}{(}\PY{l+m+mf}{1.5}\PY{p}{,} \PY{l+m+mf}{3.0}\PY{p}{,} \PY{l+m+mi}{60000}\PY{p}{)}
\PY{n+nb}{print}\PY{p}{(}\PY{l+s+sa}{f}\PY{l+s+s2}{\PYZdq{}}\PY{l+s+s2}{Gamma(1.5,3): aceptación≈}\PY{l+s+si}{\PYZob{}}\PY{n}{acc\PYZus{}a}\PY{l+s+si}{:}\PY{l+s+s2}{.3f}\PY{l+s+si}{\PYZcb{}}\PY{l+s+s2}{, 1/c≈}\PY{l+s+si}{\PYZob{}}\PY{l+m+mi}{1}\PY{o}{/}\PY{n}{c\PYZus{}a}\PY{l+s+si}{:}\PY{l+s+s2}{.3f}\PY{l+s+si}{\PYZcb{}}\PY{l+s+s2}{\PYZdq{}}\PY{p}{)}

\PY{c+c1}{\PYZsh{} Comparación con densidad teórica}
\PY{n}{xs} \PY{o}{=} \PY{n}{np}\PY{o}{.}\PY{n}{linspace}\PY{p}{(}\PY{l+m+mi}{0}\PY{p}{,} \PY{n}{np}\PY{o}{.}\PY{n}{quantile}\PY{p}{(}\PY{n}{x\PYZus{}a}\PY{p}{,} \PY{l+m+mf}{0.995}\PY{p}{)}\PY{p}{,} \PY{l+m+mi}{400}\PY{p}{)}\PY{p}{[}\PY{l+m+mi}{1}\PY{p}{:}\PY{p}{]}
\PY{n}{f} \PY{o}{=} \PY{p}{(}\PY{l+m+mf}{3.0}\PY{o}{*}\PY{o}{*}\PY{l+m+mf}{1.5} \PY{o}{/} \PY{n}{math}\PY{o}{.}\PY{n}{gamma}\PY{p}{(}\PY{l+m+mf}{1.5}\PY{p}{)}\PY{p}{)} \PY{o}{*} \PY{n}{xs}\PY{o}{*}\PY{o}{*}\PY{p}{(}\PY{l+m+mf}{1.5}\PY{o}{\PYZhy{}}\PY{l+m+mi}{1}\PY{p}{)} \PY{o}{*} \PY{n}{np}\PY{o}{.}\PY{n}{exp}\PY{p}{(}\PY{o}{\PYZhy{}}\PY{l+m+mf}{3.0}\PY{o}{*}\PY{n}{xs}\PY{p}{)}
\PY{n}{plt}\PY{o}{.}\PY{n}{figure}\PY{p}{(}\PY{p}{)}\PY{p}{;} \PY{n}{plt}\PY{o}{.}\PY{n}{hist}\PY{p}{(}\PY{n}{x\PYZus{}a}\PY{p}{,} \PY{n}{bins}\PY{o}{=}\PY{l+m+mi}{100}\PY{p}{,} \PY{n}{density}\PY{o}{=}\PY{k+kc}{True}\PY{p}{,} \PY{n}{alpha}\PY{o}{=}\PY{l+m+mf}{0.6}\PY{p}{,} \PY{n}{label}\PY{o}{=}\PY{l+s+s2}{\PYZdq{}}\PY{l+s+s2}{Sim}\PY{l+s+s2}{\PYZdq{}}\PY{p}{)}
\PY{n}{plt}\PY{o}{.}\PY{n}{plot}\PY{p}{(}\PY{n}{xs}\PY{p}{,} \PY{n}{f}\PY{p}{,} \PY{n}{label}\PY{o}{=}\PY{l+s+s2}{\PYZdq{}}\PY{l+s+s2}{Teórica}\PY{l+s+s2}{\PYZdq{}}\PY{p}{)}\PY{p}{;} \PY{n}{plt}\PY{o}{.}\PY{n}{legend}\PY{p}{(}\PY{p}{)}\PY{p}{;} \PY{n}{plt}\PY{o}{.}\PY{n}{title}\PY{p}{(}\PY{l+s+s2}{\PYZdq{}}\PY{l+s+s2}{Gamma(1.5,3)}\PY{l+s+s2}{\PYZdq{}}\PY{p}{)}
\PY{n}{plt}\PY{o}{.}\PY{n}{show}\PY{p}{(}\PY{p}{)}
\end{Verbatim}
\end{tcolorbox}

    \begin{Verbatim}[commandchars=\\\{\}]
Gamma(1.5,3): aceptación≈0.797, 1/c≈0.795
    \end{Verbatim}

    \begin{center}
    \adjustimage{max size={0.9\linewidth}{0.9\paperheight}}{notebook_files/notebook_53_1.png}
    \end{center}
    { \hspace*{\fill} \\}
    
    \hypertarget{b-textgamma0.5-6}{%
\subsection{\texorpdfstring{b)
\(\text{Gamma}(0.5, 6)\)}{b) \textbackslash text\{Gamma\}(0.5, 6)}}\label{b-textgamma0.5-6}}

    \[
\text{Gamma}(0.5, 6)
\]

    \begin{tcolorbox}[breakable, size=fbox, boxrule=1pt, pad at break*=1mm,colback=cellbackground, colframe=cellborder]
\prompt{In}{incolor}{6}{\boxspacing}
\begin{Verbatim}[commandchars=\\\{\}]
\PY{c+c1}{\PYZsh{} (b) Gamma(0.5, 6) por dos vías}
\PY{n}{x\PYZus{}b1} \PY{o}{=} \PY{n}{rgamma\PYZus{}alpha\PYZus{}lt1}\PY{p}{(}\PY{l+m+mf}{0.5}\PY{p}{,} \PY{l+m+mf}{6.0}\PY{p}{,} \PY{l+m+mi}{60000}\PY{p}{)}   \PY{c+c1}{\PYZsh{} reducción de orden}

\PY{c+c1}{\PYZsh{} vía chi\PYZhy{}cuadrado: Z\PYZca{}2 \PYZti{} Gamma(0.5, 0.5)  =\PYZgt{}  (Z\PYZca{}2)/12 \PYZti{} Gamma(0.5, 6)}
\PY{n}{Z} \PY{o}{=} \PY{n}{rng}\PY{o}{.}\PY{n}{standard\PYZus{}normal}\PY{p}{(}\PY{l+m+mi}{60000}\PY{p}{)}
\PY{n}{x\PYZus{}b2} \PY{o}{=} \PY{p}{(}\PY{n}{Z}\PY{o}{*}\PY{o}{*}\PY{l+m+mi}{2}\PY{p}{)} \PY{o}{/} \PY{l+m+mf}{12.0}

\PY{c+c1}{\PYZsh{} Densidad teórica}
\PY{n}{xs} \PY{o}{=} \PY{n}{np}\PY{o}{.}\PY{n}{linspace}\PY{p}{(}\PY{l+m+mi}{0}\PY{p}{,} \PY{n}{np}\PY{o}{.}\PY{n}{quantile}\PY{p}{(}\PY{n}{x\PYZus{}b1}\PY{p}{,} \PY{l+m+mf}{0.995}\PY{p}{)}\PY{p}{,} \PY{l+m+mi}{400}\PY{p}{)}\PY{p}{[}\PY{l+m+mi}{1}\PY{p}{:}\PY{p}{]}
\PY{n}{f} \PY{o}{=} \PY{p}{(}\PY{l+m+mf}{6.0}\PY{o}{*}\PY{o}{*}\PY{l+m+mf}{0.5} \PY{o}{/} \PY{n}{math}\PY{o}{.}\PY{n}{gamma}\PY{p}{(}\PY{l+m+mf}{0.5}\PY{p}{)}\PY{p}{)} \PY{o}{*} \PY{n}{xs}\PY{o}{*}\PY{o}{*}\PY{p}{(}\PY{l+m+mf}{0.5}\PY{o}{\PYZhy{}}\PY{l+m+mi}{1}\PY{p}{)} \PY{o}{*} \PY{n}{np}\PY{o}{.}\PY{n}{exp}\PY{p}{(}\PY{o}{\PYZhy{}}\PY{l+m+mf}{6.0}\PY{o}{*}\PY{n}{xs}\PY{p}{)}

\PY{n}{plt}\PY{o}{.}\PY{n}{figure}\PY{p}{(}\PY{p}{)}
\PY{n}{plt}\PY{o}{.}\PY{n}{hist}\PY{p}{(}\PY{n}{x\PYZus{}b1}\PY{p}{,} \PY{n}{bins}\PY{o}{=}\PY{l+m+mi}{100}\PY{p}{,} \PY{n}{density}\PY{o}{=}\PY{k+kc}{True}\PY{p}{,} \PY{n}{alpha}\PY{o}{=}\PY{l+m+mf}{0.5}\PY{p}{,} \PY{n}{label}\PY{o}{=}\PY{l+s+s2}{\PYZdq{}}\PY{l+s+s2}{Reducción de orden}\PY{l+s+s2}{\PYZdq{}}\PY{p}{)}
\PY{n}{plt}\PY{o}{.}\PY{n}{hist}\PY{p}{(}\PY{n}{x\PYZus{}b2}\PY{p}{,} \PY{n}{bins}\PY{o}{=}\PY{l+m+mi}{100}\PY{p}{,} \PY{n}{density}\PY{o}{=}\PY{k+kc}{True}\PY{p}{,} \PY{n}{alpha}\PY{o}{=}\PY{l+m+mf}{0.5}\PY{p}{,} \PY{n}{label}\PY{o}{=}\PY{l+s+sa}{r}\PY{l+s+s2}{\PYZdq{}}\PY{l+s+s2}{Escala de \PYZdl{}Z\PYZca{}2\PYZdl{}}\PY{l+s+s2}{\PYZdq{}}\PY{p}{)}
\PY{n}{plt}\PY{o}{.}\PY{n}{plot}\PY{p}{(}\PY{n}{xs}\PY{p}{,} \PY{n}{f}\PY{p}{,} \PY{n}{label}\PY{o}{=}\PY{l+s+s2}{\PYZdq{}}\PY{l+s+s2}{Teórica}\PY{l+s+s2}{\PYZdq{}}\PY{p}{)}\PY{p}{;} \PY{n}{plt}\PY{o}{.}\PY{n}{legend}\PY{p}{(}\PY{p}{)}\PY{p}{;} \PY{n}{plt}\PY{o}{.}\PY{n}{title}\PY{p}{(}\PY{l+s+s2}{\PYZdq{}}\PY{l+s+s2}{Gamma(0.5,6)}\PY{l+s+s2}{\PYZdq{}}\PY{p}{)}
\PY{n}{plt}\PY{o}{.}\PY{n}{show}\PY{p}{(}\PY{p}{)}

\PY{n+nb}{print}\PY{p}{(}\PY{l+s+sa}{f}\PY{l+s+s2}{\PYZdq{}}\PY{l+s+s2}{Medias: b1≈}\PY{l+s+si}{\PYZob{}}\PY{n}{x\PYZus{}b1}\PY{o}{.}\PY{n}{mean}\PY{p}{(}\PY{p}{)}\PY{l+s+si}{:}\PY{l+s+s2}{.4f}\PY{l+s+si}{\PYZcb{}}\PY{l+s+s2}{, b2≈}\PY{l+s+si}{\PYZob{}}\PY{n}{x\PYZus{}b2}\PY{o}{.}\PY{n}{mean}\PY{p}{(}\PY{p}{)}\PY{l+s+si}{:}\PY{l+s+s2}{.4f}\PY{l+s+si}{\PYZcb{}}\PY{l+s+s2}{, teórica=}\PY{l+s+si}{\PYZob{}}\PY{l+m+mf}{0.5}\PY{o}{/}\PY{l+m+mi}{6}\PY{l+s+si}{:}\PY{l+s+s2}{.4f}\PY{l+s+si}{\PYZcb{}}\PY{l+s+s2}{\PYZdq{}}\PY{p}{)}
\PY{n+nb}{print}\PY{p}{(}\PY{l+s+sa}{f}\PY{l+s+s2}{\PYZdq{}}\PY{l+s+s2}{Varianzas: b1≈}\PY{l+s+si}{\PYZob{}}\PY{n}{x\PYZus{}b1}\PY{o}{.}\PY{n}{var}\PY{p}{(}\PY{p}{)}\PY{l+s+si}{:}\PY{l+s+s2}{.4f}\PY{l+s+si}{\PYZcb{}}\PY{l+s+s2}{, b2≈}\PY{l+s+si}{\PYZob{}}\PY{n}{x\PYZus{}b2}\PY{o}{.}\PY{n}{var}\PY{p}{(}\PY{p}{)}\PY{l+s+si}{:}\PY{l+s+s2}{.4f}\PY{l+s+si}{\PYZcb{}}\PY{l+s+s2}{, teórica=}\PY{l+s+si}{\PYZob{}}\PY{l+m+mf}{0.5}\PY{o}{/}\PY{l+m+mi}{6}\PY{o}{*}\PY{o}{*}\PY{l+m+mi}{2}\PY{l+s+si}{:}\PY{l+s+s2}{.4f}\PY{l+s+si}{\PYZcb{}}\PY{l+s+s2}{\PYZdq{}}\PY{p}{)}
\end{Verbatim}
\end{tcolorbox}

    \begin{center}
    \adjustimage{max size={0.9\linewidth}{0.9\paperheight}}{notebook_files/notebook_56_0.png}
    \end{center}
    { \hspace*{\fill} \\}
    
    \begin{Verbatim}[commandchars=\\\{\}]
Medias: b1≈0.0832, b2≈0.0836, teórica=0.0833
Varianzas: b1≈0.0139, b2≈0.0143, teórica=0.0139
    \end{Verbatim}


    % Add a bibliography block to the postdoc
    
    
    
\end{document}
