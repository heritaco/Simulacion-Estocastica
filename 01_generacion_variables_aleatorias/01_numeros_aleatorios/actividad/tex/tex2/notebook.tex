\documentclass[11pt]{article}

    \usepackage[breakable]{tcolorbox}
    \usepackage{parskip} % Stop auto-indenting (to mimic markdown behaviour)
    

    % Basic figure setup, for now with no caption control since it's done
    % automatically by Pandoc (which extracts ![](path) syntax from Markdown).
    \usepackage{graphicx}
    % Keep aspect ratio if custom image width or height is specified
    \setkeys{Gin}{keepaspectratio}
    % Maintain compatibility with old templates. Remove in nbconvert 6.0
    \let\Oldincludegraphics\includegraphics
    % Ensure that by default, figures have no caption (until we provide a
    % proper Figure object with a Caption API and a way to capture that
    % in the conversion process - todo).
    \usepackage{caption}
    \DeclareCaptionFormat{nocaption}{}
    \captionsetup{format=nocaption,aboveskip=0pt,belowskip=0pt}

    \usepackage{float}
    \floatplacement{figure}{H} % forces figures to be placed at the correct location
    \usepackage{xcolor} % Allow colors to be defined
    \usepackage{enumerate} % Needed for markdown enumerations to work
    \usepackage{geometry} % Used to adjust the document margins
    \usepackage{amsmath} % Equations
    \usepackage{amssymb} % Equations
    \usepackage{textcomp} % defines textquotesingle
    % Hack from http://tex.stackexchange.com/a/47451/13684:
    \AtBeginDocument{%
        \def\PYZsq{\textquotesingle}% Upright quotes in Pygmentized code
    }
    \usepackage{upquote} % Upright quotes for verbatim code
    \usepackage{eurosym} % defines \euro

    \usepackage{iftex}
    \ifPDFTeX
        \usepackage[T1]{fontenc}
        \IfFileExists{alphabeta.sty}{
              \usepackage{alphabeta}
          }{
              \usepackage[mathletters]{ucs}
              \usepackage[utf8x]{inputenc}
          }
    \else
        \usepackage{fontspec}
        \usepackage{unicode-math}
    \fi

    \usepackage{fancyvrb} % verbatim replacement that allows latex
    \usepackage{grffile} % extends the file name processing of package graphics
                         % to support a larger range
    \makeatletter % fix for old versions of grffile with XeLaTeX
    \@ifpackagelater{grffile}{2019/11/01}
    {
      % Do nothing on new versions
    }
    {
      \def\Gread@@xetex#1{%
        \IfFileExists{"\Gin@base".bb}%
        {\Gread@eps{\Gin@base.bb}}%
        {\Gread@@xetex@aux#1}%
      }
    }
    \makeatother
    \usepackage[Export]{adjustbox} % Used to constrain images to a maximum size
    \adjustboxset{max size={0.9\linewidth}{0.9\paperheight}}

    % The hyperref package gives us a pdf with properly built
    % internal navigation ('pdf bookmarks' for the table of contents,
    % internal cross-reference links, web links for URLs, etc.)
    \usepackage{hyperref}
    % The default LaTeX title has an obnoxious amount of whitespace. By default,
    % titling removes some of it. It also provides customization options.
    \usepackage{titling}
    \usepackage{longtable} % longtable support required by pandoc >1.10
    \usepackage{booktabs}  % table support for pandoc > 1.12.2
    \usepackage{array}     % table support for pandoc >= 2.11.3
    \usepackage{calc}      % table minipage width calculation for pandoc >= 2.11.1
    \usepackage[inline]{enumitem} % IRkernel/repr support (it uses the enumerate* environment)
    \usepackage[normalem]{ulem} % ulem is needed to support strikethroughs (\sout)
                                % normalem makes italics be italics, not underlines
    \usepackage{soul}      % strikethrough (\st) support for pandoc >= 3.0.0
    \usepackage{mathrsfs}
    

    
    % Colors for the hyperref package
    \definecolor{urlcolor}{rgb}{0,.145,.698}
    \definecolor{linkcolor}{rgb}{.71,0.21,0.01}
    \definecolor{citecolor}{rgb}{.12,.54,.11}

    % ANSI colors
    \definecolor{ansi-black}{HTML}{3E424D}
    \definecolor{ansi-black-intense}{HTML}{282C36}
    \definecolor{ansi-red}{HTML}{E75C58}
    \definecolor{ansi-red-intense}{HTML}{B22B31}
    \definecolor{ansi-green}{HTML}{00A250}
    \definecolor{ansi-green-intense}{HTML}{007427}
    \definecolor{ansi-yellow}{HTML}{DDB62B}
    \definecolor{ansi-yellow-intense}{HTML}{B27D12}
    \definecolor{ansi-blue}{HTML}{208FFB}
    \definecolor{ansi-blue-intense}{HTML}{0065CA}
    \definecolor{ansi-magenta}{HTML}{D160C4}
    \definecolor{ansi-magenta-intense}{HTML}{A03196}
    \definecolor{ansi-cyan}{HTML}{60C6C8}
    \definecolor{ansi-cyan-intense}{HTML}{258F8F}
    \definecolor{ansi-white}{HTML}{C5C1B4}
    \definecolor{ansi-white-intense}{HTML}{A1A6B2}
    \definecolor{ansi-default-inverse-fg}{HTML}{FFFFFF}
    \definecolor{ansi-default-inverse-bg}{HTML}{000000}

    % common color for the border for error outputs.
    \definecolor{outerrorbackground}{HTML}{FFDFDF}

    % commands and environments needed by pandoc snippets
    % extracted from the output of `pandoc -s`
    \providecommand{\tightlist}{%
      \setlength{\itemsep}{0pt}\setlength{\parskip}{0pt}}
    \DefineVerbatimEnvironment{Highlighting}{Verbatim}{commandchars=\\\{\}}
    % Add ',fontsize=\small' for more characters per line
    \newenvironment{Shaded}{}{}
    \newcommand{\KeywordTok}[1]{\textcolor[rgb]{0.00,0.44,0.13}{\textbf{{#1}}}}
    \newcommand{\DataTypeTok}[1]{\textcolor[rgb]{0.56,0.13,0.00}{{#1}}}
    \newcommand{\DecValTok}[1]{\textcolor[rgb]{0.25,0.63,0.44}{{#1}}}
    \newcommand{\BaseNTok}[1]{\textcolor[rgb]{0.25,0.63,0.44}{{#1}}}
    \newcommand{\FloatTok}[1]{\textcolor[rgb]{0.25,0.63,0.44}{{#1}}}
    \newcommand{\CharTok}[1]{\textcolor[rgb]{0.25,0.44,0.63}{{#1}}}
    \newcommand{\StringTok}[1]{\textcolor[rgb]{0.25,0.44,0.63}{{#1}}}
    \newcommand{\CommentTok}[1]{\textcolor[rgb]{0.38,0.63,0.69}{\textit{{#1}}}}
    \newcommand{\OtherTok}[1]{\textcolor[rgb]{0.00,0.44,0.13}{{#1}}}
    \newcommand{\AlertTok}[1]{\textcolor[rgb]{1.00,0.00,0.00}{\textbf{{#1}}}}
    \newcommand{\FunctionTok}[1]{\textcolor[rgb]{0.02,0.16,0.49}{{#1}}}
    \newcommand{\RegionMarkerTok}[1]{{#1}}
    \newcommand{\ErrorTok}[1]{\textcolor[rgb]{1.00,0.00,0.00}{\textbf{{#1}}}}
    \newcommand{\NormalTok}[1]{{#1}}

    % Additional commands for more recent versions of Pandoc
    \newcommand{\ConstantTok}[1]{\textcolor[rgb]{0.53,0.00,0.00}{{#1}}}
    \newcommand{\SpecialCharTok}[1]{\textcolor[rgb]{0.25,0.44,0.63}{{#1}}}
    \newcommand{\VerbatimStringTok}[1]{\textcolor[rgb]{0.25,0.44,0.63}{{#1}}}
    \newcommand{\SpecialStringTok}[1]{\textcolor[rgb]{0.73,0.40,0.53}{{#1}}}
    \newcommand{\ImportTok}[1]{{#1}}
    \newcommand{\DocumentationTok}[1]{\textcolor[rgb]{0.73,0.13,0.13}{\textit{{#1}}}}
    \newcommand{\AnnotationTok}[1]{\textcolor[rgb]{0.38,0.63,0.69}{\textbf{\textit{{#1}}}}}
    \newcommand{\CommentVarTok}[1]{\textcolor[rgb]{0.38,0.63,0.69}{\textbf{\textit{{#1}}}}}
    \newcommand{\VariableTok}[1]{\textcolor[rgb]{0.10,0.09,0.49}{{#1}}}
    \newcommand{\ControlFlowTok}[1]{\textcolor[rgb]{0.00,0.44,0.13}{\textbf{{#1}}}}
    \newcommand{\OperatorTok}[1]{\textcolor[rgb]{0.40,0.40,0.40}{{#1}}}
    \newcommand{\BuiltInTok}[1]{{#1}}
    \newcommand{\ExtensionTok}[1]{{#1}}
    \newcommand{\PreprocessorTok}[1]{\textcolor[rgb]{0.74,0.48,0.00}{{#1}}}
    \newcommand{\AttributeTok}[1]{\textcolor[rgb]{0.49,0.56,0.16}{{#1}}}
    \newcommand{\InformationTok}[1]{\textcolor[rgb]{0.38,0.63,0.69}{\textbf{\textit{{#1}}}}}
    \newcommand{\WarningTok}[1]{\textcolor[rgb]{0.38,0.63,0.69}{\textbf{\textit{{#1}}}}}
    \makeatletter
    \newsavebox\pandoc@box
    \newcommand*\pandocbounded[1]{%
      \sbox\pandoc@box{#1}%
      % scaling factors for width and height
      \Gscale@div\@tempa\textheight{\dimexpr\ht\pandoc@box+\dp\pandoc@box\relax}%
      \Gscale@div\@tempb\linewidth{\wd\pandoc@box}%
      % select the smaller of both
      \ifdim\@tempb\p@<\@tempa\p@
        \let\@tempa\@tempb
      \fi
      % scaling accordingly (\@tempa < 1)
      \ifdim\@tempa\p@<\p@
        \scalebox{\@tempa}{\usebox\pandoc@box}%
      % scaling not needed, use as it is
      \else
        \usebox{\pandoc@box}%
      \fi
    }
    \makeatother

    % Define a nice break command that doesn't care if a line doesn't already
    % exist.
    \def\br{\hspace*{\fill} \\* }
    % Math Jax compatibility definitions
    \def\gt{>}
    \def\lt{<}
    \let\Oldtex\TeX
    \let\Oldlatex\LaTeX
    \renewcommand{\TeX}{\textrm{\Oldtex}}
    \renewcommand{\LaTeX}{\textrm{\Oldlatex}}
    % Document parameters
    % Document title
    \title{notebook}
    
    
    
    
    
    
    
% Pygments definitions
\makeatletter
\def\PY@reset{\let\PY@it=\relax \let\PY@bf=\relax%
    \let\PY@ul=\relax \let\PY@tc=\relax%
    \let\PY@bc=\relax \let\PY@ff=\relax}
\def\PY@tok#1{\csname PY@tok@#1\endcsname}
\def\PY@toks#1+{\ifx\relax#1\empty\else%
    \PY@tok{#1}\expandafter\PY@toks\fi}
\def\PY@do#1{\PY@bc{\PY@tc{\PY@ul{%
    \PY@it{\PY@bf{\PY@ff{#1}}}}}}}
\def\PY#1#2{\PY@reset\PY@toks#1+\relax+\PY@do{#2}}

\@namedef{PY@tok@w}{\def\PY@tc##1{\textcolor[rgb]{0.73,0.73,0.73}{##1}}}
\@namedef{PY@tok@c}{\let\PY@it=\textit\def\PY@tc##1{\textcolor[rgb]{0.24,0.48,0.48}{##1}}}
\@namedef{PY@tok@cp}{\def\PY@tc##1{\textcolor[rgb]{0.61,0.40,0.00}{##1}}}
\@namedef{PY@tok@k}{\let\PY@bf=\textbf\def\PY@tc##1{\textcolor[rgb]{0.00,0.50,0.00}{##1}}}
\@namedef{PY@tok@kp}{\def\PY@tc##1{\textcolor[rgb]{0.00,0.50,0.00}{##1}}}
\@namedef{PY@tok@kt}{\def\PY@tc##1{\textcolor[rgb]{0.69,0.00,0.25}{##1}}}
\@namedef{PY@tok@o}{\def\PY@tc##1{\textcolor[rgb]{0.40,0.40,0.40}{##1}}}
\@namedef{PY@tok@ow}{\let\PY@bf=\textbf\def\PY@tc##1{\textcolor[rgb]{0.67,0.13,1.00}{##1}}}
\@namedef{PY@tok@nb}{\def\PY@tc##1{\textcolor[rgb]{0.00,0.50,0.00}{##1}}}
\@namedef{PY@tok@nf}{\def\PY@tc##1{\textcolor[rgb]{0.00,0.00,1.00}{##1}}}
\@namedef{PY@tok@nc}{\let\PY@bf=\textbf\def\PY@tc##1{\textcolor[rgb]{0.00,0.00,1.00}{##1}}}
\@namedef{PY@tok@nn}{\let\PY@bf=\textbf\def\PY@tc##1{\textcolor[rgb]{0.00,0.00,1.00}{##1}}}
\@namedef{PY@tok@ne}{\let\PY@bf=\textbf\def\PY@tc##1{\textcolor[rgb]{0.80,0.25,0.22}{##1}}}
\@namedef{PY@tok@nv}{\def\PY@tc##1{\textcolor[rgb]{0.10,0.09,0.49}{##1}}}
\@namedef{PY@tok@no}{\def\PY@tc##1{\textcolor[rgb]{0.53,0.00,0.00}{##1}}}
\@namedef{PY@tok@nl}{\def\PY@tc##1{\textcolor[rgb]{0.46,0.46,0.00}{##1}}}
\@namedef{PY@tok@ni}{\let\PY@bf=\textbf\def\PY@tc##1{\textcolor[rgb]{0.44,0.44,0.44}{##1}}}
\@namedef{PY@tok@na}{\def\PY@tc##1{\textcolor[rgb]{0.41,0.47,0.13}{##1}}}
\@namedef{PY@tok@nt}{\let\PY@bf=\textbf\def\PY@tc##1{\textcolor[rgb]{0.00,0.50,0.00}{##1}}}
\@namedef{PY@tok@nd}{\def\PY@tc##1{\textcolor[rgb]{0.67,0.13,1.00}{##1}}}
\@namedef{PY@tok@s}{\def\PY@tc##1{\textcolor[rgb]{0.73,0.13,0.13}{##1}}}
\@namedef{PY@tok@sd}{\let\PY@it=\textit\def\PY@tc##1{\textcolor[rgb]{0.73,0.13,0.13}{##1}}}
\@namedef{PY@tok@si}{\let\PY@bf=\textbf\def\PY@tc##1{\textcolor[rgb]{0.64,0.35,0.47}{##1}}}
\@namedef{PY@tok@se}{\let\PY@bf=\textbf\def\PY@tc##1{\textcolor[rgb]{0.67,0.36,0.12}{##1}}}
\@namedef{PY@tok@sr}{\def\PY@tc##1{\textcolor[rgb]{0.64,0.35,0.47}{##1}}}
\@namedef{PY@tok@ss}{\def\PY@tc##1{\textcolor[rgb]{0.10,0.09,0.49}{##1}}}
\@namedef{PY@tok@sx}{\def\PY@tc##1{\textcolor[rgb]{0.00,0.50,0.00}{##1}}}
\@namedef{PY@tok@m}{\def\PY@tc##1{\textcolor[rgb]{0.40,0.40,0.40}{##1}}}
\@namedef{PY@tok@gh}{\let\PY@bf=\textbf\def\PY@tc##1{\textcolor[rgb]{0.00,0.00,0.50}{##1}}}
\@namedef{PY@tok@gu}{\let\PY@bf=\textbf\def\PY@tc##1{\textcolor[rgb]{0.50,0.00,0.50}{##1}}}
\@namedef{PY@tok@gd}{\def\PY@tc##1{\textcolor[rgb]{0.63,0.00,0.00}{##1}}}
\@namedef{PY@tok@gi}{\def\PY@tc##1{\textcolor[rgb]{0.00,0.52,0.00}{##1}}}
\@namedef{PY@tok@gr}{\def\PY@tc##1{\textcolor[rgb]{0.89,0.00,0.00}{##1}}}
\@namedef{PY@tok@ge}{\let\PY@it=\textit}
\@namedef{PY@tok@gs}{\let\PY@bf=\textbf}
\@namedef{PY@tok@ges}{\let\PY@bf=\textbf\let\PY@it=\textit}
\@namedef{PY@tok@gp}{\let\PY@bf=\textbf\def\PY@tc##1{\textcolor[rgb]{0.00,0.00,0.50}{##1}}}
\@namedef{PY@tok@go}{\def\PY@tc##1{\textcolor[rgb]{0.44,0.44,0.44}{##1}}}
\@namedef{PY@tok@gt}{\def\PY@tc##1{\textcolor[rgb]{0.00,0.27,0.87}{##1}}}
\@namedef{PY@tok@err}{\def\PY@bc##1{{\setlength{\fboxsep}{\string -\fboxrule}\fcolorbox[rgb]{1.00,0.00,0.00}{1,1,1}{\strut ##1}}}}
\@namedef{PY@tok@kc}{\let\PY@bf=\textbf\def\PY@tc##1{\textcolor[rgb]{0.00,0.50,0.00}{##1}}}
\@namedef{PY@tok@kd}{\let\PY@bf=\textbf\def\PY@tc##1{\textcolor[rgb]{0.00,0.50,0.00}{##1}}}
\@namedef{PY@tok@kn}{\let\PY@bf=\textbf\def\PY@tc##1{\textcolor[rgb]{0.00,0.50,0.00}{##1}}}
\@namedef{PY@tok@kr}{\let\PY@bf=\textbf\def\PY@tc##1{\textcolor[rgb]{0.00,0.50,0.00}{##1}}}
\@namedef{PY@tok@bp}{\def\PY@tc##1{\textcolor[rgb]{0.00,0.50,0.00}{##1}}}
\@namedef{PY@tok@fm}{\def\PY@tc##1{\textcolor[rgb]{0.00,0.00,1.00}{##1}}}
\@namedef{PY@tok@vc}{\def\PY@tc##1{\textcolor[rgb]{0.10,0.09,0.49}{##1}}}
\@namedef{PY@tok@vg}{\def\PY@tc##1{\textcolor[rgb]{0.10,0.09,0.49}{##1}}}
\@namedef{PY@tok@vi}{\def\PY@tc##1{\textcolor[rgb]{0.10,0.09,0.49}{##1}}}
\@namedef{PY@tok@vm}{\def\PY@tc##1{\textcolor[rgb]{0.10,0.09,0.49}{##1}}}
\@namedef{PY@tok@sa}{\def\PY@tc##1{\textcolor[rgb]{0.73,0.13,0.13}{##1}}}
\@namedef{PY@tok@sb}{\def\PY@tc##1{\textcolor[rgb]{0.73,0.13,0.13}{##1}}}
\@namedef{PY@tok@sc}{\def\PY@tc##1{\textcolor[rgb]{0.73,0.13,0.13}{##1}}}
\@namedef{PY@tok@dl}{\def\PY@tc##1{\textcolor[rgb]{0.73,0.13,0.13}{##1}}}
\@namedef{PY@tok@s2}{\def\PY@tc##1{\textcolor[rgb]{0.73,0.13,0.13}{##1}}}
\@namedef{PY@tok@sh}{\def\PY@tc##1{\textcolor[rgb]{0.73,0.13,0.13}{##1}}}
\@namedef{PY@tok@s1}{\def\PY@tc##1{\textcolor[rgb]{0.73,0.13,0.13}{##1}}}
\@namedef{PY@tok@mb}{\def\PY@tc##1{\textcolor[rgb]{0.40,0.40,0.40}{##1}}}
\@namedef{PY@tok@mf}{\def\PY@tc##1{\textcolor[rgb]{0.40,0.40,0.40}{##1}}}
\@namedef{PY@tok@mh}{\def\PY@tc##1{\textcolor[rgb]{0.40,0.40,0.40}{##1}}}
\@namedef{PY@tok@mi}{\def\PY@tc##1{\textcolor[rgb]{0.40,0.40,0.40}{##1}}}
\@namedef{PY@tok@il}{\def\PY@tc##1{\textcolor[rgb]{0.40,0.40,0.40}{##1}}}
\@namedef{PY@tok@mo}{\def\PY@tc##1{\textcolor[rgb]{0.40,0.40,0.40}{##1}}}
\@namedef{PY@tok@ch}{\let\PY@it=\textit\def\PY@tc##1{\textcolor[rgb]{0.24,0.48,0.48}{##1}}}
\@namedef{PY@tok@cm}{\let\PY@it=\textit\def\PY@tc##1{\textcolor[rgb]{0.24,0.48,0.48}{##1}}}
\@namedef{PY@tok@cpf}{\let\PY@it=\textit\def\PY@tc##1{\textcolor[rgb]{0.24,0.48,0.48}{##1}}}
\@namedef{PY@tok@c1}{\let\PY@it=\textit\def\PY@tc##1{\textcolor[rgb]{0.24,0.48,0.48}{##1}}}
\@namedef{PY@tok@cs}{\let\PY@it=\textit\def\PY@tc##1{\textcolor[rgb]{0.24,0.48,0.48}{##1}}}

\def\PYZbs{\char`\\}
\def\PYZus{\char`\_}
\def\PYZob{\char`\{}
\def\PYZcb{\char`\}}
\def\PYZca{\char`\^}
\def\PYZam{\char`\&}
\def\PYZlt{\char`\<}
\def\PYZgt{\char`\>}
\def\PYZsh{\char`\#}
\def\PYZpc{\char`\%}
\def\PYZdl{\char`\$}
\def\PYZhy{\char`\-}
\def\PYZsq{\char`\'}
\def\PYZdq{\char`\"}
\def\PYZti{\char`\~}
% for compatibility with earlier versions
\def\PYZat{@}
\def\PYZlb{[}
\def\PYZrb{]}
\makeatother


    % For linebreaks inside Verbatim environment from package fancyvrb.
    \makeatletter
        \newbox\Wrappedcontinuationbox
        \newbox\Wrappedvisiblespacebox
        \newcommand*\Wrappedvisiblespace {\textcolor{red}{\textvisiblespace}}
        \newcommand*\Wrappedcontinuationsymbol {\textcolor{red}{\llap{\tiny$\m@th\hookrightarrow$}}}
        \newcommand*\Wrappedcontinuationindent {3ex }
        \newcommand*\Wrappedafterbreak {\kern\Wrappedcontinuationindent\copy\Wrappedcontinuationbox}
        % Take advantage of the already applied Pygments mark-up to insert
        % potential linebreaks for TeX processing.
        %        {, <, #, %, $, ' and ": go to next line.
        %        _, }, ^, &, >, - and ~: stay at end of broken line.
        % Use of \textquotesingle for straight quote.
        \newcommand*\Wrappedbreaksatspecials {%
            \def\PYGZus{\discretionary{\char`\_}{\Wrappedafterbreak}{\char`\_}}%
            \def\PYGZob{\discretionary{}{\Wrappedafterbreak\char`\{}{\char`\{}}%
            \def\PYGZcb{\discretionary{\char`\}}{\Wrappedafterbreak}{\char`\}}}%
            \def\PYGZca{\discretionary{\char`\^}{\Wrappedafterbreak}{\char`\^}}%
            \def\PYGZam{\discretionary{\char`\&}{\Wrappedafterbreak}{\char`\&}}%
            \def\PYGZlt{\discretionary{}{\Wrappedafterbreak\char`\<}{\char`\<}}%
            \def\PYGZgt{\discretionary{\char`\>}{\Wrappedafterbreak}{\char`\>}}%
            \def\PYGZsh{\discretionary{}{\Wrappedafterbreak\char`\#}{\char`\#}}%
            \def\PYGZpc{\discretionary{}{\Wrappedafterbreak\char`\%}{\char`\%}}%
            \def\PYGZdl{\discretionary{}{\Wrappedafterbreak\char`\$}{\char`\$}}%
            \def\PYGZhy{\discretionary{\char`\-}{\Wrappedafterbreak}{\char`\-}}%
            \def\PYGZsq{\discretionary{}{\Wrappedafterbreak\textquotesingle}{\textquotesingle}}%
            \def\PYGZdq{\discretionary{}{\Wrappedafterbreak\char`\"}{\char`\"}}%
            \def\PYGZti{\discretionary{\char`\~}{\Wrappedafterbreak}{\char`\~}}%
        }
        % Some characters . , ; ? ! / are not pygmentized.
        % This macro makes them "active" and they will insert potential linebreaks
        \newcommand*\Wrappedbreaksatpunct {%
            \lccode`\~`\.\lowercase{\def~}{\discretionary{\hbox{\char`\.}}{\Wrappedafterbreak}{\hbox{\char`\.}}}%
            \lccode`\~`\,\lowercase{\def~}{\discretionary{\hbox{\char`\,}}{\Wrappedafterbreak}{\hbox{\char`\,}}}%
            \lccode`\~`\;\lowercase{\def~}{\discretionary{\hbox{\char`\;}}{\Wrappedafterbreak}{\hbox{\char`\;}}}%
            \lccode`\~`\:\lowercase{\def~}{\discretionary{\hbox{\char`\:}}{\Wrappedafterbreak}{\hbox{\char`\:}}}%
            \lccode`\~`\?\lowercase{\def~}{\discretionary{\hbox{\char`\?}}{\Wrappedafterbreak}{\hbox{\char`\?}}}%
            \lccode`\~`\!\lowercase{\def~}{\discretionary{\hbox{\char`\!}}{\Wrappedafterbreak}{\hbox{\char`\!}}}%
            \lccode`\~`\/\lowercase{\def~}{\discretionary{\hbox{\char`\/}}{\Wrappedafterbreak}{\hbox{\char`\/}}}%
            \catcode`\.\active
            \catcode`\,\active
            \catcode`\;\active
            \catcode`\:\active
            \catcode`\?\active
            \catcode`\!\active
            \catcode`\/\active
            \lccode`\~`\~
        }
    \makeatother

    \let\OriginalVerbatim=\Verbatim
    \makeatletter
    \renewcommand{\Verbatim}[1][1]{%
        %\parskip\z@skip
        \sbox\Wrappedcontinuationbox {\Wrappedcontinuationsymbol}%
        \sbox\Wrappedvisiblespacebox {\FV@SetupFont\Wrappedvisiblespace}%
        \def\FancyVerbFormatLine ##1{\hsize\linewidth
            \vtop{\raggedright\hyphenpenalty\z@\exhyphenpenalty\z@
                \doublehyphendemerits\z@\finalhyphendemerits\z@
                \strut ##1\strut}%
        }%
        % If the linebreak is at a space, the latter will be displayed as visible
        % space at end of first line, and a continuation symbol starts next line.
        % Stretch/shrink are however usually zero for typewriter font.
        \def\FV@Space {%
            \nobreak\hskip\z@ plus\fontdimen3\font minus\fontdimen4\font
            \discretionary{\copy\Wrappedvisiblespacebox}{\Wrappedafterbreak}
            {\kern\fontdimen2\font}%
        }%

        % Allow breaks at special characters using \PYG... macros.
        \Wrappedbreaksatspecials
        % Breaks at punctuation characters . , ; ? ! and / need catcode=\active
        \OriginalVerbatim[#1,codes*=\Wrappedbreaksatpunct]%
    }
    \makeatother

    % Exact colors from NB
    \definecolor{incolor}{HTML}{303F9F}
    \definecolor{outcolor}{HTML}{D84315}
    \definecolor{cellborder}{HTML}{CFCFCF}
    \definecolor{cellbackground}{HTML}{F7F7F7}

    % prompt
    \makeatletter
    \newcommand{\boxspacing}{\kern\kvtcb@left@rule\kern\kvtcb@boxsep}
    \makeatother
    \newcommand{\prompt}[4]{
        {\ttfamily\llap{{\color{#2}[#3]:\hspace{3pt}#4}}\vspace{-\baselineskip}}
    }
    

    
    % Prevent overflowing lines due to hard-to-break entities
    \sloppy
    % Setup hyperref package
    \hypersetup{
      breaklinks=true,  % so long urls are correctly broken across lines
      colorlinks=true,
      urlcolor=urlcolor,
      linkcolor=linkcolor,
      citecolor=citecolor,
      }
    % Slightly bigger margins than the latex defaults
    
    \geometry{verbose,tmargin=1in,bmargin=1in,lmargin=1in,rmargin=1in}
    
    

    %
    %
    %
    %
    %
    %
    %
    %
    %
    %
    %
    %
    %
    %
    %
    %
    %
    %
    %
    %
    %
    %
    %



    \usepackage{newunicodechar}
    \newunicodechar{∼}{\ensuremath{\sim}}
    \newunicodechar{←}{\ensuremath{\leftarrow}}
    \newunicodechar{…}{\ldots}


    % --- header con título de subsección + número de 

    \usepackage{etoolbox}
    \pretocmd{\section}{\clearpage}{}{}
    \pretocmd{\subsection}{\clearpage}{}{}
    
    

    \begin{document}
        
        

    % --- Portada académica ---
    \begin{titlepage}
    \newgeometry{top=5cm,bottom=3.0cm,left=3.0cm,right=3.0cm}
    \centering

            \LARGE{\textbf{Actividad 1: Simulación
    Estocástica}}\\[0.5cm]

    \small
    \textbf{Curso:} TEMAS SELECTOS 1 (O25-LAT4032-1)\\

    \textbf{Profesor:} Rubén Blancas Rivera\\

    \textbf{Alumnos:} jujuju jajaja jojojo\\

    \textbf{Universidad:} Universidad de las Américas Puebla\\

    \textbf{Fecha:} 2025-08-15
        
    \restoregeometry
    \end{titlepage}
    % --- Fin portada ---
    \newpage



    %
    %
    %
    %
    %
    %
    %
    %
    %
    %
    %
    %
    %
    %
    %
    %
    %
    %
    %
    %
    %
    %

\begin{quote}
Export this notebook to PDF with LaTeX using the provided
\texttt{amsart\_template.tplx} for a Times-like, AMS-style layout.

Command:

\begin{Shaded}
\begin{Highlighting}[]
\ExtensionTok{jupyter}\NormalTok{ nbconvert }\AttributeTok{{-}{-}to}\NormalTok{ pdf }\AttributeTok{{-}{-}template}\NormalTok{ amsart\_template.tplx actividad1\_template.ipynb}
\end{Highlighting}
\end{Shaded}
\end{quote}

\textbf{Selection of exercises:} \emph{Indicate here whether you solved
\textbf{evens} or \textbf{odds} only (teams max 3).}

\newpage

    \hypertarget{ejercicio}{%
\section{Ejercicio}\label{ejercicio}}

Si \(x_0 = 5\) y \(x_n = 2x_{{n-1}} \bmod 150\). Encontrar
\(x_1, \dots, x_{{10}}\).

    \[
x_n = ax_{n-1} \bmod m
\]

    \[
10 = 2*5 \bmod 150 \\
20 = 10*5 \bmod 150 \\
40 = 20*5 \bmod 150 \\
80 = 40*5 \bmod 150 \\
10 = 80*5 \bmod 150 \\
\vdots
\]

    \begin{tcolorbox}[breakable, size=fbox, boxrule=1pt, pad at break*=1mm,colback=cellbackground, colframe=cellborder]
\prompt{In}{incolor}{ }{\boxspacing}
\begin{Verbatim}[commandchars=\\\{\}]
\PY{n}{pseudoaleatorios} \PY{o}{=} \PY{p}{[}\PY{p}{]}

\PY{n}{x0} \PY{o}{=} \PY{l+m+mi}{5}
\PY{n}{a} \PY{o}{=} \PY{l+m+mi}{2}
\PY{n}{m} \PY{o}{=} \PY{l+m+mi}{150}

\PY{k}{for} \PY{n}{i} \PY{o+ow}{in} \PY{n+nb}{range}\PY{p}{(}\PY{l+m+mi}{10}\PY{p}{)}\PY{p}{:}
    \PY{n}{xn} \PY{o}{=} \PY{p}{(}\PY{n}{a} \PY{o}{*} \PY{n}{x0}\PY{p}{)} \PY{o}{\PYZpc{}} \PY{n}{m}
    \PY{n}{x0} \PY{o}{=} \PY{n}{xn}
    \PY{n}{pseudoaleatorios}\PY{o}{.}\PY{n}{append}\PY{p}{(}\PY{n}{xn}\PY{p}{)}

\PY{n+nb}{print}\PY{p}{(}\PY{n}{pseudoaleatorios}\PY{p}{)}
\end{Verbatim}
\end{tcolorbox}

    \begin{Verbatim}[commandchars=\\\{\}]
[10, 20, 40, 80, 10, 20, 40, 80, 10, 20]
    \end{Verbatim}

    \hypertarget{ejercicio}{%
\section{Ejercicio}\label{ejercicio}}

\[\int_{0}^{1} \exp(e^x)\,dx\]

    Sea \[
\theta = \int_{0}^{1} \exp\!\big(e^{x}\big)\,dx.
\]

Reescritura como valor esperado con \(U\sim \mathrm{Unif}(0,1)\): \[
\theta=\mathbb{E}\!\left[\exp\!\big(e^{U}\big)\right].
\]

Estimador Monte Carlo con
\(u_1,\dots,u_K \stackrel{\text{iid}}{\sim} \mathrm{Unif}(0,1)\): \[
\widehat{\theta}_K=\frac{1}{K}\sum_{i=1}^{K} \exp\!\big(e^{u_i}\big).
\]

    \begin{tcolorbox}[breakable, size=fbox, boxrule=1pt, pad at break*=1mm,colback=cellbackground, colframe=cellborder]
\prompt{In}{incolor}{ }{\boxspacing}
\begin{Verbatim}[commandchars=\\\{\}]
\PY{k+kn}{import}\PY{+w}{ }\PY{n+nn}{numpy}\PY{+w}{ }\PY{k}{as}\PY{+w}{ }\PY{n+nn}{np}

\PY{k}{def}\PY{+w}{ }\PY{n+nf}{h}\PY{p}{(}\PY{n}{u}\PY{p}{)}\PY{p}{:}
    \PY{k}{return} \PY{n}{np}\PY{o}{.}\PY{n}{exp}\PY{p}{(}\PY{n}{np}\PY{o}{.}\PY{n}{exp}\PY{p}{(}\PY{n}{u}\PY{p}{)}\PY{p}{)}

\PY{n}{k} \PY{o}{=} \PY{l+m+mi}{1000}

\PY{n}{u} \PY{o}{=} \PY{n}{np}\PY{o}{.}\PY{n}{random}\PY{o}{.}\PY{n}{random}\PY{p}{(}\PY{n}{k}\PY{p}{)}

\PY{n}{h}\PY{p}{(}\PY{n}{u}\PY{p}{)}\PY{o}{.}\PY{n}{mean}\PY{p}{(}\PY{p}{)}
\end{Verbatim}
\end{tcolorbox}

            \begin{tcolorbox}[breakable, size=fbox, boxrule=.5pt, pad at break*=1mm, opacityfill=0]
\prompt{Out}{outcolor}{ }{\boxspacing}
\begin{Verbatim}[commandchars=\\\{\}]
np.float64(6.229521839385485)
\end{Verbatim}
\end{tcolorbox}
        
    \hypertarget{ejercicio}{%
\section{Ejercicio}\label{ejercicio}}

\[\int_{-2}^{2} e^{x+x^2} \, dx\]

    Sea \[
\theta=\int_{-2}^{2} e^{x+x^2}\,dx.
\]

Cambio de variable a ({[}0,1{]}): \[
u=\frac{x-(-2)}{2-(-2)}=\frac{x+2}{4},\qquad x=-2+4u,\qquad dx=4\,du.
\]

Entonces \[
\theta=\int_{0}^{1} 4\,\exp\!\big[(-2+4u)+(-2+4u)^2\big]\,du.
\]

Forma de valor esperado con \(U\sim\mathrm{Unif}(0,1)\): \[
\theta=\mathbb{E}\!\left[g(U)\right],\qquad
g(u)=4\,\exp\!\big[(-2+4u)+(-2+4u)^2\big].
\]

Estimador Monte Carlo: \[
\widehat{\theta}_K=\frac{1}{K}\sum_{i=1}^{K} g(u_i),\qquad u_i\stackrel{iid}{\sim}\mathrm{Unif}(0,1).
\]

    \begin{tcolorbox}[breakable, size=fbox, boxrule=1pt, pad at break*=1mm,colback=cellbackground, colframe=cellborder]
\prompt{In}{incolor}{ }{\boxspacing}
\begin{Verbatim}[commandchars=\\\{\}]
\PY{k}{def}\PY{+w}{ }\PY{n+nf}{h}\PY{p}{(}\PY{n}{u}\PY{p}{)}\PY{p}{:}
    \PY{k}{return} \PY{p}{(}\PY{n}{b}\PY{o}{\PYZhy{}}\PY{n}{a}\PY{p}{)}\PY{o}{*}\PY{n}{np}\PY{o}{.}\PY{n}{exp}\PY{p}{(}\PY{n}{a}\PY{o}{+}\PY{p}{(}\PY{n}{b}\PY{o}{\PYZhy{}}\PY{n}{a}\PY{p}{)}\PY{o}{*}\PY{n}{u} \PY{o}{+} \PY{p}{(}\PY{n}{a}\PY{o}{+}\PY{p}{(}\PY{n}{b}\PY{o}{\PYZhy{}}\PY{n}{a}\PY{p}{)}\PY{o}{*}\PY{n}{u}\PY{p}{)}\PY{o}{*}\PY{o}{*}\PY{l+m+mi}{2}\PY{p}{)}

\PY{n}{k} \PY{o}{=} \PY{l+m+mi}{10000}
\PY{n}{a} \PY{o}{=} \PY{o}{\PYZhy{}}\PY{l+m+mi}{2}
\PY{n}{b} \PY{o}{=} \PY{l+m+mi}{2}
\PY{n}{u} \PY{o}{=} \PY{n}{np}\PY{o}{.}\PY{n}{random}\PY{o}{.}\PY{n}{random}\PY{p}{(}\PY{n}{k}\PY{p}{)}
\PY{n}{h}\PY{p}{(}\PY{n}{u}\PY{p}{)}\PY{o}{.}\PY{n}{mean}\PY{p}{(}\PY{p}{)}
\end{Verbatim}
\end{tcolorbox}

            \begin{tcolorbox}[breakable, size=fbox, boxrule=.5pt, pad at break*=1mm, opacityfill=0]
\prompt{Out}{outcolor}{ }{\boxspacing}
\begin{Verbatim}[commandchars=\\\{\}]
np.float64(94.60588956711096)
\end{Verbatim}
\end{tcolorbox}
        
    \hypertarget{ejercicio}{%
\section{Ejercicio}\label{ejercicio}}

\[\int_{0}^{\infty} \frac{x}{(1+x^2)^2} \, dx\]

    Sea \[
\theta=\int_{0}^{\infty}\frac{x}{(1+x^2)^2}\,dx.
\]

Cambio (pág. 21): \[
y=\frac{1}{x+1},\qquad dy=-\frac{dx}{(x+1)^2}=-y^{2}\,dx.
\] Entonces \[
\theta=\int_{0}^{1} h(y)\,dy,\qquad 
h(y)=\frac{g\!\left(\tfrac{1}{y}-1\right)}{y^{2}},\quad 
g(x)=\frac{x}{(1+x^{2})^{2}}.
\]

Cálculo explícito de \(h\): \[
x=\frac{1-y}{y}
\;\Rightarrow\;
h(y)=\frac{(1-y)\,y}{\big(1-2y+2y^{2}\big)^{2}},\qquad y\in(0,1).
\]

Forma de esperanza con \(U\sim \mathrm{Unif}(0,1)\): \[
\theta=\mathbb{E}[\,h(U)\,].
\]

Estimador Monte Carlo: \[
\widehat{\theta}_K=\frac{1}{K}\sum_{i=1}^{K} h(u_i),
\quad u_i\stackrel{\text{iid}}{\sim}\mathrm{Unif}(0,1).
\]

Chequeo analítico: \[
\theta=\int_{0}^{\infty}\frac{x}{(1+x^2)^2}\,dx
=-\tfrac{1}{2}\int_{0}^{\infty} d\!\left(\frac{1}{1+x^{2}}\right)
=\tfrac{1}{2}.
\]

    \begin{tcolorbox}[breakable, size=fbox, boxrule=1pt, pad at break*=1mm,colback=cellbackground, colframe=cellborder]
\prompt{In}{incolor}{ }{\boxspacing}
\begin{Verbatim}[commandchars=\\\{\}]
\PY{k}{def}\PY{+w}{ }\PY{n+nf}{h}\PY{p}{(}\PY{n}{u}\PY{p}{)}\PY{p}{:}
    \PY{k}{return} \PY{p}{(}\PY{p}{(}\PY{p}{(}\PY{p}{(}\PY{l+m+mi}{1}\PY{o}{/}\PY{n}{u}\PY{p}{)}\PY{o}{\PYZhy{}}\PY{l+m+mi}{1}\PY{p}{)}\PY{o}{/}\PY{p}{(}\PY{l+m+mi}{1}\PY{o}{+}\PY{p}{(}\PY{p}{(}\PY{l+m+mi}{1}\PY{o}{/}\PY{n}{u}\PY{p}{)}\PY{o}{\PYZhy{}}\PY{l+m+mi}{1}\PY{p}{)}\PY{o}{*}\PY{o}{*}\PY{l+m+mi}{2}\PY{p}{)}\PY{o}{*}\PY{o}{*}\PY{l+m+mi}{2}\PY{p}{)}\PY{p}{)}\PY{o}{/}\PY{p}{(}\PY{n}{u}\PY{o}{*}\PY{o}{*}\PY{l+m+mi}{2}\PY{p}{)}

\PY{n}{k} \PY{o}{=} \PY{l+m+mi}{10000000}

\PY{n}{u} \PY{o}{=} \PY{n}{np}\PY{o}{.}\PY{n}{random}\PY{o}{.}\PY{n}{random}\PY{p}{(}\PY{n}{k}\PY{p}{)}
\PY{n}{h}\PY{p}{(}\PY{n}{u}\PY{p}{)}\PY{o}{.}\PY{n}{mean}\PY{p}{(}\PY{p}{)}
\end{Verbatim}
\end{tcolorbox}

            \begin{tcolorbox}[breakable, size=fbox, boxrule=.5pt, pad at break*=1mm, opacityfill=0]
\prompt{Out}{outcolor}{ }{\boxspacing}
\begin{Verbatim}[commandchars=\\\{\}]
np.float64(0.5000960390593421)
\end{Verbatim}
\end{tcolorbox}
        
    Sea \[
\theta=\int_{0}^{\infty}\frac{x}{(1+x^2)^2}\,dx.
\]

\begin{enumerate}
\def\labelenumi{\arabic{enumi})}
\item
  Integral impropia: \[
  \theta=\lim_{b\to\infty}\int_{0}^{b}\frac{x}{(1+x^2)^2}\,dx.
  \]
\item
  Sustitución \(u=1+x^2\Rightarrow du=2x\,dx\): cuando
  \(x=0\Rightarrow u=1\), cuando \(x=b\Rightarrow u=1+b^2\). Entonces \[
  \int_{0}^{b}\frac{x}{(1+x^2)^2}\,dx
  =\frac12\int_{1}^{\,1+b^2} u^{-2}\,du.
  \]
\item
  Primitiva: \[
  \int u^{-2}\,du=-u^{-1}+C.
  \]
\item
  Evaluación: \[
  \frac12\Big[-u^{-1}\Big]_{1}^{\,1+b^2}
  =\frac12\!\left(-\frac{1}{1+b^2}+1\right).
  \]
\item
  Límite: \[
  \theta=\lim_{b\to\infty}\frac12\!\left(1-\frac{1}{1+b^2}\right)
  =\frac12.
  \]
\end{enumerate}

(Equivalente por antiderivación directa:
\(\displaystyle \int \frac{x}{(1+x^2)^2}dx=-\frac{1}{2(1+x^2)}+C\), y
luego
\(\theta=\lim_{b\to\infty}\big[-\tfrac{1}{2(1+x^2)}\big]_{0}^{b}=\tfrac12\).)

    \hypertarget{ejercicio}{%
\section{Ejercicio}\label{ejercicio}}

\[\int_{0}^{1} \int_{0}^{1} e^{(x+y)^2} \, dy \, dx\]

    Sea \[
\theta=\int_{0}^{1}\!\!\int_{0}^{1} e^{(x+y)^2}\,dy\,dx.
\]

\textbf{Notación de ``integrales múltiples''}: \[
\theta=\int_{0}^{1}\!\!\int_{0}^{1} g(x_1,x_2)\,dx_1\,dx_2,
\qquad g(x_1,x_2)=e^{(x_1+x_2)^2}.
\] Por las diapositivas: \[
\theta=\mathbb{E}\big[g(U_1,U_2)\big],
\quad U_1,U_2\stackrel{iid}{\sim}\mathrm{Unif}(0,1). \;\; \text{(Ley de la esperanza en el hipercubo)}
\] :contentReference{oaicite:0}

\textbf{Estimador Monte Carlo} con (k) muestras: \[
\widehat{\theta}_k=\frac{1}{k}\sum_{i=1}^{k} g(u_{i1},u_{i2})
=\frac{1}{k}\sum_{i=1}^{k} \exp\!\big((u_{i1}+u_{i2})^{2}\big),
\quad (u_{i1},u_{i2})\stackrel{iid}{\sim}\mathrm{Unif}(0,1)^2.
\] Algoritmo (formato de las diapositivas): inicialice (S\leftarrow0).
Para (i=1,\dots,k): genere (u\_\{i1\},u\_\{i2\}\sim U(0,1)), actualice
(S\leftarrow S+g(u\_\{i1\},u\_\{i2\})). Devuelva
(\widehat{\theta}\_k=S/k). :contentReference{oaicite:1}

\textbf{Error de simulación}: \[
\operatorname{se}(\widehat{\theta}_k)\approx \frac{s_g}{\sqrt{k}},
\quad
s_g^2=\frac{1}{k-1}\sum_{i=1}^{k}\Big(g(u_{i1},u_{i2})-\widehat{\theta}_k\Big)^2.
\]

    Sea \[
\theta=\int_{0}^{1}\!\!\int_{0}^{1} e^{(x+y)^2}\,dy\,dx
=\mathbb{E}\big[g(U_1,U_2)\big],\quad g(x,y)=e^{(x+y)^2},
\quad U_1,U_2\stackrel{iid}{\sim}\mathrm{Unif}(0,1).
\]

Estimador Monte Carlo: \[
\widehat{\theta}_k=\frac{1}{k}\sum_{i=1}^{k} g(u_{i1},u_{i2}).
\]

    \begin{tcolorbox}[breakable, size=fbox, boxrule=1pt, pad at break*=1mm,colback=cellbackground, colframe=cellborder]
\prompt{In}{incolor}{ }{\boxspacing}
\begin{Verbatim}[commandchars=\\\{\}]
\PY{k}{def}\PY{+w}{ }\PY{n+nf}{h}\PY{p}{(}\PY{n}{u1}\PY{p}{,} \PY{n}{u2}\PY{p}{)}\PY{p}{:}
    \PY{k}{return} \PY{n}{np}\PY{o}{.}\PY{n}{exp}\PY{p}{(}\PY{p}{(}\PY{n}{u1} \PY{o}{+} \PY{n}{u2}\PY{p}{)}\PY{o}{*}\PY{o}{*}\PY{l+m+mi}{2}\PY{p}{)}

\PY{n}{k} \PY{o}{=} \PY{l+m+mi}{10000}

\PY{n}{u1} \PY{o}{=} \PY{n}{np}\PY{o}{.}\PY{n}{random}\PY{o}{.}\PY{n}{random}\PY{p}{(}\PY{n}{k}\PY{p}{)}
\PY{n}{u2} \PY{o}{=} \PY{n}{np}\PY{o}{.}\PY{n}{random}\PY{o}{.}\PY{n}{random}\PY{p}{(}\PY{n}{k}\PY{p}{)}
\PY{n}{h}\PY{p}{(}\PY{n}{u1}\PY{p}{,} \PY{n}{u2}\PY{p}{)}\PY{o}{.}\PY{n}{mean}\PY{p}{(}\PY{p}{)}
\end{Verbatim}
\end{tcolorbox}

            \begin{tcolorbox}[breakable, size=fbox, boxrule=.5pt, pad at break*=1mm, opacityfill=0]
\prompt{Out}{outcolor}{ }{\boxspacing}
\begin{Verbatim}[commandchars=\\\{\}]
np.float64(1.2140555137345483)
\end{Verbatim}
\end{tcolorbox}
        
    \hypertarget{ejercicio}{%
\section{Ejercicio}\label{ejercicio}}

Usar simulación para aproximar \(\operatorname{{Cov}}(U, e^U)\), donde
\(U \sim \mathcal U(0,1)\). Comparar con la respuesta exacta.

    Asumo \(U\sim\mathrm{Unif}(0,1)\).

\begin{center}\rule{0.5\linewidth}{0.5pt}\end{center}

\hypertarget{definiciuxf3n-y-equivalencia}{%
\subsection{Definición y
equivalencia}\label{definiciuxf3n-y-equivalencia}}

Por definición,

\[
\operatorname{Cov}(X,Y)=\mathbb{E}\big[(X-\mathbb{E}X)(Y-\mathbb{E}Y)\big].
\]

Expansión lineal:

\[
\operatorname{Cov}(X,Y)=\mathbb{E}[XY]-\mathbb{E}[X]\;\mathbb{E}[Y].
\]

Aplicando a \(X=U\) y \(Y=e^{U}\):

\[
\operatorname{Cov}(U,e^{U})=\mathbb{E}\!\big[U e^{U}\big]-\mathbb{E}[U]\;\mathbb{E}[e^{U}].
\]

\begin{center}\rule{0.5\linewidth}{0.5pt}\end{center}

\hypertarget{cuxe1lculo-analuxedtico-paso-a-paso}{%
\subsection{Cálculo analítico paso a
paso}\label{cuxe1lculo-analuxedtico-paso-a-paso}}

\[
\mathbb{E}[U]=\int_{0}^{1}u\,du=\frac12.
\]

\[
\mathbb{E}[e^{U}]=\int_{0}^{1}e^{u}\,du=e-1.
\]

\[
\mathbb{E}\!\big[U e^{U}\big]=\int_{0}^{1}u e^{u}\,du
=\big[ue^{u}\big]_{0}^{1}-\int_{0}^{1}e^{u}\,du
=e-(e-1)=1.
\]

Por tanto,

\[
\boxed{\;\operatorname{Cov}(U,e^{U})=1-\tfrac12(e-1)=\tfrac{3-e}{2}\;\approx\;0.140859086\;}
\]

\begin{center}\rule{0.5\linewidth}{0.5pt}\end{center}

\hypertarget{estimaciuxf3n-monte-carlo}{%
\subsection{Estimación Monte Carlo}\label{estimaciuxf3n-monte-carlo}}

Muestree \(u_1,\dots,u_K\stackrel{iid}{\sim}\mathrm{Unif}(0,1)\). Defina

\[
\widehat{\mu}_U=\frac{1}{K}\sum_{i=1}^{K} u_i,\qquad
\widehat{\mu}_{e}=\frac{1}{K}\sum_{i=1}^{K} e^{u_i},\qquad
\widehat{m}=\frac{1}{K}\sum_{i=1}^{K} u_i e^{u_i}.
\]

Estimador por identidad de momentos:

\[
\boxed{\;\widehat{\operatorname{Cov}}^{(MC)}=\widehat{m}-\widehat{\mu}_U\,\widehat{\mu}_{e}\;}
\]

(Consistente cuando \(K\to\infty\).)

Opcional, versión insesgada muestral:

\[
\boxed{\;\widehat{\operatorname{Cov}}^{(\text{unb})}
=\frac{1}{K-1}\sum_{i=1}^{K}(u_i-\bar u)\big(e^{u_i}-\overline{e^{u}}\big),\quad
\bar u=\widehat{\mu}_U,\ \overline{e^{u}}=\widehat{\mu}_{e}\;}
\]

Error Monte Carlo aproximado para
\(\widehat{\operatorname{Cov}}^{(MC)}\):

\[
\operatorname{se}\big(\widehat{\operatorname{Cov}}^{(MC)}\big)\approx
\frac{s}{\sqrt{K}},
\]

donde \(s^2\) es la varianza empírica de
\(u_i e^{u_i}-\widehat{\mu}_U e^{u_i}-\widehat{\mu}_{e} u_i+\widehat{\mu}_U\widehat{\mu}_{e}\).

    \begin{tcolorbox}[breakable, size=fbox, boxrule=1pt, pad at break*=1mm,colback=cellbackground, colframe=cellborder]
\prompt{In}{incolor}{47}{\boxspacing}
\begin{Verbatim}[commandchars=\\\{\}]
\PY{k}{def}\PY{+w}{ }\PY{n+nf}{valor\PYZus{}esperado\PYZus{}1}\PY{p}{(}\PY{n}{u}\PY{p}{)}\PY{p}{:}
    \PY{k}{return} \PY{n}{u} \PY{o}{*} \PY{n}{np}\PY{o}{.}\PY{n}{exp}\PY{p}{(}\PY{n}{u}\PY{p}{)}

\PY{k}{def}\PY{+w}{ }\PY{n+nf}{valor\PYZus{}esperado\PYZus{}2}\PY{p}{(}\PY{n}{u}\PY{p}{)}\PY{p}{:}
    \PY{k}{return} \PY{n}{u}

\PY{k}{def}\PY{+w}{ }\PY{n+nf}{valor\PYZus{}esperado\PYZus{}3}\PY{p}{(}\PY{n}{u}\PY{p}{)}\PY{p}{:}
    \PY{k}{return} \PY{n}{np}\PY{o}{.}\PY{n}{exp}\PY{p}{(}\PY{n}{u}\PY{p}{)}

\PY{n}{k} \PY{o}{=} \PY{l+m+mi}{1000000}

\PY{n}{u} \PY{o}{=} \PY{n}{np}\PY{o}{.}\PY{n}{random}\PY{o}{.}\PY{n}{random}\PY{p}{(}\PY{n}{k}\PY{p}{)}

\PY{n}{valor\PYZus{}esperado\PYZus{}1}\PY{p}{(}\PY{n}{u}\PY{p}{)}\PY{o}{.}\PY{n}{mean}\PY{p}{(}\PY{p}{)} \PY{o}{\PYZhy{}} \PY{n}{valor\PYZus{}esperado\PYZus{}2}\PY{p}{(}\PY{n}{u}\PY{p}{)}\PY{o}{.}\PY{n}{mean}\PY{p}{(}\PY{p}{)} \PY{o}{*} \PY{n}{valor\PYZus{}esperado\PYZus{}3}\PY{p}{(}\PY{n}{u}\PY{p}{)}\PY{o}{.}\PY{n}{mean}\PY{p}{(}\PY{p}{)}
\end{Verbatim}
\end{tcolorbox}

            \begin{tcolorbox}[breakable, size=fbox, boxrule=.5pt, pad at break*=1mm, opacityfill=0]
\prompt{Out}{outcolor}{47}{\boxspacing}
\begin{Verbatim}[commandchars=\\\{\}]
np.float64(0.1409069042870994)
\end{Verbatim}
\end{tcolorbox}
        
    \begin{tcolorbox}[breakable, size=fbox, boxrule=1pt, pad at break*=1mm,colback=cellbackground, colframe=cellborder]
\prompt{In}{incolor}{48}{\boxspacing}
\begin{Verbatim}[commandchars=\\\{\}]
\PY{l+m+mi}{1} \PY{o}{\PYZhy{}} \PY{l+m+mi}{1}\PY{o}{/}\PY{l+m+mi}{2}\PY{o}{*}\PY{p}{(}\PY{n}{np}\PY{o}{.}\PY{n}{e} \PY{o}{\PYZhy{}}\PY{l+m+mi}{1}\PY{p}{)}
\end{Verbatim}
\end{tcolorbox}

            \begin{tcolorbox}[breakable, size=fbox, boxrule=.5pt, pad at break*=1mm, opacityfill=0]
\prompt{Out}{outcolor}{48}{\boxspacing}
\begin{Verbatim}[commandchars=\\\{\}]
0.14085908577047745
\end{Verbatim}
\end{tcolorbox}
        
    \hypertarget{ejercicio}{%
\section{Ejercicio}\label{ejercicio}}

Para variables aleatorias uniformes \(U_1, U_2, \ldots\) definir
\[N = \min\left\{ n : \sum_{i=1}^{n} U_i > 1 \right\}.\] Estimar
\(\mathbb E[N]\) por simulación con: a) 100 valores, b) 1000 valores, c)
10000 valores, d) Discutir el valor esperado.

    \begin{verbatim}
PSEUDOCÓDIGO — MINIMO_N(k)
total_contadores ← 0
PARA i ← 1 HASTA k HACER:
    suma ← 0
    contador ← 0
    MIENTRAS suma < 1 HACER:
        contador ← contador + 1
        suma ← UNIFORME(0,1)
    total_contadores ← total_contadores + contador
RETORNAR total_contadores / k
\end{verbatim}

    \begin{tcolorbox}[breakable, size=fbox, boxrule=1pt, pad at break*=1mm,colback=cellbackground, colframe=cellborder]
\prompt{In}{incolor}{ }{\boxspacing}
\begin{Verbatim}[commandchars=\\\{\}]
\PY{k}{def}\PY{+w}{ }\PY{n+nf}{minimo\PYZus{}N}\PY{p}{(}\PY{n}{k}\PY{p}{)}\PY{p}{:}
    \PY{n}{lista\PYZus{}contadores} \PY{o}{=} \PY{p}{[}\PY{p}{]}
    \PY{k}{for} \PY{n}{\PYZus{}} \PY{o+ow}{in} \PY{n+nb}{range}\PY{p}{(}\PY{n}{k}\PY{p}{)}\PY{p}{:}
        \PY{n}{suma} \PY{o}{=} \PY{l+m+mi}{0}
        \PY{n}{contador} \PY{o}{=} \PY{l+m+mi}{0}
        \PY{k}{while} \PY{n}{suma} \PY{o}{\PYZlt{}} \PY{l+m+mi}{1}\PY{p}{:}
            \PY{n}{contador} \PY{o}{+}\PY{o}{=} \PY{l+m+mi}{1}
            \PY{n}{suma} \PY{o}{+}\PY{o}{=} \PY{n}{np}\PY{o}{.}\PY{n}{random}\PY{o}{.}\PY{n}{random}\PY{p}{(}\PY{p}{)}
        \PY{n}{lista\PYZus{}contadores}\PY{o}{.}\PY{n}{append}\PY{p}{(}\PY{n}{contador}\PY{p}{)}
    \PY{k}{return} \PY{n}{lista\PYZus{}contadores}
\end{Verbatim}
\end{tcolorbox}

    \begin{tcolorbox}[breakable, size=fbox, boxrule=1pt, pad at break*=1mm,colback=cellbackground, colframe=cellborder]
\prompt{In}{incolor}{89}{\boxspacing}
\begin{Verbatim}[commandchars=\\\{\}]
\PY{n}{lista\PYZus{}para\PYZus{}dist} \PY{o}{=} \PY{n}{minimo\PYZus{}N}\PY{p}{(}\PY{l+m+mi}{1000000}\PY{p}{)}
\end{Verbatim}
\end{tcolorbox}

    \begin{tcolorbox}[breakable, size=fbox, boxrule=1pt, pad at break*=1mm,colback=cellbackground, colframe=cellborder]
\prompt{In}{incolor}{90}{\boxspacing}
\begin{Verbatim}[commandchars=\\\{\}]
\PY{k+kn}{import}\PY{+w}{ }\PY{n+nn}{matplotlib}\PY{n+nn}{.}\PY{n+nn}{pyplot}\PY{+w}{ }\PY{k}{as}\PY{+w}{ }\PY{n+nn}{plt}
\PY{n}{plt}\PY{o}{.}\PY{n}{hist}\PY{p}{(}\PY{n}{lista\PYZus{}para\PYZus{}dist}\PY{p}{,} \PY{n}{bins}\PY{o}{=}\PY{l+m+mi}{30}\PY{p}{,} \PY{n}{density}\PY{o}{=}\PY{k+kc}{True}\PY{p}{)}   
\PY{n}{plt}\PY{o}{.}\PY{n}{show}\PY{p}{(}\PY{p}{)}
\end{Verbatim}
\end{tcolorbox}

    \begin{center}
    \adjustimage{max size={0.9\linewidth}{0.9\paperheight}}{notebook_files/notebook_27_0.png}
    \end{center}
    { \hspace*{\fill} \\}
    
    \begin{tcolorbox}[breakable, size=fbox, boxrule=1pt, pad at break*=1mm,colback=cellbackground, colframe=cellborder]
\prompt{In}{incolor}{80}{\boxspacing}
\begin{Verbatim}[commandchars=\\\{\}]
\PY{n}{minimo\PYZus{}N}\PY{p}{(}\PY{l+m+mi}{1\PYZus{}000}\PY{p}{)}
\end{Verbatim}
\end{tcolorbox}

            \begin{tcolorbox}[breakable, size=fbox, boxrule=.5pt, pad at break*=1mm, opacityfill=0]
\prompt{Out}{outcolor}{80}{\boxspacing}
\begin{Verbatim}[commandchars=\\\{\}]
np.float64(2.725)
\end{Verbatim}
\end{tcolorbox}
        
    \begin{tcolorbox}[breakable, size=fbox, boxrule=1pt, pad at break*=1mm,colback=cellbackground, colframe=cellborder]
\prompt{In}{incolor}{81}{\boxspacing}
\begin{Verbatim}[commandchars=\\\{\}]
\PY{n}{minimo\PYZus{}N}\PY{p}{(}\PY{l+m+mi}{10\PYZus{}000}\PY{p}{)}
\end{Verbatim}
\end{tcolorbox}

            \begin{tcolorbox}[breakable, size=fbox, boxrule=.5pt, pad at break*=1mm, opacityfill=0]
\prompt{Out}{outcolor}{81}{\boxspacing}
\begin{Verbatim}[commandchars=\\\{\}]
np.float64(2.7227)
\end{Verbatim}
\end{tcolorbox}
        

    % Add a bibliography block to the postdoc
    
    
    
\end{document}
