\documentclass[11pt]{article}
\usepackage[margin=.1in]{geometry}
\usepackage{amsmath,amssymb,mathtools,bm}
\usepackage{tcolorbox}
\tcbuselibrary{breakable,skins}
\usepackage{ocgx2}
\usepackage{hyperref}
\tcbset{colback=white,colframe=black!15,boxrule=0.5pt,arc=2pt,enhanced,breakable}

\newcommand{\E}{\mathbb{E}}
\newcommand{\Var}{\operatorname{Var}}
\newcommand{\ind}{\mathbf{1}}
\newcommand{\R}{\mathbb{R}}
\newcommand{\N}{\mathbb{N}}
\newcommand{\PhiStd}{\Phi}

\begin{document}

  ===================== BLOQUE 1 =====================
\begin{tcolorbox}
\[
\boxed{\text{Método clásico (1D)}}
\]
\[
0 \le g(x) \le c,\quad x\in(a,b),\qquad
\theta \coloneqq \int_a^b g(x)\,dx
\]
\[
(X,Y)\sim \mathrm{Unif}(a,b)\times \mathrm{Unif}(0,c),\qquad
p \coloneqq \mathbb{P}\!\big(Y\le g(X)\big)
\]
\[
p=\dfrac{1}{c(b-a)}\int_a^b g(x)\,dx=\dfrac{\theta}{c(b-a)}
\quad\Longrightarrow\quad
\theta=c(b-a)\,p
\]
\[
\{(X_i,Y_i)\}_{i=1}^n\ \text{i.i.d.},\quad
N\coloneqq \sum_{i=1}^n \mathbf{1}\{Y_i\le g(X_i)\}\sim \mathrm{Bin}(n,p)
\]
\[
\widehat{p}=\frac{N}{n},
\qquad
\boxed{\ \widehat{\theta}=c(b-a)\,\widehat{p}=c(b-a)\,\frac{N}{n}\ }
\]
\end{tcolorbox}

\begin{tcolorbox}
\[
\boxed{\text{Propiedades}}
\]
\[
\mathbb{E}[\widehat{\theta}]=\theta
\]
\[
\mathrm{Var}(\widehat{\theta})
=\mathrm{Var}\!\Big(c(b-a)\tfrac{N}{n}\Big)
=c^2(b-a)^2\,\frac{p(1-p)}{n}
=\frac{1}{n}\,\theta\big(c(b-a)-\theta\big)
\]
\[
\widehat{\theta}\xrightarrow[]{\;\mathbb{P}\;}\theta
\quad\text{y}\quad
\frac{\widehat{\theta}-\theta}{\sqrt{\mathrm{Var}(\widehat{\theta})}}
\ \approx\ \mathcal{N}(0,1)
\]
\end{tcolorbox}

\begin{tcolorbox}
\[
\boxed{\text{Tamaño de muestra (cota tipo Chebyshev)}}
\]
\[
\forall\,\varepsilon>0,\ 0<\alpha<1:\quad
n\ \ge\ \frac{c^2(b-a)^2}{4\,\alpha\,\varepsilon^2}
\ \Longrightarrow\
\mathbb{P}\big(|\widehat{\theta}-\theta|<\varepsilon\big)\ \ge\ 1-\alpha
\]
\end{tcolorbox}

\begin{tcolorbox}
\[
\boxed{\text{Intervalos de confianza}}
\]
(1) Forma con parámetro $\varepsilon$:
\[
n\ \ge\ \frac{c^2(b-a)^2}{2\,\alpha\,\varepsilon^2}
\ \Longrightarrow\
\mathbb{P}\big(\widehat{\theta}-\varepsilon<\theta<\widehat{\theta}+\varepsilon\big)\ \ge\ 1-\alpha
\]
\[
\varepsilon^2=\frac{c^2(b-a)^2}{2\alpha n}
\ \Longrightarrow\
\mathbb{P}\!\left(
\widehat{\theta}-\frac{1}{\sqrt{\alpha}}\frac{c(b-a)}{\sqrt{2n}}
<\theta<
\widehat{\theta}+\frac{1}{\sqrt{\alpha}}\frac{c(b-a)}{\sqrt{2n}}
\right)\ \ge\ 1-\alpha
\]
\text{(2) Aproximación normal asintótica:}
\[
\mathbb{P}\!\left(
\widehat{\theta}-z_{\alpha/2}\sqrt{\mathrm{Var}(\widehat{\theta})}
<\theta<
\widehat{\theta}+z_{\alpha/2}\sqrt{\mathrm{Var}(\widehat{\theta})}
\right)\ \approx\ 1-\alpha
\]
$\text{(3) Cota ampliada con } \mathrm{Var}(\widehat{\theta})\le \dfrac{c^2(b-a)^2}{2n}:$
\[
\mathbb{P}\!\left(
\widehat{\theta}-z_{\alpha/2}\frac{c(b-a)}{\sqrt{2n}}
<\theta<
\widehat{\theta}+z_{\alpha/2}\frac{c(b-a)}{\sqrt{2n}}
\right)\ \approx\ 1-\alpha
\]
\end{tcolorbox}

\begin{tcolorbox}
\[
\boxed{\text{Caso multidimensional (2D)}}
\]
\[
0\le g(x,y)\le c,\quad (x,y)\in(a_1,b_1)\times(a_2,b_2)
\]
\[
\theta\coloneqq \int_{a_1}^{b_1}\int_{a_2}^{b_2} g(x,y)\,dy\,dx
\]
\[
(X,Y,Z)\sim \mathrm{Unif}(a_1,b_1)\times \mathrm{Unif}(a_2,b_2)\times \mathrm{Unif}(0,c),
\quad
p=\mathbb{P}\big(Z\le g(X,Y)\big)=\frac{\theta}{c(b_1-a_1)(b_2-a_2)}
\]
\[
N=\sum_{i=1}^n \mathbf{1}\{Z_i\le g(X_i,Y_i)\}\sim \mathrm{Bin}(n,p),
\qquad
\boxed{\ \widehat{\theta}=c(b_1-a_1)(b_2-a_2)\,\frac{N}{n}\ }
\]
\end{tcolorbox}

\begin{tcolorbox}
\[
\boxed{\text{Algoritmo en símbolos}}
\]
\[
X_i\sim \mathrm{Unif}(a,b),\quad
Y_i\sim \mathrm{Unif}(0,c),\quad
I_i=\mathbf{1}\{Y_i\le g(X_i)\},\quad
N=\sum_{i=1}^n I_i,\quad
\widehat{\theta}=c(b-a)\frac{N}{n}
\]
\end{tcolorbox}

\begin{tcolorbox}
\[
\boxed{\text{Ejercicio con cómputo a mano}}
\]
\[
g(x)=\sin x+1,\quad a=0,\ b=\pi,\ c=2
\]
\[
\theta=\int_0^{\pi}(\sin x+1)\,dx=\Big[-\cos x\Big]_0^{\pi}+\Big[x\Big]_0^{\pi}=2+\pi
\]
\[
p=\frac{\theta}{c(b-a)}=\frac{\pi+2}{2\pi}
\]
$\text{Escenario muestral con } n=1000,\ N=818:\ (\text{aprox. } \widehat{p}=0.818)$
\[
\widehat{\theta}=c(b-a)\frac{N}{n}=2\pi\cdot 0.818
\]
\[
\widehat{\theta}\approx 2\pi\cdot 0.818\approx 5.14
\quad\text{y}\quad
\theta=\pi+2\approx 5.14
\]
\[
\mathrm{Var}(\widehat{\theta})
=\frac{1}{n}\,\theta\big(c(b-a)-\theta\big)
=\frac{1}{1000}\,(\pi+2)\big(2\pi-(\pi+2)\big)
\]
\[
\text{IC}_{95\ }\ \text{(aprox. normal)}:\ 
\widehat{\theta}\ \pm\ 1.96\,\sqrt{\mathrm{Var}(\widehat{\theta})}
\]
\end{tcolorbox}

  ===================== BLOQUE 2 (OCGX2) =====================
\begin{tcolorbox}
\[
\boxed{\mathrm{General}}
\]
\begin{ocg}{General}{ocgG}{1}
\[
\theta=\int_a^b g(x)\,dx,\qquad
\theta=\int_a^b \frac{g(x)}{f(x)}\,f(x)\,dx=\mathbb{E}\!\left[\frac{g(X)}{f(X)}\right].
\]
\[
\hat\theta=\frac{1}{n}\sum_{i=1}^{n}\frac{g(X_i)}{f(X_i)},\qquad
\mathbb{E}[\hat\theta]=\theta.
\]
\end{ocg}
\end{tcolorbox}

\begin{tcolorbox}
\[
\boxed{\mathrm{Uniforme}}
\]
\begin{ocg}{Uniforme}{ocgU}{1}
\[
X\sim \mathrm{Unif}(a,b),\quad f(x)=\frac{1}{b-a}.
\]
\[
\theta=(b-a)\,\mathbb{E}[g(X)],\qquad
\hat\theta_{\mathrm{u}}=(b-a)\,\frac{1}{n}\sum_{i=1}^{n} g(X_i).
\]
\[
\mathrm{Var}(\hat\theta_{\mathrm{u}})=\frac{1}{n}\!\left[(b-a)\int_a^b g^2(x)\,dx-\theta^{2}\right].
\]
\[
\mathrm{Var}(\hat\theta_{\mathrm{u}})\le \mathrm{Var}(\hat\theta_{\mathrm{cl}}).
\]
\end{ocg}
\end{tcolorbox}

\begin{tcolorbox}
\[
\boxed{\mathrm{Exponencial}}
\]
\begin{ocg}{Exponencial}{ocgE}{1}
\[
X\sim \mathrm{Exp}(\lambda),\quad f(x)=\lambda e^{-\lambda x},\quad \lambda=1.
\]
\[
\theta=\mathbb{E}\!\left[e^{X} \, g(X)\right],\qquad
\hat\theta_{\mathrm{exp}}=\frac{1}{n}\sum_{i=1}^{n} e^{x_i}\,g(x_i).
\]
\end{ocg}
\end{tcolorbox}

\begin{tcolorbox}
\[
\boxed{\mathrm{Normal}}
\]
\begin{ocg}{Normal}{ocgN}{1}
\[
\varphi(x)=\frac{1}{\sqrt{2\pi}}e^{-x^{2}/2},\qquad X\sim \mathcal{N}(0,1).
\]
\[
\theta=\sqrt{2\pi}\,\mathbb{E}\!\left[e^{X^{2}/2}\,g(X)\right],\qquad
\hat\theta_{\mathrm{norm}}=\sqrt{2\pi}\,\frac{1}{n}\sum_{i=1}^{n} e^{x_i^{2}/2}\,g(x_i).
\]
\end{ocg}
\end{tcolorbox}

\begin{tcolorbox}
\[
\boxed{\mathrm{Ejercicio}}
\]
\begin{ocg}{Ejercicio}{ocgEx}{1}
\[
g(x)=x,\quad a=0,\ b=1,\quad x_1=\tfrac{1}{4},\ x_2=\tfrac{1}{2},\ x_3=\tfrac{3}{4},\ n=3.
\]
\[
\theta=\int_{0}^{1} x\,dx=\tfrac{1}{2}.
\]
\[
\hat\theta_{\mathrm{u}}=(b-a)\frac{1}{n}\sum_{i=1}^{n}g(x_i)
=1\cdot\frac{1}{3}\!\left(\tfrac{1}{4}+\tfrac{1}{2}+\tfrac{3}{4}\right)
=\frac{1}{3}\cdot\frac{3}{2}=\tfrac{1}{2}.
\]
\[
\bigl|\hat\theta_{\mathrm{u}}-\theta\bigr|=0.
\]
\end{ocg}
\end{tcolorbox}

  ===================== BLOQUE 3: MUESTREO CONDICIONAL =====================

  \begin{tcolorbox}
\[
\boxed{\text{Varianza y esperanza condicional}}
\]
\textbf{Definición.} Sea $X$ con $\E[X^2]<\infty$ y $Y$ una v.a.\ cualquiera. Se define
\[
\Var(X\mid Y)\;=\;\E\!\big[(X-\E[X\mid Y])^{2}\,\big|\,Y\big].
\]

\textbf{Propiedades.}
\[
\Var(X\mid Y)\ \ge\ 0\quad \text{c.s.}
\]
\[
\Var(X\mid Y)\;=\;\E\!\left[X^{2}\mid Y\right]\;-\;\Big(\E\!\left[X\mid Y\right]\Big)^{2}.
\]
\[
\boxed{\ \Var(X)\;=\;\E\!\left[\Var(X\mid Y)\right]\;+\;\Var\!\left(\E[X\mid Y]\right)\ }\quad
\text{(ley de la varianza total).}
\]
\[
\Var\!\left(\E[X\mid Y]\right)\ \le\ \Var(X).
\]
\end{tcolorbox}

\begin{tcolorbox}
  ========= MU ESTREO CONDICIONAL: FORMULAS =========
  Marco básico
\[
\theta=\int_a^b g(x)\,dx=\mathbb{E}\!\left[\frac{g(X)}{f(X)}\right],\qquad
X\sim f
\]
\[
\hat\theta_{1}=\frac{1}{n}\sum_{i=1}^{n}\frac{g(X_i)}{f(X_i)},\qquad
\hat\theta_{2}=\frac{1}{n}\sum_{i=1}^{n}\mathbb{E}\!\left[\frac{g(X)}{f(X)}\mid Y_i\right]
\]
\[
\mathbb{E}[\hat\theta_{1}]=\mathbb{E}[\hat\theta_{2}]=\theta
\]
\[
\operatorname{Var}(\hat\theta_{1})=\frac{1}{n}\operatorname{Var}\!\left(\frac{g(X)}{f(X)}\right),
\quad
\operatorname{Var}(\hat\theta_{2})=\frac{1}{n}\operatorname{Var}\!\left(\mathbb{E}\!\left[\frac{g(X)}{f(X)}\mid Y\right]\right)
\]
\[
\operatorname{Var}(X)=\mathbb{E}\!\left[\operatorname{Var}(X\mid Y)\right]+\operatorname{Var}\!\left(\mathbb{E}[X\mid Y]\right)
\]
\[
\operatorname{Var}(\hat\theta_{2})\le \operatorname{Var}(\hat\theta_{1})
\]
\end{tcolorbox}

\begin{tcolorbox}
  ========= EJEMPLO 1 =========
  $\theta = \iint (x+y)$
\[
\theta=\int_{0}^{1}\!\!\int_{0}^{1}(x+y)\,dx\,dy=1
\]
\[
\hat\theta_{1}=\frac{1}{n}\sum_{i=1}^{n}(X_i+Y_i),\quad (X_i,Y_i)\stackrel{\text{i.i.d.}}{\sim}U([0,1]^2)
\]
\[
\mathbb{E}[X\mid Y=y]=\tfrac12\;\Rightarrow\;
\hat\theta_{2}=\frac{1}{n}\sum_{i=1}^{n}\!\left(\tfrac12+Y_i\right),\quad Y_i\stackrel{\text{i.i.d.}}{\sim}U(0,1)
\]
\end{tcolorbox}

\begin{tcolorbox}
  ========= EJEMPLO 2 =========
  $\theta = P(X+Y \le u)$
\[
\theta=\mathbb{P}(X+Y\le u)=\mathbb{E}\!\left[\mathbf{1}_{\{X+Y\le u\}}\right]
\]
\[
\hat\theta_{1}=\frac{1}{n}\sum_{i=1}^{n}\mathbf{1}_{\{X_i+Y_i\le u\}},\quad (X_i,Y_i)\stackrel{\text{i.i.d.}}{\sim}F
\]
\[
\theta=\mathbb{E}\!\left[\mathbb{E}\!\left(\mathbf{1}_{\{X\le u-Y\}}\mid Y\right)\right]
=\mathbb{E}\!\left[F_X(u-Y)\right]
\]
\[
\hat\theta_{2}=\frac{1}{n}\sum_{i=1}^{n}F_X(u-Y_i),\quad Y_i\stackrel{\text{i.d.}}{\sim}F_Y
\]
\end{tcolorbox}

  ========= EJEMPLO 3 =========
  Suma compuesta SN
\begin{tcolorbox}

\[
S_N=\sum_{i=1}^{N}X_i,\quad N\perp\!\!\!\perp \{X_i\},\quad X_i\stackrel{\text{i.d.}}{\sim}F
\]
\[
\theta=\mathbb{P}(S_N\le x)=\mathbb{E}\!\left[\mathbf{1}_{\{S_N\le x\}}\right]
\]
\[
\hat\theta_{1}=\frac{1}{n}\sum_{k=1}^{n}\mathbf{1}_{\{S^{(k)}_{N_k}\le x\}},
\qquad
\operatorname{Var}(\hat\theta_{1})=\frac{1}{n}\theta(1-\theta)
\]
\[
F^{*n}(x)=\mathbb{P}\!\left(\sum_{i=1}^{n}X_i\le x\right)=
\mathbb{E}\!\left[F\!\left(x-\sum_{i=2}^{n}X_i\right)\right]
\]
\[
F^{*N}(x)=\mathbb{E}\!\left[F\!\left(x-\sum_{i=2}^{N}X_i\right)\right]
\]
\[
\hat\theta_{2}=\frac{1}{n}\sum_{k=1}^{n}F\!\left(x-\sum_{i=2}^{N_k}X^{(k)}_i\right)
\]
\end{tcolorbox}

\begin{tcolorbox}
  ========= EJERCICIO (CÁLCULO A MANO) =========
  Uniformes y u=1
\[
X,Y\stackrel{\text{i.i.d.}}{\sim}U(0,1),\quad u=1
\]
\[
\theta=\mathbb{P}(X+Y\le 1)=\int_{0}^{1}\!\!\int_{0}^{1}\mathbf{1}_{\{x+y\le 1\}}\,dx\,dy
=\int_{0}^{1}\!\left(\int_{0}^{1-y}dx\right)dy
=\int_{0}^{1}(1-y)\,dy=\tfrac{1}{2}
\]
\[
F_X(1-Y)=(1-Y)\,\mathbf{1}_{\{0\le Y\le 1\}}
\]
\[
\theta=\mathbb{E}[F_X(1-Y)]=\mathbb{E}[1-Y]=1-\mathbb{E}[Y]=1-\tfrac12=\tfrac12
\]
\end{tcolorbox}

  ===================== BLOQUE 4: MUESTREO POR IMPORTANCIA =====================
\begin{tcolorbox}
=== Muestreo por Importancia (resumen solo fórmulas) ===

 Objetivo
\[
\boxed{\;\theta=\int_a^b g(x)\,dx\;}
\qquad
\boxed{\;\theta=\int_a^b \frac{g(x)}{f(x)}\,f(x)\,dx=\mathbb{E}\!\left[\frac{g(X)}{f(X)}\right]\;}
\]

Estimador Monte Carlo
\[
\boxed{\;\hat\theta=\frac{1}{n}\sum_{i=1}^n \frac{g(X_i)}{f(X_i)}\;}
\qquad
\boxed{\;\mathbb{E}[\hat\theta]=\theta\;}
\qquad
\boxed{\;\mathrm{Var}(\hat\theta)=\frac{1}{n}\,\mathrm{Var}\!\left(\frac{g(X)}{f(X)}\right)\;}
\]

Cota por Cauchy–Schwarz y f óptima
\[
\boxed{\;\mathrm{Var}(\hat\theta)\;\ge\;\frac{1}{n}\Bigg[\!\left(\int_a^b |g(x)|\,dx\right)^{\!2}-\theta^2\Bigg]\;}
\qquad
\boxed{\;f^\star(x)=\dfrac{|g(x)|}{\int_a^b |g(t)|\,dt}\;}
\qquad
\boxed{\;\mathrm{Var}(\hat\theta;f^\star)=\frac{1}{n}\Bigg[\!\left(\int_a^b |g(x)|\,dx\right)^{\!2}-\theta^2\Bigg]\;}
\]

Requisito de integrabilidad
\[
\boxed{\;\int_a^b |g(x)|\,dx<\infty\;}
\]

Aproximación por partición (discreta por tramos)
$\text{Partición: } a=u_0<u_1<\cdots<u_m=b,\;\; \Delta_i=u_i-u_{i-1},\;\; x_i\in(u_{i-1},u_i)
\qquad
\bar g(x)=\sum_{i=1}^m g_i\,\mathbf{1}_{(u_{i-1},u_i)}(x),\;\; g_i:=g(x_i)$
\[
\boxed{\;p_i=\mathbb{P}(X=x_i)=\dfrac{|g_i|\Delta_i}{\sum_{j=1}^m |g_j|\Delta_j}\;}
\qquad
\boxed{\;\theta=\int_a^b g(x)\,dx\;\approx\;\mathbb{E}\!\left[\frac{g(X)\,\Delta(X)}{p(X)}\right]\;}
\]
\[
\boxed{\;\hat\theta_{\!\Delta}=\frac{1}{n}\sum_{k=1}^n \frac{g(X_k)\,\Delta(X_k)}{p(X_k)}\;}
\]

Caso multidimensional
\[
\boxed{\;\theta=\int_{a_1}^{b_1}\!\!\cdots\!\int_{a_d}^{b_d} g(\mathbf{x})\,d\mathbf{x}
=\mathbb{E}\!\left[\frac{g(\mathbf{X})}{f(\mathbf{X})}\right]\;}
\qquad
\boxed{\;f^\star(\mathbf{x})=\dfrac{|g(\mathbf{x})|}{\int |g(\mathbf{t})|\,d\mathbf{t}}\;}
\qquad
\boxed{\;\hat\theta=\frac{1}{n}\sum_{i=1}^n \frac{g(\mathbf{X}_i)}{f(\mathbf{X}_i)}\;}
\]

  ===== Ejercicio simple con cálculos a mano =====
  Integral objetivo y partición m=2
$\text{Ejercicio: } \theta=\int_0^1 e^{-x}\,dx,\;\; m=2,\;\; u_0=0,\;u_1=\tfrac{1}{2},\;u_2=1,\;\; \Delta_1=\Delta_2=\tfrac{1}{2}
\qquad
x_1=\tfrac{1}{4},\;x_2=\tfrac{3}{4},\;\; g_1=e^{-1/4},\; g_2=e^{-3/4}$
\[
p_1=\dfrac{|g_1|\Delta_1}{|g_1|\Delta_1+|g_2|\Delta_2}
=\dfrac{e^{-1/4}\cdot \tfrac{1}{2}}{e^{-1/4}\cdot \tfrac{1}{2}+e^{-3/4}\cdot \tfrac{1}{2}}
=\dfrac{e^{-1/4}}{e^{-1/4}+e^{-3/4}},
\qquad
p_2=\dfrac{e^{-3/4}}{e^{-1/4}+e^{-3/4}}
\]
\[
w_1=\frac{g_1\Delta_1}{p_1}
=\frac{e^{-1/4}\cdot \tfrac{1}{2}}{e^{-1/4}/(e^{-1/4}+e^{-3/4})}
=\tfrac{1}{2}\big(e^{-1/4}+e^{-3/4}\big),
\qquad
w_2=\frac{g_2\Delta_2}{p_2}
=\frac{e^{-3/4}\cdot \tfrac{1}{2}}{e^{-3/4}/(e^{-1/4}+e^{-3/4})}
=\tfrac{1}{2}\big(e^{-1/4}+e^{-3/4}\big)
\]
\[
\boxed{\;\hat\theta_{\!\Delta}=\frac{1}{n}\sum_{k=1}^n w_{I_k}=\tfrac{1}{2}\big(e^{-1/4}+e^{-3/4}\big)\;}
\qquad
e^{-1/4}\approx 0.77880078,\;\; e^{-3/4}\approx 0.47236655
\qquad
\hat\theta_{\!\Delta}\approx \tfrac{1}{2}(0.77880078+0.47236655)\approx 0.62558366
\]
\[
\boxed{\;\theta_{\text{exacta}}=1-e^{-1}\approx 1-0.36787944\approx 0.63212056\;}
\qquad
|\hat\theta_{\!\Delta}-\theta_{\text{exacta}}|\approx 0.00653690
\]

  ===== Recap simbólico =====
\[
\boxed{\;\text{IS:}\;\; \hat\theta=\frac{1}{n}\sum_{i=1}^n \frac{g(X_i)}{f(X_i)},\quad
f^\star\propto |g|,\quad
\mathrm{Var}(\hat\theta)\ge \frac{1}{n}\Big[(\int |g|)^2-\theta^2\Big]\;}
\]
\end{tcolorbox}

  ===================== BLOQUE 5: VARIABLES COMUNES, ANTITÉTICAS Y POISSON =====================
\begin{tcolorbox}
  Resumen en fórmulas: Variables Comunes
\[
\boxed{\text{Variables Comunes}}
\]
\[
\theta=\mathbb{E}(X-Y),\qquad
\hat\theta=\frac{1}{n}\sum_{i=1}^n(X_i-Y_i)
\]
\[
\operatorname{Var}(\hat\theta)=\frac{1}{n}\bigl(\operatorname{Var}(X)+\operatorname{Var}(Y)\bigr)
\]
\[
U\sim\mathrm{Unif}(0,1),\quad X=F^{-1}(U),\quad Y=G^{-1}(U)
\]
\[
\operatorname{Var}(\hat\theta)=\frac{1}{n}\!\left[\operatorname{Var}(X)+\operatorname{Var}(Y)-2\,\operatorname{Cov}\!\bigl(F^{-1}(U),G^{-1}(U)\bigr)\right]
\]
\[
\operatorname{Cov}\!\bigl(F^{-1}(U),G^{-1}(U)\bigr)\ge 0
\]
\end{tcolorbox}

\begin{tcolorbox}
  Variables Antitéticas
\[
\boxed{\text{Variables Antitéticas}}
\]
\[
\theta=\mathbb{E}(X+Y),\qquad
\hat\theta=\frac{1}{n}\sum_{i=1}^n(X_i+Y_i)
\]
\[
\operatorname{Var}(\hat\theta)=\frac{1}{n}\bigl(\operatorname{Var}(X)+\operatorname{Var}(Y)\bigr)
\]
\[
U\sim\mathrm{Unif}(0,1),\quad X=F^{-1}(U),\quad Y=G^{-1}(1-U)
\]
\[
\operatorname{Var}(\hat\theta)=\frac{1}{n}\!\left[\operatorname{Var}(X)+\operatorname{Var}(Y)+2\,\operatorname{Cov}\!\bigl(F^{-1}(U),G^{-1}(1-U)\bigr)\right]
\]
\[
\operatorname{Cov}\!\bigl(F^{-1}(U),G^{-1}(1-U)\bigr)\le 0
\]
\end{tcolorbox}

\begin{tcolorbox}
  Procesos de Poisson
\[
\boxed{\text{Procesos de Poisson}}
\]
\[
T\sim\mathrm{Exp}(\lambda),\qquad F_T^{-1}(U)=-\frac{1}{\lambda}\ln(1-U)\;(=\!-\tfrac{1}{\lambda}\ln U)
\]
\[
\tau_0=0,\quad \tau_n=\sum_{k=1}^{n}T_k,\qquad N(s)=\max\{n:\tau_n\le s\}
\]
\end{tcolorbox}

\begin{tcolorbox}
  Ejercicio (cálculo a mano: comunes)
\[
\boxed{\text{Ejercicio (cálculo a mano: comunes)}}
\]
\[
\mu=0.05,\quad U=0.8,\quad Z=0.5
\]
\[
T=-\frac{1}{\mu}\ln U=-20\ln(0.8)=4.4629
\]
\[
R_A=0.03+0.01Z=0.035,\qquad R_B=0.05+0.01Z=0.055
\]
\[
V_A=\frac{1-e^{-R_AT}}{R_A}=\frac{1-e^{-0.035\cdot 4.4629}}{0.035}
=\frac{1-e^{-0.1562}}{0.035}=4.1318
\]
\[
V_B=\frac{1-e^{-R_BT}}{R_B}=\frac{1-e^{-0.055\cdot 4.4629}}{0.055}
=\frac{1-e^{-0.2455}}{0.055}=3.9573
\]
\[
\hat\delta=V_B-V_A=3.9573-4.1318=-0.1744
\]
\[
\text{(mismo }Z\text{ en }R_A,R_B\Rightarrow \operatorname{Cov}(V_B,V_A)>0\Rightarrow \operatorname{Var}(\hat\delta)\text{ reducida)}
\]
\end{tcolorbox}


  --- Chuleta de distribuciones (f, F, media, var, MGF) ---
\[
\text{Def.: } \PhiStd(z)=\frac{1}{\sqrt{2\pi}}\int_{-\infty}^z e^{-u^2/2}\,du,\quad
B(\alpha,\beta)=\frac{\Gamma(\alpha)\Gamma(\beta)}{\Gamma(\alpha+\beta)},\quad
I_x(\alpha,\beta)=\frac{1}{B(\alpha,\beta)}\int_{0}^{x} u^{\alpha-1}(1-u)^{\beta-1}\,du.
\]

 -----------------------------------------------------------
  Normal
 -----------------------------------------------------------
\[
\textbf{Normal } X\sim \mathcal{N}(\mu,\sigma^2),\ \sigma>0,\ x\in\R:
\]
\[
f(x)=\frac{1}{\sigma\sqrt{2\pi}}\exp\!\left(-\frac{(x-\mu)^2}{2\sigma^2}\right),\quad
F(x)=\PhiStd\!\left(\frac{x-\mu}{\sigma}\right),
\]
\[
\E[X]=\mu,\quad \Var(X)=\sigma^2,\quad
M_X(t)=\exp\!\left(\mu t+\tfrac{1}{2}\sigma^2 t^2\right),\ t\in\R.
\]

 -----------------------------------------------------------
  Exponencial
 -----------------------------------------------------------
\[
\textbf{Exponencial } X\sim \mathrm{Exp}(\lambda),\ \lambda>0,\ x\ge 0:
\]
\[
f(x)=\lambda e^{-\lambda x}\,\ind_{[0,\infty)}(x),\quad
F(x)=\left(1-e^{-\lambda x}\right)\,\ind_{[0,\infty)}(x),
\]
\[
\E[X]=\frac{1}{\lambda},\quad \Var(X)=\frac{1}{\lambda^2},\quad
M_X(t)=\frac{\lambda}{\lambda-t}\ \text{para } t<\lambda.
\]

 -----------------------------------------------------------
  Uniforme
 -----------------------------------------------------------
\[
\textbf{Uniforme } X\sim \mathrm{Unif}(a,b),\ a<b,\ x\in[a,b]:
\]
\[
f(x)=\frac{1}{b-a}\,\ind_{[a,b]}(x),\quad
F(x)=
\begin{cases}
0,& x<a,\\[2pt]
\dfrac{x-a}{b-a},& a\le x\le b,\\[6pt]
1,& x>b,
\end{cases}
\]
\[
\E[X]=\frac{a+b}{2},\quad \Var(X)=\frac{(b-a)^2}{12},\quad
M_X(t)=
\begin{cases}
\dfrac{e^{tb}-e^{ta}}{t(b-a)}, & t\neq 0,\\
1,& t=0.
\end{cases}
\]

 -----------------------------------------------------------
  Bernoulli
 -----------------------------------------------------------
\[
\textbf{Bernoulli } X\sim \mathrm{Bern}(p),\ p\in[0,1],\ x\in\{0,1\}:
\]
\[
\Pr(X=x)=p^x(1-p)^{1-x},\quad
F(x)=
\begin{cases}
0,& x<0,\\
1-p,& 0\le x<1,\\
1,& x\ge 1,
\end{cases}
\]
\[
\E[X]=p,\quad \Var(X)=p(1-p),\quad
M_X(t)=1-p+pe^{t},\ t\in\R.
\]

 -----------------------------------------------------------
  Binomial
 -----------------------------------------------------------
\[
\textbf{Binomial } X\sim \mathrm{Bin}(n,p),\ n\in\N,\ p\in[0,1],\ x\in\{0,\dots,n\}:
\]
\[
\Pr(X=k)=\binom{n}{k}p^k(1-p)^{n-k},\quad
F(k)=\sum_{j=0}^{k}\binom{n}{j}p^j(1-p)^{n-j},
\]
\[
\E[X]=np,\quad \Var(X)=np(1-p),\quad
M_X(t)=\left(1-p+pe^{t}\right)^n,\ t\in\R.
\]

 -----------------------------------------------------------
  Poisson
 -----------------------------------------------------------
\[
\textbf{Poisson } X\sim \mathrm{Pois}(\lambda),\ \lambda>0,\ x\in\{0,1,2,\dots\}:
\]
\[
\Pr(X=k)=e^{-\lambda}\frac{\lambda^k}{k!},\quad
F(k)=e^{-\lambda}\sum_{j=0}^{k}\frac{\lambda^j}{j!},
\]
\[
\E[X]=\lambda,\quad \Var(X)=\lambda,\quad
M_X(t)=\exp\!\big(\lambda(e^{t}-1)\big),\ t\in\R.
\]

 -----------------------------------------------------------
  Beta
 -----------------------------------------------------------
\[
\textbf{Beta } X\sim \mathrm{Beta}(\alpha,\beta),\ \alpha,\beta>0,\ x\in(0,1):
\]
\[
f(x)=\frac{x^{\alpha-1}(1-x)^{\beta-1}}{B(\alpha,\beta)},\quad
F(x)=I_x(\alpha,\beta),
\]
\[
\E[X]=\frac{\alpha}{\alpha+\beta},\quad
\Var(X)=\frac{\alpha\beta}{(\alpha+\beta)^2(\alpha+\beta+1)},
\]
\[
M_X(t)={}_1F_1\!\left(\alpha;\ \alpha+\beta;\ t\right),\ t\in\R.
\]


\end{document}
