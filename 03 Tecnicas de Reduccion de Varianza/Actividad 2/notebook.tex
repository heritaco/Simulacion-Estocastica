% jupyter nbconvert 'notebook.ipynb' --to latex

\documentclass[10pt,a4paper]{article}

    \usepackage[spanish]{babel} % Idioma del documento, fundamental para la correcta separación de sílabas y traducción de términos
    \usepackage[unicode]{hyperref}
    \usepackage[breakable]{tcolorbox}
    \usepackage{parskip}
    \usepackage{graphicx}

    % \usepackage{amssymb}
    \AtBeginDocument{%
      \let\le\leqslant
      \let\ge\geqslant
      \let\leq\leqslant
      \let\geq\geqslant
    }



    \setkeys{Gin}{keepaspectratio}

    \let\Oldincludegraphics\includegraphics
  
    \usepackage{caption}
    \DeclareCaptionFormat{nocaption}{}
    \captionsetup{format=nocaption,aboveskip=0pt,belowskip=0pt}

    \usepackage{float}
    \floatplacement{figure}{H} % forces figures to be placed at the correct location
    \usepackage{xcolor} % Allow colors to be defined
    \usepackage{enumerate} % Needed for markdown enumerations to work
    \usepackage{geometry} % Used to adjust the document margins
    \usepackage{amsmath} % Equations
    \usepackage{amssymb}
    \usepackage{textcomp}
    \AtBeginDocument{%
        \def\PYZsq{\textquotesingle}% Upright quotes in Pygmentized code
    }
    \usepackage{upquote} % Upright quotes for verbatim code
    \usepackage{eurosym} % defines \euro

    \usepackage{iftex}
    \ifPDFTeX
        \usepackage[T1]{fontenc}
        \IfFileExists{alphabeta.sty}{
              \usepackage{alphabeta}
          }{
              \usepackage[mathletters]{ucs}
              \usepackage[utf8x]{inputenc}
          }
    \else
        \usepackage{fontspec}
        \usepackage{unicode-math}
    \fi

    % --- Change the cell font ---
    % \usepackage{fontspec} 
    \newfontface\myfont{Inconsolata}[Contextuals={WordInitial,WordFinal}]{}  

    \usepackage{fancyvrb}
    \fvset{
    commandchars=\\\{\},
    fontsize=\scriptsize,
    formatcom=\myfont
    }

    \usepackage{grffile} 
    \makeatletter 
    \@ifpackagelater{grffile}{2019/11/01}
    {
      % Do nothing on new versions
    }
    {
      \def\Gread@@xetex#1{%
        \IfFileExists{"\Gin@base".bb}%
        {\Gread@eps{\Gin@base.bb}}%
        {\Gread@@xetex@aux#1}%
      }
    }
    \makeatother

    \usepackage[Export]{adjustbox} % Used to constrain images to a maximum size
    \adjustboxset{max size={0.9\linewidth}{0.9\paperheight}}


    
    \usepackage{titling}
    \usepackage{longtable}
    \usepackage{booktabs}
    \usepackage{array}
    \usepackage{calc}
    \usepackage[inline]{enumitem}
    \usepackage[normalem]{ulem}
    \usepackage{soul} 
    \usepackage{mathrsfs}

    
    % Colors for the hyperref package
    \definecolor{urlcolor}{rgb}{0.00,0.00,1.00}
    \definecolor{linkcolor}{rgb}{0.00,0.00,0.00}
    \definecolor{citecolor}{rgb}{0.149,0.498,0.600}

    % ANSI colors
    \definecolor{ansi-black}{HTML}{3E424D}
    \definecolor{ansi-black-intense}{HTML}{282C36}
    \definecolor{ansi-red}{HTML}{E75C58}
    \definecolor{ansi-red-intense}{HTML}{B22B31}
    \definecolor{ansi-green}{HTML}{00A250}
    \definecolor{ansi-green-intense}{HTML}{007427}
    \definecolor{ansi-yellow}{HTML}{DDB62B}
    \definecolor{ansi-yellow-intense}{HTML}{B27D12}
    \definecolor{ansi-blue}{HTML}{208FFB}
    \definecolor{ansi-blue-intense}{HTML}{0065CA}
    \definecolor{ansi-magenta}{HTML}{D160C4}
    \definecolor{ansi-magenta-intense}{HTML}{A03196}
    \definecolor{ansi-cyan}{HTML}{60C6C8}
    \definecolor{ansi-cyan-intense}{HTML}{258F8F}
    \definecolor{ansi-white}{HTML}{C5C1B4}
    \definecolor{ansi-white-intense}{HTML}{A1A6B2}
    \definecolor{ansi-default-inverse-fg}{HTML}{FFFFFF}
    \definecolor{ansi-default-inverse-bg}{HTML}{000000}

    % common color for the border for error outputs.
    \definecolor{outerrorbackground}{HTML}{FFFFFF}


    % commands and environments needed by pandoc snippets
    % extracted from the output of `pandoc -s`
    \providecommand{\tightlist}{%
      \setlength{\itemsep}{0pt}\setlength{\parskip}{0pt}}
    \DefineVerbatimEnvironment{Highlighting}{Verbatim}{
        commandchars=\\\{\}
    }

    \newenvironment{Shaded}{}{}
    \newcommand{\KeywordTok}[1]{\textcolor[rgb]{0.00,0.00,1.00}{\textbf{{#1}}}}
    \newcommand{\DataTypeTok}[1]{\textcolor[rgb]{0.149,0.498,0.600}{{#1}}}
    \newcommand{\DecValTok}[1]{\textcolor[rgb]{0.035,0.525,0.345}{{#1}}}
    \newcommand{\BaseNTok}[1]{\textcolor[rgb]{0.035,0.525,0.345}{{#1}}}
    \newcommand{\FloatTok}[1]{\textcolor[rgb]{0.035,0.525,0.345}{{#1}}}
    \newcommand{\CharTok}[1]{\textcolor[rgb]{0.639,0.082,0.082}{{#1}}}
    \newcommand{\StringTok}[1]{\textcolor[rgb]{0.639,0.082,0.082}{{#1}}}
    \newcommand{\CommentTok}[1]{\textcolor[rgb]{0.000,0.502,0.000}{\textit{{#1}}}}
    \newcommand{\OtherTok}[1]{\textcolor[rgb]{0.00,0.00,1.00}{{#1}}}
    \newcommand{\AlertTok}[1]{\textcolor[rgb]{1.00,0.00,0.00}{\textbf{{#1}}}}
    \newcommand{\FunctionTok}[1]{\textcolor[rgb]{0.475,0.369,0.149}{{#1}}}
    \newcommand{\RegionMarkerTok}[1]{{#1}}
    \newcommand{\ErrorTok}[1]{\textcolor[rgb]{1.00,0.00,0.00}{\textbf{{#1}}}}
    \newcommand{\NormalTok}[1]{{#1}}

    % Additional commands for more recent versions of Pandoc
    \newcommand{\ConstantTok}[1]{\textcolor[rgb]{0.00,0.00,1.00}{{#1}}}
    \newcommand{\SpecialCharTok}[1]{\textcolor[rgb]{0.639,0.082,0.082}{{#1}}}
    \newcommand{\VerbatimStringTok}[1]{\textcolor[rgb]{0.639,0.082,0.082}{{#1}}}
    \newcommand{\SpecialStringTok}[1]{\textcolor[rgb]{0.639,0.082,0.082}{{#1}}}
    \newcommand{\ImportTok}[1]{{#1}}
    \newcommand{\DocumentationTok}[1]{\textcolor[rgb]{0.000,0.502,0.000}{\textit{{#1}}}}
    \newcommand{\AnnotationTok}[1]{\textcolor[rgb]{0.706,0.000,0.620}{\textbf{\textit{{#1}}}}}
    \newcommand{\CommentVarTok}[1]{\textcolor[rgb]{0.000,0.502,0.000}{\textbf{\textit{{#1}}}}}
    \newcommand{\VariableTok}[1]{\textcolor[rgb]{0.000,0.063,0.502}{{#1}}}
    \newcommand{\ControlFlowTok}[1]{\textcolor[rgb]{0.00,0.00,1.00}{\textbf{{#1}}}}
    \newcommand{\OperatorTok}[1]{\textcolor[rgb]{0.000,0.000,0.000}{{#1}}}
    \newcommand{\BuiltInTok}[1]{{#1}}
    \newcommand{\ExtensionTok}[1]{{#1}}
    \newcommand{\PreprocessorTok}[1]{\textcolor[rgb]{0.706,0.000,0.620}{{#1}}}
    \newcommand{\AttributeTok}[1]{\textcolor[rgb]{0.149,0.498,0.600}{{#1}}}
    \newcommand{\InformationTok}[1]{\textcolor[rgb]{0.000,0.502,0.000}{\textbf{\textit{{#1}}}}}
    \newcommand{\WarningTok}[1]{\textcolor[rgb]{0.000,0.502,0.000}{\textbf{\textit{{#1}}}}}
    \makeatletter

    \newsavebox\pandoc@box
    \newcommand*\pandocbounded[1]{%
      \sbox\pandoc@box{#1}%
      % scaling factors for width and height
      \Gscale@div\@tempa\textheight{\dimexpr\ht\pandoc@box+\dp\pandoc@box\relax}%
      \Gscale@div\@tempb\linewidth{\wd\pandoc@box}%
      % select the smaller of both
      \ifdim\@tempb\p@<\@tempa\p@
        \let\@tempa\@tempb
      \fi
      % scaling accordingly (\@tempa < 1)
      \ifdim\@tempa\p@<\p@
        \scalebox{\@tempa}{\usebox\pandoc@box}%
      % scaling not needed, use as it is
      \else
        \usebox{\pandoc@box}%
      \fi
    }
    \makeatother


    \def\br{\hspace*{\fill} \\* }
    % Math Jax compatibility definitions
    \def\gt{>}
    \def\lt{<}
    % define /in, the E element of symbol
    \def\inset{\mathbin{\in}}
    \let\Oldtex\TeX
    \let\Oldlatex\LaTeX
    \renewcommand{\TeX}{\textrm{\Oldtex}}
    \renewcommand{\LaTeX}{\textrm{\Oldlatex}}

    % Pygments definitions
    \makeatletter
    \def\PY@reset{\let\PY@it=\relax \let\PY@bf=\relax%
        \let\PY@ul=\relax \let\PY@tc=\relax%
        \let\PY@bc=\relax \let\PY@ff=\relax}
    \def\PY@tok#1{\csname PY@tok@#1\endcsname}
    \def\PY@toks#1+{\ifx\relax#1\empty\else%
        \PY@tok{#1}\expandafter\PY@toks\fi}
    \def\PY@do#1{\PY@bc{\PY@tc{\PY@ul{%
        \PY@it{\PY@bf{\PY@ff{#1}}}}}}}
    \def\PY#1#2{\PY@reset\PY@toks#1+\relax+\PY@do{#2}}

    \@namedef{PY@tok@w}{\def\PY@tc##1{\textcolor[rgb]{0.73,0.73,0.73}{##1}}}
    \@namedef{PY@tok@c}{\let\PY@it=\textit\def\PY@tc##1{\textcolor[rgb]{0.24,0.48,0.48}{##1}}}
    \@namedef{PY@tok@cp}{\def\PY@tc##1{\textcolor[rgb]{0.61,0.40,0.00}{##1}}}
    \@namedef{PY@tok@k}{\let\PY@bf=\textbf\def\PY@tc##1{\textcolor[rgb]{0.00,0.50,0.00}{##1}}}
    \@namedef{PY@tok@kp}{\def\PY@tc##1{\textcolor[rgb]{0.00,0.50,0.00}{##1}}}
    \@namedef{PY@tok@kt}{\def\PY@tc##1{\textcolor[rgb]{0.69,0.00,0.25}{##1}}}
    \@namedef{PY@tok@o}{\def\PY@tc##1{\textcolor[rgb]{0.40,0.40,0.40}{##1}}}
    \@namedef{PY@tok@ow}{\let\PY@bf=\textbf\def\PY@tc##1{\textcolor[rgb]{0.67,0.13,1.00}{##1}}}
    \@namedef{PY@tok@nb}{\def\PY@tc##1{\textcolor[rgb]{0.00,0.50,0.00}{##1}}}
    \@namedef{PY@tok@nf}{\def\PY@tc##1{\textcolor[rgb]{0.00,0.00,1.00}{##1}}}
    \@namedef{PY@tok@nc}{\let\PY@bf=\textbf\def\PY@tc##1{\textcolor[rgb]{0.00,0.00,1.00}{##1}}}
    \@namedef{PY@tok@nn}{\let\PY@bf=\textbf\def\PY@tc##1{\textcolor[rgb]{0.00,0.00,1.00}{##1}}}
    \@namedef{PY@tok@ne}{\let\PY@bf=\textbf\def\PY@tc##1{\textcolor[rgb]{0.80,0.25,0.22}{##1}}}
    \@namedef{PY@tok@nv}{\def\PY@tc##1{\textcolor[rgb]{0.10,0.09,0.49}{##1}}}
    \@namedef{PY@tok@no}{\def\PY@tc##1{\textcolor[rgb]{0.53,0.00,0.00}{##1}}}
    \@namedef{PY@tok@nl}{\def\PY@tc##1{\textcolor[rgb]{0.46,0.46,0.00}{##1}}}
    \@namedef{PY@tok@ni}{\let\PY@bf=\textbf\def\PY@tc##1{\textcolor[rgb]{0.44,0.44,0.44}{##1}}}
    \@namedef{PY@tok@na}{\def\PY@tc##1{\textcolor[rgb]{0.41,0.47,0.13}{##1}}}
    \@namedef{PY@tok@nt}{\let\PY@bf=\textbf\def\PY@tc##1{\textcolor[rgb]{0.00,0.50,0.00}{##1}}}
    \@namedef{PY@tok@nd}{\def\PY@tc##1{\textcolor[rgb]{0.67,0.13,1.00}{##1}}}
    \@namedef{PY@tok@s}{\def\PY@tc##1{\textcolor[rgb]{0.73,0.13,0.13}{##1}}}
    \@namedef{PY@tok@sd}{\let\PY@it=\textit\def\PY@tc##1{\textcolor[rgb]{0.73,0.13,0.13}{##1}}}
    \@namedef{PY@tok@si}{\let\PY@bf=\textbf\def\PY@tc##1{\textcolor[rgb]{0.64,0.35,0.47}{##1}}}
    \@namedef{PY@tok@se}{\let\PY@bf=\textbf\def\PY@tc##1{\textcolor[rgb]{0.67,0.36,0.12}{##1}}}
    \@namedef{PY@tok@sr}{\def\PY@tc##1{\textcolor[rgb]{0.64,0.35,0.47}{##1}}}
    \@namedef{PY@tok@ss}{\def\PY@tc##1{\textcolor[rgb]{0.10,0.09,0.49}{##1}}}
    \@namedef{PY@tok@sx}{\def\PY@tc##1{\textcolor[rgb]{0.00,0.50,0.00}{##1}}}
    \@namedef{PY@tok@m}{\def\PY@tc##1{\textcolor[rgb]{0.40,0.40,0.40}{##1}}}
    \@namedef{PY@tok@gh}{\let\PY@bf=\textbf\def\PY@tc##1{\textcolor[rgb]{0.00,0.00,0.50}{##1}}}
    \@namedef{PY@tok@gu}{\let\PY@bf=\textbf\def\PY@tc##1{\textcolor[rgb]{0.50,0.00,0.50}{##1}}}
    \@namedef{PY@tok@gd}{\def\PY@tc##1{\textcolor[rgb]{0.63,0.00,0.00}{##1}}}
    \@namedef{PY@tok@gi}{\def\PY@tc##1{\textcolor[rgb]{0.00,0.52,0.00}{##1}}}
    \@namedef{PY@tok@gr}{\def\PY@tc##1{\textcolor[rgb]{0.89,0.00,0.00}{##1}}}
    \@namedef{PY@tok@ge}{\let\PY@it=\textit}
    \@namedef{PY@tok@gs}{\let\PY@bf=\textbf}
    \@namedef{PY@tok@ges}{\let\PY@bf=\textbf\let\PY@it=\textit}
    \@namedef{PY@tok@gp}{\let\PY@bf=\textbf\def\PY@tc##1{\textcolor[rgb]{0.00,0.00,0.50}{##1}}}
    \@namedef{PY@tok@go}{\def\PY@tc##1{\textcolor[rgb]{0.44,0.44,0.44}{##1}}}
    \@namedef{PY@tok@gt}{\def\PY@tc##1{\textcolor[rgb]{0.00,0.27,0.87}{##1}}}
    \@namedef{PY@tok@err}{\def\PY@bc##1{{\setlength{\fboxsep}{\string -\fboxrule}\fcolorbox[rgb]{1.00,0.00,0.00}{1,1,1}{\strut ##1}}}}
    \@namedef{PY@tok@kc}{\let\PY@bf=\textbf\def\PY@tc##1{\textcolor[rgb]{0.00,0.50,0.00}{##1}}}
    \@namedef{PY@tok@kd}{\let\PY@bf=\textbf\def\PY@tc##1{\textcolor[rgb]{0.00,0.50,0.00}{##1}}}
    \@namedef{PY@tok@kn}{\let\PY@bf=\textbf\def\PY@tc##1{\textcolor[rgb]{0.00,0.50,0.00}{##1}}}
    \@namedef{PY@tok@kr}{\let\PY@bf=\textbf\def\PY@tc##1{\textcolor[rgb]{0.00,0.50,0.00}{##1}}}
    \@namedef{PY@tok@bp}{\def\PY@tc##1{\textcolor[rgb]{0.00,0.50,0.00}{##1}}}
    \@namedef{PY@tok@fm}{\def\PY@tc##1{\textcolor[rgb]{0.00,0.00,1.00}{##1}}}
    \@namedef{PY@tok@vc}{\def\PY@tc##1{\textcolor[rgb]{0.10,0.09,0.49}{##1}}}
    \@namedef{PY@tok@vg}{\def\PY@tc##1{\textcolor[rgb]{0.10,0.09,0.49}{##1}}}
    \@namedef{PY@tok@vi}{\def\PY@tc##1{\textcolor[rgb]{0.10,0.09,0.49}{##1}}}
    \@namedef{PY@tok@vm}{\def\PY@tc##1{\textcolor[rgb]{0.10,0.09,0.49}{##1}}}
    \@namedef{PY@tok@sa}{\def\PY@tc##1{\textcolor[rgb]{0.73,0.13,0.13}{##1}}}
    \@namedef{PY@tok@sb}{\def\PY@tc##1{\textcolor[rgb]{0.73,0.13,0.13}{##1}}}
    \@namedef{PY@tok@sc}{\def\PY@tc##1{\textcolor[rgb]{0.73,0.13,0.13}{##1}}}
    \@namedef{PY@tok@dl}{\def\PY@tc##1{\textcolor[rgb]{0.73,0.13,0.13}{##1}}}
    \@namedef{PY@tok@s2}{\def\PY@tc##1{\textcolor[rgb]{0.73,0.13,0.13}{##1}}}
    \@namedef{PY@tok@sh}{\def\PY@tc##1{\textcolor[rgb]{0.73,0.13,0.13}{##1}}}
    \@namedef{PY@tok@s1}{\def\PY@tc##1{\textcolor[rgb]{0.73,0.13,0.13}{##1}}}
    \@namedef{PY@tok@mb}{\def\PY@tc##1{\textcolor[rgb]{0.40,0.40,0.40}{##1}}}
    \@namedef{PY@tok@mf}{\def\PY@tc##1{\textcolor[rgb]{0.40,0.40,0.40}{##1}}}
    \@namedef{PY@tok@mh}{\def\PY@tc##1{\textcolor[rgb]{0.40,0.40,0.40}{##1}}}
    \@namedef{PY@tok@mi}{\def\PY@tc##1{\textcolor[rgb]{0.40,0.40,0.40}{##1}}}
    \@namedef{PY@tok@il}{\def\PY@tc##1{\textcolor[rgb]{0.40,0.40,0.40}{##1}}}
    \@namedef{PY@tok@mo}{\def\PY@tc##1{\textcolor[rgb]{0.40,0.40,0.40}{##1}}}
    \@namedef{PY@tok@ch}{\let\PY@it=\textit\def\PY@tc##1{\textcolor[rgb]{0.24,0.48,0.48}{##1}}}
    \@namedef{PY@tok@cm}{\let\PY@it=\textit\def\PY@tc##1{\textcolor[rgb]{0.24,0.48,0.48}{##1}}}
    \@namedef{PY@tok@cpf}{\let\PY@it=\textit\def\PY@tc##1{\textcolor[rgb]{0.24,0.48,0.48}{##1}}}
    \@namedef{PY@tok@c1}{\let\PY@it=\textit\def\PY@tc##1{\textcolor[rgb]{0.24,0.48,0.48}{##1}}}
    \@namedef{PY@tok@cs}{\let\PY@it=\textit\def\PY@tc##1{\textcolor[rgb]{0.24,0.48,0.48}{##1}}}

    \def\PYZbs{\char`\\}
    \def\PYZus{\char`\_}
    \def\PYZob{\char`\{}
    \def\PYZcb{\char`\}}
    \def\PYZca{\char`\^}
    \def\PYZam{\char`\&}
    \def\PYZlt{\char`\<}
    \def\PYZgt{\char`\>}
    \def\PYZsh{\char`\#}
    \def\PYZpc{\char`\%}
    \def\PYZdl{\char`\$}
    \def\PYZhy{\char`\-}
    \def\PYZsq{\char`\'}
    \def\PYZdq{\char`\"}
    \def\PYZti{\char`\~}
    \def\PYZat{@}
    \def\PYZlb{[}
    \def\PYZrb{]}
    \makeatother


    % For linebreaks inside Verbatim environment from package fancyvrb.
    \makeatletter
        \newbox\Wrappedcontinuationbox
        \newbox\Wrappedvisiblespacebox
        \newcommand*\Wrappedvisiblespace {\textcolor{red}{\textvisiblespace}}
        \newcommand*\Wrappedcontinuationsymbol {\textcolor{red}{\llap{\tiny$\m@th\hookrightarrow$}}}
        \newcommand*\Wrappedcontinuationindent {3ex }
        \newcommand*\Wrappedafterbreak {\kern\Wrappedcontinuationindent\copy\Wrappedcontinuationbox}
        \newcommand*\Wrappedbreaksatspecials {%
            \def\PYGZus{\discretionary{\char`\_}{\Wrappedafterbreak}{\char`\_}}%
            \def\PYGZob{\discretionary{}{\Wrappedafterbreak\char`\{}{\char`\{}}%
            \def\PYGZcb{\discretionary{\char`\}}{\Wrappedafterbreak}{\char`\}}}%
            \def\PYGZca{\discretionary{\char`\^}{\Wrappedafterbreak}{\char`\^}}%
            \def\PYGZam{\discretionary{\char`\&}{\Wrappedafterbreak}{\char`\&}}%
            \def\PYGZlt{\discretionary{}{\Wrappedafterbreak\char`\<}{\char`\<}}%
            \def\PYGZgt{\discretionary{\char`\>}{\Wrappedafterbreak}{\char`\>}}%
            \def\PYGZsh{\discretionary{}{\Wrappedafterbreak\char`\#}{\char`\#}}%
            \def\PYGZpc{\discretionary{}{\Wrappedafterbreak\char`\%}{\char`\%}}%
            \def\PYGZdl{\discretionary{}{\Wrappedafterbreak\char`\$}{\char`\$}}%
            \def\PYGZhy{\discretionary{\char`\-}{\Wrappedafterbreak}{\char`\-}}%
            \def\PYGZsq{\discretionary{}{\Wrappedafterbreak\textquotesingle}{\textquotesingle}}%
            \def\PYGZdq{\discretionary{}{\Wrappedafterbreak\char`\"}{\char`\"}}%
            \def\PYGZti{\discretionary{\char`\~}{\Wrappedafterbreak}{\char`\~}}%
        }
        \newcommand*\Wrappedbreaksatpunct {%
            \lccode`\~`\.\lowercase{\def~}{\discretionary{\hbox{\char`\.}}{\Wrappedafterbreak}{\hbox{\char`\.}}}%
            \lccode`\~`\,\lowercase{\def~}{\discretionary{\hbox{\char`\,}}{\Wrappedafterbreak}{\hbox{\char`\,}}}%
            \lccode`\~`\;\lowercase{\def~}{\discretionary{\hbox{\char`\;}}{\Wrappedafterbreak}{\hbox{\char`\;}}}%
            \lccode`\~`\:\lowercase{\def~}{\discretionary{\hbox{\char`\:}}{\Wrappedafterbreak}{\hbox{\char`\:}}}%
            \lccode`\~`\?\lowercase{\def~}{\discretionary{\hbox{\char`\?}}{\Wrappedafterbreak}{\hbox{\char`\?}}}%
            \lccode`\~`\!\lowercase{\def~}{\discretionary{\hbox{\char`\!}}{\Wrappedafterbreak}{\hbox{\char`\!}}}%
            \lccode`\~`\/\lowercase{\def~}{\discretionary{\hbox{\char`\/}}{\Wrappedafterbreak}{\hbox{\char`\/}}}%
            \catcode`\.\active
            \catcode`\,\active
            \catcode`\;\active
            \catcode`\:\active
            \catcode`\?\active
            \catcode`\!\active
            \catcode`\/\active
            \lccode`\~`\~
        }
    \makeatother

    \let\OriginalVerbatim=\Verbatim
    \makeatletter
    \renewcommand{\Verbatim}[1][1]{%
        %\parskip\z@skip
        \sbox\Wrappedcontinuationbox {\Wrappedcontinuationsymbol}%
        \sbox\Wrappedvisiblespacebox {\FV@SetupFont\Wrappedvisiblespace}%
        \def\FancyVerbFormatLine ##1{\hsize\linewidth
            \vtop{\raggedright\hyphenpenalty\z@\exhyphenpenalty\z@
                \doublehyphendemerits\z@\finalhyphendemerits\z@
                \strut ##1\strut}%
        }%
        % If the linebreak is at a space, the latter will be displayed as visible
        % space at end of first line, and a continuation symbol starts next line.
        % Stretch/shrink are however usually zero for typewriter font.
        \def\FV@Space {%
            \nobreak\hskip\z@ plus\fontdimen3\font minus\fontdimen4\font
            \discretionary{\copy\Wrappedvisiblespacebox}{\Wrappedafterbreak}
            {\kern\fontdimen2\font}%
        }%

        % Allow breaks at special characters using \PYG... macros.
        \Wrappedbreaksatspecials
        % Breaks at punctuation characters . , ; ? ! and / need catcode=\active
        \OriginalVerbatim[#1,codes*=\Wrappedbreaksatpunct]%
    }
    \makeatother

    % Exact colors from NB
    \definecolor{incolor}{HTML}{303F9F}
    \definecolor{outcolor}{HTML}{D84315}
    \definecolor{cellborder}{HTML}{F7F7F7}
    \definecolor{cellbackground}{HTML}{F7F7F7}

    % prompt
    \makeatletter
    \newcommand{\boxspacing}{\kern\kvtcb@left@rule\kern\kvtcb@boxsep}
    \makeatother
    \newcommand{\prompt}[4]{
        {\ttfamily\llap{{\color{#2}[#3]:\hspace{3pt}#4}}\vspace{-\baselineskip}}
    }
    

    
    % Prevent overflowing lines due to hard-to-break entities
    \sloppy
    % Setup hyperref package
    \hypersetup{
      breaklinks=true,  % so long urls are correctly broken across lines
      colorlinks=true,
      urlcolor=urlcolor,
      linkcolor=linkcolor,
      citecolor=citecolor,
      }
    % Slightly bigger margins than the latex defaults
        
    \geometry{verbose,tmargin=1.5in,bmargin=1in,lmargin=1.125in,rmargin=1.125in}
    


    % --- Nuevos caracteres Unicode ----
    \usepackage{newunicodechar}
    \newunicodechar{∼}{\ensuremath{\sim}}
    \newunicodechar{←}{\ensuremath{\leftarrow}}
    \newunicodechar{…}{\ldots}
    % \le y \ge se vuelven \leq y \geq
    \let\le\leq
    \let\ge\geq


    % --- Fuentes (texto y matemáticas) ---
    \usepackage{libertinust1math}
    \usepackage{fontspec}
    \setmainfont{EB Garamond}[
        UprightFont = * Regular,
        ItalicFont = * Italic,
        BoldFont = * SemiBold,
        BoldItalicFont = * SemiBold Italic,
    ]


    % --- Secciones y subsecciones ---
    \usepackage{titlesec}
    \newfontface\boldd{EB Garamond Bold}
    \newfontface\bolditalic{EB Garamond  Bold Italic}
    \newfontface\extrabold{EB Garamond ExtraBold}
    \newfontface\medium{EB Garamond Medium}



    \usepackage{bookmark}              % mejora anchors
    \hypersetup{hypertexnames=false}   % evita destinos idénticos

    \usepackage{etoolbox,needspace}

    % Rompe página solo en \section y crea ancla propia
    \pretocmd{\section}{\clearpage\phantomsection\needspace{6\baselineskip}}{}{}

    % No rompas página en \section; solo asegura espacio
    \pretocmd{\section}{\phantomsection\needspace{4\baselineskip}\clearpage}{}{}

    \setcounter{secnumdepth}{0}


    % --- Hacer los títulos de sección y subsección más grandes ---
    \titleformat{\section}
      {\boldd\fontsize{34pt}{128pt}\selectfont} % el primer {} es el formato, el segundo {} es el tamaño de línea
      {\thesection}{18em}{} % el primer {} es el formato, el segundo {} es la separación entre número y título
      \titleformat{\section}
        {\boldd\fontsize{18pt}{18pt}\selectfont}
        {\thesection}{10em}{}
    \titlespacing*{\section}{0pt}{0pt}{24pt} % el primer {} es la sangría, el segundo {} es el espacio antes, el tercero {} es el espacio después
    \titlespacing*{\section}{0pt}{0pt}{18pt} % el primer {} es la sangría, el segundo {} es el espacio antes, el tercero {} es el espacio después

    % --- Encabezado con nombre de la sección ---
    \usepackage{fancyhdr}
    \pagestyle{fancy}
    \fancyhf{}
    \fancyhead[L]{\small\leftmark}
    \fancyhead[R]{\small\thepage}
    \fancyhead[C]{\small\textit{\rightmark}}
    \renewcommand{\headrulewidth}{0.1pt}

    %\textsc{} para versalitas

    % --- Crear un nuevo estilo de pagina con el numero arriba a la derecha ---
    \fancypagestyle{myfancy}{
      \fancyhf{}
      \fancyhead[R]{\small\thepage}
      \renewcommand{\headrulewidth}{0pt}
    }

        % --- Cambia el pagestyle a 'plain' en cada \section ---
    \let\oldsection\section
    \renewcommand{\section}{%
      \clearpage
      \thispagestyle{myfancy}%
      \oldsection
    }

    % --- Espaciado ---
    \usepackage{setspace}
    \setstretch{1.2}
    \setlength{\parskip}{0.2em}

    \begin{document}

    % --- Portada ---
    \begin{titlepage}
    \centering
    \vspace*{4cm}
    {\extrabold\fontsize{28pt}{28pt}\selectfont Actividad 3.2\par}
    \vspace{1cm}
    
    {\medium\fontsize{12pt}{12pt}\selectfont
    {\large Curso:  Temas Selectos I: O25 LAT4032 1\par}
    \vspace{.5em}
    {\large Profesor:  Rubén Blancas Rivera\par}
    \vspace{.5em}
    {\large Mayren Herrera Vargas, ID: 173802\\ Sofia Graham Coello, ID: 174291\\Heriberto Espino Montelongo, ID: 175199\par}
    \vspace{.5em}
    {\large Universidad de las Américas Puebla\par}
    \vfill
    {\large 25 de Octubre de 2025 \par}}
    \end{titlepage}


    % --- Tabla de contenidos ---
    \renewcommand{\contentsname}{Índice}
    \tableofcontents
    \thispagestyle{empty}
    \newpage

    % --- Inicio del documento ---
    


    \hypertarget{Importación de Librerías}{%
\section{Importación de Librerías}\label{importaciuxf3n-de-libreruxedas}}

    \begin{tcolorbox}[breakable, size=fbox, boxrule=1pt, pad at break*=1mm,colback=cellbackground, colframe=cellborder]
\prompt{In}{incolor}{1}{\boxspacing}
\begin{Verbatim}[commandchars=\\\{\}]
\PY{k+kn}{import}\PY{+w}{ }\PY{n+nn}{numpy}\PY{+w}{ }\PY{k}{as}\PY{+w}{ }\PY{n+nn}{np}
\PY{k+kn}{from}\PY{+w}{ }\PY{n+nn}{math}\PY{+w}{ }\PY{k+kn}{import} \PY{n}{erf}\PY{p}{,} \PY{n}{sqrt}\PY{p}{,} \PY{n}{comb}\PY{p}{,} \PY{n}{floor}
\end{Verbatim}
\end{tcolorbox}

    \hypertarget{ejercicio-1}{%
\section{Ejercicio 1}\label{ejercicio-1}}


Sea $\phi(x)$ la densidad de $\mathcal{N}(0,1)$. Se sabe que su cuantil del $95\%$ satisface
\[
\int_{-\infty}^{c_p} \phi(x)\,dx = 0.95, \qquad c_p \approx 1.65.
\]
Escriba un programa que utilice el \emph{método clásico de Monte Carlo} para corroborar este resultado, 
realizando la integración numérica en el intervalo acotado $(-4,\,1.65)$.

\subsection{Planteamiento}
Queremos aproximar
\[
\theta=\int_{-4}^{1.65}\phi(x)\,dx \approx \int_{-\infty}^{1.65}\phi(x)\,dx,
\]
pues $\Phi(-4)$ es despreciable. Para aplicar Monte Carlo clásico (hit-or-miss) se elige un rectángulo
\[
R=[a,b]\times[0,c], \quad a=-4,\; b=1.65,\; c=\max_{x\in[a,b]}\phi(x)=\phi(0)=\frac{1}{\sqrt{2\pi}}.
\]
Sea $(X,Y)\sim \mathrm{Unif}(a,b)\times\mathrm{Unif}(0,c)$; entonces
\[
p=\mathbb{P}\{Y\leq \phi(X)\}=\frac{1}{c(b-a)}\int_{a}^{b}\phi(x)\,dx=\frac{\theta}{c(b-a)}.
\]
Con $N$ puntos bajo la curva en una muestra de tamaño $n$, el estimador clásico es
\[
\hat\theta=c(b-a)\,\hat p = c(b-a)\,\frac{N}{n},
\]
insesgado y con varianza $\mathrm{Var}(\hat\theta)=\theta\,[c(b-a)-\theta]/n$.

\subsection{Pseudocódigo (Monte Carlo clásico)}
\begin{enumerate}
  \item \textbf{Entradas:} $a=-4$, $b=1.65$, $c=\phi(0)$, tamaño $n$ grande.
  \item \textbf{Para} $i=1,\dots,n$:
  \begin{enumerate}
    \item Generar $X_i\sim\mathrm{Unif}(a,b)$ y $Y_i\sim\mathrm{Unif}(0,c)$ de forma independiente.
    \item Definir $I_i=\mathbf{1}\{Y_i\leq \phi(X_i)\}$.
  \end{enumerate}
  \item Calcular $N=\sum_{i=1}^n I_i$, $\hat p = N/n$ y
        \[
        \hat\theta = c\,(b-a)\,\hat p.
        \]
  \item Reportar $\hat\theta$ y un IC al $95\%$: $\hat\theta \pm 1.96\,\widehat{\mathrm{SE}}$.
\end{enumerate}


\subsection{Código}
    \begin{tcolorbox}[breakable, size=fbox, boxrule=1pt, pad at break*=1mm,colback=cellbackground, colframe=cellborder]
\prompt{In}{incolor}{2}{\boxspacing}
\begin{Verbatim}[commandchars=\\\{\}]
\PY{k}{def}\PY{+w}{ }\PY{n+nf}{phi}\PY{p}{(}\PY{n}{x}\PY{p}{)}\PY{p}{:}
\PY{+w}{    }\PY{l+s+sd}{\PYZdq{}\PYZdq{}\PYZdq{}Densidad normal estándar φ(x).\PYZdq{}\PYZdq{}\PYZdq{}}
    \PY{k}{return} \PY{p}{(}\PY{l+m+mi}{1}\PY{o}{/}\PY{n}{np}\PY{o}{.}\PY{n}{sqrt}\PY{p}{(}\PY{l+m+mi}{2}\PY{o}{*}\PY{n}{np}\PY{o}{.}\PY{n}{pi}\PY{p}{)}\PY{p}{)} \PY{o}{*} \PY{n}{np}\PY{o}{.}\PY{n}{exp}\PY{p}{(}\PY{o}{\PYZhy{}}\PY{n}{x}\PY{o}{*}\PY{o}{*}\PY{l+m+mi}{2}\PY{o}{/}\PY{l+m+mi}{2}\PY{p}{)}

\PY{c+c1}{\PYZsh{} Parámetros del rectángulo}
\PY{n}{a}\PY{p}{,} \PY{n}{b} \PY{o}{=} \PY{o}{\PYZhy{}}\PY{l+m+mf}{4.0}\PY{p}{,} \PY{l+m+mf}{1.65}                      \PY{c+c1}{\PYZsh{} intervalo acotado}
\PY{n}{c} \PY{o}{=} \PY{l+m+mi}{1}\PY{o}{/}\PY{n}{np}\PY{o}{.}\PY{n}{sqrt}\PY{p}{(}\PY{l+m+mi}{2}\PY{o}{*}\PY{n}{np}\PY{o}{.}\PY{n}{pi}\PY{p}{)}                 \PY{c+c1}{\PYZsh{} c = max φ(x) = φ(0)}
\PY{n}{n} \PY{o}{=} \PY{l+m+mi}{300\PYZus{}000}                            \PY{c+c1}{\PYZsh{} número de puntos}
\PY{n}{rng} \PY{o}{=} \PY{n}{np}\PY{o}{.}\PY{n}{random}\PY{o}{.}\PY{n}{default\PYZus{}rng}\PY{p}{(}\PY{l+m+mi}{7}\PY{p}{)}         \PY{c+c1}{\PYZsh{} semilla para reproducibilidad}

\PY{c+c1}{\PYZsh{} Muestreo uniforme en el rectángulo [a,b]×[0,c]}
\PY{n}{x} \PY{o}{=} \PY{n}{rng}\PY{o}{.}\PY{n}{uniform}\PY{p}{(}\PY{n}{a}\PY{p}{,} \PY{n}{b}\PY{p}{,} \PY{n}{n}\PY{p}{)}
\PY{n}{y} \PY{o}{=} \PY{n}{rng}\PY{o}{.}\PY{n}{uniform}\PY{p}{(}\PY{l+m+mf}{0.0}\PY{p}{,} \PY{n}{c}\PY{p}{,} \PY{n}{n}\PY{p}{)}

\PY{c+c1}{\PYZsh{} Conteo de puntos bajo la curva}
\PY{n}{n0} \PY{o}{=} \PY{n}{np}\PY{o}{.}\PY{n}{sum}\PY{p}{(}\PY{n}{y} \PY{o}{\PYZlt{}}\PY{o}{=} \PY{n}{phi}\PY{p}{(}\PY{n}{x}\PY{p}{)}\PY{p}{)}

\PY{c+c1}{\PYZsh{} Estimador clásico: θ̂ = c (b−a) · (n0/n)}
\PY{n}{rect\PYZus{}area} \PY{o}{=} \PY{n}{c} \PY{o}{*} \PY{p}{(}\PY{n}{b} \PY{o}{\PYZhy{}} \PY{n}{a}\PY{p}{)}
\PY{n}{p\PYZus{}hat} \PY{o}{=} \PY{n}{n0} \PY{o}{/} \PY{n}{n}
\PY{n}{theta\PYZus{}hat} \PY{o}{=} \PY{n}{rect\PYZus{}area} \PY{o}{*} \PY{n}{p\PYZus{}hat}

\PY{c+c1}{\PYZsh{} Error estándar plug\PYZhy{}in usando Var(θ̂) = θ/n [ c(b−a) − θ ] (desconocido ⇒ sustituimos θ̂)}
\PY{n}{var\PYZus{}hat} \PY{o}{=} \PY{p}{(}\PY{n}{theta\PYZus{}hat} \PY{o}{/} \PY{n}{n}\PY{p}{)} \PY{o}{*} \PY{p}{(}\PY{n}{rect\PYZus{}area} \PY{o}{\PYZhy{}} \PY{n}{theta\PYZus{}hat}\PY{p}{)}
\PY{n}{se\PYZus{}hat} \PY{o}{=} \PY{n}{np}\PY{o}{.}\PY{n}{sqrt}\PY{p}{(}\PY{n}{var\PYZus{}hat}\PY{p}{)}

\PY{c+c1}{\PYZsh{} Valor \PYZdq{}teórico\PYZdq{} con CDF normal vía erf (para comparar)}
\PY{k}{def}\PY{+w}{ }\PY{n+nf}{Phi}\PY{p}{(}\PY{n}{z}\PY{p}{)}\PY{p}{:}  \PY{c+c1}{\PYZsh{} CDF normal estándar}
    \PY{k}{return} \PY{l+m+mf}{0.5}\PY{o}{*}\PY{p}{(}\PY{l+m+mi}{1} \PY{o}{+} \PY{n}{erf}\PY{p}{(}\PY{n}{z}\PY{o}{/}\PY{n}{sqrt}\PY{p}{(}\PY{l+m+mi}{2}\PY{p}{)}\PY{p}{)}\PY{p}{)}

\PY{n}{true\PYZus{}full} \PY{o}{=} \PY{n}{Phi}\PY{p}{(}\PY{l+m+mf}{1.65}\PY{p}{)}           \PY{c+c1}{\PYZsh{} ∫\PYZus{}\PYZob{}\PYZhy{}∞\PYZcb{}\PYZca{}\PYZob{}1.65\PYZcb{} φ}
\PY{n}{left\PYZus{}tail} \PY{o}{=} \PY{n}{Phi}\PY{p}{(}\PY{o}{\PYZhy{}}\PY{l+m+mf}{4.0}\PY{p}{)}           \PY{c+c1}{\PYZsh{} ∫\PYZus{}\PYZob{}\PYZhy{}∞\PYZcb{}\PYZca{}\PYZob{}\PYZhy{}4\PYZcb{} φ (muy pequeño)}
\PY{n}{true\PYZus{}trunc} \PY{o}{=} \PY{n}{true\PYZus{}full} \PY{o}{\PYZhy{}} \PY{n}{left\PYZus{}tail}   \PY{c+c1}{\PYZsh{} ∫\PYZus{}\PYZob{}\PYZhy{}4\PYZcb{}\PYZca{}\PYZob{}1.65\PYZcb{} φ}

\PY{c+c1}{\PYZsh{} IC aproximado al 95\PYZpc{}}
\PY{n}{ci\PYZus{}low} \PY{o}{=} \PY{n}{theta\PYZus{}hat} \PY{o}{\PYZhy{}} \PY{l+m+mf}{1.96}\PY{o}{*}\PY{n}{se\PYZus{}hat}
\PY{n}{ci\PYZus{}high} \PY{o}{=} \PY{n}{theta\PYZus{}hat} \PY{o}{+} \PY{l+m+mf}{1.96}\PY{o}{*}\PY{n}{se\PYZus{}hat}

\PY{n+nb}{print}\PY{p}{(}\PY{l+s+sa}{f}\PY{l+s+s2}{\PYZdq{}}\PY{l+s+s2}{Estimación Monte Carlo (truncada [\PYZhy{}4,1.65]): }\PY{l+s+si}{\PYZob{}}\PY{n}{theta\PYZus{}hat}\PY{l+s+si}{:}\PY{l+s+s2}{.6f}\PY{l+s+si}{\PYZcb{}}\PY{l+s+s2}{\PYZdq{}}\PY{p}{)}
\PY{n+nb}{print}\PY{p}{(}\PY{l+s+sa}{f}\PY{l+s+s2}{\PYZdq{}}\PY{l+s+s2}{Error estándar (plug\PYZhy{}in): }\PY{l+s+si}{\PYZob{}}\PY{n}{se\PYZus{}hat}\PY{l+s+si}{:}\PY{l+s+s2}{.6f}\PY{l+s+si}{\PYZcb{}}\PY{l+s+s2}{\PYZdq{}}\PY{p}{)}
\PY{n+nb}{print}\PY{p}{(}\PY{l+s+sa}{f}\PY{l+s+s2}{\PYZdq{}}\PY{l+s+s2}{IC 95\PYZpc{} (θ̂ ± 1.96·SE): [}\PY{l+s+si}{\PYZob{}}\PY{n}{ci\PYZus{}low}\PY{l+s+si}{:}\PY{l+s+s2}{.6f}\PY{l+s+si}{\PYZcb{}}\PY{l+s+s2}{, }\PY{l+s+si}{\PYZob{}}\PY{n}{ci\PYZus{}high}\PY{l+s+si}{:}\PY{l+s+s2}{.6f}\PY{l+s+si}{\PYZcb{}}\PY{l+s+s2}{]}\PY{l+s+s2}{\PYZdq{}}\PY{p}{)}
\PY{n+nb}{print}\PY{p}{(}\PY{l+s+sa}{f}\PY{l+s+s2}{\PYZdq{}}\PY{l+s+s2}{Valor teórico ∫\PYZus{}(\PYZhy{}∞)\PYZca{}(1.65) φ = }\PY{l+s+si}{\PYZob{}}\PY{n}{true\PYZus{}full}\PY{l+s+si}{:}\PY{l+s+s2}{.6f}\PY{l+s+si}{\PYZcb{}}\PY{l+s+s2}{\PYZdq{}}\PY{p}{)}
\PY{n+nb}{print}\PY{p}{(}\PY{l+s+sa}{f}\PY{l+s+s2}{\PYZdq{}}\PY{l+s+s2}{Cola izquierda Φ(\PYZhy{}4) = }\PY{l+s+si}{\PYZob{}}\PY{n}{left\PYZus{}tail}\PY{l+s+si}{:}\PY{l+s+s2}{.8f}\PY{l+s+si}{\PYZcb{}}\PY{l+s+s2}{  ⇒  valor truncado ≈ }\PY{l+s+si}{\PYZob{}}\PY{n}{true\PYZus{}trunc}\PY{l+s+si}{:}\PY{l+s+s2}{.6f}\PY{l+s+si}{\PYZcb{}}\PY{l+s+s2}{\PYZdq{}}\PY{p}{)}
\end{Verbatim}
\end{tcolorbox}

    \begin{Verbatim}[commandchars=\\\{\}]
Estimación Monte Carlo (truncada [-4,1.65]): 0.953715
Error estándar (plug-in): 0.002033
IC 95\% (θ̂ ± 1.96·SE): [0.949730, 0.957700]
Valor teórico ∫\_(-∞)\^{}(1.65) φ = 0.950529
Cola izquierda Φ(-4) = 0.00003167  ⇒  valor truncado ≈ 0.950497
    \end{Verbatim}

    \hypertarget{ejercicio-2}{%
\section{Ejercicio 2}\label{ejercicio-2}}


Escriba el algoritmo para llevar a cabo la integración Monte Carlo en su procedimiento clásico cuando la función a integrar es:
\begin{enumerate}[label=(\alph*)]
    \item $g(x)=c$ (constante), para $a<x<b$.
    \item $g(x)=2x$ para $0<x<1$, cuando se toma $c=1$.
\end{enumerate}

\subsection{Planteamiento}
Sea $g:[a,b]\to\mathbb{R}_+$ acotada por arriba, es decir, existe $c>0$ tal que $0\le g(x)\le c$ para todo $x\in(a,b)$. 
Considere una variable aleatoria $(X,Y)$ con distribución uniforme independiente en el rectángulo
\[
R=[a,b]\times[0,c], 
\qquad X\sim \mathrm{Unif}(a,b),\;\; Y\sim \mathrm{Unif}(0,c).
\]
Entonces, por geometría,
\[
p \;=\; \Pr\{Y\le g(X)\}
= \frac{1}{c(b-a)}\int_a^b g(x)\,dx
= \frac{\theta}{c(b-a)}, 
\qquad \theta=\int_a^b g(x)\,dx.
\]
Si tomamos una muestra $\{(X_i,Y_i)\}_{i=1}^n$ i.i.d.\ de $(X,Y)$ y definimos 
$N=\sum_{i=1}^n \mathbf{1}\{Y_i\le g(X_i)\}\sim\mathrm{Bin}(n,p)$, el estimador clásico es
\[
\widehat{\theta}=c(b-a)\,\widehat{p}
= c(b-a)\,\frac{N}{n}.
\]
Es insesgado y su varianza es
\[
\mathrm{Var}(\widehat{\theta})
= \frac{\theta}{n}\,\big(c(b-a)-\theta\big).
\]
En la práctica, como $\theta$ es desconocido, se sustituye por $\widehat{\theta}$ para obtener un error estándar tipo plug-in:
\[
\widehat{\mathrm{SE}}(\widehat{\theta})
= \sqrt{\frac{\widehat{\theta}}{n}\,\big(c(b-a)-\widehat{\theta}\big)}.
\]
\emph{Regla crucial:} el método clásico requiere una cota válida $c$ con $0\le g(x)\le c$ en todo $(a,b)$. Elegir $c$ lo más pequeño posible (respetando $g\le c$) reduce la varianza.

\subsection{Inciso (a): $g(x)=c$ constante en $(a,b)$}
\textbf{Elección del rectángulo.} Como $g(x)\equiv c$, la cota mínima válida es $c$ mismo. Tomamos $R=[a,b]\times[0,c]$.

\textbf{Probabilidad y estimación.}
Para todo $i$, $Y_i\le g(X_i)$ ocurre con probabilidad $1$ porque $Y_i\in[0,c]$ y $g(X_i)=c$.
Entonces $N=n$ y
\[
\widehat{\theta}=c(b-a)\,\frac{N}{n}=c(b-a),
\]
que coincide exactamente con el valor verdadero 
\(\displaystyle \theta=\int_a^b c\,dx=c(b-a)\).
En consecuencia, $\mathrm{Var}(\widehat{\theta})=0$.

\textbf{Algoritmo (a).}
\begin{enumerate}
  \item Entradas: $a<b$, constante $c\ge0$, tamaño $n$.
  \item Para $i=1,\dots,n$: generar $X_i\sim\mathrm{Unif}(a,b)$, $Y_i\sim\mathrm{Unif}(0,c)$; $I_i=\mathbf{1}\{Y_i\le c\}$ (siempre $1$).
  \item $N=\sum I_i=n$ y $\widehat{\theta}=c(b-a)$.
\end{enumerate}

\subsection{Inciso (b): $g(x)=2x$ en $(0,1)$}
Aquí $\max_{x\in(0,1)} g(x)=2$, de modo que la \emph{cota mínima válida} es $c=2$. 
Con $c=2$:
\[
\theta=\int_0^1 2x\,dx=1, 
\quad c(b-a)=2\cdot 1=2, 
\quad p=\frac{\theta}{c(b-a)}=\frac{1}{2}.
\]
El estimador es $\widehat{\theta}=2\,\frac{N}{n}$ con $N\sim\mathrm{Bin}(n,\tfrac12)$, 
y $\mathrm{Var}(\widehat{\theta})=\frac{1}{n}\big(2-1\big)=\frac1n$.

\textbf{Qué sucede si ``se toma $c=1$''.}
La condición $g\le c$ se viola porque $g(x)=2x>1$ cuando $x>0.5$. 
El rectángulo $[0,1]\times[0,1]$ \emph{recorta} la región bajo la curva y lo que se estima es
\[
\tilde{\theta}=\int_0^1 \min\{2x,1\}\,dx
=\int_0^{1/2} 2x\,dx+\int_{1/2}^1 1\,dx
=\frac{1}{4}+\frac{1}{2}=\frac{3}{4}\neq \theta,
\]
es decir, el procedimiento resulta sesgado. Por lo tanto, para el método clásico \textbf{debe} usarse $c\ge2$.

\textbf{Algoritmo (b) correcto, con $c=2$.}
\begin{enumerate}
  \item Entradas: $a=0$, $b=1$, $c=2$, tamaño $n$.
  \item Para $i=1,\dots,n$: generar $X_i\sim\mathrm{Unif}(0,1)$, $Y_i\sim\mathrm{Unif}(0,2)$; $I_i=\mathbf{1}\{Y_i\le 2X_i\}$.
  \item $N=\sum I_i$, $\widehat{\theta}=2\cdot (N/n)$; 
        opcionalmente, estimar $\widehat{\mathrm{SE}}$ y un IC al $95\%$ vía $\widehat{\theta}\pm 1.96\,\widehat{\mathrm{SE}}$.
\end{enumerate}

\subsection{Pseudocódigo general}
\begin{enumerate}
  \item \textbf{Entradas:} intervalo $(a,b)$; cota $c>0$ con $0\le g(x)\le c$ en $(a,b)$; tamaño muestral $n$.
  \item \textbf{Para $i=1,\dots,n$:}
  \begin{enumerate}
    \item Generar $X_i\sim\mathrm{Unif}(a,b)$ y $Y_i\sim\mathrm{Unif}(0,c)$, independientes.
    \item Definir $I_i=\mathbf{1}\{Y_i\le g(X_i)\}$.
  \end{enumerate}
  \item \textbf{Conteo:} $N=\sum_{i=1}^n I_i$ y $\widehat{p}=N/n$.
  \item \textbf{Estimación:} $\widehat{\theta}=c\,(b-a)\,\widehat{p}$.
  \item \textbf{Error (opcional):} $\widehat{\mathrm{SE}}=\sqrt{\frac{\widehat{\theta}}{n}\big(c(b-a)-\widehat{\theta}\big)}$; 
        \textbf{IC 95\%:} $\widehat{\theta}\pm 1.96\,\widehat{\mathrm{SE}}$.
\end{enumerate}


\subsection{Código}

    \begin{tcolorbox}[breakable, size=fbox, boxrule=1pt, pad at break*=1mm,colback=cellbackground, colframe=cellborder]
\prompt{In}{incolor}{3}{\boxspacing}
\begin{Verbatim}[commandchars=\\\{\}]
\PY{k}{def}\PY{+w}{ }\PY{n+nf}{mc\PYZus{}clasico\PYZus{}hit\PYZus{}or\PYZus{}miss}\PY{p}{(}\PY{n}{g}\PY{p}{,} \PY{n}{a}\PY{p}{,} \PY{n}{b}\PY{p}{,} \PY{n}{c}\PY{p}{,} \PY{n}{n}\PY{o}{=}\PY{l+m+mi}{200\PYZus{}000}\PY{p}{,} \PY{n}{seed}\PY{o}{=}\PY{l+m+mi}{123}\PY{p}{)}\PY{p}{:}
\PY{+w}{    }\PY{l+s+sd}{\PYZdq{}\PYZdq{}\PYZdq{}}
\PY{l+s+sd}{    Método Monte Carlo clásico (hit\PYZhy{}or\PYZhy{}miss) para ∫\PYZus{}a\PYZca{}b g(x) dx con 0 \PYZlt{}= g(x) \PYZlt{}= c.}
\PY{l+s+sd}{    Devuelve (theta\PYZus{}hat, se\PYZus{}hat, n0, rect\PYZus{}area).}
\PY{l+s+sd}{    \PYZdq{}\PYZdq{}\PYZdq{}}
    \PY{n}{rng} \PY{o}{=} \PY{n}{np}\PY{o}{.}\PY{n}{random}\PY{o}{.}\PY{n}{default\PYZus{}rng}\PY{p}{(}\PY{n}{seed}\PY{p}{)}
    \PY{n}{x} \PY{o}{=} \PY{n}{rng}\PY{o}{.}\PY{n}{uniform}\PY{p}{(}\PY{n}{a}\PY{p}{,} \PY{n}{b}\PY{p}{,} \PY{n}{size}\PY{o}{=}\PY{n}{n}\PY{p}{)}
    \PY{n}{y} \PY{o}{=} \PY{n}{rng}\PY{o}{.}\PY{n}{uniform}\PY{p}{(}\PY{l+m+mf}{0.0}\PY{p}{,} \PY{n}{c}\PY{p}{,} \PY{n}{size}\PY{o}{=}\PY{n}{n}\PY{p}{)}
    \PY{n}{under} \PY{o}{=} \PY{n}{y} \PY{o}{\PYZlt{}}\PY{o}{=} \PY{n}{g}\PY{p}{(}\PY{n}{x}\PY{p}{)}
    \PY{n}{n0} \PY{o}{=} \PY{n+nb}{int}\PY{p}{(}\PY{n}{np}\PY{o}{.}\PY{n}{sum}\PY{p}{(}\PY{n}{under}\PY{p}{)}\PY{p}{)}
    \PY{n}{rect\PYZus{}area} \PY{o}{=} \PY{n}{c} \PY{o}{*} \PY{p}{(}\PY{n}{b} \PY{o}{\PYZhy{}} \PY{n}{a}\PY{p}{)}
    \PY{n}{p\PYZus{}hat} \PY{o}{=} \PY{n}{n0} \PY{o}{/} \PY{n}{n}
    \PY{n}{theta\PYZus{}hat} \PY{o}{=} \PY{n}{rect\PYZus{}area} \PY{o}{*} \PY{n}{p\PYZus{}hat}
    \PY{c+c1}{\PYZsh{} Var(θ̂) ≈ θ̂/n * (c(b−a) − θ̂)  (plug\PYZhy{}in)}
    \PY{n}{var\PYZus{}hat} \PY{o}{=} \PY{p}{(}\PY{n}{theta\PYZus{}hat} \PY{o}{/} \PY{n}{n}\PY{p}{)} \PY{o}{*} \PY{p}{(}\PY{n}{rect\PYZus{}area} \PY{o}{\PYZhy{}} \PY{n}{theta\PYZus{}hat}\PY{p}{)}
    \PY{n}{se\PYZus{}hat} \PY{o}{=} \PY{n}{np}\PY{o}{.}\PY{n}{sqrt}\PY{p}{(}\PY{n+nb}{max}\PY{p}{(}\PY{n}{var\PYZus{}hat}\PY{p}{,} \PY{l+m+mf}{0.0}\PY{p}{)}\PY{p}{)}
    \PY{k}{return} \PY{n}{theta\PYZus{}hat}\PY{p}{,} \PY{n}{se\PYZus{}hat}\PY{p}{,} \PY{n}{n0}\PY{p}{,} \PY{n}{rect\PYZus{}area}

\PY{c+c1}{\PYZsh{} (a) g(x)=c0 constante}
\PY{n}{a}\PY{p}{,} \PY{n}{b}\PY{p}{,} \PY{n}{c0} \PY{o}{=} \PY{l+m+mf}{0.0}\PY{p}{,} \PY{l+m+mf}{3.0}\PY{p}{,} \PY{l+m+mf}{5.0}
\PY{n}{g\PYZus{}const} \PY{o}{=} \PY{k}{lambda} \PY{n}{x}\PY{p}{:} \PY{n}{c0}\PY{o}{*}\PY{n}{np}\PY{o}{.}\PY{n}{ones\PYZus{}like}\PY{p}{(}\PY{n}{x}\PY{p}{)}
\PY{n}{theta\PYZus{}hat\PYZus{}a}\PY{p}{,} \PY{n}{se\PYZus{}hat\PYZus{}a}\PY{p}{,} \PY{n}{\PYZus{}}\PY{p}{,} \PY{n}{\PYZus{}} \PY{o}{=} \PY{n}{mc\PYZus{}clasico\PYZus{}hit\PYZus{}or\PYZus{}miss}\PY{p}{(}\PY{n}{g\PYZus{}const}\PY{p}{,} \PY{n}{a}\PY{p}{,} \PY{n}{b}\PY{p}{,} \PY{n}{c}\PY{o}{=}\PY{n}{c0}\PY{p}{,} \PY{n}{n}\PY{o}{=}\PY{l+m+mi}{100\PYZus{}000}\PY{p}{,} \PY{n}{seed}\PY{o}{=}\PY{l+m+mi}{7}\PY{p}{)}
\PY{n+nb}{print}\PY{p}{(}\PY{l+s+s2}{\PYZdq{}}\PY{l+s+s2}{(a) g(x)=c0 en (a,b):   θ̂ =}\PY{l+s+s2}{\PYZdq{}}\PY{p}{,} \PY{n}{theta\PYZus{}hat\PYZus{}a}\PY{p}{,} \PY{l+s+s2}{\PYZdq{}}\PY{l+s+s2}{  SE ≈}\PY{l+s+s2}{\PYZdq{}}\PY{p}{,} \PY{n}{se\PYZus{}hat\PYZus{}a}\PY{p}{,} \PY{l+s+s2}{\PYZdq{}}\PY{l+s+s2}{  (valor real =}\PY{l+s+s2}{\PYZdq{}}\PY{p}{,} \PY{n}{c0}\PY{o}{*}\PY{p}{(}\PY{n}{b}\PY{o}{\PYZhy{}}\PY{n}{a}\PY{p}{)}\PY{p}{,} \PY{l+s+s2}{\PYZdq{}}\PY{l+s+s2}{)}\PY{l+s+s2}{\PYZdq{}}\PY{p}{)}

\PY{c+c1}{\PYZsh{} (b) g(x)=2x en (0,1) — caso correcto: c=2}
\PY{n}{g\PYZus{}lin} \PY{o}{=} \PY{k}{lambda} \PY{n}{x}\PY{p}{:} \PY{l+m+mf}{2.0}\PY{o}{*}\PY{n}{x}
\PY{n}{theta\PYZus{}hat\PYZus{}b\PYZus{}ok}\PY{p}{,} \PY{n}{se\PYZus{}hat\PYZus{}b\PYZus{}ok}\PY{p}{,} \PY{n}{\PYZus{}}\PY{p}{,} \PY{n}{\PYZus{}} \PY{o}{=} \PY{n}{mc\PYZus{}clasico\PYZus{}hit\PYZus{}or\PYZus{}miss}\PY{p}{(}\PY{n}{g\PYZus{}lin}\PY{p}{,} \PY{l+m+mf}{0.0}\PY{p}{,} \PY{l+m+mf}{1.0}\PY{p}{,} \PY{n}{c}\PY{o}{=}\PY{l+m+mf}{2.0}\PY{p}{,} \PY{n}{n}\PY{o}{=}\PY{l+m+mi}{200\PYZus{}000}\PY{p}{,} \PY{n}{seed}\PY{o}{=}\PY{l+m+mi}{7}\PY{p}{)}
\PY{n+nb}{print}\PY{p}{(}\PY{l+s+s2}{\PYZdq{}}\PY{l+s+s2}{(b) g(x)=2x, c=2:       θ̂ ≈}\PY{l+s+s2}{\PYZdq{}}\PY{p}{,} \PY{n+nb}{round}\PY{p}{(}\PY{n}{theta\PYZus{}hat\PYZus{}b\PYZus{}ok}\PY{p}{,}\PY{l+m+mi}{6}\PY{p}{)}\PY{p}{,} \PY{l+s+s2}{\PYZdq{}}\PY{l+s+s2}{  SE ≈}\PY{l+s+s2}{\PYZdq{}}\PY{p}{,} \PY{n+nb}{round}\PY{p}{(}\PY{n}{se\PYZus{}hat\PYZus{}b\PYZus{}ok}\PY{p}{,}\PY{l+m+mi}{6}\PY{p}{)}\PY{p}{,} \PY{l+s+s2}{\PYZdq{}}\PY{l+s+s2}{  (valor real = 1)}\PY{l+s+s2}{\PYZdq{}}\PY{p}{)}

\PY{c+c1}{\PYZsh{} (b) Si alguien usa c=1 (no válido): muestra el sesgo hacia 3/4}
\PY{n}{theta\PYZus{}hat\PYZus{}b\PYZus{}bad}\PY{p}{,} \PY{n}{se\PYZus{}hat\PYZus{}b\PYZus{}bad}\PY{p}{,} \PY{n}{\PYZus{}}\PY{p}{,} \PY{n}{\PYZus{}} \PY{o}{=} \PY{n}{mc\PYZus{}clasico\PYZus{}hit\PYZus{}or\PYZus{}miss}\PY{p}{(}\PY{n}{g\PYZus{}lin}\PY{p}{,} \PY{l+m+mf}{0.0}\PY{p}{,} \PY{l+m+mf}{1.0}\PY{p}{,} \PY{n}{c}\PY{o}{=}\PY{l+m+mf}{1.0}\PY{p}{,} \PY{n}{n}\PY{o}{=}\PY{l+m+mi}{200\PYZus{}000}\PY{p}{,} \PY{n}{seed}\PY{o}{=}\PY{l+m+mi}{7}\PY{p}{)}
\PY{n+nb}{print}\PY{p}{(}\PY{l+s+s2}{\PYZdq{}}\PY{l+s+s2}{(b) g(x)=2x, c=1 (!)    θ̂ ≈}\PY{l+s+s2}{\PYZdq{}}\PY{p}{,} \PY{n+nb}{round}\PY{p}{(}\PY{n}{theta\PYZus{}hat\PYZus{}b\PYZus{}bad}\PY{p}{,}\PY{l+m+mi}{6}\PY{p}{)}\PY{p}{,} \PY{l+s+s2}{\PYZdq{}}\PY{l+s+s2}{  SE ≈}\PY{l+s+s2}{\PYZdq{}}\PY{p}{,} \PY{n+nb}{round}\PY{p}{(}\PY{n}{se\PYZus{}hat\PYZus{}b\PYZus{}bad}\PY{p}{,}\PY{l+m+mi}{6}\PY{p}{)}\PY{p}{,} \PY{l+s+s2}{\PYZdq{}}\PY{l+s+s2}{  (valor }\PY{l+s+s2}{\PYZsq{}}\PY{l+s+s2}{truncado}\PY{l+s+s2}{\PYZsq{}}\PY{l+s+s2}{ = 0.75)}\PY{l+s+s2}{\PYZdq{}}\PY{p}{)}
\end{Verbatim}
\end{tcolorbox}

    \begin{Verbatim}[commandchars=\\\{\}]
(a) g(x)=c0 en (a,b):   θ̂ = 15.0   SE ≈ 0.0   (valor real = 15.0 )
(b) g(x)=2x, c=2:       θ̂ ≈ 1.00249   SE ≈ 0.002236   (valor real = 1)
(b) g(x)=2x, c=1 (!)    θ̂ ≈ 0.75134   SE ≈ 0.000967   (valor 'truncado' = 0.75)
    \end{Verbatim}

    \hypertarget{ejercicio-3}{%
\section{Ejercicio 3}\label{ejercicio-3}}

 Elabore un programa de cómputo que utilice el método clásico de integración Monte Carlo para calcular el volumen de una esfera de radio $1$. Para corroborar su resultado, recuerde que el volumen de una esfera de radio $r$ es $\frac{4}{3}\pi r^{3}$.
\\

\paragraph{Planteamineto.}Sea $B=\{(x,y,z)\in\mathbb{R}^3:\;x^2+y^2+z^2\le r^2\}$ la esfera de radio $r$ y encajémosla en el cubo
$C=[-r,r]^3$ de volumen $V_C=(2r)^3$. Si tomamos $(X,Y,Z)\sim\mathrm{Unif}([-r,r]^3)$ y definimos
$N=\#\{(X_i,Y_i,Z_i)\in B\}$ sobre una muestra i.i.d.\ de tamaño $n$, entonces
\[
p=\Pr\{(X,Y,Z)\in B\}=\frac{\mathrm{Vol}(B)}{\mathrm{Vol}(C)}=\frac{V}{V_C},
\quad\text{de modo que}\quad
V=V_C\,p.
\]
El estimador clásico \emph{hit-or-miss} resulta de reemplazar $p$ por $\widehat p=N/n$:
\[
\widehat V \;=\; V_C\,\widehat p \;=\; V_C\,\frac{N}{n}.
\]
Esta construcción es la extensión multidimensional del método clásico: en general, si
$0\le g(x,y)\le c$ en un rectángulo $(a_1,b_1)\times(a_2,b_2)$, se tiene
$p=\frac{1}{c(b_1-a_1)(b_2-a_2)}\iint g$ y $\theta=c(b_1-a_1)(b_2-a_2)p$; por analogía, en 3D el volumen buscado
es el volumen del contenedor por la fracción de puntos “favorables”.

Como $N\sim\mathrm{Bin}(n,p)$, la varianza del estimador es
\[
\mathrm{Var}(\widehat V)\;=\;V_C^2\,\frac{p(1-p)}{n}\;\;\;\text{(equiv. plug-in)}\;\;
\widehat{\mathrm{Var}}(\widehat V)\;=\;\frac{\widehat V}{n}\,\big(V_C-\widehat V\big),
\]
y un IC aproximado al $95\%$ es $\widehat V\pm 1.96\,\widehat{\mathrm{SE}}$, con
$\widehat{\mathrm{SE}}=\sqrt{\widehat{\mathrm{Var}}(\widehat V)}$. Estas propiedades provienen directamente del
caso 1D del método clásico (insesgado, varianza $\theta\,[c(b-a)-\theta]/n$) y se trasladan a la versión geométrica
en 3D.

Para $r=1$ se usa $C=[-1,1]^3$ con $V_C=8$ y se compara con el valor exacto $V_{\text{exacto}}=\frac{4}{3}\pi r^3$. 

\subsection{Código}
    \begin{tcolorbox}[breakable, size=fbox, boxrule=1pt, pad at break*=1mm,colback=cellbackground, colframe=cellborder]
\prompt{In}{incolor}{4}{\boxspacing}
\begin{Verbatim}[commandchars=\\\{\}]
\PY{k}{def}\PY{+w}{ }\PY{n+nf}{volumen\PYZus{}esfera\PYZus{}mc}\PY{p}{(}\PY{n}{r}\PY{o}{=}\PY{l+m+mf}{1.0}\PY{p}{,} \PY{n}{n}\PY{o}{=}\PY{l+m+mi}{1\PYZus{}000\PYZus{}000}\PY{p}{,} \PY{n}{seed}\PY{o}{=}\PY{l+m+mi}{42}\PY{p}{)}\PY{p}{:}
\PY{+w}{    }\PY{l+s+sd}{\PYZdq{}\PYZdq{}\PYZdq{}}
\PY{l+s+sd}{    Volumen de una esfera de radio r mediante Monte Carlo clásico (hit\PYZhy{}or\PYZhy{}miss) en 3D.}
\PY{l+s+sd}{    Idea: muestrear uniforme en el cubo [\PYZhy{}r, r]\PYZca{}3 y contar la fracción de puntos}
\PY{l+s+sd}{    que cae dentro de la esfera x\PYZca{}2 + y\PYZca{}2 + z\PYZca{}2 \PYZlt{}= r\PYZca{}2.}
\PY{l+s+sd}{    \PYZdq{}\PYZdq{}\PYZdq{}}
    \PY{n}{rng} \PY{o}{=} \PY{n}{np}\PY{o}{.}\PY{n}{random}\PY{o}{.}\PY{n}{default\PYZus{}rng}\PY{p}{(}\PY{n}{seed}\PY{p}{)}

    \PY{c+c1}{\PYZsh{} 1) Volumen del cubo contenedor: V\PYZus{}cubo = (2r)\PYZca{}3}
    \PY{n}{a}\PY{p}{,} \PY{n}{b} \PY{o}{=} \PY{o}{\PYZhy{}}\PY{n}{r}\PY{p}{,} \PY{n}{r}
    \PY{n}{V\PYZus{}cubo} \PY{o}{=} \PY{p}{(}\PY{n}{b} \PY{o}{\PYZhy{}} \PY{n}{a}\PY{p}{)} \PY{o}{*}\PY{o}{*} \PY{l+m+mi}{3}

    \PY{c+c1}{\PYZsh{} 2) Muestreo uniforme en el cubo}
    \PY{n}{x} \PY{o}{=} \PY{n}{rng}\PY{o}{.}\PY{n}{uniform}\PY{p}{(}\PY{n}{a}\PY{p}{,} \PY{n}{b}\PY{p}{,} \PY{n}{size}\PY{o}{=}\PY{n}{n}\PY{p}{)}
    \PY{n}{y} \PY{o}{=} \PY{n}{rng}\PY{o}{.}\PY{n}{uniform}\PY{p}{(}\PY{n}{a}\PY{p}{,} \PY{n}{b}\PY{p}{,} \PY{n}{size}\PY{o}{=}\PY{n}{n}\PY{p}{)}
    \PY{n}{z} \PY{o}{=} \PY{n}{rng}\PY{o}{.}\PY{n}{uniform}\PY{p}{(}\PY{n}{a}\PY{p}{,} \PY{n}{b}\PY{p}{,} \PY{n}{size}\PY{o}{=}\PY{n}{n}\PY{p}{)}

    \PY{c+c1}{\PYZsh{} 3) Indicador: ¿cayó dentro de la esfera?}
    \PY{n}{inside} \PY{o}{=} \PY{p}{(}\PY{n}{x}\PY{o}{*}\PY{n}{x} \PY{o}{+} \PY{n}{y}\PY{o}{*}\PY{n}{y} \PY{o}{+} \PY{n}{z}\PY{o}{*}\PY{n}{z}\PY{p}{)} \PY{o}{\PYZlt{}}\PY{o}{=} \PY{p}{(}\PY{n}{r}\PY{o}{*}\PY{n}{r}\PY{p}{)}
    \PY{n}{n0} \PY{o}{=} \PY{n+nb}{int}\PY{p}{(}\PY{n}{np}\PY{o}{.}\PY{n}{sum}\PY{p}{(}\PY{n}{inside}\PY{p}{)}\PY{p}{)}

    \PY{c+c1}{\PYZsh{} 4) Estimador clásico: V̂ = V\PYZus{}cubo * (n0/n)}
    \PY{n}{p\PYZus{}hat} \PY{o}{=} \PY{n}{n0} \PY{o}{/} \PY{n}{n}
    \PY{n}{V\PYZus{}hat} \PY{o}{=} \PY{n}{V\PYZus{}cubo} \PY{o}{*} \PY{n}{p\PYZus{}hat}

    \PY{c+c1}{\PYZsh{} 5) Error estándar (plug\PYZhy{}in) por binomial: Var(V̂) = V\PYZus{}cubo\PYZca{}2 * p(1\PYZhy{}p)/n}
    \PY{c+c1}{\PYZsh{}    Forma equivalente: Var(V̂) = (V\PYZus{}hat/n) * (V\PYZus{}cubo \PYZhy{} V\PYZus{}hat)}
    \PY{n}{var\PYZus{}hat} \PY{o}{=} \PY{p}{(}\PY{n}{V\PYZus{}hat} \PY{o}{/} \PY{n}{n}\PY{p}{)} \PY{o}{*} \PY{p}{(}\PY{n}{V\PYZus{}cubo} \PY{o}{\PYZhy{}} \PY{n}{V\PYZus{}hat}\PY{p}{)}
    \PY{n}{se\PYZus{}hat} \PY{o}{=} \PY{n}{np}\PY{o}{.}\PY{n}{sqrt}\PY{p}{(}\PY{n+nb}{max}\PY{p}{(}\PY{n}{var\PYZus{}hat}\PY{p}{,} \PY{l+m+mf}{0.0}\PY{p}{)}\PY{p}{)}

    \PY{c+c1}{\PYZsh{} 6) Valor exacto y un IC 95\PYZpc{} (approx)}
    \PY{n}{V\PYZus{}true} \PY{o}{=} \PY{p}{(}\PY{l+m+mf}{4.0}\PY{o}{/}\PY{l+m+mf}{3.0}\PY{p}{)} \PY{o}{*} \PY{n}{np}\PY{o}{.}\PY{n}{pi} \PY{o}{*} \PY{p}{(}\PY{n}{r}\PY{o}{*}\PY{o}{*}\PY{l+m+mi}{3}\PY{p}{)}
    \PY{n}{ci} \PY{o}{=} \PY{p}{(}\PY{n}{V\PYZus{}hat} \PY{o}{\PYZhy{}} \PY{l+m+mf}{1.96}\PY{o}{*}\PY{n}{se\PYZus{}hat}\PY{p}{,} \PY{n}{V\PYZus{}hat} \PY{o}{+} \PY{l+m+mf}{1.96}\PY{o}{*}\PY{n}{se\PYZus{}hat}\PY{p}{)}

    \PY{k}{return} \PY{p}{\PYZob{}}
        \PY{l+s+s2}{\PYZdq{}}\PY{l+s+s2}{V\PYZus{}hat}\PY{l+s+s2}{\PYZdq{}}\PY{p}{:} \PY{n}{V\PYZus{}hat}\PY{p}{,}
        \PY{l+s+s2}{\PYZdq{}}\PY{l+s+s2}{SE}\PY{l+s+s2}{\PYZdq{}}\PY{p}{:} \PY{n}{se\PYZus{}hat}\PY{p}{,}
        \PY{l+s+s2}{\PYZdq{}}\PY{l+s+s2}{n}\PY{l+s+s2}{\PYZdq{}}\PY{p}{:} \PY{n}{n}\PY{p}{,}
        \PY{l+s+s2}{\PYZdq{}}\PY{l+s+s2}{r}\PY{l+s+s2}{\PYZdq{}}\PY{p}{:} \PY{n}{r}\PY{p}{,}
        \PY{l+s+s2}{\PYZdq{}}\PY{l+s+s2}{V\PYZus{}cubo}\PY{l+s+s2}{\PYZdq{}}\PY{p}{:} \PY{n}{V\PYZus{}cubo}\PY{p}{,}
        \PY{l+s+s2}{\PYZdq{}}\PY{l+s+s2}{n0}\PY{l+s+s2}{\PYZdq{}}\PY{p}{:} \PY{n}{n0}\PY{p}{,}
        \PY{l+s+s2}{\PYZdq{}}\PY{l+s+s2}{p\PYZus{}hat}\PY{l+s+s2}{\PYZdq{}}\PY{p}{:} \PY{n}{p\PYZus{}hat}\PY{p}{,}
        \PY{l+s+s2}{\PYZdq{}}\PY{l+s+s2}{IC95}\PY{l+s+s2}{\PYZdq{}}\PY{p}{:} \PY{n}{ci}\PY{p}{,}
        \PY{l+s+s2}{\PYZdq{}}\PY{l+s+s2}{V\PYZus{}true}\PY{l+s+s2}{\PYZdq{}}\PY{p}{:} \PY{n}{V\PYZus{}true}\PY{p}{,}
    \PY{p}{\PYZcb{}}

\PY{c+c1}{\PYZsh{} Ejemplo de uso:}
\PY{n}{res} \PY{o}{=} \PY{n}{volumen\PYZus{}esfera\PYZus{}mc}\PY{p}{(}\PY{n}{r}\PY{o}{=}\PY{l+m+mf}{1.0}\PY{p}{,} \PY{n}{n}\PY{o}{=}\PY{l+m+mi}{1\PYZus{}000\PYZus{}000}\PY{p}{,} \PY{n}{seed}\PY{o}{=}\PY{l+m+mi}{7}\PY{p}{)}
\PY{n+nb}{print}\PY{p}{(}\PY{l+s+sa}{f}\PY{l+s+s2}{\PYZdq{}}\PY{l+s+s2}{Volumen estimado = }\PY{l+s+si}{\PYZob{}}\PY{n}{res}\PY{p}{[}\PY{l+s+s1}{\PYZsq{}}\PY{l+s+s1}{V\PYZus{}hat}\PY{l+s+s1}{\PYZsq{}}\PY{p}{]}\PY{l+s+si}{:}\PY{l+s+s2}{.5f}\PY{l+s+si}{\PYZcb{}}\PY{l+s+s2}{\PYZdq{}}\PY{p}{)}
\PY{n+nb}{print}\PY{p}{(}\PY{l+s+sa}{f}\PY{l+s+s2}{\PYZdq{}}\PY{l+s+s2}{Error estándar   = }\PY{l+s+si}{\PYZob{}}\PY{n}{res}\PY{p}{[}\PY{l+s+s1}{\PYZsq{}}\PY{l+s+s1}{SE}\PY{l+s+s1}{\PYZsq{}}\PY{p}{]}\PY{l+s+si}{:}\PY{l+s+s2}{.6f}\PY{l+s+si}{\PYZcb{}}\PY{l+s+s2}{\PYZdq{}}\PY{p}{)}
\PY{n+nb}{print}\PY{p}{(}\PY{l+s+sa}{f}\PY{l+s+s2}{\PYZdq{}}\PY{l+s+s2}{IC 95\PYZpc{}           = [}\PY{l+s+si}{\PYZob{}}\PY{n}{res}\PY{p}{[}\PY{l+s+s1}{\PYZsq{}}\PY{l+s+s1}{IC95}\PY{l+s+s1}{\PYZsq{}}\PY{p}{]}\PY{p}{[}\PY{l+m+mi}{0}\PY{p}{]}\PY{l+s+si}{:}\PY{l+s+s2}{.5f}\PY{l+s+si}{\PYZcb{}}\PY{l+s+s2}{, }\PY{l+s+si}{\PYZob{}}\PY{n}{res}\PY{p}{[}\PY{l+s+s1}{\PYZsq{}}\PY{l+s+s1}{IC95}\PY{l+s+s1}{\PYZsq{}}\PY{p}{]}\PY{p}{[}\PY{l+m+mi}{1}\PY{p}{]}\PY{l+s+si}{:}\PY{l+s+s2}{.5f}\PY{l+s+si}{\PYZcb{}}\PY{l+s+s2}{]}\PY{l+s+s2}{\PYZdq{}}\PY{p}{)}
\PY{n+nb}{print}\PY{p}{(}\PY{l+s+sa}{f}\PY{l+s+s2}{\PYZdq{}}\PY{l+s+s2}{Volumen exacto   = }\PY{l+s+si}{\PYZob{}}\PY{n}{res}\PY{p}{[}\PY{l+s+s1}{\PYZsq{}}\PY{l+s+s1}{V\PYZus{}true}\PY{l+s+s1}{\PYZsq{}}\PY{p}{]}\PY{l+s+si}{:}\PY{l+s+s2}{.5f}\PY{l+s+si}{\PYZcb{}}\PY{l+s+s2}{\PYZdq{}}\PY{p}{)}
\end{Verbatim}
\end{tcolorbox}

    \begin{Verbatim}[commandchars=\\\{\}]
Volumen estimado = 4.19170
Error estándar   = 0.003995
IC 95\%           = [4.18387, 4.19953]
Volumen exacto   = 4.18879
    \end{Verbatim}

    \hypertarget{ejercicio-4}{%
\section{Ejercicio 4}\label{ejercicio-4}}


Usando simulación, aproxime el área de la región
\[
\left\{(x,y):\,-1<x<1,\; y>0,\; \sqrt{1-2x^{2}}<y<\sqrt{1-2x^{4}}\right\}.
\]
\paragraph{Planteamiento.}
Sea $R=[a,b]\times[0,1]$ un rectángulo que contenga a la región pedida; 
por ejemplo podemos reducir la varianza tomando $a=-1/\sqrt{2}$, $b=1/\sqrt{2}$
(pues fuera de ese rango $1-2x^2<0$ y la región es vacía).
Si $(X,Y)\sim \mathrm{Unif}(R)$ y $I=\mathbf{1}\{\sqrt{1-2X^{2}}<Y<\sqrt{1-2X^{4}}\}$, entonces
\[
p=\Pr(I=1)=\frac{\text{Área}(\text{región})}{\text{Área}(R)}\qquad\Longrightarrow\qquad
\widehat{A}=\text{Área}(R)\cdot \frac{1}{n}\sum_{i=1}^n I_i.
\]
Este es el método clásico \emph{hit-or-miss}: muestrear puntos uniformes en $R$ 
y contar cuántos caen dentro de la región; es insesgado y 
\(\mathrm{Var}(\widehat{A})= \widehat{A}\,( \text{Área}(R)-\widehat{A})/n\) (plug-in). 
También puede usarse la media muestral en $[a,b]$ con 
$h(x)=\sqrt{1-2x^4}-\sqrt{1-2x^2}$ (no negativa en $[a,b]$): 
\(\widehat{A}_{\text{mm}}=(b-a)\frac{1}{n}\sum_{i=1}^n h(X_i)\), que típicamente tiene menor varianza.

\subsection{Código}
    \begin{tcolorbox}[breakable, size=fbox, boxrule=1pt, pad at break*=1mm,colback=cellbackground, colframe=cellborder]
\prompt{In}{incolor}{5}{\boxspacing}
\begin{Verbatim}[commandchars=\\\{\}]
\PY{k}{def}\PY{+w}{ }\PY{n+nf}{area\PYZus{}region\PYZus{}hit\PYZus{}or\PYZus{}miss}\PY{p}{(}\PY{n}{n}\PY{o}{=}\PY{l+m+mi}{300\PYZus{}000}\PY{p}{,} \PY{n}{seed}\PY{o}{=}\PY{l+m+mi}{123}\PY{p}{,} \PY{n}{tight}\PY{o}{=}\PY{k+kc}{True}\PY{p}{)}\PY{p}{:}
\PY{+w}{    }\PY{l+s+sd}{\PYZdq{}\PYZdq{}\PYZdq{}}
\PY{l+s+sd}{    Estima el área de \PYZob{}(x,y): \PYZhy{}1\PYZlt{}x\PYZlt{}1, y\PYZgt{}0, sqrt(1\PYZhy{}2x\PYZca{}2) \PYZlt{} y \PYZlt{} sqrt(1\PYZhy{}2x\PYZca{}4)\PYZcb{}}
\PY{l+s+sd}{    por \PYZsq{}hit\PYZhy{}or\PYZhy{}miss\PYZsq{}.}

\PY{l+s+sd}{    Si tight=True usa el rectángulo [\PYZhy{}1/sqrt(2), 1/sqrt(2)] x [0, 1],}
\PY{l+s+sd}{    que contiene a la región y reduce varianza.}
\PY{l+s+sd}{    \PYZdq{}\PYZdq{}\PYZdq{}}
    \PY{n}{rng} \PY{o}{=} \PY{n}{np}\PY{o}{.}\PY{n}{random}\PY{o}{.}\PY{n}{default\PYZus{}rng}\PY{p}{(}\PY{n}{seed}\PY{p}{)}
    \PY{k}{if} \PY{n}{tight}\PY{p}{:}
        \PY{n}{a}\PY{p}{,} \PY{n}{b} \PY{o}{=} \PY{o}{\PYZhy{}}\PY{l+m+mi}{1}\PY{o}{/}\PY{n}{np}\PY{o}{.}\PY{n}{sqrt}\PY{p}{(}\PY{l+m+mi}{2}\PY{p}{)}\PY{p}{,} \PY{l+m+mi}{1}\PY{o}{/}\PY{n}{np}\PY{o}{.}\PY{n}{sqrt}\PY{p}{(}\PY{l+m+mi}{2}\PY{p}{)}
    \PY{k}{else}\PY{p}{:}
        \PY{n}{a}\PY{p}{,} \PY{n}{b} \PY{o}{=} \PY{o}{\PYZhy{}}\PY{l+m+mf}{1.0}\PY{p}{,} \PY{l+m+mf}{1.0}
    \PY{n}{c}\PY{p}{,} \PY{n}{d} \PY{o}{=} \PY{l+m+mf}{0.0}\PY{p}{,} \PY{l+m+mf}{1.0}                      \PY{c+c1}{\PYZsh{} en y basta [0,1], pues y\PYZlt{}sqrt(1\PYZhy{}2x\PYZca{}4)≤1}
    \PY{n}{area\PYZus{}rect} \PY{o}{=} \PY{p}{(}\PY{n}{b} \PY{o}{\PYZhy{}} \PY{n}{a}\PY{p}{)} \PY{o}{*} \PY{p}{(}\PY{n}{d} \PY{o}{\PYZhy{}} \PY{n}{c}\PY{p}{)}

    \PY{c+c1}{\PYZsh{} Muestras uniformes en el rectángulo}
    \PY{n}{x} \PY{o}{=} \PY{n}{rng}\PY{o}{.}\PY{n}{uniform}\PY{p}{(}\PY{n}{a}\PY{p}{,} \PY{n}{b}\PY{p}{,} \PY{n}{size}\PY{o}{=}\PY{n}{n}\PY{p}{)}
    \PY{n}{y} \PY{o}{=} \PY{n}{rng}\PY{o}{.}\PY{n}{uniform}\PY{p}{(}\PY{n}{c}\PY{p}{,} \PY{n}{d}\PY{p}{,} \PY{n}{size}\PY{o}{=}\PY{n}{n}\PY{p}{)}

    \PY{c+c1}{\PYZsh{} Radicandos (para evitar sqrt de negativos)}
    \PY{n}{r1} \PY{o}{=} \PY{l+m+mi}{1} \PY{o}{\PYZhy{}} \PY{l+m+mi}{2}\PY{o}{*}\PY{n}{x}\PY{o}{*}\PY{o}{*}\PY{l+m+mi}{2}
    \PY{n}{r2} \PY{o}{=} \PY{l+m+mi}{1} \PY{o}{\PYZhy{}} \PY{l+m+mi}{2}\PY{o}{*}\PY{n}{x}\PY{o}{*}\PY{o}{*}\PY{l+m+mi}{4}
    \PY{n}{valid} \PY{o}{=} \PY{p}{(}\PY{n}{r1} \PY{o}{\PYZgt{}}\PY{o}{=} \PY{l+m+mi}{0}\PY{p}{)} \PY{o}{\PYZam{}} \PY{p}{(}\PY{n}{r2} \PY{o}{\PYZgt{}}\PY{o}{=} \PY{l+m+mi}{0}\PY{p}{)}

    \PY{n}{y\PYZus{}low}  \PY{o}{=} \PY{n}{np}\PY{o}{.}\PY{n}{zeros\PYZus{}like}\PY{p}{(}\PY{n}{x}\PY{p}{)}
    \PY{n}{y\PYZus{}high} \PY{o}{=} \PY{n}{np}\PY{o}{.}\PY{n}{zeros\PYZus{}like}\PY{p}{(}\PY{n}{x}\PY{p}{)}
    \PY{n}{y\PYZus{}low}\PY{p}{[}\PY{n}{valid}\PY{p}{]}  \PY{o}{=} \PY{n}{np}\PY{o}{.}\PY{n}{sqrt}\PY{p}{(}\PY{n}{r1}\PY{p}{[}\PY{n}{valid}\PY{p}{]}\PY{p}{)}
    \PY{n}{y\PYZus{}high}\PY{p}{[}\PY{n}{valid}\PY{p}{]} \PY{o}{=} \PY{n}{np}\PY{o}{.}\PY{n}{sqrt}\PY{p}{(}\PY{n}{r2}\PY{p}{[}\PY{n}{valid}\PY{p}{]}\PY{p}{)}

    \PY{n}{inside} \PY{o}{=} \PY{n}{valid} \PY{o}{\PYZam{}} \PY{p}{(}\PY{n}{y} \PY{o}{\PYZgt{}} \PY{n}{y\PYZus{}low}\PY{p}{)} \PY{o}{\PYZam{}} \PY{p}{(}\PY{n}{y} \PY{o}{\PYZlt{}} \PY{n}{y\PYZus{}high}\PY{p}{)}
    \PY{n}{p\PYZus{}hat} \PY{o}{=} \PY{n}{np}\PY{o}{.}\PY{n}{mean}\PY{p}{(}\PY{n}{inside}\PY{p}{)}
    \PY{n}{area\PYZus{}hat} \PY{o}{=} \PY{n}{area\PYZus{}rect} \PY{o}{*} \PY{n}{p\PYZus{}hat}

    \PY{c+c1}{\PYZsh{} SE (plug\PYZhy{}in) para hit\PYZhy{}or\PYZhy{}miss: Var = A/n * (AreaRect \PYZhy{} A)}
    \PY{n}{var\PYZus{}hat} \PY{o}{=} \PY{p}{(}\PY{n}{area\PYZus{}hat} \PY{o}{/} \PY{n}{n}\PY{p}{)} \PY{o}{*} \PY{p}{(}\PY{n}{area\PYZus{}rect} \PY{o}{\PYZhy{}} \PY{n}{area\PYZus{}hat}\PY{p}{)}
    \PY{n}{se\PYZus{}hat} \PY{o}{=} \PY{n}{np}\PY{o}{.}\PY{n}{sqrt}\PY{p}{(}\PY{n+nb}{max}\PY{p}{(}\PY{n}{var\PYZus{}hat}\PY{p}{,} \PY{l+m+mf}{0.0}\PY{p}{)}\PY{p}{)}
    \PY{n}{ci95} \PY{o}{=} \PY{p}{(}\PY{n}{area\PYZus{}hat} \PY{o}{\PYZhy{}} \PY{l+m+mf}{1.96}\PY{o}{*}\PY{n}{se\PYZus{}hat}\PY{p}{,} \PY{n}{area\PYZus{}hat} \PY{o}{+} \PY{l+m+mf}{1.96}\PY{o}{*}\PY{n}{se\PYZus{}hat}\PY{p}{)}

    \PY{k}{return} \PY{p}{\PYZob{}}\PY{l+s+s2}{\PYZdq{}}\PY{l+s+s2}{area\PYZus{}hat}\PY{l+s+s2}{\PYZdq{}}\PY{p}{:} \PY{n}{area\PYZus{}hat}\PY{p}{,} \PY{l+s+s2}{\PYZdq{}}\PY{l+s+s2}{SE}\PY{l+s+s2}{\PYZdq{}}\PY{p}{:} \PY{n}{se\PYZus{}hat}\PY{p}{,} \PY{l+s+s2}{\PYZdq{}}\PY{l+s+s2}{IC95}\PY{l+s+s2}{\PYZdq{}}\PY{p}{:} \PY{n}{ci95}\PY{p}{,}
            \PY{l+s+s2}{\PYZdq{}}\PY{l+s+s2}{n}\PY{l+s+s2}{\PYZdq{}}\PY{p}{:} \PY{n}{n}\PY{p}{,} \PY{l+s+s2}{\PYZdq{}}\PY{l+s+s2}{rect\PYZus{}area}\PY{l+s+s2}{\PYZdq{}}\PY{p}{:} \PY{n}{area\PYZus{}rect}\PY{p}{,} \PY{l+s+s2}{\PYZdq{}}\PY{l+s+s2}{p\PYZus{}hat}\PY{l+s+s2}{\PYZdq{}}\PY{p}{:} \PY{n}{p\PYZus{}hat}\PY{p}{\PYZcb{}}

\PY{c+c1}{\PYZsh{} Ejemplo rápido:}
\PY{n}{res} \PY{o}{=} \PY{n}{area\PYZus{}region\PYZus{}hit\PYZus{}or\PYZus{}miss}\PY{p}{(}\PY{n}{n}\PY{o}{=}\PY{l+m+mi}{400\PYZus{}000}\PY{p}{,} \PY{n}{seed}\PY{o}{=}\PY{l+m+mi}{7}\PY{p}{,} \PY{n}{tight}\PY{o}{=}\PY{k+kc}{True}\PY{p}{)}
\PY{n}{res}
\end{Verbatim}
\end{tcolorbox}

            \begin{tcolorbox}[breakable, size=fbox, boxrule=.5pt, pad at break*=1mm, opacityfill=0]
\prompt{Out}{outcolor}{5}{\boxspacing}
\begin{Verbatim}[commandchars=\\\{\}]
\{'area\_hat': np.float64(0.2264580177428037),
 'SE': np.float64(0.0008200255578334128),
 'IC95': (np.float64(0.2248507676494502), np.float64(0.2280652678361572)),
 'n': 400000,
 'rect\_area': np.float64(1.414213562373095),
 'p\_hat': np.float64(0.16013)\}
\end{Verbatim}
\end{tcolorbox}
        
    \hypertarget{ejercicio-5}{%
\section{Ejercicio 5}\label{ejercicio-5}}

Suponga que se tienen 3 tipos de riesgos en una aseguradora. Cada riesgo puede provocar
una reclamación cuya cantidad en miles de pesos se modela con una variable aleatoria
$X_i$, para $i=1,2,3$. Los tres riesgos se consideran independientes y, en cada caso,
$X_i \sim \mathrm{Unif}(0,15)$. La aseguradora desea calcular el valor $x$ tal que
\[
\Pr(X_1 + X_2 + X_3 \le x) \;=\; 0.95.
\]
Exprese la probabilidad anterior como la integral triple que corresponde. Aproxime el valor
de $x$ realizando simulaciones de $X_1,X_2,X_3$ y probando distintos valores de $x$.

\subsection{Planteamiento}
Sean $X_1,X_2,X_3 \stackrel{\text{i.i.d.}}{\sim}\mathrm{Unif}(0,15)$ y $S=X_1+X_2+X_3$.
Para $x\in\mathbb{R}$, la probabilidad requerida puede escribirse como volumen bajo
el indicador en el cubo $[0,15]^3$:
\[
\Pr(S\le x)
= \frac{1}{15^3}
\iiint_{[0,15]^3} \mathbf{1}\{u+v+w \le x\}\,du\,dv\,dw,
\]
es decir, la \emph{integral triple} de la función $g(u,v,w)=\mathbf{1}\{u+v+w\le x\}$
en el dominio acotado con densidad uniforme (caso multidimensional del método clásico). 
\emph{Equivalencias Monte Carlo:} $\Pr(S\le x)=E[\mathbf{1}\{S\le x\}]$, de modo que
un estimador insesgado es la media muestral de los indicadores. 


\subsection{Algoritmo de simulación }
\begin{enumerate}
  \item Genere una muestra i.i.d.\ $X_{i1},X_{i2},X_{i3}\sim\mathrm{Unif}(0,15)$ para $i=1,\dots,n$
        y compute $S_i=X_{i1}+X_{i2}+X_{i3}$.
  \item Para cada valor candidato $x$ (una malla, o búsqueda dicotómica), calcule
        \[
        \widehat{p}(x)=\frac{1}{n}\sum_{i=1}^n \mathbf{1}\{S_i\le x\}.
        \]
  \item Seleccione $\hat{x}$ tal que $\widehat{p}(\hat{x})\approx 0.95$.
        Equivalentemente, tome el \emph{cuantil empírico} de orden $0.95$ de $\{S_i\}$.

\end{enumerate}
Este procedimiento es el caso de \emph{media muestral} aplicada a la función $g(u,v,w)=\mathbf{1}\{u+v+w\le x\}$ 
con densidad uniforme en el cubo; ver definición/procedimiento en tus notas.

\subsection{Código}
    \begin{tcolorbox}[breakable, size=fbox, boxrule=1pt, pad at break*=1mm,colback=cellbackground, colframe=cellborder]
\prompt{In}{incolor}{6}{\boxspacing}
\begin{Verbatim}[commandchars=\\\{\}]
\PY{c+c1}{\PYZsh{} \PYZhy{}\PYZhy{}\PYZhy{}\PYZhy{}\PYZhy{}\PYZhy{}\PYZhy{}\PYZhy{}\PYZhy{}\PYZhy{}\PYZhy{}\PYZhy{}\PYZhy{}\PYZhy{}\PYZhy{}\PYZhy{}\PYZhy{}\PYZhy{}\PYZhy{}\PYZhy{}\PYZhy{}\PYZhy{}\PYZhy{}\PYZhy{}\PYZhy{}\PYZhy{}\PYZhy{}\PYZhy{}\PYZhy{}\PYZhy{}\PYZhy{}}
\PY{c+c1}{\PYZsh{} 1) Simulación base: S = X1+X2+X3 con Xi \PYZti{} Unif(0,15)}
\PY{c+c1}{\PYZsh{} \PYZhy{}\PYZhy{}\PYZhy{}\PYZhy{}\PYZhy{}\PYZhy{}\PYZhy{}\PYZhy{}\PYZhy{}\PYZhy{}\PYZhy{}\PYZhy{}\PYZhy{}\PYZhy{}\PYZhy{}\PYZhy{}\PYZhy{}\PYZhy{}\PYZhy{}\PYZhy{}\PYZhy{}\PYZhy{}\PYZhy{}\PYZhy{}\PYZhy{}\PYZhy{}\PYZhy{}\PYZhy{}\PYZhy{}\PYZhy{}\PYZhy{}}
\PY{k}{def}\PY{+w}{ }\PY{n+nf}{simular\PYZus{}sumas}\PY{p}{(}\PY{n}{n}\PY{o}{=}\PY{l+m+mi}{1\PYZus{}000\PYZus{}000}\PY{p}{,} \PY{n}{seed}\PY{o}{=}\PY{l+m+mi}{123}\PY{p}{)}\PY{p}{:}
    \PY{n}{rng} \PY{o}{=} \PY{n}{np}\PY{o}{.}\PY{n}{random}\PY{o}{.}\PY{n}{default\PYZus{}rng}\PY{p}{(}\PY{n}{seed}\PY{p}{)}
    \PY{n}{X} \PY{o}{=} \PY{n}{rng}\PY{o}{.}\PY{n}{uniform}\PY{p}{(}\PY{l+m+mf}{0.0}\PY{p}{,} \PY{l+m+mf}{15.0}\PY{p}{,} \PY{n}{size}\PY{o}{=}\PY{p}{(}\PY{n}{n}\PY{p}{,} \PY{l+m+mi}{3}\PY{p}{)}\PY{p}{)}
    \PY{n}{S} \PY{o}{=} \PY{n}{X}\PY{o}{.}\PY{n}{sum}\PY{p}{(}\PY{n}{axis}\PY{o}{=}\PY{l+m+mi}{1}\PY{p}{)}
    \PY{n}{S}\PY{o}{.}\PY{n}{sort}\PY{p}{(}\PY{p}{)}  \PY{c+c1}{\PYZsh{} ordenar una vez \PYZhy{}\PYZgt{} ECDF eficiente/estable}
    \PY{k}{return} \PY{n}{S}

\PY{c+c1}{\PYZsh{} \PYZhy{}\PYZhy{}\PYZhy{}\PYZhy{}\PYZhy{}\PYZhy{}\PYZhy{}\PYZhy{}\PYZhy{}\PYZhy{}\PYZhy{}\PYZhy{}\PYZhy{}\PYZhy{}\PYZhy{}\PYZhy{}\PYZhy{}\PYZhy{}\PYZhy{}\PYZhy{}\PYZhy{}\PYZhy{}\PYZhy{}\PYZhy{}\PYZhy{}\PYZhy{}\PYZhy{}\PYZhy{}\PYZhy{}\PYZhy{}\PYZhy{}}
\PY{c+c1}{\PYZsh{} 2) ECDF y probabilidad estimada P(S \PYZlt{}= x)}
\PY{c+c1}{\PYZsh{} \PYZhy{}\PYZhy{}\PYZhy{}\PYZhy{}\PYZhy{}\PYZhy{}\PYZhy{}\PYZhy{}\PYZhy{}\PYZhy{}\PYZhy{}\PYZhy{}\PYZhy{}\PYZhy{}\PYZhy{}\PYZhy{}\PYZhy{}\PYZhy{}\PYZhy{}\PYZhy{}\PYZhy{}\PYZhy{}\PYZhy{}\PYZhy{}\PYZhy{}\PYZhy{}\PYZhy{}\PYZhy{}\PYZhy{}\PYZhy{}\PYZhy{}}
\PY{k}{def}\PY{+w}{ }\PY{n+nf}{p\PYZus{}hat\PYZus{}ecdf}\PY{p}{(}\PY{n}{S\PYZus{}sorted}\PY{p}{,} \PY{n}{x}\PY{p}{)}\PY{p}{:}
\PY{+w}{    }\PY{l+s+sd}{\PYZdq{}\PYZdq{}\PYZdq{}}
\PY{l+s+sd}{    Devuelve hat\PYZob{}p\PYZcb{}(x) = (1/n) * \PYZsh{}\PYZob{} S\PYZus{}i \PYZlt{}= x \PYZcb{} usando búsqueda binaria}
\PY{l+s+sd}{    sobre el arreglo ordenado S\PYZus{}sorted.}
\PY{l+s+sd}{    \PYZdq{}\PYZdq{}\PYZdq{}}
    \PY{k+kn}{import}\PY{+w}{ }\PY{n+nn}{bisect}
    \PY{n}{n} \PY{o}{=} \PY{n+nb}{len}\PY{p}{(}\PY{n}{S\PYZus{}sorted}\PY{p}{)}
    \PY{n}{idx} \PY{o}{=} \PY{n}{bisect}\PY{o}{.}\PY{n}{bisect\PYZus{}right}\PY{p}{(}\PY{n}{S\PYZus{}sorted}\PY{p}{,} \PY{n}{x}\PY{p}{)}
    \PY{k}{return} \PY{n}{idx} \PY{o}{/} \PY{n}{n}

\PY{c+c1}{\PYZsh{} \PYZhy{}\PYZhy{}\PYZhy{}\PYZhy{}\PYZhy{}\PYZhy{}\PYZhy{}\PYZhy{}\PYZhy{}\PYZhy{}\PYZhy{}\PYZhy{}\PYZhy{}\PYZhy{}\PYZhy{}\PYZhy{}\PYZhy{}\PYZhy{}\PYZhy{}\PYZhy{}\PYZhy{}\PYZhy{}\PYZhy{}\PYZhy{}\PYZhy{}\PYZhy{}\PYZhy{}\PYZhy{}\PYZhy{}\PYZhy{}\PYZhy{}}
\PY{c+c1}{\PYZsh{} 3) Buscar x con p\PYZus{}hat(x) ≈ objetivo (búsqueda dicotómica sobre ECDF)}
\PY{c+c1}{\PYZsh{} \PYZhy{}\PYZhy{}\PYZhy{}\PYZhy{}\PYZhy{}\PYZhy{}\PYZhy{}\PYZhy{}\PYZhy{}\PYZhy{}\PYZhy{}\PYZhy{}\PYZhy{}\PYZhy{}\PYZhy{}\PYZhy{}\PYZhy{}\PYZhy{}\PYZhy{}\PYZhy{}\PYZhy{}\PYZhy{}\PYZhy{}\PYZhy{}\PYZhy{}\PYZhy{}\PYZhy{}\PYZhy{}\PYZhy{}\PYZhy{}\PYZhy{}}
\PY{k}{def}\PY{+w}{ }\PY{n+nf}{buscar\PYZus{}x\PYZus{}para\PYZus{}prob}\PY{p}{(}\PY{n}{S\PYZus{}sorted}\PY{p}{,} \PY{n}{objetivo}\PY{o}{=}\PY{l+m+mf}{0.95}\PY{p}{,} \PY{n}{lo}\PY{o}{=}\PY{l+m+mf}{0.0}\PY{p}{,} \PY{n}{hi}\PY{o}{=}\PY{l+m+mf}{45.0}\PY{p}{,} \PY{n}{tol}\PY{o}{=}\PY{l+m+mf}{1e\PYZhy{}4}\PY{p}{,} \PY{n}{maxit}\PY{o}{=}\PY{l+m+mi}{60}\PY{p}{)}\PY{p}{:}
    \PY{n}{p\PYZus{}lo} \PY{o}{=} \PY{n}{p\PYZus{}hat\PYZus{}ecdf}\PY{p}{(}\PY{n}{S\PYZus{}sorted}\PY{p}{,} \PY{n}{lo}\PY{p}{)}
    \PY{n}{p\PYZus{}hi} \PY{o}{=} \PY{n}{p\PYZus{}hat\PYZus{}ecdf}\PY{p}{(}\PY{n}{S\PYZus{}sorted}\PY{p}{,} \PY{n}{hi}\PY{p}{)}
    \PY{k}{if} \PY{n}{objetivo} \PY{o}{\PYZlt{}}\PY{o}{=} \PY{n}{p\PYZus{}lo}\PY{p}{:}
        \PY{k}{return} \PY{n}{lo}
    \PY{k}{if} \PY{n}{objetivo} \PY{o}{\PYZgt{}}\PY{o}{=} \PY{n}{p\PYZus{}hi}\PY{p}{:}
        \PY{k}{return} \PY{n}{hi}

    \PY{k}{for} \PY{n}{\PYZus{}} \PY{o+ow}{in} \PY{n+nb}{range}\PY{p}{(}\PY{n}{maxit}\PY{p}{)}\PY{p}{:}
        \PY{n}{mid} \PY{o}{=} \PY{l+m+mf}{0.5} \PY{o}{*} \PY{p}{(}\PY{n}{lo} \PY{o}{+} \PY{n}{hi}\PY{p}{)}
        \PY{n}{p\PYZus{}mid} \PY{o}{=} \PY{n}{p\PYZus{}hat\PYZus{}ecdf}\PY{p}{(}\PY{n}{S\PYZus{}sorted}\PY{p}{,} \PY{n}{mid}\PY{p}{)}
        \PY{k}{if} \PY{n+nb}{abs}\PY{p}{(}\PY{n}{p\PYZus{}mid} \PY{o}{\PYZhy{}} \PY{n}{objetivo}\PY{p}{)} \PY{o}{\PYZlt{}}\PY{o}{=} \PY{n}{tol}\PY{p}{:}
            \PY{k}{return} \PY{n}{mid}
        \PY{k}{if} \PY{n}{p\PYZus{}mid} \PY{o}{\PYZlt{}} \PY{n}{objetivo}\PY{p}{:}
            \PY{n}{lo} \PY{o}{=} \PY{n}{mid}
        \PY{k}{else}\PY{p}{:}
            \PY{n}{hi} \PY{o}{=} \PY{n}{mid}
    \PY{k}{return} \PY{l+m+mf}{0.5} \PY{o}{*} \PY{p}{(}\PY{n}{lo} \PY{o}{+} \PY{n}{hi}\PY{p}{)}

\PY{c+c1}{\PYZsh{} \PYZhy{}\PYZhy{}\PYZhy{}\PYZhy{}\PYZhy{}\PYZhy{}\PYZhy{}\PYZhy{}\PYZhy{}\PYZhy{}\PYZhy{}\PYZhy{}\PYZhy{}\PYZhy{}\PYZhy{}\PYZhy{}\PYZhy{}\PYZhy{}\PYZhy{}\PYZhy{}\PYZhy{}\PYZhy{}\PYZhy{}\PYZhy{}\PYZhy{}\PYZhy{}\PYZhy{}\PYZhy{}\PYZhy{}\PYZhy{}\PYZhy{}}
\PY{c+c1}{\PYZsh{} 4) Cuantil empírico (ECDF): alternativa directa}
\PY{c+c1}{\PYZsh{} \PYZhy{}\PYZhy{}\PYZhy{}\PYZhy{}\PYZhy{}\PYZhy{}\PYZhy{}\PYZhy{}\PYZhy{}\PYZhy{}\PYZhy{}\PYZhy{}\PYZhy{}\PYZhy{}\PYZhy{}\PYZhy{}\PYZhy{}\PYZhy{}\PYZhy{}\PYZhy{}\PYZhy{}\PYZhy{}\PYZhy{}\PYZhy{}\PYZhy{}\PYZhy{}\PYZhy{}\PYZhy{}\PYZhy{}\PYZhy{}\PYZhy{}}
\PY{k}{def}\PY{+w}{ }\PY{n+nf}{cuantil\PYZus{}empirico}\PY{p}{(}\PY{n}{S\PYZus{}sorted}\PY{p}{,} \PY{n}{q}\PY{o}{=}\PY{l+m+mf}{0.95}\PY{p}{)}\PY{p}{:}
\PY{+w}{    }\PY{l+s+sd}{\PYZdq{}\PYZdq{}\PYZdq{}}
\PY{l+s+sd}{    Devuelve el cuantil empírico de orden q de S\PYZus{}sorted (arreglo ordenado).}
\PY{l+s+sd}{    \PYZdq{}\PYZdq{}\PYZdq{}}
    \PY{n}{n} \PY{o}{=} \PY{n+nb}{len}\PY{p}{(}\PY{n}{S\PYZus{}sorted}\PY{p}{)}
    \PY{c+c1}{\PYZsh{} índice \PYZdq{}linear\PYZdq{} equivalente a np.quantile con método \PYZsq{}linear\PYZsq{}}
    \PY{n}{pos} \PY{o}{=} \PY{n}{q} \PY{o}{*} \PY{p}{(}\PY{n}{n} \PY{o}{\PYZhy{}} \PY{l+m+mi}{1}\PY{p}{)}
    \PY{n}{i} \PY{o}{=} \PY{n+nb}{int}\PY{p}{(}\PY{n}{np}\PY{o}{.}\PY{n}{floor}\PY{p}{(}\PY{n}{pos}\PY{p}{)}\PY{p}{)}
    \PY{n}{frac} \PY{o}{=} \PY{n}{pos} \PY{o}{\PYZhy{}} \PY{n}{i}
    \PY{k}{if} \PY{n}{i} \PY{o}{+} \PY{l+m+mi}{1} \PY{o}{\PYZlt{}} \PY{n}{n}\PY{p}{:}
        \PY{k}{return} \PY{p}{(}\PY{l+m+mi}{1} \PY{o}{\PYZhy{}} \PY{n}{frac}\PY{p}{)} \PY{o}{*} \PY{n}{S\PYZus{}sorted}\PY{p}{[}\PY{n}{i}\PY{p}{]} \PY{o}{+} \PY{n}{frac} \PY{o}{*} \PY{n}{S\PYZus{}sorted}\PY{p}{[}\PY{n}{i} \PY{o}{+} \PY{l+m+mi}{1}\PY{p}{]}
    \PY{k}{else}\PY{p}{:}
        \PY{k}{return} \PY{n}{S\PYZus{}sorted}\PY{p}{[}\PY{o}{\PYZhy{}}\PY{l+m+mi}{1}\PY{p}{]}

\PY{c+c1}{\PYZsh{} \PYZhy{}\PYZhy{}\PYZhy{}\PYZhy{}\PYZhy{}\PYZhy{}\PYZhy{}\PYZhy{}\PYZhy{}\PYZhy{}\PYZhy{}\PYZhy{}\PYZhy{}\PYZhy{}\PYZhy{}\PYZhy{}\PYZhy{}\PYZhy{}\PYZhy{}\PYZhy{}\PYZhy{}\PYZhy{}\PYZhy{}\PYZhy{}\PYZhy{}\PYZhy{}\PYZhy{}\PYZhy{}\PYZhy{}\PYZhy{}\PYZhy{}}
\PY{c+c1}{\PYZsh{} Ejecución de ejemplo}
\PY{c+c1}{\PYZsh{} \PYZhy{}\PYZhy{}\PYZhy{}\PYZhy{}\PYZhy{}\PYZhy{}\PYZhy{}\PYZhy{}\PYZhy{}\PYZhy{}\PYZhy{}\PYZhy{}\PYZhy{}\PYZhy{}\PYZhy{}\PYZhy{}\PYZhy{}\PYZhy{}\PYZhy{}\PYZhy{}\PYZhy{}\PYZhy{}\PYZhy{}\PYZhy{}\PYZhy{}\PYZhy{}\PYZhy{}\PYZhy{}\PYZhy{}\PYZhy{}\PYZhy{}}
\PY{k}{if} \PY{n+nv+vm}{\PYZus{}\PYZus{}name\PYZus{}\PYZus{}} \PY{o}{==} \PY{l+s+s2}{\PYZdq{}}\PY{l+s+s2}{\PYZus{}\PYZus{}main\PYZus{}\PYZus{}}\PY{l+s+s2}{\PYZdq{}}\PY{p}{:}
    \PY{c+c1}{\PYZsh{} Generar una sola muestra grande \PYZhy{}\PYZgt{} se usa para todos los x (comparaciones estables)}
    \PY{n}{S} \PY{o}{=} \PY{n}{simular\PYZus{}sumas}\PY{p}{(}\PY{n}{n}\PY{o}{=}\PY{l+m+mi}{1\PYZus{}000\PYZus{}000}\PY{p}{,} \PY{n}{seed}\PY{o}{=}\PY{l+m+mi}{7}\PY{p}{)}

    \PY{c+c1}{\PYZsh{} Aproximar x con P(S\PYZlt{}=x)=0.95 probando distintos x vía búsqueda dicotómica}
    \PY{n}{x95\PYZus{}biseccion} \PY{o}{=} \PY{n}{buscar\PYZus{}x\PYZus{}para\PYZus{}prob}\PY{p}{(}\PY{n}{S}\PY{p}{,} \PY{n}{objetivo}\PY{o}{=}\PY{l+m+mf}{0.95}\PY{p}{)}

    \PY{c+c1}{\PYZsh{} Alternativa: cuantil empírico 0.95 directamente}
    \PY{n}{x95\PYZus{}empirico} \PY{o}{=} \PY{n}{cuantil\PYZus{}empirico}\PY{p}{(}\PY{n}{S}\PY{p}{,} \PY{n}{q}\PY{o}{=}\PY{l+m+mf}{0.95}\PY{p}{)}

    \PY{c+c1}{\PYZsh{} Verificación rápida con normal aproximada}
    \PY{n}{mean\PYZus{}S} \PY{o}{=} \PY{l+m+mi}{3} \PY{o}{*} \PY{l+m+mi}{15} \PY{o}{/} \PY{l+m+mi}{2}         \PY{c+c1}{\PYZsh{} = 22.5}
    \PY{n}{sd\PYZus{}S}   \PY{o}{=} \PY{n}{np}\PY{o}{.}\PY{n}{sqrt}\PY{p}{(}\PY{l+m+mi}{3} \PY{o}{*} \PY{p}{(}\PY{l+m+mi}{15}\PY{o}{*}\PY{o}{*}\PY{l+m+mi}{2}\PY{p}{)} \PY{o}{/} \PY{l+m+mi}{12}\PY{p}{)}  \PY{c+c1}{\PYZsh{} = 7.5}
    \PY{n}{x95\PYZus{}normal} \PY{o}{=} \PY{n}{mean\PYZus{}S} \PY{o}{+} \PY{l+m+mf}{1.645} \PY{o}{*} \PY{n}{sd\PYZus{}S}


    \PY{n+nb}{print}\PY{p}{(}\PY{l+s+sa}{f}\PY{l+s+s2}{\PYZdq{}}\PY{l+s+s2}{x (búsqueda sobre ECDF)  ≈ }\PY{l+s+si}{\PYZob{}}\PY{n}{x95\PYZus{}biseccion}\PY{l+s+si}{:}\PY{l+s+s2}{.6f}\PY{l+s+si}{\PYZcb{}}\PY{l+s+s2}{\PYZdq{}}\PY{p}{)}
    \PY{n+nb}{print}\PY{p}{(}\PY{l+s+sa}{f}\PY{l+s+s2}{\PYZdq{}}\PY{l+s+s2}{x (cuantil empírico 0.95)≈ }\PY{l+s+si}{\PYZob{}}\PY{n}{x95\PYZus{}empirico}\PY{l+s+si}{:}\PY{l+s+s2}{.6f}\PY{l+s+si}{\PYZcb{}}\PY{l+s+s2}{\PYZdq{}}\PY{p}{)}
    \PY{n+nb}{print}\PY{p}{(}\PY{l+s+sa}{f}\PY{l+s+s2}{\PYZdq{}}\PY{l+s+s2}{check normal aprox       ≈ }\PY{l+s+si}{\PYZob{}}\PY{n}{x95\PYZus{}normal}\PY{l+s+si}{:}\PY{l+s+s2}{.6f}\PY{l+s+si}{\PYZcb{}}\PY{l+s+s2}{\PYZdq{}}\PY{p}{)}
\end{Verbatim}
\end{tcolorbox}

    \begin{Verbatim}[commandchars=\\\{\}]
x (búsqueda sobre ECDF)  ≈ 34.958496
x (cuantil empírico 0.95)≈ 34.954878
check normal aprox       ≈ 34.837500
    \end{Verbatim}

    \hypertarget{ejercicio-6}{%
\section{Ejercicio 6}\label{ejercicio-6}}


Se tiene el siguiente modelo para el comportamiento de una partícula al tiempo $t$:
\[
X_t \;=\; 3\sqrt{t} + B_t, \qquad t>0,
\]
en donde $B_t$ es una variable aleatoria con distribución $\mathcal{N}(0.2t,\;0.32t)$.
Exprese la probabilidad siguiente como una integral y después apróximela usando simulación:
\[
\Pr\!\big(X_{10} > 12\big).
\]
\subsection{Modelo y distribución de $X_t$}
Dado que $B_t \sim \mathcal{N}(0.2t,\;0.32t)$ y $X_t=3\sqrt{t}+B_t$, se sigue que
\[
X_t \sim \mathcal{N}\!\big(\mu(t),\sigma^2(t)\big), 
\qquad \mu(t)=3\sqrt{t}+0.2t,\quad \sigma^2(t)=0.32t.
\]
Para $t=10$:
\[
\mu = 3\sqrt{10} + 0.2\cdot 10 = 3\sqrt{10} + 2,\qquad
\sigma^2 = 0.32\cdot 10 = 3.2=\frac{16}{5},\qquad
\sigma=\sqrt{\frac{16}{5}}=\frac{4}{\sqrt{5}}.
\]

\subsection{Probabilidad como integral y cambio de variable}
Como $X_{10}\sim\mathcal{N}(\mu,\sigma^2)$, entonces
\[
\Pr(X_{10}>12)\;=\;\int_{12}^{\infty}
\frac{1}{\sigma\sqrt{2\pi}}\exp\!\left(-\frac{(x-\mu)^2}{2\sigma^2}\right)\,dx.
\]
Estandarizando con $z=\dfrac{x-\mu}{\sigma}$ (luego $dx=\sigma\,dz$), el límite inferior
se convierte en
\[
z_0=\frac{12-\mu}{\sigma}
=\frac{12-(3\sqrt{10}+2)}{\,4/\sqrt{5}\,}
=\frac{10-3\sqrt{10}}{4/\sqrt{5}}
=\frac{\sqrt{5}}{4}\,\big(10-3\sqrt{10}\big).
\]
Por tanto,
\[
\Pr(X_{10}>12)\;=\;\int_{z_0}^{\infty}\frac{1}{\sqrt{2\pi}}e^{-z^2/2}\,dz
\;=\;1-\Phi(z_0),
\]
donde $\Phi$ es la CDF de la normal estándar.

\paragraph{Evaluación numérica.}
Usando $\sqrt{10}\approx 3.16228$ y $\sqrt{5}\approx 2.23607$:
\[
\mu\approx 3(3.16228)+2=11.48683,\qquad
\sigma=\frac{4}{\sqrt{5}}\approx 1.78885,
\]
\[
z_0=\frac{\sqrt{5}}{4}(10-3\sqrt{10})\approx 0.28696,
\qquad
\Pr(X_{10}>12)=1-\Phi(0.28696)\approx 0.387.
\]

\subsection{Aproximación por simulación}
La probabilidad como esperanza de un indicador:
\[
\Pr(X_{10}>12)=\mathbb{E}\big[\mathbf{1}\{X_{10}>12\}\big].
\]
Simulamos $B_{10}^{(i)}\sim \mathcal{N}(0.2\cdot 10,\;0.32\cdot 10)=\mathcal{N}(2,\;3.2)$ e
definimos $X_{10}^{(i)}=3\sqrt{10}+B_{10}^{(i)}$. Con $I_i=\mathbf{1}\{X_{10}^{(i)}>12\}$,
el estimador insesgado es
\[
\widehat{p}=\frac{1}{n}\sum_{i=1}^n I_i,
\qquad
\widehat{\mathrm{SE}}(\widehat{p})=\sqrt{\frac{\widehat{p}(1-\widehat{p})}{n}}.
\]
Un IC aproximado al $95\%$ es $\widehat{p}\pm 1.96\,\widehat{\mathrm{SE}}$.

\subsection{Código}

    \begin{tcolorbox}[breakable, size=fbox, boxrule=1pt, pad at break*=1mm,colback=cellbackground, colframe=cellborder]
\prompt{In}{incolor}{7}{\boxspacing}
\begin{Verbatim}[commandchars=\\\{\}]
\PY{c+c1}{\PYZsh{} \PYZhy{}\PYZhy{}\PYZhy{}\PYZhy{}\PYZhy{}\PYZhy{}\PYZhy{}\PYZhy{}\PYZhy{}\PYZhy{} Analítico \PYZhy{}\PYZhy{}\PYZhy{}\PYZhy{}\PYZhy{}\PYZhy{}\PYZhy{}\PYZhy{}\PYZhy{}\PYZhy{}}
\PY{k}{def}\PY{+w}{ }\PY{n+nf}{normal\PYZus{}cdf}\PY{p}{(}\PY{n}{z}\PY{p}{:} \PY{n+nb}{float}\PY{p}{)} \PY{o}{\PYZhy{}}\PY{o}{\PYZgt{}} \PY{n+nb}{float}\PY{p}{:}
\PY{+w}{    }\PY{l+s+sd}{\PYZdq{}\PYZdq{}\PYZdq{}Φ(z) usando erf (sin dependencias externas).\PYZdq{}\PYZdq{}\PYZdq{}}
    \PY{k}{return} \PY{l+m+mf}{0.5} \PY{o}{*} \PY{p}{(}\PY{l+m+mf}{1.0} \PY{o}{+} \PY{n}{erf}\PY{p}{(}\PY{n}{z} \PY{o}{/} \PY{n}{sqrt}\PY{p}{(}\PY{l+m+mf}{2.0}\PY{p}{)}\PY{p}{)}\PY{p}{)}

\PY{k}{def}\PY{+w}{ }\PY{n+nf}{prob\PYZus{}analitica}\PY{p}{(}\PY{p}{)}\PY{p}{:}
\PY{+w}{    }\PY{l+s+sd}{\PYZdq{}\PYZdq{}\PYZdq{}}
\PY{l+s+sd}{    P(X10 \PYZgt{} 12) con X\PYZus{}t = 3*sqrt(t) + B\PYZus{}t,  B\PYZus{}t \PYZti{} N(0.2 t, 0.32 t), t=10.}
\PY{l+s+sd}{    Devuelve (p, mu, sigma, z0).}
\PY{l+s+sd}{    \PYZdq{}\PYZdq{}\PYZdq{}}
    \PY{n}{t} \PY{o}{=} \PY{l+m+mf}{10.0}
    \PY{n}{mu} \PY{o}{=} \PY{l+m+mf}{3.0}\PY{o}{*}\PY{n}{sqrt}\PY{p}{(}\PY{n}{t}\PY{p}{)} \PY{o}{+} \PY{l+m+mf}{0.2}\PY{o}{*}\PY{n}{t}          \PY{c+c1}{\PYZsh{} = 3*sqrt(10) + 2}
    \PY{n}{sigma} \PY{o}{=} \PY{n}{sqrt}\PY{p}{(}\PY{l+m+mf}{0.32}\PY{o}{*}\PY{n}{t}\PY{p}{)}              \PY{c+c1}{\PYZsh{} = sqrt(3.2) = 4/sqrt(5)}
    \PY{n}{z0} \PY{o}{=} \PY{p}{(}\PY{l+m+mf}{12.0} \PY{o}{\PYZhy{}} \PY{n}{mu}\PY{p}{)} \PY{o}{/} \PY{n}{sigma}
    \PY{n}{p} \PY{o}{=} \PY{l+m+mf}{1.0} \PY{o}{\PYZhy{}} \PY{n}{normal\PYZus{}cdf}\PY{p}{(}\PY{n}{z0}\PY{p}{)}
    \PY{k}{return} \PY{n}{p}\PY{p}{,} \PY{n}{mu}\PY{p}{,} \PY{n}{sigma}\PY{p}{,} \PY{n}{z0}

\PY{c+c1}{\PYZsh{} \PYZhy{}\PYZhy{}\PYZhy{}\PYZhy{}\PYZhy{}\PYZhy{}\PYZhy{}\PYZhy{}\PYZhy{}\PYZhy{} Simulación (media muestral) \PYZhy{}\PYZhy{}\PYZhy{}\PYZhy{}\PYZhy{}\PYZhy{}\PYZhy{}\PYZhy{}\PYZhy{}\PYZhy{}}
\PY{k}{def}\PY{+w}{ }\PY{n+nf}{prob\PYZus{}mc}\PY{p}{(}\PY{n}{n}\PY{o}{=}\PY{l+m+mi}{500\PYZus{}000}\PY{p}{,} \PY{n}{seed}\PY{o}{=}\PY{l+m+mi}{7}\PY{p}{)}\PY{p}{:}
\PY{+w}{    }\PY{l+s+sd}{\PYZdq{}\PYZdq{}\PYZdq{}}
\PY{l+s+sd}{    Estima P(X10 \PYZgt{} 12) por simulación.}
\PY{l+s+sd}{    \PYZdq{}\PYZdq{}\PYZdq{}}
    \PY{n}{t} \PY{o}{=} \PY{l+m+mf}{10.0}
    \PY{n}{mu\PYZus{}B}\PY{p}{,} \PY{n}{var\PYZus{}B} \PY{o}{=} \PY{l+m+mf}{0.2}\PY{o}{*}\PY{n}{t}\PY{p}{,} \PY{l+m+mf}{0.32}\PY{o}{*}\PY{n}{t}       \PY{c+c1}{\PYZsh{} B10 \PYZti{} N(2, 3.2)}
    \PY{n}{sd\PYZus{}B} \PY{o}{=} \PY{n}{sqrt}\PY{p}{(}\PY{n}{var\PYZus{}B}\PY{p}{)}
    \PY{n}{drift} \PY{o}{=} \PY{l+m+mf}{3.0}\PY{o}{*}\PY{n}{sqrt}\PY{p}{(}\PY{n}{t}\PY{p}{)}               \PY{c+c1}{\PYZsh{} 3*sqrt(10)}

    \PY{n}{rng} \PY{o}{=} \PY{n}{np}\PY{o}{.}\PY{n}{random}\PY{o}{.}\PY{n}{default\PYZus{}rng}\PY{p}{(}\PY{n}{seed}\PY{p}{)}
    \PY{n}{Z} \PY{o}{=} \PY{n}{rng}\PY{o}{.}\PY{n}{standard\PYZus{}normal}\PY{p}{(}\PY{n}{n}\PY{p}{)}
    \PY{n}{B10} \PY{o}{=} \PY{n}{mu\PYZus{}B} \PY{o}{+} \PY{n}{sd\PYZus{}B} \PY{o}{*} \PY{n}{Z}
    \PY{n}{X10} \PY{o}{=} \PY{n}{drift} \PY{o}{+} \PY{n}{B10}

    \PY{n}{I} \PY{o}{=} \PY{p}{(}\PY{n}{X10} \PY{o}{\PYZgt{}} \PY{l+m+mf}{12.0}\PY{p}{)}
    \PY{n}{p\PYZus{}hat} \PY{o}{=} \PY{n+nb}{float}\PY{p}{(}\PY{n}{np}\PY{o}{.}\PY{n}{mean}\PY{p}{(}\PY{n}{I}\PY{p}{)}\PY{p}{)}
    \PY{n}{se} \PY{o}{=} \PY{n+nb}{float}\PY{p}{(}\PY{n}{np}\PY{o}{.}\PY{n}{sqrt}\PY{p}{(}\PY{n}{p\PYZus{}hat} \PY{o}{*} \PY{p}{(}\PY{l+m+mf}{1.0} \PY{o}{\PYZhy{}} \PY{n}{p\PYZus{}hat}\PY{p}{)} \PY{o}{/} \PY{n}{n}\PY{p}{)}\PY{p}{)}
    \PY{n}{ci} \PY{o}{=} \PY{p}{(}\PY{n}{p\PYZus{}hat} \PY{o}{\PYZhy{}} \PY{l+m+mf}{1.96}\PY{o}{*}\PY{n}{se}\PY{p}{,} \PY{n}{p\PYZus{}hat} \PY{o}{+} \PY{l+m+mf}{1.96}\PY{o}{*}\PY{n}{se}\PY{p}{)}
    \PY{k}{return} \PY{n}{p\PYZus{}hat}\PY{p}{,} \PY{n}{se}\PY{p}{,} \PY{n}{ci}

\PY{c+c1}{\PYZsh{} \PYZhy{}\PYZhy{}\PYZhy{}\PYZhy{}\PYZhy{}\PYZhy{}\PYZhy{}\PYZhy{}\PYZhy{}\PYZhy{} Ejecución de ejemplo \PYZhy{}\PYZhy{}\PYZhy{}\PYZhy{}\PYZhy{}\PYZhy{}\PYZhy{}\PYZhy{}\PYZhy{}\PYZhy{}}
\PY{k}{if} \PY{n+nv+vm}{\PYZus{}\PYZus{}name\PYZus{}\PYZus{}} \PY{o}{==} \PY{l+s+s2}{\PYZdq{}}\PY{l+s+s2}{\PYZus{}\PYZus{}main\PYZus{}\PYZus{}}\PY{l+s+s2}{\PYZdq{}}\PY{p}{:}
    \PY{n}{p\PYZus{}ex}\PY{p}{,} \PY{n}{mu}\PY{p}{,} \PY{n}{sigma}\PY{p}{,} \PY{n}{z0} \PY{o}{=} \PY{n}{prob\PYZus{}analitica}\PY{p}{(}\PY{p}{)}
    \PY{n}{p\PYZus{}mc}\PY{p}{,} \PY{n}{se\PYZus{}mc}\PY{p}{,} \PY{n}{ci\PYZus{}mc} \PY{o}{=} \PY{n}{prob\PYZus{}mc}\PY{p}{(}\PY{n}{n}\PY{o}{=}\PY{l+m+mi}{500\PYZus{}000}\PY{p}{,} \PY{n}{seed}\PY{o}{=}\PY{l+m+mi}{7}\PY{p}{)}

    \PY{n+nb}{print}\PY{p}{(}\PY{l+s+sa}{f}\PY{l+s+s2}{\PYZdq{}}\PY{l+s+s2}{Analítico:  P(X10\PYZgt{}12) = }\PY{l+s+si}{\PYZob{}}\PY{n}{p\PYZus{}ex}\PY{l+s+si}{:}\PY{l+s+s2}{.6f}\PY{l+s+si}{\PYZcb{}}\PY{l+s+s2}{  }\PY{l+s+s2}{\PYZdq{}}
          \PY{l+s+sa}{f}\PY{l+s+s2}{\PYZdq{}}\PY{l+s+s2}{(mu=}\PY{l+s+si}{\PYZob{}}\PY{n}{mu}\PY{l+s+si}{:}\PY{l+s+s2}{.6f}\PY{l+s+si}{\PYZcb{}}\PY{l+s+s2}{, sigma=}\PY{l+s+si}{\PYZob{}}\PY{n}{sigma}\PY{l+s+si}{:}\PY{l+s+s2}{.6f}\PY{l+s+si}{\PYZcb{}}\PY{l+s+s2}{, z0=}\PY{l+s+si}{\PYZob{}}\PY{p}{(}\PY{n}{z0}\PY{p}{)}\PY{l+s+si}{:}\PY{l+s+s2}{.6f}\PY{l+s+si}{\PYZcb{}}\PY{l+s+s2}{)}\PY{l+s+s2}{\PYZdq{}}\PY{p}{)}
    \PY{n+nb}{print}\PY{p}{(}\PY{l+s+sa}{f}\PY{l+s+s2}{\PYZdq{}}\PY{l+s+s2}{Monte Carlo: p\PYZus{}hat=}\PY{l+s+si}{\PYZob{}}\PY{n}{p\PYZus{}mc}\PY{l+s+si}{:}\PY{l+s+s2}{.6f}\PY{l+s+si}{\PYZcb{}}\PY{l+s+s2}{, SE=}\PY{l+s+si}{\PYZob{}}\PY{n}{se\PYZus{}mc}\PY{l+s+si}{:}\PY{l+s+s2}{.6f}\PY{l+s+si}{\PYZcb{}}\PY{l+s+s2}{, }\PY{l+s+s2}{\PYZdq{}}
          \PY{l+s+sa}{f}\PY{l+s+s2}{\PYZdq{}}\PY{l+s+s2}{IC95\PYZpc{}=[}\PY{l+s+si}{\PYZob{}}\PY{n}{ci\PYZus{}mc}\PY{p}{[}\PY{l+m+mi}{0}\PY{p}{]}\PY{l+s+si}{:}\PY{l+s+s2}{.6f}\PY{l+s+si}{\PYZcb{}}\PY{l+s+s2}{, }\PY{l+s+si}{\PYZob{}}\PY{n}{ci\PYZus{}mc}\PY{p}{[}\PY{l+m+mi}{1}\PY{p}{]}\PY{l+s+si}{:}\PY{l+s+s2}{.6f}\PY{l+s+si}{\PYZcb{}}\PY{l+s+s2}{]}\PY{l+s+s2}{\PYZdq{}}\PY{p}{)}
\end{Verbatim}
\end{tcolorbox}

    \begin{Verbatim}[commandchars=\\\{\}]
Analítico:  P(X10>12) = 0.387106  (mu=11.486833, sigma=1.788854, z0=0.286869)
Monte Carlo: p\_hat=0.387350, SE=0.000689, IC95\%=[0.386000, 0.388700]
    \end{Verbatim}

    \hypertarget{ejercicio-7}{%
\section{Ejercicio 7}\label{ejercicio-7}}


Dentro de la teoría del riesgo se tiene el siguiente modelo para el capital $C_t$ de una
compañía aseguradora al tiempo $t\ge0$:
\[
C_t \;=\; u + c\,t \;-\; \sum_{j=0}^{N_t} Y_j, \qquad t\ge0,
\]
en donde $u\ge0$ y $c>0$ son constantes, $N_t\sim\mathrm{Poisson}(\lambda t)$ con $\lambda>0$,
y $Y_1,Y_2,\dots$ son variables aleatorias independientes entre sí e independientes de $N_t$.
Suponga que $Y_0:=0$ y que $Y_j\sim\mathrm{Exp}(1)$ para $j\ge1$. Suponga también que
$u=1$, $c=3$ y $\lambda=2$. Usando simulación, aproxime:
\begin{enumerate}[label=(\alph*)]
\item $\Pr(C_5<0)$.
\item $\Pr(C_5<0 \mid Y_1<1)$.
\end{enumerate}

\paragraph{Modelo y reexpresión.}
Para $t=5$ se tiene $N_5\sim\mathrm{Poisson}(\lambda t)=\mathrm{Poisson}(10)$ y
\[
C_5 \;=\; u + c\cdot 5 \;-\; \sum_{j=1}^{N_5} Y_j 
\;=\; 16 \;-\; S_{N_5}, \qquad Y_j\stackrel{iid}{\sim}\mathrm{Exp}(1).
\]
Aquí $S_{N_5}$ es la suma compuesta de $N_5$ reclamaciones exponenciales. Por tanto
\[
\Pr(C_5<0) \;=\; \Pr\big(S_{N_5}>16\big) \;=\; 
\mathbb{E}\big[\mathbf{1}\{S_{N_5}>16\}\big],
\]
lo cual se estima por \emph{media muestral} simulando copias independientes del par $(N_5,S_{N_5})$.

\paragraph{Condicional.}
Como $Y_1\sim\mathrm{Exp}(1)$ es independiente de $N_5$ y de $\{Y_j\}_{j\ge2}$, se puede escribir
\[
\Pr(C_5<0 \mid Y_1<1)
\;=\; \mathbb{E}\!\left[\mathbf{1}\{C_5<0\}\,\middle|\, Y_1<1\right]
\;\approx\; \frac{1}{m}\sum_{i:\,Y_1^{(i)}<1} \mathbf{1}\{C_5^{(i)}<0\}.
\]
En la simulación se acopla $Y_1$ dentro de la suma de siniestros (si $N_5\ge1$, se toma
$S_{N_5}=Y_1+\sum_{j=2}^{N_5}Y_j$; si $N_5=0$, entonces $S_{0}=0$). 
\textit{Nota:} si se desea, puede añadirse la lectura alternativa 
$\Pr(C_5<0 \mid N_5\ge1,\;Y_1<1)$, condicionando además a que ocurra al menos un siniestro antes de $t=5$.

\subsection{Código}
    \begin{tcolorbox}[breakable, size=fbox, boxrule=1pt, pad at break*=1mm,colback=cellbackground, colframe=cellborder]
\prompt{In}{incolor}{8}{\boxspacing}
\begin{Verbatim}[commandchars=\\\{\}]
\PY{k}{def}\PY{+w}{ }\PY{n+nf}{sim\PYZus{}ej7}\PY{p}{(}\PY{n}{n}\PY{o}{=}\PY{l+m+mi}{300\PYZus{}000}\PY{p}{,} \PY{n}{u}\PY{o}{=}\PY{l+m+mf}{1.0}\PY{p}{,} \PY{n}{c}\PY{o}{=}\PY{l+m+mf}{3.0}\PY{p}{,} \PY{n}{lam}\PY{o}{=}\PY{l+m+mf}{2.0}\PY{p}{,} \PY{n}{t}\PY{o}{=}\PY{l+m+mf}{5.0}\PY{p}{,} \PY{n}{seed}\PY{o}{=}\PY{l+m+mi}{7}\PY{p}{)}\PY{p}{:}
\PY{+w}{    }\PY{l+s+sd}{\PYZdq{}\PYZdq{}\PYZdq{}}
\PY{l+s+sd}{    Proceso de riesgo clásico:}
\PY{l+s+sd}{      C\PYZus{}t = u + c t \PYZhy{} sum\PYZus{}\PYZob{}j=1\PYZcb{}\PYZca{}\PYZob{}N\PYZus{}t\PYZcb{} Y\PYZus{}j,   N\PYZus{}t \PYZti{} Poisson(lam * t),  Y\PYZus{}j \PYZti{} Exp(1).}

\PY{l+s+sd}{    Devuelve un dict con:}
\PY{l+s+sd}{      \PYZhy{} p\PYZus{}ruin:           estimación de P(C5 \PYZlt{} 0)}
\PY{l+s+sd}{      \PYZhy{} se\PYZus{}ruin:          error estándar (binomial) de p\PYZus{}ruin}
\PY{l+s+sd}{      \PYZhy{} p\PYZus{}cond\PYZus{}uncond:    estimación de P(C5 \PYZlt{} 0 | Y1 \PYZlt{} 1)  (condición \PYZsq{}solo Y1\PYZlt{}1\PYZsq{})}
\PY{l+s+sd}{      \PYZhy{} se\PYZus{}cond\PYZus{}uncond:   su error estándar sobre m = \PYZsh{} \PYZob{}Y1\PYZlt{}1\PYZcb{}}
\PY{l+s+sd}{      \PYZhy{} p\PYZus{}cond\PYZus{}atleast1:  estimación de P(C5 \PYZlt{} 0 | N5\PYZgt{}=1, Y1 \PYZlt{} 1)  (lectura alternativa)}
\PY{l+s+sd}{      \PYZhy{} se\PYZus{}cond\PYZus{}atleast1: su error estándar sobre m = \PYZsh{} \PYZob{}N5\PYZgt{}=1, Y1\PYZlt{}1\PYZcb{}}
\PY{l+s+sd}{      \PYZhy{} mean\PYZus{}N:           media muestral de N5 (control de calidad; \PYZti{} 10)}
\PY{l+s+sd}{    \PYZdq{}\PYZdq{}\PYZdq{}}
    \PY{n}{rng} \PY{o}{=} \PY{n}{np}\PY{o}{.}\PY{n}{random}\PY{o}{.}\PY{n}{default\PYZus{}rng}\PY{p}{(}\PY{n}{seed}\PY{p}{)}
    \PY{n}{premium} \PY{o}{=} \PY{n}{u} \PY{o}{+} \PY{n}{c} \PY{o}{*} \PY{n}{t}  \PY{c+c1}{\PYZsh{} u + c t}

    \PY{c+c1}{\PYZsh{} 1) Número de siniestros}
    \PY{n}{N} \PY{o}{=} \PY{n}{rng}\PY{o}{.}\PY{n}{poisson}\PY{p}{(}\PY{n}{lam} \PY{o}{*} \PY{n}{t}\PY{p}{,} \PY{n}{size}\PY{o}{=}\PY{n}{n}\PY{p}{)}   \PY{c+c1}{\PYZsh{} N5}

    \PY{c+c1}{\PYZsh{} 2) Primer siniestro Y1 (definido siempre; independiente de N)}
    \PY{n}{Y1} \PY{o}{=} \PY{n}{rng}\PY{o}{.}\PY{n}{exponential}\PY{p}{(}\PY{n}{scale}\PY{o}{=}\PY{l+m+mf}{1.0}\PY{p}{,} \PY{n}{size}\PY{o}{=}\PY{n}{n}\PY{p}{)}

    \PY{c+c1}{\PYZsh{} 3) Suma total S = Y1 + Gamma(N\PYZhy{}1,1) si N\PYZgt{}=1; S=0 si N=0}
    \PY{n}{S} \PY{o}{=} \PY{n}{np}\PY{o}{.}\PY{n}{zeros}\PY{p}{(}\PY{n}{n}\PY{p}{)}
    \PY{n}{mask\PYZus{}ge1} \PY{o}{=} \PY{p}{(}\PY{n}{N} \PY{o}{\PYZgt{}}\PY{o}{=} \PY{l+m+mi}{1}\PY{p}{)}
    \PY{n}{mask\PYZus{}ge2} \PY{o}{=} \PY{p}{(}\PY{n}{N} \PY{o}{\PYZgt{}}\PY{o}{=} \PY{l+m+mi}{2}\PY{p}{)}
    \PY{n}{S}\PY{p}{[}\PY{n}{mask\PYZus{}ge1}\PY{p}{]} \PY{o}{=} \PY{n}{Y1}\PY{p}{[}\PY{n}{mask\PYZus{}ge1}\PY{p}{]}                 \PY{c+c1}{\PYZsh{} incluye Y1}
    \PY{k}{if} \PY{n}{np}\PY{o}{.}\PY{n}{any}\PY{p}{(}\PY{n}{mask\PYZus{}ge2}\PY{p}{)}\PY{p}{:}                       \PY{c+c1}{\PYZsh{} faltan N\PYZhy{}1 ≥ 1 reclamaciones}
        \PY{n}{k\PYZus{}rest} \PY{o}{=} \PY{n}{N}\PY{p}{[}\PY{n}{mask\PYZus{}ge2}\PY{p}{]} \PY{o}{\PYZhy{}} \PY{l+m+mi}{1}              \PY{c+c1}{\PYZsh{} parámetros forma (enteros)}
        \PY{n}{S}\PY{p}{[}\PY{n}{mask\PYZus{}ge2}\PY{p}{]} \PY{o}{+}\PY{o}{=} \PY{n}{rng}\PY{o}{.}\PY{n}{gamma}\PY{p}{(}\PY{n}{shape}\PY{o}{=}\PY{n}{k\PYZus{}rest}\PY{p}{,} \PY{n}{scale}\PY{o}{=}\PY{l+m+mf}{1.0}\PY{p}{)}

    \PY{c+c1}{\PYZsh{} 4) Capital al tiempo t}
    \PY{n}{C} \PY{o}{=} \PY{n}{premium} \PY{o}{\PYZhy{}} \PY{n}{S}

    \PY{c+c1}{\PYZsh{} (a) P(C5 \PYZlt{} 0)}
    \PY{n}{ruin} \PY{o}{=} \PY{p}{(}\PY{n}{C} \PY{o}{\PYZlt{}} \PY{l+m+mf}{0.0}\PY{p}{)}
    \PY{n}{p\PYZus{}ruin} \PY{o}{=} \PY{n+nb}{float}\PY{p}{(}\PY{n}{np}\PY{o}{.}\PY{n}{mean}\PY{p}{(}\PY{n}{ruin}\PY{p}{)}\PY{p}{)}
    \PY{n}{se\PYZus{}ruin} \PY{o}{=} \PY{n+nb}{float}\PY{p}{(}\PY{n}{np}\PY{o}{.}\PY{n}{sqrt}\PY{p}{(}\PY{n}{p\PYZus{}ruin} \PY{o}{*} \PY{p}{(}\PY{l+m+mf}{1.0} \PY{o}{\PYZhy{}} \PY{n}{p\PYZus{}ruin}\PY{p}{)} \PY{o}{/} \PY{n}{n}\PY{p}{)}\PY{p}{)}
    \PY{n}{ic95\PYZus{}ruin} \PY{o}{=} \PY{p}{(}\PY{n}{p\PYZus{}ruin} \PY{o}{\PYZhy{}} \PY{l+m+mf}{1.96}\PY{o}{*}\PY{n}{se\PYZus{}ruin}\PY{p}{,} \PY{n}{p\PYZus{}ruin} \PY{o}{+} \PY{l+m+mf}{1.96}\PY{o}{*}\PY{n}{se\PYZus{}ruin}\PY{p}{)}

    \PY{c+c1}{\PYZsh{} (b1) Condición \PYZdq{}solo Y1\PYZlt{}1\PYZdq{}}
    \PY{n}{den1} \PY{o}{=} \PY{p}{(}\PY{n}{Y1} \PY{o}{\PYZlt{}} \PY{l+m+mf}{1.0}\PY{p}{)}
    \PY{n}{m1} \PY{o}{=} \PY{n+nb}{int}\PY{p}{(}\PY{n}{np}\PY{o}{.}\PY{n}{count\PYZus{}nonzero}\PY{p}{(}\PY{n}{den1}\PY{p}{)}\PY{p}{)}
    \PY{n}{p\PYZus{}cond\PYZus{}uncond} \PY{o}{=} \PY{n+nb}{float}\PY{p}{(}\PY{n}{np}\PY{o}{.}\PY{n}{mean}\PY{p}{(}\PY{n}{ruin}\PY{p}{[}\PY{n}{den1}\PY{p}{]}\PY{p}{)}\PY{p}{)} \PY{k}{if} \PY{n}{m1} \PY{o}{\PYZgt{}} \PY{l+m+mi}{0} \PY{k}{else} \PY{n}{np}\PY{o}{.}\PY{n}{nan}
    \PY{n}{se\PYZus{}cond\PYZus{}uncond} \PY{o}{=} \PY{n+nb}{float}\PY{p}{(}\PY{n}{np}\PY{o}{.}\PY{n}{sqrt}\PY{p}{(}\PY{n}{p\PYZus{}cond\PYZus{}uncond} \PY{o}{*} \PY{p}{(}\PY{l+m+mf}{1.0} \PY{o}{\PYZhy{}} \PY{n}{p\PYZus{}cond\PYZus{}uncond}\PY{p}{)} \PY{o}{/} \PY{n}{m1}\PY{p}{)}\PY{p}{)} \PY{k}{if} \PY{n}{m1} \PY{o}{\PYZgt{}} \PY{l+m+mi}{0} \PY{k}{else} \PY{n}{np}\PY{o}{.}\PY{n}{nan}
    \PY{n}{ic95\PYZus{}cu} \PY{o}{=} \PY{p}{(}\PY{n}{p\PYZus{}cond\PYZus{}uncond} \PY{o}{\PYZhy{}} \PY{l+m+mf}{1.96}\PY{o}{*}\PY{n}{se\PYZus{}cond\PYZus{}uncond}\PY{p}{,} \PY{n}{p\PYZus{}cond\PYZus{}uncond} \PY{o}{+} \PY{l+m+mf}{1.96}\PY{o}{*}\PY{n}{se\PYZus{}cond\PYZus{}uncond}\PY{p}{)} \PY{k}{if} \PY{n}{m1} \PY{o}{\PYZgt{}} \PY{l+m+mi}{0} \PY{k}{else} \PY{p}{(}\PY{n}{np}\PY{o}{.}\PY{n}{nan}\PY{p}{,} \PY{n}{np}\PY{o}{.}\PY{n}{nan}\PY{p}{)}

    \PY{c+c1}{\PYZsh{} (b2) Condición \PYZdq{}Y1\PYZlt{}1 y N5\PYZgt{}=1\PYZdq{}}
    \PY{n}{den2} \PY{o}{=} \PY{n}{den1} \PY{o}{\PYZam{}} \PY{p}{(}\PY{n}{N} \PY{o}{\PYZgt{}}\PY{o}{=} \PY{l+m+mi}{1}\PY{p}{)}
    \PY{n}{m2} \PY{o}{=} \PY{n+nb}{int}\PY{p}{(}\PY{n}{np}\PY{o}{.}\PY{n}{count\PYZus{}nonzero}\PY{p}{(}\PY{n}{den2}\PY{p}{)}\PY{p}{)}
    \PY{n}{p\PYZus{}cond\PYZus{}atleast1} \PY{o}{=} \PY{n+nb}{float}\PY{p}{(}\PY{n}{np}\PY{o}{.}\PY{n}{mean}\PY{p}{(}\PY{n}{ruin}\PY{p}{[}\PY{n}{den2}\PY{p}{]}\PY{p}{)}\PY{p}{)} \PY{k}{if} \PY{n}{m2} \PY{o}{\PYZgt{}} \PY{l+m+mi}{0} \PY{k}{else} \PY{n}{np}\PY{o}{.}\PY{n}{nan}
    \PY{n}{se\PYZus{}cond\PYZus{}atleast1} \PY{o}{=} \PY{n+nb}{float}\PY{p}{(}\PY{n}{np}\PY{o}{.}\PY{n}{sqrt}\PY{p}{(}\PY{n}{p\PYZus{}cond\PYZus{}atleast1} \PY{o}{*} \PY{p}{(}\PY{l+m+mf}{1.0} \PY{o}{\PYZhy{}} \PY{n}{p\PYZus{}cond\PYZus{}atleast1}\PY{p}{)} \PY{o}{/} \PY{n}{m2}\PY{p}{)}\PY{p}{)} \PY{k}{if} \PY{n}{m2} \PY{o}{\PYZgt{}} \PY{l+m+mi}{0} \PY{k}{else} \PY{n}{np}\PY{o}{.}\PY{n}{nan}
    \PY{n}{ic95\PYZus{}ca} \PY{o}{=} \PY{p}{(}\PY{n}{p\PYZus{}cond\PYZus{}atleast1} \PY{o}{\PYZhy{}} \PY{l+m+mf}{1.96}\PY{o}{*}\PY{n}{se\PYZus{}cond\PYZus{}atleast1}\PY{p}{,} \PY{n}{p\PYZus{}cond\PYZus{}atleast1} \PY{o}{+} \PY{l+m+mf}{1.96}\PY{o}{*}\PY{n}{se\PYZus{}cond\PYZus{}atleast1}\PY{p}{)} \PY{k}{if} \PY{n}{m2} \PY{o}{\PYZgt{}} \PY{l+m+mi}{0} \PY{k}{else} \PY{p}{(}\PY{n}{np}\PY{o}{.}\PY{n}{nan}\PY{p}{,} \PY{n}{np}\PY{o}{.}\PY{n}{nan}\PY{p}{)}

    \PY{k}{return} \PY{p}{\PYZob{}}
        \PY{l+s+s2}{\PYZdq{}}\PY{l+s+s2}{n}\PY{l+s+s2}{\PYZdq{}}\PY{p}{:} \PY{n}{n}\PY{p}{,}
        \PY{l+s+s2}{\PYZdq{}}\PY{l+s+s2}{premium}\PY{l+s+s2}{\PYZdq{}}\PY{p}{:} \PY{n}{premium}\PY{p}{,}
        \PY{l+s+s2}{\PYZdq{}}\PY{l+s+s2}{mean\PYZus{}N}\PY{l+s+s2}{\PYZdq{}}\PY{p}{:} \PY{n+nb}{float}\PY{p}{(}\PY{n}{np}\PY{o}{.}\PY{n}{mean}\PY{p}{(}\PY{n}{N}\PY{p}{)}\PY{p}{)}\PY{p}{,}
        \PY{l+s+s2}{\PYZdq{}}\PY{l+s+s2}{p\PYZus{}ruin}\PY{l+s+s2}{\PYZdq{}}\PY{p}{:} \PY{n}{p\PYZus{}ruin}\PY{p}{,} \PY{l+s+s2}{\PYZdq{}}\PY{l+s+s2}{se\PYZus{}ruin}\PY{l+s+s2}{\PYZdq{}}\PY{p}{:} \PY{n}{se\PYZus{}ruin}\PY{p}{,} \PY{l+s+s2}{\PYZdq{}}\PY{l+s+s2}{IC95\PYZus{}ruin}\PY{l+s+s2}{\PYZdq{}}\PY{p}{:} \PY{n}{ic95\PYZus{}ruin}\PY{p}{,}
        \PY{l+s+s2}{\PYZdq{}}\PY{l+s+s2}{p\PYZus{}cond\PYZus{}uncond}\PY{l+s+s2}{\PYZdq{}}\PY{p}{:} \PY{n}{p\PYZus{}cond\PYZus{}uncond}\PY{p}{,} \PY{l+s+s2}{\PYZdq{}}\PY{l+s+s2}{se\PYZus{}cond\PYZus{}uncond}\PY{l+s+s2}{\PYZdq{}}\PY{p}{:} \PY{n}{se\PYZus{}cond\PYZus{}uncond}\PY{p}{,} \PY{l+s+s2}{\PYZdq{}}\PY{l+s+s2}{IC95\PYZus{}cond\PYZus{}uncond}\PY{l+s+s2}{\PYZdq{}}\PY{p}{:} \PY{n}{ic95\PYZus{}cu}\PY{p}{,} \PY{l+s+s2}{\PYZdq{}}\PY{l+s+s2}{m\PYZus{}uncond}\PY{l+s+s2}{\PYZdq{}}\PY{p}{:} \PY{n}{m1}\PY{p}{,}
        \PY{l+s+s2}{\PYZdq{}}\PY{l+s+s2}{p\PYZus{}cond\PYZus{}atleast1}\PY{l+s+s2}{\PYZdq{}}\PY{p}{:} \PY{n}{p\PYZus{}cond\PYZus{}atleast1}\PY{p}{,} \PY{l+s+s2}{\PYZdq{}}\PY{l+s+s2}{se\PYZus{}cond\PYZus{}atleast1}\PY{l+s+s2}{\PYZdq{}}\PY{p}{:} \PY{n}{se\PYZus{}cond\PYZus{}atleast1}\PY{p}{,} \PY{l+s+s2}{\PYZdq{}}\PY{l+s+s2}{IC95\PYZus{}cond\PYZus{}atleast1}\PY{l+s+s2}{\PYZdq{}}\PY{p}{:} \PY{n}{ic95\PYZus{}ca}\PY{p}{,} \PY{l+s+s2}{\PYZdq{}}\PY{l+s+s2}{m\PYZus{}atleast1}\PY{l+s+s2}{\PYZdq{}}\PY{p}{:} \PY{n}{m2}
    \PY{p}{\PYZcb{}}

\PY{c+c1}{\PYZsh{} Ejemplo de uso:}
\PY{n}{res} \PY{o}{=} \PY{n}{sim\PYZus{}ej7}\PY{p}{(}\PY{n}{n}\PY{o}{=}\PY{l+m+mi}{300\PYZus{}000}\PY{p}{,} \PY{n}{seed}\PY{o}{=}\PY{l+m+mi}{7}\PY{p}{)}
\PY{k}{for} \PY{n}{k}\PY{p}{,} \PY{n}{v} \PY{o+ow}{in} \PY{n}{res}\PY{o}{.}\PY{n}{items}\PY{p}{(}\PY{p}{)}\PY{p}{:}
    \PY{n+nb}{print}\PY{p}{(}\PY{l+s+sa}{f}\PY{l+s+s2}{\PYZdq{}}\PY{l+s+si}{\PYZob{}}\PY{n}{k}\PY{l+s+si}{\PYZcb{}}\PY{l+s+s2}{: }\PY{l+s+si}{\PYZob{}}\PY{n}{v}\PY{l+s+si}{\PYZcb{}}\PY{l+s+s2}{\PYZdq{}}\PY{p}{)}
\end{Verbatim}
\end{tcolorbox}

    \begin{Verbatim}[commandchars=\\\{\}]
n: 300000
premium: 16.0
mean\_N: 9.992766666666666
p\_ruin: 0.09878
se\_ruin: 0.0005447400346342587
IC95\_ruin: (0.09771230953211686, 0.09984769046788315)
p\_cond\_uncond: 0.07879677337612974
se\_cond\_uncond: 0.0006190339111531578
IC95\_cond\_uncond: (0.07758346691026954, 0.08001007984198993)
m\_uncond: 189424
p\_cond\_atleast1: 0.07880301358435977
se\_cond\_atleast1: 0.0006190808379106504
IC95\_cond\_atleast1: (0.0775896151420549, 0.08001641202666465)
m\_atleast1: 189409
    \end{Verbatim}


\section{Ejercicio 8}


Sea $X\sim\mathcal{N}(0,1)$. Estime $\theta=\Pr(Z>2)$ mediante muestreo por importancia
utilizando la función de densidad exponencial trasladada $f(x)$ que aparece abajo. Tome un
tamaño de muestra suficientemente grande para obtener precisión de dos dígitos. Compare con
el valor que se obtiene de una tabla de probabilidades de la distribución normal.
\[
f(x)=
\begin{cases}
e^{-(x-2)}, & x>2,\\
0, & \text{en otro caso.}
\end{cases}
\]

Sea $Z\sim\mathcal{N}(0,1)$ y
\[
\theta \;=\; \Pr(Z>2)\;=\;\int_{2}^{\infty}\phi(x)\,dx,
\qquad
\phi(x)=\frac{1}{\sqrt{2\pi}}\,e^{-x^{2}/2}.
\]

Usaremos la densidad exponencial trasladada
\[
f(x)=e^{-(x-2)}\,\mathbf{1}_{\{x>2\}}
\quad\Longleftrightarrow\quad
X \sim 2+\mathrm{Exp}(1).
\]

Como el soporte de $f$ es $(2,\infty)$,
\[
\theta
=\int_{2}^{\infty}\phi(x)\,dx
=\int_{2}^{\infty}\frac{\phi(x)}{f(x)}\,f(x)\,dx
=\mathbb{E}_{f}\!\big[w(X)\big],
\]
donde el \emph{peso de importancia} es
\[
w(x)\;=\;\frac{\phi(x)}{f(x)}
=\frac{1}{\sqrt{2\pi}}\exp\!\Big(-\tfrac{x^{2}}{2} + x - 2\Big)
=\phi(x)\,e^{\,x-2},\qquad x>2.
\]
Por lo tanto, un estimador insesgado de $\theta$ es la media muestral de los pesos:
\[
\widehat{\theta}_{\mathrm{IS}}
=\frac{1}{n}\sum_{i=1}^{n} w(X_i),
\qquad X_i \stackrel{iid}{\sim} f.
\]

Sea $s_w^{2}=\tfrac{1}{n-1}\sum_{i=1}^{n}\big(w(X_i)-\overline{w}\big)^{2}$ con
$\overline{w}=\widehat{\theta}_{\mathrm{IS}}$. Entonces
\[
\widehat{\mathrm{SE}}\!\left(\widehat{\theta}_{\mathrm{IS}}\right)=\frac{s_w}{\sqrt{n}},
\qquad
\text{IC 95\%:}\quad
\widehat{\theta}_{\mathrm{IS}}\;\pm\;1.96\,\widehat{\mathrm{SE}}.
\]

El valor “de tabla” es
\[
\theta_{\text{tab}}=1-\Phi(2)\approx 0.02275013,
\]
donde $\Phi$ es la CDF normal estándar. También puede compararse con el
\emph{Monte Carlo crudo} $\widehat{\theta}_{\text{crudo}}=\tfrac{1}{n}\sum\mathbf{1}\{Z_i>2\}$,
$Z_i\sim\mathcal{N}(0,1)$, cuyo error estándar es
$\sqrt{\widehat{\theta}_{\text{crudo}}\,(1-\widehat{\theta}_{\text{crudo}})/n}$.

Si se desea un \emph{half-width} (radio del IC) no mayor que $\varepsilon$
(p.\,ej.\ $\varepsilon=5\times10^{-4}$ para estabilizar dos dígitos en $0.02$),
una aproximación es
\[
n \;\gtrsim\; \frac{(1.96)^{2}\,\mathrm{Var}_{f}(w)}{\varepsilon^{2}}
\quad\text{(usar un ``piloto'' para estimar $\mathrm{Var}_{f}(w)$ y redondear hacia arriba).}
\]

\paragraph{Algoritmo (pseudocódigo).}
\begin{enumerate}
  \item Fijar $n$ grande (o calcularlo con un piloto).
  \item Generar $X_i = 2 + E_i$, $E_i\sim\mathrm{Exp}(1)$, $i=1,\dots,n$.
  \item Calcular $w_i = \phi(X_i)\,e^{\,X_i-2}$.
  \item Estimar $\widehat{\theta}_{\mathrm{IS}}=\frac{1}{n}\sum w_i$,
        su $\widehat{\mathrm{SE}} = s_w/\sqrt{n}$ e IC95\%.
  \item Comparar con $\theta_{\text{tab}}=1-\Phi(2)$ (y, si se desea, con el MC crudo).
\end{enumerate}


    \hypertarget{ejercicio-9}{%
\section{Ejercicio 9}\label{ejercicio-9}}


Sea $(X,Y)$ un vector aleatorio con distribución $\mathrm{Unif}(-1,1)\times\mathrm{Unif}(-1,1)$.
Use muestreo condicional para encontrar una aproximación a las probabilidades que aparecen abajo.
Calcule el valor exacto de estas probabilidades y compruebe si las aproximaciones obtenidas son razonables.
\begin{enumerate}[label=(\alph*)]
\item $\Pr(4X-2Y>0)$.
\item $\Pr(X^2+Y^2<1)$.
\item $\Pr\!\left(\dfrac{X^2}{3}+\dfrac{Y^2}{2}<1\right)$.
\end{enumerate}


\subsection{(a) \texorpdfstring{$\Pr(4X-2Y>0)$}{Pr(4X-2Y>0)}}
El evento es $Y<2X$. Condicionando en $X=x$ y usando que $Y\sim\mathrm{Unif}(-1,1)$,
\[
p(x)=\Pr(Y<2x\mid X=x)
=\begin{cases}
0, & x\le -\tfrac12,\\[4pt]
\displaystyle \frac{2x+1}{2}, & -\tfrac12<x<\tfrac12,\\[8pt]
1, & x\ge \tfrac12 .
\end{cases}
\]
Entonces
\[
\Pr(4X-2Y>0)
=\int_{-1}^{1} p(x)\,\frac12\,dx
=\frac12\left[\int_{-1/2}^{1/2}\frac{2x+1}{2}\,dx+\int_{1/2}^{1} 1\,dx\right].
\]
Cálculo:
\[
\int_{-1/2}^{1/2}\!\!\frac{2x+1}{2}\,dx
=\frac14\!\int_{-1/2}^{1/2}\!(2x+1)\,dx
=\frac14\Big[x^2+x\Big]_{-1/2}^{1/2}
=\frac14\Big(\tfrac34-(-\tfrac14)\Big)=\frac14,
\]
\[
\int_{1/2}^{1}\!1\,dx=\frac12,
\qquad
\Rightarrow\quad
\Pr(4X-2Y>0)=\frac12\left(\frac14+\frac12\right)=\boxed{\tfrac12}.
\]

\subsection{(b) \texorpdfstring{$\Pr(X^2+Y^2<1)$}{Pr(X^2+Y^2<1)}}
Dado $X=x$, el evento es $\lvert Y\rvert<\sqrt{1-x^2}$ (si $|x|<1$; si $|x|\ge1$ la probabilidad es $0$). 
Como $Y\sim\mathrm{Unif}(-1,1)$, la longitud del intervalo admisible para $Y$ es $2\sqrt{1-x^2}$, y por tanto
\[
p(x)=\Pr(|Y|<\sqrt{1-x^2}\mid X=x)=\frac{2\sqrt{1-x^2}}{2}=\sqrt{1-x^2},\qquad |x|<1.
\]
Luego
\[
\Pr(X^2+Y^2<1)=\int_{-1}^{1}\sqrt{1-x^2}\,\frac12\,dx
=\frac12\int_{-1}^{1}\sqrt{1-x^2}\,dx.
\]
Usando una primitiva estándar 
$\displaystyle \int \sqrt{1-x^2}\,dx=\tfrac12\!\left(x\sqrt{1-x^2}+\arcsin x\right)+C$,
\[
\int_{-1}^{1}\sqrt{1-x^2}\,dx
=\left[\tfrac12\!\left(x\sqrt{1-x^2}+\arcsin x\right)\right]_{-1}^{1}
=\tfrac12\!\left(\tfrac{\pi}{2}-(-\tfrac{\pi}{2})\right)=\frac{\pi}{2}.
\]
Así,
\[
\Pr(X^2+Y^2<1)=\frac12\cdot\frac{\pi}{2}=\boxed{\frac{\pi}{4}}\approx 0.7854.
\]

\subsection{(c) \texorpdfstring{$\Pr\!\big(\frac{X^2}{3}+\frac{Y^2}{2}<1\big)$}{Pr(X^2/3 + Y^2/2 < 1)}}
Dado $X=x$, la condición es $\lvert Y\rvert<\sqrt{\,2\big(1-\tfrac{x^2}{3}\big)}$.
Para todo $|x|<1$,
\[
1-\frac{x^2}{3}>\frac{2}{3}
\;\Rightarrow\;
2\!\left(1-\frac{x^2}{3}\right)>\frac{4}{3}
\;\Rightarrow\;
\sqrt{\,2\!\left(1-\frac{x^2}{3}\right)}>\sqrt{\frac{4}{3}}>1.
\]
Como $Y\in(-1,1)$, la desigualdad se satisface \emph{siempre}; por lo tanto $p(x)\equiv1$ y
\[
\Pr\!\left(\frac{X^2}{3}+\frac{Y^2}{2}<1\right)=\int_{-1}^{1}1\cdot \frac12\,dx=\boxed{1}.
\]

\[
\boxed{\Pr(4X-2Y>0)=\tfrac12},\qquad
\boxed{\Pr(X^2+Y^2<1)=\tfrac{\pi}{4}},\qquad
\boxed{\Pr\!\big(\tfrac{X^2}{3}+\tfrac{Y^2}{2}<1\big)=1}.
\]

\subsection{Código}\label{codigo}
    \begin{tcolorbox}[breakable, size=fbox, boxrule=1pt, pad at break*=1mm,colback=cellbackground, colframe=cellborder]
\prompt{In}{incolor}{9}{\boxspacing}
\begin{Verbatim}[commandchars=\\\{\}]
\PY{c+c1}{\PYZsh{} \PYZhy{}\PYZhy{}\PYZhy{}\PYZhy{}\PYZhy{}\PYZhy{}\PYZhy{}\PYZhy{}\PYZhy{}\PYZhy{}\PYZhy{}\PYZhy{}\PYZhy{}\PYZhy{}\PYZhy{}\PYZhy{}\PYZhy{}\PYZhy{}\PYZhy{}\PYZhy{}\PYZhy{}\PYZhy{}\PYZhy{}\PYZhy{}\PYZhy{}\PYZhy{}\PYZhy{}}
\PY{c+c1}{\PYZsh{} Probabilidades exactas}
\PY{c+c1}{\PYZsh{} \PYZhy{}\PYZhy{}\PYZhy{}\PYZhy{}\PYZhy{}\PYZhy{}\PYZhy{}\PYZhy{}\PYZhy{}\PYZhy{}\PYZhy{}\PYZhy{}\PYZhy{}\PYZhy{}\PYZhy{}\PYZhy{}\PYZhy{}\PYZhy{}\PYZhy{}\PYZhy{}\PYZhy{}\PYZhy{}\PYZhy{}\PYZhy{}\PYZhy{}\PYZhy{}\PYZhy{}}
\PY{k}{def}\PY{+w}{ }\PY{n+nf}{exact\PYZus{}a}\PY{p}{(}\PY{p}{)}\PY{p}{:} \PY{k}{return} \PY{l+m+mf}{0.5}
\PY{k}{def}\PY{+w}{ }\PY{n+nf}{exact\PYZus{}b}\PY{p}{(}\PY{p}{)}\PY{p}{:} \PY{k}{return} \PY{n}{np}\PY{o}{.}\PY{n}{pi} \PY{o}{/} \PY{l+m+mf}{4.0}
\PY{k}{def}\PY{+w}{ }\PY{n+nf}{exact\PYZus{}c}\PY{p}{(}\PY{p}{)}\PY{p}{:} \PY{k}{return} \PY{l+m+mf}{1.0}

\PY{c+c1}{\PYZsh{} \PYZhy{}\PYZhy{}\PYZhy{}\PYZhy{}\PYZhy{}\PYZhy{}\PYZhy{}\PYZhy{}\PYZhy{}\PYZhy{}\PYZhy{}\PYZhy{}\PYZhy{}\PYZhy{}\PYZhy{}\PYZhy{}\PYZhy{}\PYZhy{}\PYZhy{}\PYZhy{}\PYZhy{}\PYZhy{}\PYZhy{}\PYZhy{}\PYZhy{}\PYZhy{}\PYZhy{}}
\PY{c+c1}{\PYZsh{} p(x) condicionales}
\PY{c+c1}{\PYZsh{} \PYZhy{}\PYZhy{}\PYZhy{}\PYZhy{}\PYZhy{}\PYZhy{}\PYZhy{}\PYZhy{}\PYZhy{}\PYZhy{}\PYZhy{}\PYZhy{}\PYZhy{}\PYZhy{}\PYZhy{}\PYZhy{}\PYZhy{}\PYZhy{}\PYZhy{}\PYZhy{}\PYZhy{}\PYZhy{}\PYZhy{}\PYZhy{}\PYZhy{}\PYZhy{}\PYZhy{}}
\PY{k}{def}\PY{+w}{ }\PY{n+nf}{p\PYZus{}a\PYZus{}of\PYZus{}x}\PY{p}{(}\PY{n}{x}\PY{p}{)}\PY{p}{:}
    \PY{c+c1}{\PYZsh{} piecewise: 0, (2x+1)/2, 1}
    \PY{n}{out} \PY{o}{=} \PY{n}{np}\PY{o}{.}\PY{n}{zeros\PYZus{}like}\PY{p}{(}\PY{n}{x}\PY{p}{,} \PY{n}{dtype}\PY{o}{=}\PY{n+nb}{float}\PY{p}{)}
    \PY{n}{mask\PYZus{}mid} \PY{o}{=} \PY{p}{(}\PY{n}{x} \PY{o}{\PYZgt{}} \PY{o}{\PYZhy{}}\PY{l+m+mf}{0.5}\PY{p}{)} \PY{o}{\PYZam{}} \PY{p}{(}\PY{n}{x} \PY{o}{\PYZlt{}} \PY{l+m+mf}{0.5}\PY{p}{)}
    \PY{n}{mask\PYZus{}hi}  \PY{o}{=} \PY{p}{(}\PY{n}{x} \PY{o}{\PYZgt{}}\PY{o}{=} \PY{l+m+mf}{0.5}\PY{p}{)}
    \PY{n}{out}\PY{p}{[}\PY{n}{mask\PYZus{}mid}\PY{p}{]} \PY{o}{=} \PY{p}{(}\PY{l+m+mf}{2.0} \PY{o}{*} \PY{n}{x}\PY{p}{[}\PY{n}{mask\PYZus{}mid}\PY{p}{]} \PY{o}{+} \PY{l+m+mf}{1.0}\PY{p}{)} \PY{o}{/} \PY{l+m+mf}{2.0}
    \PY{n}{out}\PY{p}{[}\PY{n}{mask\PYZus{}hi}\PY{p}{]}  \PY{o}{=} \PY{l+m+mf}{1.0}
    \PY{k}{return} \PY{n}{out}

\PY{k}{def}\PY{+w}{ }\PY{n+nf}{p\PYZus{}b\PYZus{}of\PYZus{}x}\PY{p}{(}\PY{n}{x}\PY{p}{)}\PY{p}{:}
    \PY{c+c1}{\PYZsh{} sqrt(1 \PYZhy{} x\PYZca{}2), safe}
    \PY{n}{val} \PY{o}{=} \PY{l+m+mf}{1.0} \PY{o}{\PYZhy{}} \PY{n}{x}\PY{o}{*}\PY{n}{x}
    \PY{n}{val}\PY{p}{[}\PY{n}{val} \PY{o}{\PYZlt{}} \PY{l+m+mi}{0}\PY{p}{]} \PY{o}{=} \PY{l+m+mf}{0.0}
    \PY{k}{return} \PY{n}{np}\PY{o}{.}\PY{n}{sqrt}\PY{p}{(}\PY{n}{val}\PY{p}{)}

\PY{k}{def}\PY{+w}{ }\PY{n+nf}{p\PYZus{}c\PYZus{}of\PYZus{}x}\PY{p}{(}\PY{n}{x}\PY{p}{)}\PY{p}{:}
    \PY{c+c1}{\PYZsh{} always 1 on (\PYZhy{}1,1)}
    \PY{k}{return} \PY{n}{np}\PY{o}{.}\PY{n}{ones\PYZus{}like}\PY{p}{(}\PY{n}{x}\PY{p}{,} \PY{n}{dtype}\PY{o}{=}\PY{n+nb}{float}\PY{p}{)}

\PY{c+c1}{\PYZsh{} \PYZhy{}\PYZhy{}\PYZhy{}\PYZhy{}\PYZhy{}\PYZhy{}\PYZhy{}\PYZhy{}\PYZhy{}\PYZhy{}\PYZhy{}\PYZhy{}\PYZhy{}\PYZhy{}\PYZhy{}\PYZhy{}\PYZhy{}\PYZhy{}\PYZhy{}\PYZhy{}\PYZhy{}\PYZhy{}\PYZhy{}\PYZhy{}\PYZhy{}\PYZhy{}\PYZhy{}}
\PY{c+c1}{\PYZsh{} Estimadores por muestreo condicional}
\PY{c+c1}{\PYZsh{} \PYZhy{}\PYZhy{}\PYZhy{}\PYZhy{}\PYZhy{}\PYZhy{}\PYZhy{}\PYZhy{}\PYZhy{}\PYZhy{}\PYZhy{}\PYZhy{}\PYZhy{}\PYZhy{}\PYZhy{}\PYZhy{}\PYZhy{}\PYZhy{}\PYZhy{}\PYZhy{}\PYZhy{}\PYZhy{}\PYZhy{}\PYZhy{}\PYZhy{}\PYZhy{}\PYZhy{}}
\PY{k}{def}\PY{+w}{ }\PY{n+nf}{cond\PYZus{}estimator}\PY{p}{(}\PY{n}{p\PYZus{}of\PYZus{}x}\PY{p}{,} \PY{n}{n}\PY{o}{=}\PY{l+m+mi}{200\PYZus{}000}\PY{p}{,} \PY{n}{seed}\PY{o}{=}\PY{l+m+mi}{7}\PY{p}{)}\PY{p}{:}
    \PY{n}{rng} \PY{o}{=} \PY{n}{np}\PY{o}{.}\PY{n}{random}\PY{o}{.}\PY{n}{default\PYZus{}rng}\PY{p}{(}\PY{n}{seed}\PY{p}{)}
    \PY{n}{x} \PY{o}{=} \PY{n}{rng}\PY{o}{.}\PY{n}{uniform}\PY{p}{(}\PY{o}{\PYZhy{}}\PY{l+m+mf}{1.0}\PY{p}{,} \PY{l+m+mf}{1.0}\PY{p}{,} \PY{n}{size}\PY{o}{=}\PY{n}{n}\PY{p}{)}      \PY{c+c1}{\PYZsh{} X \PYZti{} Unif(\PYZhy{}1,1)}
    \PY{n}{p\PYZus{}vals} \PY{o}{=} \PY{n}{p\PYZus{}of\PYZus{}x}\PY{p}{(}\PY{n}{x}\PY{p}{)}                      \PY{c+c1}{\PYZsh{} p(X)}
    \PY{n}{theta\PYZus{}hat} \PY{o}{=} \PY{n+nb}{float}\PY{p}{(}\PY{n}{np}\PY{o}{.}\PY{n}{mean}\PY{p}{(}\PY{n}{p\PYZus{}vals}\PY{p}{)}\PY{p}{)} \PY{o}{/} \PY{l+m+mf}{1.0}  \PY{c+c1}{\PYZsh{} E[p(X)] (f\PYZus{}X=1/2 se refleja en p(x) ya definido? \PYZhy{}\PYZgt{} p(x) arriba ya es prob condicional; su promedio directo es la prob, porque usamos media muestral sobre X \PYZti{} Unif(\PYZhy{}1,1) y la densidad 1/2 se incorpora en la esperanza \PYZhy{}\PYZgt{} tomar mean(p\PYZus{}vals) está bien si x \PYZti{} Unif(\PYZhy{}1,1) con peso uniforme; estrictamente E[p(X)] = ∫ p(x) f\PYZus{}X dx; discretamente es el promedio simple al muestrear de f\PYZus{}X)}
    \PY{c+c1}{\PYZsh{} Error estándar (de la media de p(X))}
    \PY{n}{se} \PY{o}{=} \PY{n+nb}{float}\PY{p}{(}\PY{n}{np}\PY{o}{.}\PY{n}{std}\PY{p}{(}\PY{n}{p\PYZus{}vals}\PY{p}{,} \PY{n}{ddof}\PY{o}{=}\PY{l+m+mi}{1}\PY{p}{)} \PY{o}{/} \PY{n}{np}\PY{o}{.}\PY{n}{sqrt}\PY{p}{(}\PY{n}{n}\PY{p}{)}\PY{p}{)}
    \PY{n}{ci} \PY{o}{=} \PY{p}{(}\PY{n}{theta\PYZus{}hat} \PY{o}{\PYZhy{}} \PY{l+m+mf}{1.96} \PY{o}{*} \PY{n}{se}\PY{p}{,} \PY{n}{theta\PYZus{}hat} \PY{o}{+} \PY{l+m+mf}{1.96} \PY{o}{*} \PY{n}{se}\PY{p}{)}
    \PY{k}{return} \PY{n}{theta\PYZus{}hat}\PY{p}{,} \PY{n}{se}\PY{p}{,} \PY{n}{ci}

\PY{c+c1}{\PYZsh{} \PYZhy{}\PYZhy{}\PYZhy{}\PYZhy{}\PYZhy{}\PYZhy{}\PYZhy{}\PYZhy{}\PYZhy{}\PYZhy{}\PYZhy{}\PYZhy{}\PYZhy{}\PYZhy{}\PYZhy{}\PYZhy{}\PYZhy{}\PYZhy{}\PYZhy{}\PYZhy{}\PYZhy{}\PYZhy{}\PYZhy{}\PYZhy{}\PYZhy{}\PYZhy{}\PYZhy{}}
\PY{c+c1}{\PYZsh{} Estimadores “crudos” para comparar (opcional)}
\PY{c+c1}{\PYZsh{} \PYZhy{}\PYZhy{}\PYZhy{}\PYZhy{}\PYZhy{}\PYZhy{}\PYZhy{}\PYZhy{}\PYZhy{}\PYZhy{}\PYZhy{}\PYZhy{}\PYZhy{}\PYZhy{}\PYZhy{}\PYZhy{}\PYZhy{}\PYZhy{}\PYZhy{}\PYZhy{}\PYZhy{}\PYZhy{}\PYZhy{}\PYZhy{}\PYZhy{}\PYZhy{}\PYZhy{}}
\PY{k}{def}\PY{+w}{ }\PY{n+nf}{crude\PYZus{}estimator}\PY{p}{(}\PY{n}{event\PYZus{}fn}\PY{p}{,} \PY{n}{n}\PY{o}{=}\PY{l+m+mi}{200\PYZus{}000}\PY{p}{,} \PY{n}{seed}\PY{o}{=}\PY{l+m+mi}{7}\PY{p}{)}\PY{p}{:}
    \PY{n}{rng} \PY{o}{=} \PY{n}{np}\PY{o}{.}\PY{n}{random}\PY{o}{.}\PY{n}{default\PYZus{}rng}\PY{p}{(}\PY{n}{seed}\PY{p}{)}
    \PY{n}{x} \PY{o}{=} \PY{n}{rng}\PY{o}{.}\PY{n}{uniform}\PY{p}{(}\PY{o}{\PYZhy{}}\PY{l+m+mf}{1.0}\PY{p}{,} \PY{l+m+mf}{1.0}\PY{p}{,} \PY{n}{size}\PY{o}{=}\PY{n}{n}\PY{p}{)}
    \PY{n}{y} \PY{o}{=} \PY{n}{rng}\PY{o}{.}\PY{n}{uniform}\PY{p}{(}\PY{o}{\PYZhy{}}\PY{l+m+mf}{1.0}\PY{p}{,} \PY{l+m+mf}{1.0}\PY{p}{,} \PY{n}{size}\PY{o}{=}\PY{n}{n}\PY{p}{)}
    \PY{n}{I} \PY{o}{=} \PY{n}{event\PYZus{}fn}\PY{p}{(}\PY{n}{x}\PY{p}{,} \PY{n}{y}\PY{p}{)}\PY{o}{.}\PY{n}{astype}\PY{p}{(}\PY{n+nb}{float}\PY{p}{)}
    \PY{n}{p} \PY{o}{=} \PY{n+nb}{float}\PY{p}{(}\PY{n}{np}\PY{o}{.}\PY{n}{mean}\PY{p}{(}\PY{n}{I}\PY{p}{)}\PY{p}{)}
    \PY{n}{se} \PY{o}{=} \PY{n+nb}{float}\PY{p}{(}\PY{n}{np}\PY{o}{.}\PY{n}{sqrt}\PY{p}{(}\PY{n}{p} \PY{o}{*} \PY{p}{(}\PY{l+m+mi}{1} \PY{o}{\PYZhy{}} \PY{n}{p}\PY{p}{)} \PY{o}{/} \PY{n}{n}\PY{p}{)}\PY{p}{)}
    \PY{n}{ci} \PY{o}{=} \PY{p}{(}\PY{n}{p} \PY{o}{\PYZhy{}} \PY{l+m+mf}{1.96} \PY{o}{*} \PY{n}{se}\PY{p}{,} \PY{n}{p} \PY{o}{+} \PY{l+m+mf}{1.96} \PY{o}{*} \PY{n}{se}\PY{p}{)}
    \PY{k}{return} \PY{n}{p}\PY{p}{,} \PY{n}{se}\PY{p}{,} \PY{n}{ci}

\PY{k}{def}\PY{+w}{ }\PY{n+nf}{event\PYZus{}a}\PY{p}{(}\PY{n}{x}\PY{p}{,} \PY{n}{y}\PY{p}{)}\PY{p}{:} \PY{k}{return} \PY{p}{(}\PY{l+m+mf}{4.0}\PY{o}{*}\PY{n}{x} \PY{o}{\PYZhy{}} \PY{l+m+mf}{2.0}\PY{o}{*}\PY{n}{y}\PY{p}{)} \PY{o}{\PYZgt{}} \PY{l+m+mf}{0.0}
\PY{k}{def}\PY{+w}{ }\PY{n+nf}{event\PYZus{}b}\PY{p}{(}\PY{n}{x}\PY{p}{,} \PY{n}{y}\PY{p}{)}\PY{p}{:} \PY{k}{return} \PY{p}{(}\PY{n}{x}\PY{o}{*}\PY{n}{x} \PY{o}{+} \PY{n}{y}\PY{o}{*}\PY{n}{y}\PY{p}{)} \PY{o}{\PYZlt{}} \PY{l+m+mf}{1.0}
\PY{k}{def}\PY{+w}{ }\PY{n+nf}{event\PYZus{}c}\PY{p}{(}\PY{n}{x}\PY{p}{,} \PY{n}{y}\PY{p}{)}\PY{p}{:} \PY{k}{return} \PY{p}{(}\PY{n}{x}\PY{o}{*}\PY{n}{x}\PY{o}{/}\PY{l+m+mf}{3.0} \PY{o}{+} \PY{n}{y}\PY{o}{*}\PY{n}{y}\PY{o}{/}\PY{l+m+mf}{2.0}\PY{p}{)} \PY{o}{\PYZlt{}} \PY{l+m+mf}{1.0}

\PY{c+c1}{\PYZsh{} \PYZhy{}\PYZhy{}\PYZhy{}\PYZhy{}\PYZhy{}\PYZhy{}\PYZhy{}\PYZhy{}\PYZhy{}\PYZhy{}\PYZhy{}\PYZhy{}\PYZhy{}\PYZhy{}\PYZhy{}\PYZhy{}\PYZhy{}\PYZhy{}\PYZhy{}\PYZhy{}\PYZhy{}\PYZhy{}\PYZhy{}\PYZhy{}\PYZhy{}\PYZhy{}\PYZhy{}}
\PY{c+c1}{\PYZsh{} Ejemplo de uso}
\PY{c+c1}{\PYZsh{} \PYZhy{}\PYZhy{}\PYZhy{}\PYZhy{}\PYZhy{}\PYZhy{}\PYZhy{}\PYZhy{}\PYZhy{}\PYZhy{}\PYZhy{}\PYZhy{}\PYZhy{}\PYZhy{}\PYZhy{}\PYZhy{}\PYZhy{}\PYZhy{}\PYZhy{}\PYZhy{}\PYZhy{}\PYZhy{}\PYZhy{}\PYZhy{}\PYZhy{}\PYZhy{}\PYZhy{}}
\PY{k}{if} \PY{n+nv+vm}{\PYZus{}\PYZus{}name\PYZus{}\PYZus{}} \PY{o}{==} \PY{l+s+s2}{\PYZdq{}}\PY{l+s+s2}{\PYZus{}\PYZus{}main\PYZus{}\PYZus{}}\PY{l+s+s2}{\PYZdq{}}\PY{p}{:}
    \PY{n}{n} \PY{o}{=} \PY{l+m+mi}{200\PYZus{}000}

    \PY{c+c1}{\PYZsh{} (a)}
    \PY{n}{th\PYZus{}a}\PY{p}{,} \PY{n}{se\PYZus{}a}\PY{p}{,} \PY{n}{ci\PYZus{}a} \PY{o}{=} \PY{n}{cond\PYZus{}estimator}\PY{p}{(}\PY{n}{p\PYZus{}a\PYZus{}of\PYZus{}x}\PY{p}{,} \PY{n}{n}\PY{o}{=}\PY{n}{n}\PY{p}{,} \PY{n}{seed}\PY{o}{=}\PY{l+m+mi}{1}\PY{p}{)}
    \PY{n}{cr\PYZus{}a}\PY{p}{,} \PY{n}{secr\PYZus{}a}\PY{p}{,} \PY{n}{ccr\PYZus{}a} \PY{o}{=} \PY{n}{crude\PYZus{}estimator}\PY{p}{(}\PY{n}{event\PYZus{}a}\PY{p}{,} \PY{n}{n}\PY{o}{=}\PY{n}{n}\PY{p}{,} \PY{n}{seed}\PY{o}{=}\PY{l+m+mi}{1}\PY{p}{)}
    \PY{n+nb}{print}\PY{p}{(}\PY{l+s+s2}{\PYZdq{}}\PY{l+s+s2}{(a) cond:}\PY{l+s+s2}{\PYZdq{}}\PY{p}{,} \PY{n}{th\PYZus{}a}\PY{p}{,} \PY{l+s+s2}{\PYZdq{}}\PY{l+s+s2}{SE:}\PY{l+s+s2}{\PYZdq{}}\PY{p}{,} \PY{n}{se\PYZus{}a}\PY{p}{,} \PY{l+s+s2}{\PYZdq{}}\PY{l+s+s2}{IC95:}\PY{l+s+s2}{\PYZdq{}}\PY{p}{,} \PY{n}{ci\PYZus{}a}\PY{p}{,} \PY{l+s+s2}{\PYZdq{}}\PY{l+s+s2}{| exacto:}\PY{l+s+s2}{\PYZdq{}}\PY{p}{,} \PY{n}{exact\PYZus{}a}\PY{p}{(}\PY{p}{)}\PY{p}{)}
    \PY{n+nb}{print}\PY{p}{(}\PY{l+s+s2}{\PYZdq{}}\PY{l+s+s2}{    crudo:}\PY{l+s+s2}{\PYZdq{}}\PY{p}{,} \PY{n}{cr\PYZus{}a}\PY{p}{,} \PY{l+s+s2}{\PYZdq{}}\PY{l+s+s2}{SE:}\PY{l+s+s2}{\PYZdq{}}\PY{p}{,} \PY{n}{secr\PYZus{}a}\PY{p}{,} \PY{l+s+s2}{\PYZdq{}}\PY{l+s+s2}{IC95:}\PY{l+s+s2}{\PYZdq{}}\PY{p}{,} \PY{n}{ccr\PYZus{}a}\PY{p}{)}

    \PY{c+c1}{\PYZsh{} (b)}
    \PY{n}{th\PYZus{}b}\PY{p}{,} \PY{n}{se\PYZus{}b}\PY{p}{,} \PY{n}{ci\PYZus{}b} \PY{o}{=} \PY{n}{cond\PYZus{}estimator}\PY{p}{(}\PY{n}{p\PYZus{}b\PYZus{}of\PYZus{}x}\PY{p}{,} \PY{n}{n}\PY{o}{=}\PY{n}{n}\PY{p}{,} \PY{n}{seed}\PY{o}{=}\PY{l+m+mi}{2}\PY{p}{)}
    \PY{n}{cr\PYZus{}b}\PY{p}{,} \PY{n}{secr\PYZus{}b}\PY{p}{,} \PY{n}{ccr\PYZus{}b} \PY{o}{=} \PY{n}{crude\PYZus{}estimator}\PY{p}{(}\PY{n}{event\PYZus{}b}\PY{p}{,} \PY{n}{n}\PY{o}{=}\PY{n}{n}\PY{p}{,} \PY{n}{seed}\PY{o}{=}\PY{l+m+mi}{2}\PY{p}{)}
    \PY{n+nb}{print}\PY{p}{(}\PY{l+s+s2}{\PYZdq{}}\PY{l+s+s2}{(b) cond:}\PY{l+s+s2}{\PYZdq{}}\PY{p}{,} \PY{n}{th\PYZus{}b}\PY{p}{,} \PY{l+s+s2}{\PYZdq{}}\PY{l+s+s2}{SE:}\PY{l+s+s2}{\PYZdq{}}\PY{p}{,} \PY{n}{se\PYZus{}b}\PY{p}{,} \PY{l+s+s2}{\PYZdq{}}\PY{l+s+s2}{IC95:}\PY{l+s+s2}{\PYZdq{}}\PY{p}{,} \PY{n}{ci\PYZus{}b}\PY{p}{,} \PY{l+s+s2}{\PYZdq{}}\PY{l+s+s2}{| exacto:}\PY{l+s+s2}{\PYZdq{}}\PY{p}{,} \PY{n}{exact\PYZus{}b}\PY{p}{(}\PY{p}{)}\PY{p}{)}
    \PY{n+nb}{print}\PY{p}{(}\PY{l+s+s2}{\PYZdq{}}\PY{l+s+s2}{    crudo:}\PY{l+s+s2}{\PYZdq{}}\PY{p}{,} \PY{n}{cr\PYZus{}b}\PY{p}{,} \PY{l+s+s2}{\PYZdq{}}\PY{l+s+s2}{SE:}\PY{l+s+s2}{\PYZdq{}}\PY{p}{,} \PY{n}{secr\PYZus{}b}\PY{p}{,} \PY{l+s+s2}{\PYZdq{}}\PY{l+s+s2}{IC95:}\PY{l+s+s2}{\PYZdq{}}\PY{p}{,} \PY{n}{ccr\PYZus{}b}\PY{p}{)}

    \PY{c+c1}{\PYZsh{} (c)}
    \PY{n}{th\PYZus{}c}\PY{p}{,} \PY{n}{se\PYZus{}c}\PY{p}{,} \PY{n}{ci\PYZus{}c} \PY{o}{=} \PY{n}{cond\PYZus{}estimator}\PY{p}{(}\PY{n}{p\PYZus{}c\PYZus{}of\PYZus{}x}\PY{p}{,} \PY{n}{n}\PY{o}{=}\PY{n}{n}\PY{p}{,} \PY{n}{seed}\PY{o}{=}\PY{l+m+mi}{3}\PY{p}{)}
    \PY{n}{cr\PYZus{}c}\PY{p}{,} \PY{n}{secr\PYZus{}c}\PY{p}{,} \PY{n}{ccr\PYZus{}c} \PY{o}{=} \PY{n}{crude\PYZus{}estimator}\PY{p}{(}\PY{n}{event\PYZus{}c}\PY{p}{,} \PY{n}{n}\PY{o}{=}\PY{n}{n}\PY{p}{,} \PY{n}{seed}\PY{o}{=}\PY{l+m+mi}{3}\PY{p}{)}
    \PY{n+nb}{print}\PY{p}{(}\PY{l+s+s2}{\PYZdq{}}\PY{l+s+s2}{(c) cond:}\PY{l+s+s2}{\PYZdq{}}\PY{p}{,} \PY{n}{th\PYZus{}c}\PY{p}{,} \PY{l+s+s2}{\PYZdq{}}\PY{l+s+s2}{SE:}\PY{l+s+s2}{\PYZdq{}}\PY{p}{,} \PY{n}{se\PYZus{}c}\PY{p}{,} \PY{l+s+s2}{\PYZdq{}}\PY{l+s+s2}{IC95:}\PY{l+s+s2}{\PYZdq{}}\PY{p}{,} \PY{n}{ci\PYZus{}c}\PY{p}{,} \PY{l+s+s2}{\PYZdq{}}\PY{l+s+s2}{| exacto:}\PY{l+s+s2}{\PYZdq{}}\PY{p}{,} \PY{n}{exact\PYZus{}c}\PY{p}{(}\PY{p}{)}\PY{p}{)}
    \PY{n+nb}{print}\PY{p}{(}\PY{l+s+s2}{\PYZdq{}}\PY{l+s+s2}{    crudo:}\PY{l+s+s2}{\PYZdq{}}\PY{p}{,} \PY{n}{cr\PYZus{}c}\PY{p}{,} \PY{l+s+s2}{\PYZdq{}}\PY{l+s+s2}{SE:}\PY{l+s+s2}{\PYZdq{}}\PY{p}{,} \PY{n}{secr\PYZus{}c}\PY{p}{,} \PY{l+s+s2}{\PYZdq{}}\PY{l+s+s2}{IC95:}\PY{l+s+s2}{\PYZdq{}}\PY{p}{,} \PY{n}{ccr\PYZus{}c}\PY{p}{)}
\end{Verbatim}
\end{tcolorbox}

    \begin{Verbatim}[commandchars=\\\{\}]
(a) cond: 0.49944514018736547 SE: 0.0009129076713279941 IC95:
(0.4976558411515626, 0.5012344392231683) | exacto: 0.5
    crudo: 0.49887 SE: 0.0011180311335110486 IC95: (0.4966786589783183,
0.5010613410216816)
(b) cond: 0.7847524382953979 SE: 0.0004990404544118926 IC95:
(0.7837743190047506, 0.7857305575860452) | exacto: 0.7853981633974483
    crudo: 0.785135 SE: 0.0009184172030591543 IC95: (0.7833349022820041,
0.786935097717996)
(c) cond: 1.0 SE: 0.0 IC95: (1.0, 1.0) | exacto: 1.0
    crudo: 1.0 SE: 0.0 IC95: (1.0, 1.0)
    \end{Verbatim}

    \hypertarget{ejercicio-10}{%
\section{Ejercicio 10}\label{ejercicio-10}}


Sea $(X,N)$ un vector aleatorio discreto con función de probabilidad
\[
f(x,n) \;=\; \frac{3}{10}\,\Big(\tfrac12\Big)^{n x},
\qquad n=1,2;\; x=0,1,2,\ldots
\]
y suponga que $X_1,X_2$ son dos copias independientes de $X$ (independientes entre sí y de $N$).
\begin{enumerate}[label=(\alph*)]
\item Usando muestreo condicional sobre $N$, encuentre una aproximación a la esperanza de
\[
S=\sum_{i=1}^{N} X_i.
\]
\item Encuentre el valor exacto de $\mathbb{E}(S)$ y compruebe si la aproximación obtenida es razonable.
\end{enumerate}


La marginal de $N$ se obtiene con series geométricas:
\[
\Pr(N=n)=\sum_{x\ge0} f(x,n)=\frac{3}{10}\sum_{x\ge0}\Big(\tfrac12\Big)^{nx}
=\frac{3}{10}\,\frac{1}{1-(1/2)^n}
=\begin{cases}
\frac{3}{5}, & n=1,\\[2pt]
\frac{2}{5}, & n=2.
\end{cases}
\]
La condicional de $X$ dado $N=n$ es
\[
\Pr(X=x\mid N=n)=\frac{f(x,n)}{\Pr(N=n)}
=\Big(1-\big(\tfrac12\big)^n\Big)\Big(\tfrac12\Big)^{nx},\qquad x=0,1,\ldots
\]
que es \emph{geométrica} en $\{0,1,\ldots\}$ con parámetro $p_n=(1/2)^n$
(en la convención ``número de fallas antes del primer éxito'').
Así,
\[
\mathbb{E}[X\mid N=n]=\frac{p_n}{1-p_n}=\frac{(1/2)^n}{1-(1/2)^n},
\quad\Rightarrow\quad
\mathbb{E}[X\mid N=1]=1,\;\; \mathbb{E}[X\mid N=2]=\tfrac{1}{3}.
\]

2) Media de $X$ y de $N$.
Por la ley de la esperanza total,
\[
\mu_X:=\mathbb{E}[X]
=\sum_{n}\Pr(N=n)\,\mathbb{E}[X\mid N=n]
=\frac{3}{5}\cdot 1+\frac{2}{5}\cdot\frac{1}{3}
=\boxed{\frac{11}{15}}.
\]
La media de $N$ (dos puntos):
\[
\mu_N:=\mathbb{E}[N]=1\cdot\frac{3}{5}+2\cdot\frac{2}{5}
=\boxed{\frac{7}{5}}.
\]

3)Valor exacto de $\mathbb{E}[S]$.
Como $X_i$ son i.i.d.\ con ley $X$ e independientes de $N$,
\[
\mathbb{E}[S\mid N]=N\,\mu_X
\quad\Longrightarrow\quad
\mathbb{E}[S]=\mathbb{E}\!\big[\mathbb{E}[S\mid N]\big]=\mu_X\,\mathbb{E}[N]
=\boxed{\frac{11}{15}\cdot\frac{7}{5}=\frac{77}{75}\approx 1.026\overline6}.
\]

4) Estimación por \emph{muestreo condicional} sobre $N$.
Para aproximar $\mathbb{E}[S]$ por simulación con menor varianza que el método crudo:
\begin{enumerate}
  \item Generar $N_1,\dots,N_m$ i.i.d.\ con $\Pr(N=1)=3/5$, $\Pr(N=2)=2/5$.
  \item Usar $\mu_X=\mathbb{E}[X]=11/15$ (o estimarla con un piloto independiente).
  \item Estimador CMC:
  \[
     \widehat{\mu}_S^{(\text{cond})}
     = \frac{1}{m}\sum_{k=1}^{m} N_k\,\mu_X,
  \]
  con error estándar $\widehat{\mathrm{SE}}=\mathrm{SD}(N\,\mu_X)/\sqrt{m}$ e
  intervalo $95\%$: $\widehat{\mu}_S^{(\text{cond})}\pm 1.96\,\widehat{\mathrm{SE}}$.
\end{enumerate}
Este estimador implementa $\mathbb{E}[S]=\mathbb{E}\{\mathbb{E}[S\mid N]\}$ y evita simular todos los $X_i$ en cada réplica, logrando \emph{reducción de varianza}.



\subsection{Código}
    \begin{tcolorbox}[breakable, size=fbox, boxrule=1pt, pad at break*=1mm,colback=cellbackground, colframe=cellborder]
\prompt{In}{incolor}{10}{\boxspacing}
\begin{Verbatim}[commandchars=\\\{\}]
\PY{n}{pN1}\PY{p}{,} \PY{n}{pN2} \PY{o}{=} \PY{l+m+mi}{3}\PY{o}{/}\PY{l+m+mi}{5}\PY{p}{,} \PY{l+m+mi}{2}\PY{o}{/}\PY{l+m+mi}{5}
\PY{n}{E\PYZus{}N} \PY{o}{=} \PY{l+m+mi}{1}\PY{o}{*}\PY{n}{pN1} \PY{o}{+} \PY{l+m+mi}{2}\PY{o}{*}\PY{n}{pN2}                 \PY{c+c1}{\PYZsh{} = 7/5}
\PY{c+c1}{\PYZsh{} E[X] por series: (3/10) * [ sum x (1/2)\PYZca{}x + sum x (1/4)\PYZca{}x ]}
\PY{n}{E\PYZus{}X} \PY{o}{=} \PY{p}{(}\PY{l+m+mi}{3}\PY{o}{/}\PY{l+m+mi}{10}\PY{p}{)} \PY{o}{*} \PY{p}{(}\PY{p}{(}\PY{l+m+mf}{0.5}\PY{p}{)} \PY{o}{/} \PY{p}{(}\PY{l+m+mi}{1}\PY{o}{\PYZhy{}}\PY{l+m+mf}{0.5}\PY{p}{)}\PY{o}{*}\PY{o}{*}\PY{l+m+mi}{2} \PY{o}{+} \PY{p}{(}\PY{l+m+mf}{0.25}\PY{p}{)} \PY{o}{/} \PY{p}{(}\PY{l+m+mi}{1}\PY{o}{\PYZhy{}}\PY{l+m+mf}{0.25}\PY{p}{)}\PY{o}{*}\PY{o}{*}\PY{l+m+mi}{2}\PY{p}{)}  \PY{c+c1}{\PYZsh{} = 11/15}
\PY{n}{E\PYZus{}S\PYZus{}exact} \PY{o}{=} \PY{n}{E\PYZus{}N} \PY{o}{*} \PY{n}{E\PYZus{}X}                                   \PY{c+c1}{\PYZsh{} = 77/75}

\PY{k}{def}\PY{+w}{ }\PY{n+nf}{sample\PYZus{}N}\PY{p}{(}\PY{n}{rng}\PY{p}{,} \PY{n}{size}\PY{p}{)}\PY{p}{:}
\PY{+w}{    }\PY{l+s+sd}{\PYZdq{}\PYZdq{}\PYZdq{}Muestra N in \PYZob{}1,2\PYZcb{} con P(N=1)=3/5, P(N=2)=2/5.\PYZdq{}\PYZdq{}\PYZdq{}}
    \PY{n}{u} \PY{o}{=} \PY{n}{rng}\PY{o}{.}\PY{n}{random}\PY{p}{(}\PY{n}{size}\PY{p}{)}
    \PY{k}{return} \PY{n}{np}\PY{o}{.}\PY{n}{where}\PY{p}{(}\PY{n}{u} \PY{o}{\PYZlt{}} \PY{n}{pN1}\PY{p}{,} \PY{l+m+mi}{1}\PY{p}{,} \PY{l+m+mi}{2}\PY{p}{)}

\PY{k}{def}\PY{+w}{ }\PY{n+nf}{sample\PYZus{}X\PYZus{}given\PYZus{}n}\PY{p}{(}\PY{n}{rng}\PY{p}{,} \PY{n}{n}\PY{p}{)}\PY{p}{:}
\PY{+w}{    }\PY{l+s+sd}{\PYZdq{}\PYZdq{}\PYZdq{}}
\PY{l+s+sd}{    Muestra X | N=n. Condicional es geométrica en \PYZob{}0,1,...\PYZcb{} con p\PYZus{}n=1\PYZhy{}(1/2)\PYZca{}n.}
\PY{l+s+sd}{    Numpy geometric devuelve \PYZob{}1,2,...\PYZcb{} con media 1/p \PYZhy{}\PYZgt{} restamos 1.}
\PY{l+s+sd}{    \PYZdq{}\PYZdq{}\PYZdq{}}
    \PY{n}{pn} \PY{o}{=} \PY{l+m+mi}{1} \PY{o}{\PYZhy{}} \PY{l+m+mf}{0.5}\PY{o}{*}\PY{o}{*}\PY{n}{n}
    \PY{k}{return} \PY{n}{rng}\PY{o}{.}\PY{n}{geometric}\PY{p}{(}\PY{n}{p}\PY{o}{=}\PY{n}{pn}\PY{p}{)} \PY{o}{\PYZhy{}} \PY{l+m+mi}{1}

\PY{k}{def}\PY{+w}{ }\PY{n+nf}{sample\PYZus{}X\PYZus{}marginal}\PY{p}{(}\PY{n}{rng}\PY{p}{,} \PY{n}{size}\PY{p}{)}\PY{p}{:}
\PY{+w}{    }\PY{l+s+sd}{\PYZdq{}\PYZdq{}\PYZdq{}}
\PY{l+s+sd}{    Muestra X de su marginal P(X=x) = (3/10)[2\PYZca{}\PYZob{}\PYZhy{}x\PYZcb{} + 4\PYZca{}\PYZob{}\PYZhy{}x\PYZcb{}].}
\PY{l+s+sd}{    Se usa el truco de mezcla: elegir N\PYZsq{} \PYZti{} \PYZob{}1,2\PYZcb{} con (3/5, 2/5) y luego X|N\PYZsq{}.}
\PY{l+s+sd}{    \PYZdq{}\PYZdq{}\PYZdq{}}
    \PY{n}{Nprime} \PY{o}{=} \PY{n}{sample\PYZus{}N}\PY{p}{(}\PY{n}{rng}\PY{p}{,} \PY{n}{size}\PY{p}{)}
    \PY{k}{return} \PY{n}{sample\PYZus{}X\PYZus{}given\PYZus{}n}\PY{p}{(}\PY{n}{rng}\PY{p}{,} \PY{n}{Nprime}\PY{p}{)}

\PY{c+c1}{\PYZsh{} \PYZhy{}\PYZhy{}\PYZhy{}\PYZhy{}\PYZhy{}\PYZhy{}\PYZhy{}\PYZhy{}\PYZhy{}\PYZhy{} Simulación condicional sobre N \PYZhy{}\PYZhy{}\PYZhy{}\PYZhy{}\PYZhy{}\PYZhy{}\PYZhy{}\PYZhy{}\PYZhy{}\PYZhy{}}
\PY{k}{def}\PY{+w}{ }\PY{n+nf}{estimate\PYZus{}ES\PYZus{}conditional}\PY{p}{(}\PY{n}{m}\PY{o}{=}\PY{l+m+mi}{200\PYZus{}000}\PY{p}{,} \PY{n}{seed}\PY{o}{=}\PY{l+m+mi}{7}\PY{p}{)}\PY{p}{:}
    \PY{n}{rng} \PY{o}{=} \PY{n}{np}\PY{o}{.}\PY{n}{random}\PY{o}{.}\PY{n}{default\PYZus{}rng}\PY{p}{(}\PY{n}{seed}\PY{p}{)}
    \PY{n}{N} \PY{o}{=} \PY{n}{sample\PYZus{}N}\PY{p}{(}\PY{n}{rng}\PY{p}{,} \PY{n}{m}\PY{p}{)}
    \PY{n}{mu\PYZus{}X} \PY{o}{=} \PY{n}{E\PYZus{}X}                        \PY{c+c1}{\PYZsh{} usar el valor exacto (puedes reemplazar por estimado si quieres)}
    \PY{n}{S\PYZus{}hat} \PY{o}{=} \PY{n}{N} \PY{o}{*} \PY{n}{mu\PYZus{}X}
    \PY{n}{mu\PYZus{}hat} \PY{o}{=} \PY{n+nb}{float}\PY{p}{(}\PY{n}{np}\PY{o}{.}\PY{n}{mean}\PY{p}{(}\PY{n}{S\PYZus{}hat}\PY{p}{)}\PY{p}{)}
    \PY{c+c1}{\PYZsh{} Error estándar de la media de N*mu\PYZus{}X:}
    \PY{n}{se} \PY{o}{=} \PY{n+nb}{float}\PY{p}{(}\PY{n}{np}\PY{o}{.}\PY{n}{std}\PY{p}{(}\PY{n}{S\PYZus{}hat}\PY{p}{,} \PY{n}{ddof}\PY{o}{=}\PY{l+m+mi}{1}\PY{p}{)} \PY{o}{/} \PY{n}{np}\PY{o}{.}\PY{n}{sqrt}\PY{p}{(}\PY{n}{m}\PY{p}{)}\PY{p}{)}
    \PY{n}{ci} \PY{o}{=} \PY{p}{(}\PY{n}{mu\PYZus{}hat} \PY{o}{\PYZhy{}} \PY{l+m+mf}{1.96}\PY{o}{*}\PY{n}{se}\PY{p}{,} \PY{n}{mu\PYZus{}hat} \PY{o}{+} \PY{l+m+mf}{1.96}\PY{o}{*}\PY{n}{se}\PY{p}{)}
    \PY{k}{return} \PY{n}{mu\PYZus{}hat}\PY{p}{,} \PY{n}{se}\PY{p}{,} \PY{n}{ci}

\PY{c+c1}{\PYZsh{} \PYZhy{}\PYZhy{}\PYZhy{}\PYZhy{}\PYZhy{}\PYZhy{}\PYZhy{}\PYZhy{}\PYZhy{}\PYZhy{} Simulación \PYZdq{}cruda\PYZdq{} (para comparar) \PYZhy{}\PYZhy{}\PYZhy{}\PYZhy{}\PYZhy{}\PYZhy{}\PYZhy{}\PYZhy{}\PYZhy{}\PYZhy{}}
\PY{k}{def}\PY{+w}{ }\PY{n+nf}{estimate\PYZus{}ES\PYZus{}crude}\PY{p}{(}\PY{n}{m}\PY{o}{=}\PY{l+m+mi}{200\PYZus{}000}\PY{p}{,} \PY{n}{seed}\PY{o}{=}\PY{l+m+mi}{8}\PY{p}{)}\PY{p}{:}
    \PY{n}{rng} \PY{o}{=} \PY{n}{np}\PY{o}{.}\PY{n}{random}\PY{o}{.}\PY{n}{default\PYZus{}rng}\PY{p}{(}\PY{n}{seed}\PY{p}{)}
    \PY{n}{N} \PY{o}{=} \PY{n}{sample\PYZus{}N}\PY{p}{(}\PY{n}{rng}\PY{p}{,} \PY{n}{m}\PY{p}{)}
    \PY{n}{S} \PY{o}{=} \PY{n}{np}\PY{o}{.}\PY{n}{zeros}\PY{p}{(}\PY{n}{m}\PY{p}{,} \PY{n}{dtype}\PY{o}{=}\PY{n+nb}{float}\PY{p}{)}
    \PY{c+c1}{\PYZsh{} casos N=1 y N=2 (únicos posibles)}
    \PY{n}{mask1} \PY{o}{=} \PY{p}{(}\PY{n}{N} \PY{o}{==} \PY{l+m+mi}{1}\PY{p}{)}
    \PY{n}{mask2} \PY{o}{=} \PY{o}{\PYZti{}}\PY{n}{mask1}
    \PY{n}{S}\PY{p}{[}\PY{n}{mask1}\PY{p}{]} \PY{o}{=} \PY{n}{sample\PYZus{}X\PYZus{}marginal}\PY{p}{(}\PY{n}{rng}\PY{p}{,} \PY{n}{np}\PY{o}{.}\PY{n}{sum}\PY{p}{(}\PY{n}{mask1}\PY{p}{)}\PY{p}{)}
    \PY{k}{if} \PY{n}{np}\PY{o}{.}\PY{n}{any}\PY{p}{(}\PY{n}{mask2}\PY{p}{)}\PY{p}{:}
        \PY{n}{X1} \PY{o}{=} \PY{n}{sample\PYZus{}X\PYZus{}marginal}\PY{p}{(}\PY{n}{rng}\PY{p}{,} \PY{n}{np}\PY{o}{.}\PY{n}{sum}\PY{p}{(}\PY{n}{mask2}\PY{p}{)}\PY{p}{)}
        \PY{n}{X2} \PY{o}{=} \PY{n}{sample\PYZus{}X\PYZus{}marginal}\PY{p}{(}\PY{n}{rng}\PY{p}{,} \PY{n}{np}\PY{o}{.}\PY{n}{sum}\PY{p}{(}\PY{n}{mask2}\PY{p}{)}\PY{p}{)}
        \PY{n}{S}\PY{p}{[}\PY{n}{mask2}\PY{p}{]} \PY{o}{=} \PY{n}{X1} \PY{o}{+} \PY{n}{X2}
    \PY{n}{mu\PYZus{}hat} \PY{o}{=} \PY{n+nb}{float}\PY{p}{(}\PY{n}{np}\PY{o}{.}\PY{n}{mean}\PY{p}{(}\PY{n}{S}\PY{p}{)}\PY{p}{)}
    \PY{n}{se} \PY{o}{=} \PY{n+nb}{float}\PY{p}{(}\PY{n}{np}\PY{o}{.}\PY{n}{std}\PY{p}{(}\PY{n}{S}\PY{p}{,} \PY{n}{ddof}\PY{o}{=}\PY{l+m+mi}{1}\PY{p}{)} \PY{o}{/} \PY{n}{np}\PY{o}{.}\PY{n}{sqrt}\PY{p}{(}\PY{n}{m}\PY{p}{)}\PY{p}{)}
    \PY{n}{ci} \PY{o}{=} \PY{p}{(}\PY{n}{mu\PYZus{}hat} \PY{o}{\PYZhy{}} \PY{l+m+mf}{1.96}\PY{o}{*}\PY{n}{se}\PY{p}{,} \PY{n}{mu\PYZus{}hat} \PY{o}{+} \PY{l+m+mf}{1.96}\PY{o}{*}\PY{n}{se}\PY{p}{)}
    \PY{k}{return} \PY{n}{mu\PYZus{}hat}\PY{p}{,} \PY{n}{se}\PY{p}{,} \PY{n}{ci}

\PY{c+c1}{\PYZsh{} \PYZhy{}\PYZhy{}\PYZhy{}\PYZhy{}\PYZhy{}\PYZhy{}\PYZhy{}\PYZhy{}\PYZhy{}\PYZhy{} Demostración \PYZhy{}\PYZhy{}\PYZhy{}\PYZhy{}\PYZhy{}\PYZhy{}\PYZhy{}\PYZhy{}\PYZhy{}\PYZhy{}}
\PY{k}{if} \PY{n+nv+vm}{\PYZus{}\PYZus{}name\PYZus{}\PYZus{}} \PY{o}{==} \PY{l+s+s2}{\PYZdq{}}\PY{l+s+s2}{\PYZus{}\PYZus{}main\PYZus{}\PYZus{}}\PY{l+s+s2}{\PYZdq{}}\PY{p}{:}
    \PY{n}{m} \PY{o}{=} \PY{l+m+mi}{300\PYZus{}000}
    \PY{n}{mu\PYZus{}c}\PY{p}{,} \PY{n}{se\PYZus{}c}\PY{p}{,} \PY{n}{ci\PYZus{}c} \PY{o}{=} \PY{n}{estimate\PYZus{}ES\PYZus{}conditional}\PY{p}{(}\PY{n}{m}\PY{o}{=}\PY{n}{m}\PY{p}{,} \PY{n}{seed}\PY{o}{=}\PY{l+m+mi}{7}\PY{p}{)}
    \PY{n}{mu\PYZus{}r}\PY{p}{,} \PY{n}{se\PYZus{}r}\PY{p}{,} \PY{n}{ci\PYZus{}r} \PY{o}{=} \PY{n}{estimate\PYZus{}ES\PYZus{}crude}\PY{p}{(}\PY{n}{m}\PY{o}{=}\PY{n}{m}\PY{p}{,} \PY{n}{seed}\PY{o}{=}\PY{l+m+mi}{8}\PY{p}{)}

    \PY{n+nb}{print}\PY{p}{(}\PY{l+s+sa}{f}\PY{l+s+s2}{\PYZdq{}}\PY{l+s+s2}{E[S] exacta             = }\PY{l+s+si}{\PYZob{}}\PY{n}{E\PYZus{}S\PYZus{}exact}\PY{l+s+si}{:}\PY{l+s+s2}{.8f}\PY{l+s+si}{\PYZcb{}}\PY{l+s+s2}{  (= 77/75)}\PY{l+s+s2}{\PYZdq{}}\PY{p}{)}
    \PY{n+nb}{print}\PY{p}{(}\PY{l+s+sa}{f}\PY{l+s+s2}{\PYZdq{}}\PY{l+s+s2}{Condicional sobre N     = }\PY{l+s+si}{\PYZob{}}\PY{n}{mu\PYZus{}c}\PY{l+s+si}{:}\PY{l+s+s2}{.8f}\PY{l+s+si}{\PYZcb{}}\PY{l+s+s2}{   SE=}\PY{l+s+si}{\PYZob{}}\PY{n}{se\PYZus{}c}\PY{l+s+si}{:}\PY{l+s+s2}{.6f}\PY{l+s+si}{\PYZcb{}}\PY{l+s+s2}{  IC95=}\PY{l+s+si}{\PYZob{}}\PY{n}{ci\PYZus{}c}\PY{l+s+si}{\PYZcb{}}\PY{l+s+s2}{\PYZdq{}}\PY{p}{)}
    \PY{n+nb}{print}\PY{p}{(}\PY{l+s+sa}{f}\PY{l+s+s2}{\PYZdq{}}\PY{l+s+s2}{Crudo (simular X\PYZus{}i)     = }\PY{l+s+si}{\PYZob{}}\PY{n}{mu\PYZus{}r}\PY{l+s+si}{:}\PY{l+s+s2}{.8f}\PY{l+s+si}{\PYZcb{}}\PY{l+s+s2}{   SE=}\PY{l+s+si}{\PYZob{}}\PY{n}{se\PYZus{}r}\PY{l+s+si}{:}\PY{l+s+s2}{.6f}\PY{l+s+si}{\PYZcb{}}\PY{l+s+s2}{  IC95=}\PY{l+s+si}{\PYZob{}}\PY{n}{ci\PYZus{}r}\PY{l+s+si}{\PYZcb{}}\PY{l+s+s2}{\PYZdq{}}\PY{p}{)}
    \PY{k}{if} \PY{n}{se\PYZus{}c} \PY{o}{\PYZgt{}} \PY{l+m+mi}{0}\PY{p}{:}
        \PY{n+nb}{print}\PY{p}{(}\PY{l+s+sa}{f}\PY{l+s+s2}{\PYZdq{}}\PY{l+s+s2}{Reducción de varianza ≈ }\PY{l+s+si}{\PYZob{}}\PY{p}{(}\PY{n}{se\PYZus{}r}\PY{o}{*}\PY{o}{*}\PY{l+m+mi}{2}\PY{p}{)}\PY{o}{/}\PY{p}{(}\PY{n}{se\PYZus{}c}\PY{o}{*}\PY{o}{*}\PY{l+m+mi}{2}\PY{p}{)}\PY{l+s+si}{:}\PY{l+s+s2}{.2f}\PY{l+s+si}{\PYZcb{}}\PY{l+s+s2}{x (SE\PYZus{}crudo\PYZca{}2 / SE\PYZus{}cond\PYZca{}2)}\PY{l+s+s2}{\PYZdq{}}\PY{p}{)}
\end{Verbatim}
\end{tcolorbox}

    \begin{Verbatim}[commandchars=\\\{\}]
E[S] exacta             = 1.02666667  (= 77/75)
Condicional sobre N     = 1.02729489   SE=0.000656  IC95=(1.0260088398921172,
1.02858093788566)
Crudo (simular X\_i)     = 1.03039000   SE=0.002725  IC95=(1.0250482811313377,
1.035731718868662)
Reducción de varianza ≈ 17.25x (SE\_crudo\^{}2 / SE\_cond\^{}2)
    \end{Verbatim}

    \hypertarget{ejercicio-11}{%
\section{Ejercicio 11}\label{ejercicio-11}}

Defina la variable aleatoria $S := X^{N}$, en donde $X$ y $N$ son independientes y son tales que
$X \sim \mathrm{Unif}(0,1)$ y $N \sim \mathrm{Unif}\{1,\dots,10\}$.
\begin{enumerate}[label=(\alph*)]
\item Usando muestreo condicional sobre $N$, encuentre una aproximación para $\mathbb{E}(S)$.
\item Encuentre el valor exacto de $\mathbb{E}(S)$ y compruebe si la aproximación obtenida es razonable.
\end{enumerate}

\paragraph{Idea (muestreo condicional).}
Como $S=X^{N}$ y $X\perp N$,
\[
\mathbb{E}(S)=\mathbb{E}\!\big[\mathbb{E}(X^{N}\mid N)\big]
=\mathbb{E}\!\big[\mathbb{E}(X^{n})\big]_{n=N}.
\]
Si $X\sim\mathrm{Unif}(0,1)$, entonces $\mathbb{E}(X^{n})=\int_0^1 x^{n}\,dx=\frac{1}{n+1}$.
Por lo tanto,
\[
\mathbb{E}(S)=\mathbb{E}\!\left(\frac{1}{N+1}\right)
=\frac{1}{10}\sum_{n=1}^{10}\frac{1}{n+1}
=\frac{1}{10}\sum_{k=2}^{11}\frac{1}{k}
=\frac{H_{11}-1}{10},
\]
donde $H_m=\sum_{k=1}^m \frac{1}{k}$ es el número armónico.
Numéricamente,
\[
H_{11}=1+\frac12+\cdots+\frac{1}{11}\approx 3.019877344,\qquad
\Rightarrow\quad \mathbb{E}(S)\approx \frac{3.019877344-1}{10}=0.2019877344.
\]

\paragraph{Estimación por CMC (Conditional Monte Carlo).}
En lugar de simular $X$ y elevarlo a un exponente aleatorio, basta simular $N$ y usar
$\mathbb{E}(X^{N}\mid N=n)=\frac{1}{n+1}$:
\[
\widehat{\mu}_{\text{CMC}}=\frac{1}{m}\sum_{i=1}^m \frac{1}{N_i+1},
\qquad N_i\stackrel{iid}{\sim}\mathrm{Unif}\{1,\dots,10\}.
\]
Este estimador es insesgado y típicamente tiene menor varianza que el “crudo”
\(
\widehat{\mu}_{\text{crudo}}=\frac{1}{m}\sum_{i=1}^m X_i^{N_i}
\),
pues sustituye $X^{N}$ por su esperanza condicional.
El error estándar de $\widehat{\mu}_{\text{CMC}}$ es
\(
\widehat{\mathrm{SE}}=\mathrm{SD}\!\left(\frac{1}{N+1}\right)/\sqrt{m}
\),
y un IC95\% es
\(
\widehat{\mu}_{\text{CMC}}\pm 1.96\,\widehat{\mathrm{SE}}.
\)

\subsection{Código}
    \begin{tcolorbox}[breakable, size=fbox, boxrule=1pt, pad at break*=1mm,colback=cellbackground, colframe=cellborder]
\prompt{In}{incolor}{11}{\boxspacing}
\begin{Verbatim}[commandchars=\\\{\}]
\PY{k}{def}\PY{+w}{ }\PY{n+nf}{exact\PYZus{}E\PYZus{}S}\PY{p}{(}\PY{p}{)}\PY{p}{:}
    \PY{c+c1}{\PYZsh{} (H\PYZus{}11 \PYZhy{} 1)/10}
    \PY{n}{H11} \PY{o}{=} \PY{n+nb}{sum}\PY{p}{(}\PY{l+m+mf}{1.0}\PY{o}{/}\PY{n}{k} \PY{k}{for} \PY{n}{k} \PY{o+ow}{in} \PY{n+nb}{range}\PY{p}{(}\PY{l+m+mi}{1}\PY{p}{,} \PY{l+m+mi}{12}\PY{p}{)}\PY{p}{)}
    \PY{k}{return} \PY{p}{(}\PY{n}{H11} \PY{o}{\PYZhy{}} \PY{l+m+mf}{1.0}\PY{p}{)} \PY{o}{/} \PY{l+m+mf}{10.0}

\PY{k}{def}\PY{+w}{ }\PY{n+nf}{E\PYZus{}S\PYZus{}cmc}\PY{p}{(}\PY{n}{m}\PY{o}{=}\PY{l+m+mi}{100\PYZus{}000}\PY{p}{,} \PY{n}{seed}\PY{o}{=}\PY{l+m+mi}{7}\PY{p}{)}\PY{p}{:}
    \PY{n}{rng} \PY{o}{=} \PY{n}{np}\PY{o}{.}\PY{n}{random}\PY{o}{.}\PY{n}{default\PYZus{}rng}\PY{p}{(}\PY{n}{seed}\PY{p}{)}
    \PY{n}{N} \PY{o}{=} \PY{n}{rng}\PY{o}{.}\PY{n}{integers}\PY{p}{(}\PY{l+m+mi}{1}\PY{p}{,} \PY{l+m+mi}{11}\PY{p}{,} \PY{n}{size}\PY{o}{=}\PY{n}{m}\PY{p}{)}          \PY{c+c1}{\PYZsh{} Unif\PYZob{}1,...,10\PYZcb{}}
    \PY{n}{vals} \PY{o}{=} \PY{l+m+mf}{1.0} \PY{o}{/} \PY{p}{(}\PY{n}{N} \PY{o}{+} \PY{l+m+mf}{1.0}\PY{p}{)}                   \PY{c+c1}{\PYZsh{} E[X\PYZca{}N | N]}
    \PY{n}{mu} \PY{o}{=} \PY{n+nb}{float}\PY{p}{(}\PY{n}{np}\PY{o}{.}\PY{n}{mean}\PY{p}{(}\PY{n}{vals}\PY{p}{)}\PY{p}{)}
    \PY{n}{se} \PY{o}{=} \PY{n+nb}{float}\PY{p}{(}\PY{n}{np}\PY{o}{.}\PY{n}{std}\PY{p}{(}\PY{n}{vals}\PY{p}{,} \PY{n}{ddof}\PY{o}{=}\PY{l+m+mi}{1}\PY{p}{)} \PY{o}{/} \PY{n}{np}\PY{o}{.}\PY{n}{sqrt}\PY{p}{(}\PY{n}{m}\PY{p}{)}\PY{p}{)}
    \PY{n}{ci} \PY{o}{=} \PY{p}{(}\PY{n}{mu} \PY{o}{\PYZhy{}} \PY{l+m+mf}{1.96}\PY{o}{*}\PY{n}{se}\PY{p}{,} \PY{n}{mu} \PY{o}{+} \PY{l+m+mf}{1.96}\PY{o}{*}\PY{n}{se}\PY{p}{)}
    \PY{k}{return} \PY{n}{mu}\PY{p}{,} \PY{n}{se}\PY{p}{,} \PY{n}{ci}

\PY{k}{def}\PY{+w}{ }\PY{n+nf}{E\PYZus{}S\PYZus{}crudo}\PY{p}{(}\PY{n}{m}\PY{o}{=}\PY{l+m+mi}{100\PYZus{}000}\PY{p}{,} \PY{n}{seed}\PY{o}{=}\PY{l+m+mi}{8}\PY{p}{)}\PY{p}{:}
    \PY{n}{rng} \PY{o}{=} \PY{n}{np}\PY{o}{.}\PY{n}{random}\PY{o}{.}\PY{n}{default\PYZus{}rng}\PY{p}{(}\PY{n}{seed}\PY{p}{)}
    \PY{n}{N} \PY{o}{=} \PY{n}{rng}\PY{o}{.}\PY{n}{integers}\PY{p}{(}\PY{l+m+mi}{1}\PY{p}{,} \PY{l+m+mi}{11}\PY{p}{,} \PY{n}{size}\PY{o}{=}\PY{n}{m}\PY{p}{)}
    \PY{n}{X} \PY{o}{=} \PY{n}{rng}\PY{o}{.}\PY{n}{uniform}\PY{p}{(}\PY{l+m+mf}{0.0}\PY{p}{,} \PY{l+m+mf}{1.0}\PY{p}{,} \PY{n}{size}\PY{o}{=}\PY{n}{m}\PY{p}{)}
    \PY{n}{S} \PY{o}{=} \PY{n}{X} \PY{o}{*}\PY{o}{*} \PY{n}{N}
    \PY{n}{mu} \PY{o}{=} \PY{n+nb}{float}\PY{p}{(}\PY{n}{np}\PY{o}{.}\PY{n}{mean}\PY{p}{(}\PY{n}{S}\PY{p}{)}\PY{p}{)}
    \PY{n}{se} \PY{o}{=} \PY{n+nb}{float}\PY{p}{(}\PY{n}{np}\PY{o}{.}\PY{n}{std}\PY{p}{(}\PY{n}{S}\PY{p}{,} \PY{n}{ddof}\PY{o}{=}\PY{l+m+mi}{1}\PY{p}{)} \PY{o}{/} \PY{n}{np}\PY{o}{.}\PY{n}{sqrt}\PY{p}{(}\PY{n}{m}\PY{p}{)}\PY{p}{)}
    \PY{n}{ci} \PY{o}{=} \PY{p}{(}\PY{n}{mu} \PY{o}{\PYZhy{}} \PY{l+m+mf}{1.96}\PY{o}{*}\PY{n}{se}\PY{p}{,} \PY{n}{mu} \PY{o}{+} \PY{l+m+mf}{1.96}\PY{o}{*}\PY{n}{se}\PY{p}{)}
    \PY{k}{return} \PY{n}{mu}\PY{p}{,} \PY{n}{se}\PY{p}{,} \PY{n}{ci}

\PY{k}{if} \PY{n+nv+vm}{\PYZus{}\PYZus{}name\PYZus{}\PYZus{}} \PY{o}{==} \PY{l+s+s2}{\PYZdq{}}\PY{l+s+s2}{\PYZus{}\PYZus{}main\PYZus{}\PYZus{}}\PY{l+s+s2}{\PYZdq{}}\PY{p}{:}
    \PY{n}{exact} \PY{o}{=} \PY{n}{exact\PYZus{}E\PYZus{}S}\PY{p}{(}\PY{p}{)}
    \PY{n}{mu\PYZus{}cmc}\PY{p}{,} \PY{n}{se\PYZus{}cmc}\PY{p}{,} \PY{n}{ci\PYZus{}cmc} \PY{o}{=} \PY{n}{E\PYZus{}S\PYZus{}cmc}\PY{p}{(}\PY{n}{m}\PY{o}{=}\PY{l+m+mi}{200\PYZus{}000}\PY{p}{,} \PY{n}{seed}\PY{o}{=}\PY{l+m+mi}{7}\PY{p}{)}
    \PY{n}{mu\PYZus{}raw}\PY{p}{,} \PY{n}{se\PYZus{}raw}\PY{p}{,} \PY{n}{ci\PYZus{}raw} \PY{o}{=} \PY{n}{E\PYZus{}S\PYZus{}crudo}\PY{p}{(}\PY{n}{m}\PY{o}{=}\PY{l+m+mi}{200\PYZus{}000}\PY{p}{,} \PY{n}{seed}\PY{o}{=}\PY{l+m+mi}{7}\PY{p}{)}

    \PY{n+nb}{print}\PY{p}{(}\PY{l+s+sa}{f}\PY{l+s+s2}{\PYZdq{}}\PY{l+s+s2}{E[S] exacta      = }\PY{l+s+si}{\PYZob{}}\PY{n}{exact}\PY{l+s+si}{:}\PY{l+s+s2}{.10f}\PY{l+s+si}{\PYZcb{}}\PY{l+s+s2}{\PYZdq{}}\PY{p}{)}
    \PY{n+nb}{print}\PY{p}{(}\PY{l+s+sa}{f}\PY{l+s+s2}{\PYZdq{}}\PY{l+s+s2}{CMC (solo N)     = }\PY{l+s+si}{\PYZob{}}\PY{n}{mu\PYZus{}cmc}\PY{l+s+si}{:}\PY{l+s+s2}{.10f}\PY{l+s+si}{\PYZcb{}}\PY{l+s+s2}{  SE=}\PY{l+s+si}{\PYZob{}}\PY{n}{se\PYZus{}cmc}\PY{l+s+si}{:}\PY{l+s+s2}{.6f}\PY{l+s+si}{\PYZcb{}}\PY{l+s+s2}{  IC95=}\PY{l+s+si}{\PYZob{}}\PY{n}{ci\PYZus{}cmc}\PY{l+s+si}{\PYZcb{}}\PY{l+s+s2}{\PYZdq{}}\PY{p}{)}
    \PY{n+nb}{print}\PY{p}{(}\PY{l+s+sa}{f}\PY{l+s+s2}{\PYZdq{}}\PY{l+s+s2}{Crudo (X\PYZca{}N)      = }\PY{l+s+si}{\PYZob{}}\PY{n}{mu\PYZus{}raw}\PY{l+s+si}{:}\PY{l+s+s2}{.10f}\PY{l+s+si}{\PYZcb{}}\PY{l+s+s2}{  SE=}\PY{l+s+si}{\PYZob{}}\PY{n}{se\PYZus{}raw}\PY{l+s+si}{:}\PY{l+s+s2}{.6f}\PY{l+s+si}{\PYZcb{}}\PY{l+s+s2}{  IC95=}\PY{l+s+si}{\PYZob{}}\PY{n}{ci\PYZus{}raw}\PY{l+s+si}{\PYZcb{}}\PY{l+s+s2}{\PYZdq{}}\PY{p}{)}
    \PY{k}{if} \PY{n}{se\PYZus{}cmc} \PY{o}{\PYZgt{}} \PY{l+m+mi}{0}\PY{p}{:}
        \PY{n+nb}{print}\PY{p}{(}\PY{l+s+sa}{f}\PY{l+s+s2}{\PYZdq{}}\PY{l+s+s2}{Reducción var. ≈ }\PY{l+s+si}{\PYZob{}}\PY{p}{(}\PY{n}{se\PYZus{}raw}\PY{o}{*}\PY{o}{*}\PY{l+m+mi}{2}\PY{p}{)}\PY{o}{/}\PY{p}{(}\PY{n}{se\PYZus{}cmc}\PY{o}{*}\PY{o}{*}\PY{l+m+mi}{2}\PY{p}{)}\PY{l+s+si}{:}\PY{l+s+s2}{.2f}\PY{l+s+si}{\PYZcb{}}\PY{l+s+s2}{x (SE\PYZus{}crudo\PYZca{}2/SE\PYZus{}CMC\PYZca{}2)}\PY{l+s+s2}{\PYZdq{}}\PY{p}{)}
\end{Verbatim}
\end{tcolorbox}

    \begin{Verbatim}[commandchars=\\\{\}]
E[S] exacta      = 0.2019877345
CMC (solo N)     = 0.2017700258  SE=0.000273  IC95=(0.2012342275226516,
0.20230582406464995)
Crudo (X\^{}N)      = 0.2015832854  SE=0.000620  IC95=(0.20036824333333011,
0.2027983275502032)
Reducción var. ≈ 5.14x (SE\_crudo\^{}2/SE\_CMC\^{}2)
    \end{Verbatim}


    % Add a bibliography block to the postdoc
    
    
    
\end{document}
