\documentclass[8pt]{article}
\usepackage[spanish,es-noshorthands]{babel}
\usepackage[utf8]{inputenc}
\usepackage[T1]{fontenc}
\usepackage{geometry}
\geometry{margin=1in}
\usepackage{setspace}
\usepackage{parskip}
\setstretch{1.1}

% ====== Variables ======
\newcommand{\Ciudad}{Puebla, Puebla}
\newcommand{\Fecha}{6 de noviembre de 2025}
\newcommand{\Recomendado}{Heriberto Espino Montelongo}
\newcommand{\Firmante}{Nuria Arroyo Bustamante}
\newcommand{\Cargo}{Estudiante}
\newcommand{\Unidad}{} % opcional
\newcommand{\Institucion}{Universidad de las Américas Puebla}
\newcommand{\Correo}{nuria.arroyobe@udlap.mx}
\newcommand{\Telefono}{4421601140}
% =======================

\usepackage{fontspec}
\setmainfont{EB Garamond}[
    UprightFont = * Regular,
    ItalicFont = * Italic,
    BoldFont = * SemiBold,
    BoldItalicFont = * SemiBold Italic,
]

\begin{document}
\thispagestyle{empty}

\begin{flushright}
\Ciudad, \Fecha
\end{flushright}

\begin{center}
\Large \textbf{Asunto: Carta de recomendación — \Recomendado}
\end{center}

A quien corresponda:

Por medio de la presente, recomiendo ampliamente a \textbf{\Recomendado}. He observado su desempeño en contextos académicos y aplicados de análisis de datos y puedo dar fe de su \textbf{pensamiento crítico}, su \textbf{ética de trabajo} y su proyección profesional.

\textbf{Pensamiento crítico en análisis de datos.} Enmarca problemas con precisión, define objetivos medibles, identifica supuestos y selecciona variables y métodos acordes con la estructura de los datos. Contrasta modelos con \emph{baselines}, realiza validación honesta (hold-out, k-fold), analiza sensibilidad y comunica incertidumbre con intervalos y diagnósticos de error. Rastrea sesgos, documenta decisiones y justifica cada paso con criterios cuantitativos.

\textbf{Ética de trabajo.} Es constante y confiable. Cumple plazos, mantiene control de versiones y bitácoras, protege la confidencialidad y respeta la reproducibilidad. Cuando detecta un error, lo reporta, corrige y documenta sin dilación. Asume tareas complejas y sostiene la calidad diaria del trabajo.

\textbf{Competencias técnicas y metodológicas.} Domina Python para ciencia de datos (pandas, NumPy, SciPy, scikit-learn), SQL para consultas eficientes y Git para trazabilidad. Se mueve entre estadística clásica y aprendizaje automático: pruebas de hipótesis, modelos lineales, árboles, \emph{gradient boosting} y calibración, evitando el sobreajuste y comunicando límites del modelo. Cuida el preprocesamiento, la ingeniería de características y la trazabilidad experimental.

\textbf{Comunicación y colaboración.} Explica resultados a públicos técnicos y no técnicos con lenguaje claro, visualizaciones pertinentes y conclusiones accionables. Integra retroalimentación, colabora con foco y puede liderar discusiones técnicas o trabajar con autonomía.

\textbf{Iniciativa y aprendizaje continuo.} Contrasta alternativas, propone mejoras y adopta herramientas y metodologías nuevas cuando el problema lo exige.

En síntesis, \Recomendado{} combina rigor analítico, responsabilidad profesional y habilidades interpersonales que lo hacen un candidato sólido para cualquier entorno que requiera pensamiento crítico en datos y entregables con estándares altos. Lo recomiendo sin reservas para el programa o puesto que corresponda.

Quedo a disposición para ampliar esta información.

\vspace{1em}
\textbf{Atentamente,}

% ====== Bloque de firma ======
\vspace{3cm} % espacio para firma manuscrita
\noindent\makebox[2.8in]{\hrulefill}

\noindent\textbf{\Firmante}\\
\Cargo\if\relax\detokenize{\Unidad}\relax\else{} \Unidad\fi\\
\Institucion\\
Correo: \Correo{},\quad Tel.: \Telefono
% ============================

\end{document}
